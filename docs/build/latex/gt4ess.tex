%% Generated by Sphinx.
\def\sphinxdocclass{report}
\documentclass[letterpaper,10pt,english]{sphinxmanual}
\ifdefined\pdfpxdimen
   \let\sphinxpxdimen\pdfpxdimen\else\newdimen\sphinxpxdimen
\fi \sphinxpxdimen=.75bp\relax

\usepackage[utf8]{inputenc}
\ifdefined\DeclareUnicodeCharacter
 \ifdefined\DeclareUnicodeCharacterAsOptional
  \DeclareUnicodeCharacter{"00A0}{\nobreakspace}
  \DeclareUnicodeCharacter{"2500}{\sphinxunichar{2500}}
  \DeclareUnicodeCharacter{"2502}{\sphinxunichar{2502}}
  \DeclareUnicodeCharacter{"2514}{\sphinxunichar{2514}}
  \DeclareUnicodeCharacter{"251C}{\sphinxunichar{251C}}
  \DeclareUnicodeCharacter{"2572}{\textbackslash}
 \else
  \DeclareUnicodeCharacter{00A0}{\nobreakspace}
  \DeclareUnicodeCharacter{2500}{\sphinxunichar{2500}}
  \DeclareUnicodeCharacter{2502}{\sphinxunichar{2502}}
  \DeclareUnicodeCharacter{2514}{\sphinxunichar{2514}}
  \DeclareUnicodeCharacter{251C}{\sphinxunichar{251C}}
  \DeclareUnicodeCharacter{2572}{\textbackslash}
 \fi
\fi
\usepackage{cmap}
\usepackage[T1]{fontenc}
\usepackage{amsmath,amssymb,amstext}
\usepackage{babel}
\usepackage{times}
\usepackage[Bjarne]{fncychap}
\usepackage[dontkeepoldnames]{sphinx}

\usepackage{geometry}

% Include hyperref last.
\usepackage{hyperref}
% Fix anchor placement for figures with captions.
\usepackage{hypcap}% it must be loaded after hyperref.
% Set up styles of URL: it should be placed after hyperref.
\urlstyle{same}
\addto\captionsenglish{\renewcommand{\contentsname}{Contents:}}

\addto\captionsenglish{\renewcommand{\figurename}{Fig.}}
\addto\captionsenglish{\renewcommand{\tablename}{Table}}
\addto\captionsenglish{\renewcommand{\literalblockname}{Listing}}

\addto\captionsenglish{\renewcommand{\literalblockcontinuedname}{continued from previous page}}
\addto\captionsenglish{\renewcommand{\literalblockcontinuesname}{continues on next page}}

\addto\extrasenglish{\def\pageautorefname{page}}

\setcounter{tocdepth}{1}



\title{gt4ess Documentation}
\date{Feb 23, 2018}
\release{0.1.0}
\author{Stefano Ubbiali}
\newcommand{\sphinxlogo}{\vbox{}}
\renewcommand{\releasename}{Release}
\makeindex

\begin{document}

\maketitle
\sphinxtableofcontents
\phantomsection\label{\detokenize{index::doc}}



\chapter{API Documentation}
\label{\detokenize{api:welcome-to-gt4ess-s-documentation}}\label{\detokenize{api:api-documentation}}\label{\detokenize{api::doc}}

\section{Axis}
\label{\detokenize{api:axis}}\index{Axis (class in grids.axis)}

\begin{fulllineitems}
\phantomsection\label{\detokenize{api:grids.axis.Axis}}\pysiglinewithargsret{\sphinxbfcode{class }\sphinxcode{grids.axis.}\sphinxbfcode{Axis}}{\emph{coords}, \emph{dims}, \emph{attrs=None}}{}
Class representing a one-dimensional axis. The class API is designed to be similar to
that provided by \sphinxhref{http://xarray.pydata.org/en/stable/generated/xarray.DataArray.html\#xarray.DataArray}{\sphinxcode{xarray.DataArray}}.
\begin{quote}\begin{description}
\item[{Variables}] \leavevmode\begin{itemize}
\item {} 
\sphinxstyleliteralstrong{coords} (\sphinxstyleliteralemphasis{list}) \textendash{} One-dimensional \sphinxhref{https://docs.scipy.org/doc/numpy-1.13.0/reference/generated/numpy.ndarray.html\#numpy.ndarray}{\sphinxcode{numpy.ndarray}} storing axis coordinates, wrapped within a list.

\item {} 
\sphinxstyleliteralstrong{values} (\sphinxstyleliteralemphasis{array\_like}) \textendash{} One-dimensional \sphinxhref{https://docs.scipy.org/doc/numpy-1.13.0/reference/generated/numpy.ndarray.html\#numpy.ndarray}{\sphinxcode{numpy.ndarray}} storing axis coordinates. This attrribute is semantically identical
to \sphinxcode{coords} and it is introduced only for the sake of compliancy with \sphinxhref{http://xarray.pydata.org/en/stable/generated/xarray.DataArray.html\#xarray.DataArray}{\sphinxcode{xarray.DataArray}}’s API.

\item {} 
\sphinxstyleliteralstrong{dims} (\sphinxstyleliteralemphasis{str}) \textendash{} Axis dimension, i.e., label.

\item {} 
\sphinxstyleliteralstrong{attrs} (\sphinxstyleliteralemphasis{dict}) \textendash{} Axis attributes, e.g., the units.

\end{itemize}

\end{description}\end{quote}
\index{\_\_getitem\_\_() (grids.axis.Axis method)}

\begin{fulllineitems}
\phantomsection\label{\detokenize{api:grids.axis.Axis.__getitem__}}\pysiglinewithargsret{\sphinxbfcode{\_\_getitem\_\_}}{\emph{i}}{}
Get direct access to the coordinate vector.
\begin{quote}\begin{description}
\item[{Parameters}] \leavevmode
\sphinxstyleliteralstrong{i} (\sphinxtitleref{int} or {\color{red}\bfseries{}{}`}array\_like) \textendash{} The index, or a sequence of indices.

\item[{Returns}] \leavevmode
The coordinate(s).

\item[{Return type}] \leavevmode
float

\end{description}\end{quote}

\end{fulllineitems}

\index{\_\_init\_\_() (grids.axis.Axis method)}

\begin{fulllineitems}
\phantomsection\label{\detokenize{api:grids.axis.Axis.__init__}}\pysiglinewithargsret{\sphinxbfcode{\_\_init\_\_}}{\emph{coords}, \emph{dims}, \emph{attrs=None}}{}
Constructor.
\begin{quote}\begin{description}
\item[{Parameters}] \leavevmode\begin{itemize}
\item {} 
\sphinxstyleliteralstrong{coords} (\sphinxstyleliteralemphasis{array\_like}) \textendash{} One-dimensional \sphinxhref{https://docs.scipy.org/doc/numpy-1.13.0/reference/generated/numpy.ndarray.html\#numpy.ndarray}{\sphinxcode{numpy.ndarray}} representing the axis values.

\item {} 
\sphinxstyleliteralstrong{dims} (\sphinxstyleliteralemphasis{str}) \textendash{} Axis label.

\item {} 
\sphinxstyleliteralstrong{attrs} (\sphinxtitleref{dict}, optional) \textendash{} 
Axis attributes. This may be used to specify, e.g., the units, which, following the
\sphinxhref{http://cfconventions.org}{CF Conventions}, may be either:
\begin{itemize}
\item {} 
’m’ (meters) or multiples, for height-based coordinates;

\item {} 
’Pa’ (Pascal) or multiples, for pressure-based coordinates;

\item {} 
’K’ (Kelvin), for temperature-based coordinates;

\item {} 
’degrees\_east’, for longitude;

\item {} 
’degrees\_north’, for latitude.

\end{itemize}


\end{itemize}

\end{description}\end{quote}

\end{fulllineitems}

\index{\_check\_arguments() (grids.axis.Axis method)}

\begin{fulllineitems}
\phantomsection\label{\detokenize{api:grids.axis.Axis._check_arguments}}\pysiglinewithargsret{\sphinxbfcode{\_check\_arguments}}{\emph{coords}, \emph{attr}}{}
Convert user-specified units to base units, e.g., km \textendash{}\textgreater{} m, hPa \textendash{}\textgreater{} Pa.
\begin{quote}\begin{description}
\item[{Parameters}] \leavevmode\begin{itemize}
\item {} 
\sphinxstyleliteralstrong{coords} (\sphinxstyleliteralemphasis{array\_like}) \textendash{} One-dimensional \sphinxhref{https://docs.scipy.org/doc/numpy-1.13.0/reference/generated/numpy.ndarray.html\#numpy.ndarray}{\sphinxcode{numpy.ndarray}} representing the axis values.

\item {} 
\sphinxstyleliteralstrong{attrs} (\sphinxstyleliteralemphasis{dict}) \textendash{} 
Axis attributes. This may be used to specify, e.g., the units, which, following the
\sphinxhref{http://cfconventions.org}{CF Conventions}, may be either:
\begin{itemize}
\item {} 
’m’ (meters) or multiples, for height-based coordinates;

\item {} 
’Pa’ (Pascal) or multiples, for pressure-based coordinates;

\item {} 
’K’ (Kelvin), for temperature-based coordinates;

\item {} 
’degrees\_east’, for longitude;

\item {} 
’degrees\_north’, for latitude.

\end{itemize}


\end{itemize}

\item[{Returns}] \leavevmode
The axis coordinates expressed in base units.

\item[{Return type}] \leavevmode
array\_like

\end{description}\end{quote}

\end{fulllineitems}


\end{fulllineitems}



\section{Dynamics}
\label{\detokenize{api:dynamics}}

\subsection{Diagnostics}
\label{\detokenize{api:diagnostics}}\index{DiagnosticIsentropic (class in dycore.diagnostic\_isentropic)}

\begin{fulllineitems}
\phantomsection\label{\detokenize{api:dycore.diagnostic_isentropic.DiagnosticIsentropic}}\pysiglinewithargsret{\sphinxbfcode{class }\sphinxcode{dycore.diagnostic\_isentropic.}\sphinxbfcode{DiagnosticIsentropic}}{\emph{grid}, \emph{imoist}, \emph{backend}}{}
Class implementing the diagnostic steps of the three-dimensional moist isentropic dynamical core
using GT4Py’s stencils.
\index{\_\_init\_\_() (dycore.diagnostic\_isentropic.DiagnosticIsentropic method)}

\begin{fulllineitems}
\phantomsection\label{\detokenize{api:dycore.diagnostic_isentropic.DiagnosticIsentropic.__init__}}\pysiglinewithargsret{\sphinxbfcode{\_\_init\_\_}}{\emph{grid}, \emph{imoist}, \emph{backend}}{}
Constructor.
\begin{quote}\begin{description}
\item[{Parameters}] \leavevmode\begin{itemize}
\item {} 
\sphinxstyleliteralstrong{grid} (\sphinxstyleliteralemphasis{obj}) \textendash{} {\hyperref[\detokenize{api:grids.grid_xyz.GridXYZ}]{\sphinxcrossref{\sphinxcode{GridXYZ}}}} representing the underlying grid.

\item {} 
\sphinxstyleliteralstrong{imoist} (\sphinxstyleliteralemphasis{bool}) \textendash{} \sphinxcode{True} for a moist dynamical core, \sphinxcode{False} otherwise.

\item {} 
\sphinxstyleliteralstrong{backend} (\sphinxstyleliteralemphasis{obj}) \textendash{} \sphinxcode{gridtools.mode} specifying the backend for the GT4Py’s stencils.

\end{itemize}

\end{description}\end{quote}

\end{fulllineitems}

\index{\_defs\_stencil\_diagnosing\_conservative\_variables() (dycore.diagnostic\_isentropic.DiagnosticIsentropic method)}

\begin{fulllineitems}
\phantomsection\label{\detokenize{api:dycore.diagnostic_isentropic.DiagnosticIsentropic._defs_stencil_diagnosing_conservative_variables}}\pysiglinewithargsret{\sphinxbfcode{\_defs\_stencil\_diagnosing\_conservative\_variables}}{\emph{in\_s}, \emph{in\_u}, \emph{in\_v}, \emph{in\_qv=None}, \emph{in\_qc=None}, \emph{in\_qr=None}}{}
GT4Py’s stencil diagnosing the conservative model variables, i.e., the momentums - \(U\) and \(V\) -
and, optionally, the mass of water constituents - \(Q_v\), \(Q_c\) and \(Q_r\).
\begin{quote}\begin{description}
\item[{Parameters}] \leavevmode\begin{itemize}
\item {} 
\sphinxstyleliteralstrong{in\_s} (\sphinxstyleliteralemphasis{obj}) \textendash{} \sphinxcode{gridtools.Equation} representing the isentropic density.

\item {} 
\sphinxstyleliteralstrong{in\_u} (\sphinxstyleliteralemphasis{obj}) \textendash{} \sphinxcode{gridtools.Equation} representing the \(x\)-velocity.

\item {} 
\sphinxstyleliteralstrong{in\_v} (\sphinxstyleliteralemphasis{obj}) \textendash{} \sphinxcode{gridtools.Equation} representing the \(y\)-velocity.

\item {} 
\sphinxstyleliteralstrong{in\_qv} (\sphinxtitleref{obj}, optional) \textendash{} \sphinxcode{gridtools.Equation} representing the mass fraction of water vapour.

\item {} 
\sphinxstyleliteralstrong{in\_qc} (\sphinxtitleref{obj}, optional) \textendash{} \sphinxcode{gridtools.Equation} representing the mass fraction of cloud water.

\item {} 
\sphinxstyleliteralstrong{in\_qr} (\sphinxtitleref{obj}, optional) \textendash{} \sphinxcode{gridtools.Equation} representing the mass fraction of precipitation water.

\end{itemize}

\item[{Returns}] \leavevmode
\begin{itemize}
\item {} 
\sphinxstylestrong{out\_U} (\sphinxstyleemphasis{obj}) \textendash{} \sphinxcode{gridtools.Equation} representing the diagnosed \(U\).

\item {} 
\sphinxstylestrong{out\_V} (\sphinxstyleemphasis{obj}) \textendash{} \sphinxcode{gridtools.Equation} representing the diagnosed \(V\).

\item {} 
\sphinxstylestrong{out\_Qv} (\sphinxtitleref{obj}, optional) \textendash{} \sphinxcode{gridtools.Equation} representing the diagnosed \(Qv\).

\item {} 
\sphinxstylestrong{out\_Qc} (\sphinxtitleref{obj}, optional) \textendash{} \sphinxcode{gridtools.Equation} representing the diagnosed \(Qc\).

\item {} 
\sphinxstylestrong{out\_Qr} (\sphinxtitleref{obj}, optional) \textendash{} \sphinxcode{gridtools.Equation} representing the diagnosed \(Qr\).

\end{itemize}


\end{description}\end{quote}

\end{fulllineitems}

\index{\_defs\_stencil\_diagnosing\_height() (dycore.diagnostic\_isentropic.DiagnosticIsentropic method)}

\begin{fulllineitems}
\phantomsection\label{\detokenize{api:dycore.diagnostic_isentropic.DiagnosticIsentropic._defs_stencil_diagnosing_height}}\pysiglinewithargsret{\sphinxbfcode{\_defs\_stencil\_diagnosing\_height}}{\emph{in\_theta}, \emph{in\_exn}, \emph{in\_p}, \emph{in\_h}}{}
GT4Py’s stencil diagnosing the geometric height of the isentropes.
\begin{quote}\begin{description}
\item[{Parameters}] \leavevmode\begin{itemize}
\item {} 
\sphinxstyleliteralstrong{in\_theta} (\sphinxstyleliteralemphasis{obj}) \textendash{} \sphinxcode{gridtools.Equation} representing the vertical half levels.

\item {} 
\sphinxstyleliteralstrong{in\_exn} (\sphinxstyleliteralemphasis{obj}) \textendash{} \sphinxcode{gridtools.Equation} representing the Exner function.

\item {} 
\sphinxstyleliteralstrong{in\_p} (\sphinxstyleliteralemphasis{obj}) \textendash{} \sphinxcode{gridtools.Equation} representing the pressure.

\item {} 
\sphinxstyleliteralstrong{in\_h} (\sphinxstyleliteralemphasis{obj}) \textendash{} \sphinxcode{gridtools.Equation} representing the geometric height of the isentropes.

\end{itemize}

\item[{Returns}] \leavevmode
\sphinxcode{gridtools.Equation} representing the diagnosed geometric height of the isentropes.

\item[{Return type}] \leavevmode
obj

\end{description}\end{quote}

\end{fulllineitems}

\index{\_defs\_stencil\_diagnosing\_montgomery() (dycore.diagnostic\_isentropic.DiagnosticIsentropic method)}

\begin{fulllineitems}
\phantomsection\label{\detokenize{api:dycore.diagnostic_isentropic.DiagnosticIsentropic._defs_stencil_diagnosing_montgomery}}\pysiglinewithargsret{\sphinxbfcode{\_defs\_stencil\_diagnosing\_montgomery}}{\emph{in\_exn}, \emph{in\_mtg}}{}
GT4Py’s stencil diagnosing the Exner function.
\begin{quote}\begin{description}
\item[{Parameters}] \leavevmode\begin{itemize}
\item {} 
\sphinxstyleliteralstrong{in\_exn} (\sphinxstyleliteralemphasis{obj}) \textendash{} \sphinxcode{gridtools.Equation} representing the Exner function.

\item {} 
\sphinxstyleliteralstrong{in\_mtg} (\sphinxstyleliteralemphasis{obj}) \textendash{} \sphinxcode{gridtools.Equation} representing the Montgomery potential.

\end{itemize}

\item[{Returns}] \leavevmode
\sphinxcode{gridtools.Equation} representing the diagnosed Montgomery potential.

\item[{Return type}] \leavevmode
obj

\end{description}\end{quote}

\end{fulllineitems}

\index{\_defs\_stencil\_diagnosing\_pressure() (dycore.diagnostic\_isentropic.DiagnosticIsentropic method)}

\begin{fulllineitems}
\phantomsection\label{\detokenize{api:dycore.diagnostic_isentropic.DiagnosticIsentropic._defs_stencil_diagnosing_pressure}}\pysiglinewithargsret{\sphinxbfcode{\_defs\_stencil\_diagnosing\_pressure}}{\emph{in\_s}, \emph{in\_p}}{}
GT4Py’s stencil diagnosing the pressure.
\begin{quote}\begin{description}
\item[{Parameters}] \leavevmode\begin{itemize}
\item {} 
\sphinxstyleliteralstrong{in\_s} (\sphinxstyleliteralemphasis{obj}) \textendash{} \sphinxcode{gridtools.Equation} representing the isentropic density.

\item {} 
\sphinxstyleliteralstrong{in\_p} (\sphinxstyleliteralemphasis{obj}) \textendash{} \sphinxcode{gridtools.Equation} representing the pressure.

\end{itemize}

\item[{Returns}] \leavevmode
\sphinxcode{gridtools.Equation} representing the diagnosed pressure.

\item[{Return type}] \leavevmode
obj

\end{description}\end{quote}

\end{fulllineitems}

\index{\_defs\_stencil\_diagnosing\_velocity\_x() (dycore.diagnostic\_isentropic.DiagnosticIsentropic method)}

\begin{fulllineitems}
\phantomsection\label{\detokenize{api:dycore.diagnostic_isentropic.DiagnosticIsentropic._defs_stencil_diagnosing_velocity_x}}\pysiglinewithargsret{\sphinxbfcode{\_defs\_stencil\_diagnosing\_velocity\_x}}{\emph{in\_s}, \emph{in\_U}}{}
GT4Py’s stencil diagnosing the \(x\)-component of the velocity.
\begin{quote}\begin{description}
\item[{Parameters}] \leavevmode\begin{itemize}
\item {} 
\sphinxstyleliteralstrong{in\_s} (\sphinxstyleliteralemphasis{obj}) \textendash{} \sphinxcode{gridtools.Equation} representing the isentropic density.

\item {} 
\sphinxstyleliteralstrong{in\_U} (\sphinxstyleliteralemphasis{obj}) \textendash{} \sphinxcode{gridtools.Equation} representing the \(x\)-momentum.

\end{itemize}

\item[{Returns}] \leavevmode
\sphinxcode{gridtools.Equation} representing the diagnosed \(x\)-velocity.

\item[{Return type}] \leavevmode
obj

\end{description}\end{quote}

\end{fulllineitems}

\index{\_defs\_stencil\_diagnosing\_velocity\_y() (dycore.diagnostic\_isentropic.DiagnosticIsentropic method)}

\begin{fulllineitems}
\phantomsection\label{\detokenize{api:dycore.diagnostic_isentropic.DiagnosticIsentropic._defs_stencil_diagnosing_velocity_y}}\pysiglinewithargsret{\sphinxbfcode{\_defs\_stencil\_diagnosing\_velocity\_y}}{\emph{in\_s}, \emph{in\_V}}{}
GT4Py’s stencil diagnosing the \(y\)-component of the velocity.
\begin{quote}\begin{description}
\item[{Parameters}] \leavevmode\begin{itemize}
\item {} 
\sphinxstyleliteralstrong{in\_s} (\sphinxstyleliteralemphasis{obj}) \textendash{} \sphinxcode{gridtools.Equation} representing the isentropic density.

\item {} 
\sphinxstyleliteralstrong{in\_V} (\sphinxstyleliteralemphasis{obj}) \textendash{} \sphinxcode{gridtools.Equation} representing the \(y\)-momentum.

\end{itemize}

\item[{Returns}] \leavevmode
\sphinxcode{gridtools.Equation} representing the diagnosed \(y\)-velocity.

\item[{Return type}] \leavevmode
obj

\end{description}\end{quote}

\end{fulllineitems}

\index{\_defs\_stencil\_diagnosing\_water\_constituents() (dycore.diagnostic\_isentropic.DiagnosticIsentropic method)}

\begin{fulllineitems}
\phantomsection\label{\detokenize{api:dycore.diagnostic_isentropic.DiagnosticIsentropic._defs_stencil_diagnosing_water_constituents}}\pysiglinewithargsret{\sphinxbfcode{\_defs\_stencil\_diagnosing\_water\_constituents}}{\emph{in\_s}, \emph{in\_U}, \emph{in\_V}, \emph{in\_Qv}, \emph{in\_Qc}, \emph{in\_Qr}}{}
GT4Py’s stencil diagnosing the water constituents.
\begin{quote}\begin{description}
\item[{Parameters}] \leavevmode\begin{itemize}
\item {} 
\sphinxstyleliteralstrong{in\_s} (\sphinxstyleliteralemphasis{obj}) \textendash{} \sphinxcode{gridtools.Equation} representing the isentropic density.

\item {} 
\sphinxstyleliteralstrong{in\_U} (\sphinxstyleliteralemphasis{obj}) \textendash{} \sphinxcode{gridtools.Equation} representing the \(x\)-momentum.

\item {} 
\sphinxstyleliteralstrong{in\_V} (\sphinxstyleliteralemphasis{obj}) \textendash{} \sphinxcode{gridtools.Equation} representing the \(y\)-momentum.

\item {} 
\sphinxstyleliteralstrong{in\_Qv} (\sphinxstyleliteralemphasis{obj}) \textendash{} \sphinxcode{gridtools.Equation} representing the mass of water vapour.

\item {} 
\sphinxstyleliteralstrong{in\_Qc} (\sphinxstyleliteralemphasis{obj}) \textendash{} \sphinxcode{gridtools.Equation} representing the mass of cloud water.

\item {} 
\sphinxstyleliteralstrong{in\_Qr} (\sphinxstyleliteralemphasis{obj}) \textendash{} \sphinxcode{gridtools.Equation} representing the mass of precipitation water.

\end{itemize}

\item[{Returns}] \leavevmode
\begin{itemize}
\item {} 
\sphinxstylestrong{out\_qv} (\sphinxstyleemphasis{obj}) \textendash{} \sphinxcode{gridtools.Equation} representing the diagnosed mass fraction of water vapour.

\item {} 
\sphinxstylestrong{out\_qc} (\sphinxstyleemphasis{obj}) \textendash{} \sphinxcode{gridtools.Equation} representing the diagnosed mass fraction of cloud water.

\item {} 
\sphinxstylestrong{out\_qr} (\sphinxstyleemphasis{obj}) \textendash{} \sphinxcode{gridtools.Equation} representing the diagnosed mass fraction of precipitation water.

\end{itemize}


\end{description}\end{quote}

\end{fulllineitems}

\index{\_initialize\_stencil\_diagnosing\_conservative\_variables() (dycore.diagnostic\_isentropic.DiagnosticIsentropic method)}

\begin{fulllineitems}
\phantomsection\label{\detokenize{api:dycore.diagnostic_isentropic.DiagnosticIsentropic._initialize_stencil_diagnosing_conservative_variables}}\pysiglinewithargsret{\sphinxbfcode{\_initialize\_stencil\_diagnosing\_conservative\_variables}}{}{}
Initialize the GT4Py’s stencil in charge of diagnosing the conservative model variables.

\end{fulllineitems}

\index{\_initialize\_stencil\_diagnosing\_height() (dycore.diagnostic\_isentropic.DiagnosticIsentropic method)}

\begin{fulllineitems}
\phantomsection\label{\detokenize{api:dycore.diagnostic_isentropic.DiagnosticIsentropic._initialize_stencil_diagnosing_height}}\pysiglinewithargsret{\sphinxbfcode{\_initialize\_stencil\_diagnosing\_height}}{}{}
Initialize the GT4Py’s stencil in charge of diagnosing the geometric height of the half-level isentropes.

\end{fulllineitems}

\index{\_initialize\_stencil\_diagnosing\_montgomery() (dycore.diagnostic\_isentropic.DiagnosticIsentropic method)}

\begin{fulllineitems}
\phantomsection\label{\detokenize{api:dycore.diagnostic_isentropic.DiagnosticIsentropic._initialize_stencil_diagnosing_montgomery}}\pysiglinewithargsret{\sphinxbfcode{\_initialize\_stencil\_diagnosing\_montgomery}}{}{}
Initialize the GT4Py’s stencil in charge of diagnosing the Montgomery potential.

\end{fulllineitems}

\index{\_initialize\_stencil\_diagnosing\_pressure() (dycore.diagnostic\_isentropic.DiagnosticIsentropic method)}

\begin{fulllineitems}
\phantomsection\label{\detokenize{api:dycore.diagnostic_isentropic.DiagnosticIsentropic._initialize_stencil_diagnosing_pressure}}\pysiglinewithargsret{\sphinxbfcode{\_initialize\_stencil\_diagnosing\_pressure}}{}{}
Initialize the GT4Py’s stencil in charge of diagnosing the pressure.

\end{fulllineitems}

\index{\_initialize\_stencil\_diagnosing\_velocity\_x() (dycore.diagnostic\_isentropic.DiagnosticIsentropic method)}

\begin{fulllineitems}
\phantomsection\label{\detokenize{api:dycore.diagnostic_isentropic.DiagnosticIsentropic._initialize_stencil_diagnosing_velocity_x}}\pysiglinewithargsret{\sphinxbfcode{\_initialize\_stencil\_diagnosing\_velocity\_x}}{}{}
Initialize the GT4Py’s stencil in charge of diagnosing the \(x\)-component of the velocity.

\end{fulllineitems}

\index{\_initialize\_stencil\_diagnosing\_velocity\_y() (dycore.diagnostic\_isentropic.DiagnosticIsentropic method)}

\begin{fulllineitems}
\phantomsection\label{\detokenize{api:dycore.diagnostic_isentropic.DiagnosticIsentropic._initialize_stencil_diagnosing_velocity_y}}\pysiglinewithargsret{\sphinxbfcode{\_initialize\_stencil\_diagnosing\_velocity\_y}}{}{}
Initialize the GT4Py’s stencil in charge of diagnosing the \(y\)-component of the velocity.

\end{fulllineitems}

\index{\_initialize\_stencil\_diagnosing\_water\_constituents() (dycore.diagnostic\_isentropic.DiagnosticIsentropic method)}

\begin{fulllineitems}
\phantomsection\label{\detokenize{api:dycore.diagnostic_isentropic.DiagnosticIsentropic._initialize_stencil_diagnosing_water_constituents}}\pysiglinewithargsret{\sphinxbfcode{\_initialize\_stencil\_diagnosing\_water\_constituents}}{}{}
Initialize the GT4Py’s stencil in charge of diagnosing the water constituents.

\end{fulllineitems}

\index{\_set\_inputs\_to\_stencil\_diagnosing\_conservative\_variables() (dycore.diagnostic\_isentropic.DiagnosticIsentropic method)}

\begin{fulllineitems}
\phantomsection\label{\detokenize{api:dycore.diagnostic_isentropic.DiagnosticIsentropic._set_inputs_to_stencil_diagnosing_conservative_variables}}\pysiglinewithargsret{\sphinxbfcode{\_set\_inputs\_to\_stencil\_diagnosing\_conservative\_variables}}{\emph{s}, \emph{u}, \emph{v}, \emph{qv}, \emph{qc}, \emph{qr}}{}
Update the private instance attributes which serve as inputs to the GT4Py’s stencil which diagnoses
the conservative variables.
\begin{quote}\begin{description}
\item[{Parameters}] \leavevmode\begin{itemize}
\item {} 
\sphinxstyleliteralstrong{s} (\sphinxstyleliteralemphasis{array\_like}) \textendash{} \sphinxhref{https://docs.scipy.org/doc/numpy-1.13.0/reference/generated/numpy.ndarray.html\#numpy.ndarray}{\sphinxcode{numpy.ndarray}} with shape (\sphinxcode{nx}, \sphinxcode{ny}, \sphinxcode{nz}) representing the isentropic density.

\item {} 
\sphinxstyleliteralstrong{u} (\sphinxstyleliteralemphasis{array\_like}) \textendash{} \sphinxhref{https://docs.scipy.org/doc/numpy-1.13.0/reference/generated/numpy.ndarray.html\#numpy.ndarray}{\sphinxcode{numpy.ndarray}} with shape (\sphinxcode{nx+1}, \sphinxcode{ny}, \sphinxcode{nz}) representing the \(x\)-velocity.

\item {} 
\sphinxstyleliteralstrong{v} (\sphinxstyleliteralemphasis{array\_like}) \textendash{} \sphinxhref{https://docs.scipy.org/doc/numpy-1.13.0/reference/generated/numpy.ndarray.html\#numpy.ndarray}{\sphinxcode{numpy.ndarray}} with shape (\sphinxcode{nx}, \sphinxcode{ny+1}, \sphinxcode{nz}) representing the \(y\)-velocity.

\item {} 
\sphinxstyleliteralstrong{qv} (\sphinxstyleliteralemphasis{array\_like}) \textendash{} \sphinxhref{https://docs.scipy.org/doc/numpy-1.13.0/reference/generated/numpy.ndarray.html\#numpy.ndarray}{\sphinxcode{numpy.ndarray}} with shape (\sphinxcode{nx}, \sphinxcode{ny}, \sphinxcode{nz}) representing the mass fraction of
water vapour.

\item {} 
\sphinxstyleliteralstrong{qc} (\sphinxstyleliteralemphasis{array\_like}) \textendash{} \sphinxhref{https://docs.scipy.org/doc/numpy-1.13.0/reference/generated/numpy.ndarray.html\#numpy.ndarray}{\sphinxcode{numpy.ndarray}} with shape (\sphinxcode{nx}, \sphinxcode{ny}, \sphinxcode{nz}) representing the mass fraction of
cloud water.

\item {} 
\sphinxstyleliteralstrong{qr} (\sphinxstyleliteralemphasis{array\_like}) \textendash{} \sphinxhref{https://docs.scipy.org/doc/numpy-1.13.0/reference/generated/numpy.ndarray.html\#numpy.ndarray}{\sphinxcode{numpy.ndarray}} with shape (\sphinxcode{nx}, \sphinxcode{ny}, \sphinxcode{nz}) representing the mass fraction of
precipitation water.

\end{itemize}

\end{description}\end{quote}

\end{fulllineitems}

\index{\_set\_inputs\_to\_stencil\_diagnosing\_pressure() (dycore.diagnostic\_isentropic.DiagnosticIsentropic method)}

\begin{fulllineitems}
\phantomsection\label{\detokenize{api:dycore.diagnostic_isentropic.DiagnosticIsentropic._set_inputs_to_stencil_diagnosing_pressure}}\pysiglinewithargsret{\sphinxbfcode{\_set\_inputs\_to\_stencil\_diagnosing\_pressure}}{\emph{s}}{}
Update the private instance attributes which serve as inputs to the GT4Py’s stencil which diagnoses the pressure.
\begin{quote}\begin{description}
\item[{Parameters}] \leavevmode
\sphinxstyleliteralstrong{s} (\sphinxstyleliteralemphasis{array\_like}) \textendash{} \sphinxhref{https://docs.scipy.org/doc/numpy-1.13.0/reference/generated/numpy.ndarray.html\#numpy.ndarray}{\sphinxcode{numpy.ndarray}} with shape (\sphinxcode{nx}, \sphinxcode{ny}, \sphinxcode{nz}) representing the isentropic density.

\end{description}\end{quote}

\end{fulllineitems}

\index{\_set\_inputs\_to\_stencils\_diagnosing\_nonconservative\_variables() (dycore.diagnostic\_isentropic.DiagnosticIsentropic method)}

\begin{fulllineitems}
\phantomsection\label{\detokenize{api:dycore.diagnostic_isentropic.DiagnosticIsentropic._set_inputs_to_stencils_diagnosing_nonconservative_variables}}\pysiglinewithargsret{\sphinxbfcode{\_set\_inputs\_to\_stencils\_diagnosing\_nonconservative\_variables}}{\emph{s}, \emph{U}, \emph{V}, \emph{Qv}, \emph{Qc}, \emph{Qr}}{}
Update the private instance attributes which serve as inputs to the GT4Py’s stencils which diagnose
the nonconservative variables.
\begin{quote}\begin{description}
\item[{Parameters}] \leavevmode\begin{itemize}
\item {} 
\sphinxstyleliteralstrong{s} (\sphinxstyleliteralemphasis{array\_like}) \textendash{} \sphinxhref{https://docs.scipy.org/doc/numpy-1.13.0/reference/generated/numpy.ndarray.html\#numpy.ndarray}{\sphinxcode{numpy.ndarray}} with shape (\sphinxcode{nx}, \sphinxcode{ny}, \sphinxcode{nz}) representing the isentropic density.

\item {} 
\sphinxstyleliteralstrong{U} (\sphinxstyleliteralemphasis{array\_like}) \textendash{} \sphinxhref{https://docs.scipy.org/doc/numpy-1.13.0/reference/generated/numpy.ndarray.html\#numpy.ndarray}{\sphinxcode{numpy.ndarray}} with shape (\sphinxcode{nx}, \sphinxcode{ny}, \sphinxcode{nz}) representing the \(x\)-velocity.

\item {} 
\sphinxstyleliteralstrong{V} (\sphinxstyleliteralemphasis{array\_like}) \textendash{} \sphinxhref{https://docs.scipy.org/doc/numpy-1.13.0/reference/generated/numpy.ndarray.html\#numpy.ndarray}{\sphinxcode{numpy.ndarray}} with shape (\sphinxcode{nx}, \sphinxcode{ny}, \sphinxcode{nz}) representing the \(y\)-velocity.

\item {} 
\sphinxstyleliteralstrong{Qv} (\sphinxtitleref{array\_like}, optional) \textendash{} \sphinxhref{https://docs.scipy.org/doc/numpy-1.13.0/reference/generated/numpy.ndarray.html\#numpy.ndarray}{\sphinxcode{numpy.ndarray}} with shape (\sphinxcode{nx}, \sphinxcode{ny}, \sphinxcode{nz}) representing the mass of water vapour.

\item {} 
\sphinxstyleliteralstrong{Qc} (\sphinxtitleref{array\_like}, optional) \textendash{} \sphinxhref{https://docs.scipy.org/doc/numpy-1.13.0/reference/generated/numpy.ndarray.html\#numpy.ndarray}{\sphinxcode{numpy.ndarray}} with shape (\sphinxcode{nx}, \sphinxcode{ny}, \sphinxcode{nz}) representing the mass of cloud water.

\item {} 
\sphinxstyleliteralstrong{Qr} (\sphinxtitleref{array\_like}, optional) \textendash{} \sphinxhref{https://docs.scipy.org/doc/numpy-1.13.0/reference/generated/numpy.ndarray.html\#numpy.ndarray}{\sphinxcode{numpy.ndarray}} with shape (\sphinxcode{nx}, \sphinxcode{ny}, \sphinxcode{nz}) representing the mass of precipitation water.

\end{itemize}

\end{description}\end{quote}

\end{fulllineitems}

\index{get\_conservative\_variables() (dycore.diagnostic\_isentropic.DiagnosticIsentropic method)}

\begin{fulllineitems}
\phantomsection\label{\detokenize{api:dycore.diagnostic_isentropic.DiagnosticIsentropic.get_conservative_variables}}\pysiglinewithargsret{\sphinxbfcode{get\_conservative\_variables}}{\emph{s}, \emph{u}, \emph{v}, \emph{qv=None}, \emph{qc=None}, \emph{qr=None}}{}
Diagnosis of the conservative model variables, i.e., the momentums - \(U\) and \(V\) -
and, optionally, the mass of water constituents - \(Q_v\), \(Q_c\) and \(Q_r\).
\begin{quote}\begin{description}
\item[{Parameters}] \leavevmode\begin{itemize}
\item {} 
\sphinxstyleliteralstrong{s} (\sphinxstyleliteralemphasis{array\_like}) \textendash{} \sphinxhref{https://docs.scipy.org/doc/numpy-1.13.0/reference/generated/numpy.ndarray.html\#numpy.ndarray}{\sphinxcode{numpy.ndarray}} with shape (\sphinxcode{nx}, \sphinxcode{ny}, \sphinxcode{nz}) representing the isentropic density.

\item {} 
\sphinxstyleliteralstrong{u} (\sphinxstyleliteralemphasis{array\_like}) \textendash{} \sphinxhref{https://docs.scipy.org/doc/numpy-1.13.0/reference/generated/numpy.ndarray.html\#numpy.ndarray}{\sphinxcode{numpy.ndarray}} with shape (\sphinxcode{nx+1}, \sphinxcode{ny}, \sphinxcode{nz}) representing the \(x\)-velocity.

\item {} 
\sphinxstyleliteralstrong{v} (\sphinxstyleliteralemphasis{array\_like}) \textendash{} \sphinxhref{https://docs.scipy.org/doc/numpy-1.13.0/reference/generated/numpy.ndarray.html\#numpy.ndarray}{\sphinxcode{numpy.ndarray}} with shape (\sphinxcode{nx}, \sphinxcode{ny+1}, \sphinxcode{nz}) representing the \(y\)-velocity.

\item {} 
\sphinxstyleliteralstrong{qv} (\sphinxtitleref{array\_like}, optional) \textendash{} \sphinxhref{https://docs.scipy.org/doc/numpy-1.13.0/reference/generated/numpy.ndarray.html\#numpy.ndarray}{\sphinxcode{numpy.ndarray}} with shape (\sphinxcode{nx}, \sphinxcode{ny}, \sphinxcode{nz}) representing the mass fraction of
water vapour.

\item {} 
\sphinxstyleliteralstrong{qc} (\sphinxtitleref{array\_like}, optional) \textendash{} \sphinxhref{https://docs.scipy.org/doc/numpy-1.13.0/reference/generated/numpy.ndarray.html\#numpy.ndarray}{\sphinxcode{numpy.ndarray}} with shape (\sphinxcode{nx}, \sphinxcode{ny}, \sphinxcode{nz}) representing the mass fraction of
cloud water.

\item {} 
\sphinxstyleliteralstrong{qr} (\sphinxtitleref{array\_like}, optional) \textendash{} \sphinxhref{https://docs.scipy.org/doc/numpy-1.13.0/reference/generated/numpy.ndarray.html\#numpy.ndarray}{\sphinxcode{numpy.ndarray}} with shape (\sphinxcode{nx}, \sphinxcode{ny}, \sphinxcode{nz}) representing the mass fraction of
precipitation water.

\end{itemize}

\item[{Returns}] \leavevmode
\begin{itemize}
\item {} 
\sphinxstylestrong{U} (\sphinxstyleemphasis{array\_like}) \textendash{} \sphinxhref{https://docs.scipy.org/doc/numpy-1.13.0/reference/generated/numpy.ndarray.html\#numpy.ndarray}{\sphinxcode{numpy.ndarray}} with shape (\sphinxcode{nx}, \sphinxcode{ny}, \sphinxcode{nz}) representing the diagnosed \(U\).

\item {} 
\sphinxstylestrong{V} (\sphinxstyleemphasis{array\_like}) \textendash{} \sphinxhref{https://docs.scipy.org/doc/numpy-1.13.0/reference/generated/numpy.ndarray.html\#numpy.ndarray}{\sphinxcode{numpy.ndarray}} with shape (\sphinxcode{nx}, \sphinxcode{ny}, \sphinxcode{nz}) representing the diagnosed \(V\).

\item {} 
\sphinxstylestrong{Qv} (\sphinxtitleref{array\_like}, optional) \textendash{} \sphinxhref{https://docs.scipy.org/doc/numpy-1.13.0/reference/generated/numpy.ndarray.html\#numpy.ndarray}{\sphinxcode{numpy.ndarray}} with shape (\sphinxcode{nx}, \sphinxcode{ny}, \sphinxcode{nz}) representing the diagnosed \(Q_v\).

\item {} 
\sphinxstylestrong{Qc} (\sphinxtitleref{array\_like}, optional) \textendash{} \sphinxhref{https://docs.scipy.org/doc/numpy-1.13.0/reference/generated/numpy.ndarray.html\#numpy.ndarray}{\sphinxcode{numpy.ndarray}} with shape (\sphinxcode{nx}, \sphinxcode{ny}, \sphinxcode{nz}) representing the diagnosed \(Q_c\).

\item {} 
\sphinxstylestrong{Qr} (\sphinxtitleref{array\_like}, optional) \textendash{} \sphinxhref{https://docs.scipy.org/doc/numpy-1.13.0/reference/generated/numpy.ndarray.html\#numpy.ndarray}{\sphinxcode{numpy.ndarray}} with shape (\sphinxcode{nx}, \sphinxcode{ny}, \sphinxcode{nz}) representing the diagnosed \(Q_r\).

\end{itemize}


\end{description}\end{quote}

\end{fulllineitems}

\index{get\_diagnostic\_variables() (dycore.diagnostic\_isentropic.DiagnosticIsentropic method)}

\begin{fulllineitems}
\phantomsection\label{\detokenize{api:dycore.diagnostic_isentropic.DiagnosticIsentropic.get_diagnostic_variables}}\pysiglinewithargsret{\sphinxbfcode{get\_diagnostic\_variables}}{\emph{s}, \emph{pt}}{}
Diagnosis of the pressure, the Exner function, the Montgomery potential, and the geometric height of the
potential temperature surfaces.
\begin{quote}\begin{description}
\item[{Parameters}] \leavevmode\begin{itemize}
\item {} 
\sphinxstyleliteralstrong{s} (\sphinxstyleliteralemphasis{array\_like}) \textendash{} \sphinxhref{https://docs.scipy.org/doc/numpy-1.13.0/reference/generated/numpy.ndarray.html\#numpy.ndarray}{\sphinxcode{numpy.ndarray}} with shape (\sphinxcode{nx}, \sphinxcode{ny}, \sphinxcode{nz}) representing the isentropic density.

\item {} 
\sphinxstyleliteralstrong{pt} (\sphinxstyleliteralemphasis{float}) \textendash{} Boundary value for the pressure at the top of the domain.

\end{itemize}

\item[{Returns}] \leavevmode
\begin{itemize}
\item {} 
\sphinxstylestrong{p} (\sphinxstyleemphasis{array\_like}) \textendash{} \sphinxhref{https://docs.scipy.org/doc/numpy-1.13.0/reference/generated/numpy.ndarray.html\#numpy.ndarray}{\sphinxcode{numpy.ndarray}} with shape (\sphinxcode{nx}, \sphinxcode{ny}, \sphinxcode{nz+1}) representing the diagnosed
pressure.

\item {} 
\sphinxstylestrong{exn} (\sphinxstyleemphasis{array\_like}) \textendash{} \sphinxhref{https://docs.scipy.org/doc/numpy-1.13.0/reference/generated/numpy.ndarray.html\#numpy.ndarray}{\sphinxcode{numpy.ndarray}} with shape (\sphinxcode{nx}, \sphinxcode{ny}, \sphinxcode{nz+1}) representing the diagnosed
Exner function.

\item {} 
\sphinxstylestrong{mtg} (\sphinxstyleemphasis{array\_like}) \textendash{} \sphinxhref{https://docs.scipy.org/doc/numpy-1.13.0/reference/generated/numpy.ndarray.html\#numpy.ndarray}{\sphinxcode{numpy.ndarray}} with shape (\sphinxcode{nx}, \sphinxcode{ny}, \sphinxcode{nz}) representing the diagnosed
Montgomery potential.

\item {} 
\sphinxstylestrong{h} (\sphinxstyleemphasis{array\_like}) \textendash{} \sphinxhref{https://docs.scipy.org/doc/numpy-1.13.0/reference/generated/numpy.ndarray.html\#numpy.ndarray}{\sphinxcode{numpy.ndarray}} with shape (\sphinxcode{nx}, \sphinxcode{ny}, \sphinxcode{nz+1}) representing the diagnosed
geometric height of the potential temperature surfaces.

\end{itemize}


\end{description}\end{quote}

\end{fulllineitems}

\index{get\_nonconservative\_variables() (dycore.diagnostic\_isentropic.DiagnosticIsentropic method)}

\begin{fulllineitems}
\phantomsection\label{\detokenize{api:dycore.diagnostic_isentropic.DiagnosticIsentropic.get_nonconservative_variables}}\pysiglinewithargsret{\sphinxbfcode{get\_nonconservative\_variables}}{\emph{s}, \emph{U}, \emph{V}, \emph{Qv=None}, \emph{Qc=None}, \emph{Qr=None}}{}
Diagnosis of the non-conservative model variables, i.e., the velocity components - \(u\) and \(v\) -
and, optionally, the mass fraction of the water constituents - \(q_v\), \(q_c\) and \(q_r\).
\begin{quote}\begin{description}
\item[{Parameters}] \leavevmode\begin{itemize}
\item {} 
\sphinxstyleliteralstrong{s} (\sphinxstyleliteralemphasis{array\_like}) \textendash{} \sphinxhref{https://docs.scipy.org/doc/numpy-1.13.0/reference/generated/numpy.ndarray.html\#numpy.ndarray}{\sphinxcode{numpy.ndarray}} with shape (\sphinxcode{nx}, \sphinxcode{ny}, \sphinxcode{nz}) representing the isentropic density.

\item {} 
\sphinxstyleliteralstrong{U} (\sphinxstyleliteralemphasis{array\_like}) \textendash{} \sphinxhref{https://docs.scipy.org/doc/numpy-1.13.0/reference/generated/numpy.ndarray.html\#numpy.ndarray}{\sphinxcode{numpy.ndarray}} with shape (\sphinxcode{nx}, \sphinxcode{ny}, \sphinxcode{nz}) representing the \(x\)-velocity.

\item {} 
\sphinxstyleliteralstrong{V} (\sphinxstyleliteralemphasis{array\_like}) \textendash{} \sphinxhref{https://docs.scipy.org/doc/numpy-1.13.0/reference/generated/numpy.ndarray.html\#numpy.ndarray}{\sphinxcode{numpy.ndarray}} with shape (\sphinxcode{nx}, \sphinxcode{ny}, \sphinxcode{nz}) representing the \(y\)-velocity.

\item {} 
\sphinxstyleliteralstrong{Qv} (\sphinxtitleref{array\_like}, optional) \textendash{} \sphinxhref{https://docs.scipy.org/doc/numpy-1.13.0/reference/generated/numpy.ndarray.html\#numpy.ndarray}{\sphinxcode{numpy.ndarray}} with shape (\sphinxcode{nx}, \sphinxcode{ny}, \sphinxcode{nz}) representing the mass of water vapour.

\item {} 
\sphinxstyleliteralstrong{Qc} (\sphinxtitleref{array\_like}, optional) \textendash{} \sphinxhref{https://docs.scipy.org/doc/numpy-1.13.0/reference/generated/numpy.ndarray.html\#numpy.ndarray}{\sphinxcode{numpy.ndarray}} with shape (\sphinxcode{nx}, \sphinxcode{ny}, \sphinxcode{nz}) representing the mass of cloud water.

\item {} 
\sphinxstyleliteralstrong{Qr} (\sphinxtitleref{array\_like}, optional) \textendash{} \sphinxhref{https://docs.scipy.org/doc/numpy-1.13.0/reference/generated/numpy.ndarray.html\#numpy.ndarray}{\sphinxcode{numpy.ndarray}} with shape (\sphinxcode{nx}, \sphinxcode{ny}, \sphinxcode{nz}) representing the mass of precipitation water.

\end{itemize}

\item[{Returns}] \leavevmode
\begin{itemize}
\item {} 
\sphinxstylestrong{u} (\sphinxstyleemphasis{array\_like}) \textendash{} \sphinxhref{https://docs.scipy.org/doc/numpy-1.13.0/reference/generated/numpy.ndarray.html\#numpy.ndarray}{\sphinxcode{numpy.ndarray}} with shape (\sphinxcode{nx+1}, \sphinxcode{ny}, \sphinxcode{nz}) representing the diagnosed \(u\).

\item {} 
\sphinxstylestrong{v} (\sphinxstyleemphasis{array\_like}) \textendash{} \sphinxhref{https://docs.scipy.org/doc/numpy-1.13.0/reference/generated/numpy.ndarray.html\#numpy.ndarray}{\sphinxcode{numpy.ndarray}} with shape (\sphinxcode{nx}, \sphinxcode{ny+1}, \sphinxcode{nz}) representing the diagnosed \(v\).

\item {} 
\sphinxstylestrong{qv} (\sphinxtitleref{array\_like}, optional) \textendash{} \sphinxhref{https://docs.scipy.org/doc/numpy-1.13.0/reference/generated/numpy.ndarray.html\#numpy.ndarray}{\sphinxcode{numpy.ndarray}} with shape (\sphinxcode{nx}, \sphinxcode{ny}, \sphinxcode{nz}) representing the diagnosed \(q_v\).

\item {} 
\sphinxstylestrong{qc} (\sphinxtitleref{array\_like}, optional) \textendash{} \sphinxhref{https://docs.scipy.org/doc/numpy-1.13.0/reference/generated/numpy.ndarray.html\#numpy.ndarray}{\sphinxcode{numpy.ndarray}} with shape (\sphinxcode{nx}, \sphinxcode{ny}, \sphinxcode{nz}) representing the diagnosed \(q_c\).

\item {} 
\sphinxstylestrong{qr} (\sphinxtitleref{array\_like}, optional) \textendash{} \sphinxhref{https://docs.scipy.org/doc/numpy-1.13.0/reference/generated/numpy.ndarray.html\#numpy.ndarray}{\sphinxcode{numpy.ndarray}} with shape (\sphinxcode{nx}, \sphinxcode{ny}, \sphinxcode{nz}) representing the diagnosed \(q_r\).

\end{itemize}


\end{description}\end{quote}

\begin{sphinxadmonition}{note}{Note:}
The first and last rows (respectively, columns) of \sphinxcode{u} (resp., \sphinxcode{v}) are not set by the method.
\end{sphinxadmonition}

\end{fulllineitems}


\end{fulllineitems}



\subsection{Dynamical cores}
\label{\detokenize{api:dynamical-cores}}\index{DynamicalCore (class in dycore.dycore)}

\begin{fulllineitems}
\phantomsection\label{\detokenize{api:dycore.dycore.DynamicalCore}}\pysiglinewithargsret{\sphinxbfcode{class }\sphinxcode{dycore.dycore.}\sphinxbfcode{DynamicalCore}}{\emph{grid}}{}
Abstract base class whose derived classes implement different dynamical cores.
The class inherits \sphinxcode{sympl.Prognostic}.
\index{\_\_call\_\_() (dycore.dycore.DynamicalCore method)}

\begin{fulllineitems}
\phantomsection\label{\detokenize{api:dycore.dycore.DynamicalCore.__call__}}\pysiglinewithargsret{\sphinxbfcode{\_\_call\_\_}}{\emph{dt}, \emph{state}}{}
Call operator advancing the input state one step forward.
As this method is marked as abstract, its implementation is delegated to the derived classes.
\begin{quote}\begin{description}
\item[{Parameters}] \leavevmode\begin{itemize}
\item {} 
\sphinxstyleliteralstrong{dt} (\sphinxstyleliteralemphasis{obj}) \textendash{} \sphinxcode{datetime.timedelta} object representing the time step.

\item {} 
\sphinxstyleliteralstrong{state} (\sphinxstyleliteralemphasis{obj}) \textendash{} The current state, as an instance of {\hyperref[\detokenize{api:storages.grid_data.GridData}]{\sphinxcrossref{\sphinxcode{GridData}}}} or one of its derived classes.

\end{itemize}

\item[{Returns}] \leavevmode
The state at the next time level. This is of the same class of \sphinxcode{state}.

\item[{Return type}] \leavevmode
obj

\end{description}\end{quote}

\end{fulllineitems}

\index{\_\_init\_\_() (dycore.dycore.DynamicalCore method)}

\begin{fulllineitems}
\phantomsection\label{\detokenize{api:dycore.dycore.DynamicalCore.__init__}}\pysiglinewithargsret{\sphinxbfcode{\_\_init\_\_}}{\emph{grid}}{}
Constructor.
\begin{quote}\begin{description}
\item[{Parameters}] \leavevmode
\sphinxstyleliteralstrong{grid} (\sphinxstyleliteralemphasis{obj}) \textendash{} The underlying grid, as an instance of {\hyperref[\detokenize{api:grids.grid_xyz.GridXYZ}]{\sphinxcrossref{\sphinxcode{GridXYZ}}}} or one of its derived classes.

\end{description}\end{quote}

\end{fulllineitems}

\index{get\_initial\_state() (dycore.dycore.DynamicalCore method)}

\begin{fulllineitems}
\phantomsection\label{\detokenize{api:dycore.dycore.DynamicalCore.get_initial_state}}\pysiglinewithargsret{\sphinxbfcode{get\_initial\_state}}{\emph{*args}}{}
Get the initial state.
As this method is marked as abstract, its implementation is delegated to the derived classes.
\begin{quote}\begin{description}
\item[{Parameters}] \leavevmode
\sphinxstyleliteralstrong{*args} \textendash{} The arguments depend on the specific dynamical core which the derived class implements.

\item[{Returns}] \leavevmode
The initial state, as an instance of {\hyperref[\detokenize{api:storages.grid_data.GridData}]{\sphinxcrossref{\sphinxcode{GridData}}}} or one of its derived classes.

\item[{Return type}] \leavevmode
obj

\end{description}\end{quote}

\end{fulllineitems}

\index{update\_topography() (dycore.dycore.DynamicalCore method)}

\begin{fulllineitems}
\phantomsection\label{\detokenize{api:dycore.dycore.DynamicalCore.update_topography}}\pysiglinewithargsret{\sphinxbfcode{update\_topography}}{\emph{time}}{}
Update the underlying (time-dependent) topography.
\begin{quote}\begin{description}
\item[{Parameters}] \leavevmode
\sphinxstyleliteralstrong{time} (\sphinxstyleliteralemphasis{obj}) \textendash{} \sphinxcode{datetime.timedelta} representing the elapsed simulation time.

\end{description}\end{quote}

\end{fulllineitems}


\end{fulllineitems}

\index{DynamicalCoreIsentropic (class in dycore.dycore\_isentropic)}

\begin{fulllineitems}
\phantomsection\label{\detokenize{api:dycore.dycore_isentropic.DynamicalCoreIsentropic}}\pysiglinewithargsret{\sphinxbfcode{class }\sphinxcode{dycore.dycore\_isentropic.}\sphinxbfcode{DynamicalCoreIsentropic}}{\emph{grid}, \emph{imoist}, \emph{horizontal\_boundary\_type}, \emph{scheme}, \emph{backend}, \emph{idamp=True}, \emph{damp\_type='rayleigh'}, \emph{damp\_depth=15}, \emph{damp\_max=0.0002}, \emph{idiff=True}, \emph{diff\_damp\_depth=10}, \emph{diff\_coeff=0.03}, \emph{diff\_coeff\_moist=0.03}, \emph{diff\_max=0.24}}{}
This class inherits {\hyperref[\detokenize{api:dycore.dycore.DynamicalCore}]{\sphinxcrossref{\sphinxcode{DynamicalCore}}}} to implement the three-dimensional
(moist) isentropic dynamical core using GT4Py’s stencils. The class offers different numerical
schemes to carry out the prognostic step of the dynamical core, and supports different types of
lateral boundary conditions.
\index{\_\_call\_\_() (dycore.dycore\_isentropic.DynamicalCoreIsentropic method)}

\begin{fulllineitems}
\phantomsection\label{\detokenize{api:dycore.dycore_isentropic.DynamicalCoreIsentropic.__call__}}\pysiglinewithargsret{\sphinxbfcode{\_\_call\_\_}}{\emph{dt}, \emph{state}}{}
Call operator advancing the state variables one step forward.
\begin{quote}\begin{description}
\item[{Parameters}] \leavevmode\begin{itemize}
\item {} 
\sphinxstyleliteralstrong{dt} (\sphinxstyleliteralemphasis{obj}) \textendash{} \sphinxcode{datetime.timedelta} representing the time step.

\item {} 
\sphinxstyleliteralstrong{state} (\sphinxstyleliteralemphasis{obj}) \textendash{} {\hyperref[\detokenize{api:storages.state_isentropic.StateIsentropic}]{\sphinxcrossref{\sphinxcode{StateIsentropic}}}} representing the current state.

\end{itemize}

\item[{Returns}] \leavevmode
{\hyperref[\detokenize{api:storages.state_isentropic.StateIsentropic}]{\sphinxcrossref{\sphinxcode{StateIsentropic}}}} representing the state at the next time level.

\item[{Return type}] \leavevmode
obj

\end{description}\end{quote}

\end{fulllineitems}

\index{\_\_init\_\_() (dycore.dycore\_isentropic.DynamicalCoreIsentropic method)}

\begin{fulllineitems}
\phantomsection\label{\detokenize{api:dycore.dycore_isentropic.DynamicalCoreIsentropic.__init__}}\pysiglinewithargsret{\sphinxbfcode{\_\_init\_\_}}{\emph{grid}, \emph{imoist}, \emph{horizontal\_boundary\_type}, \emph{scheme}, \emph{backend}, \emph{idamp=True}, \emph{damp\_type='rayleigh'}, \emph{damp\_depth=15}, \emph{damp\_max=0.0002}, \emph{idiff=True}, \emph{diff\_damp\_depth=10}, \emph{diff\_coeff=0.03}, \emph{diff\_coeff\_moist=0.03}, \emph{diff\_max=0.24}}{}
Constructor.
\begin{quote}\begin{description}
\item[{Parameters}] \leavevmode\begin{itemize}
\item {} 
\sphinxstyleliteralstrong{grid} (\sphinxstyleliteralemphasis{obj}) \textendash{} {\hyperref[\detokenize{api:grids.grid_xyz.GridXYZ}]{\sphinxcrossref{\sphinxcode{GridXYZ}}}} representing the underlying grid.

\item {} 
\sphinxstyleliteralstrong{imoist} (\sphinxstyleliteralemphasis{bool}) \textendash{} \sphinxcode{True} for a moist dynamical core, \sphinxcode{False} otherwise.

\item {} 
\sphinxstyleliteralstrong{horizontal\_boundary\_type} (\sphinxstyleliteralemphasis{str}) \textendash{} String specifying the horizontal boundary conditions.
See {\hyperref[\detokenize{api:dycore.horizontal_boundary.HorizontalBoundary}]{\sphinxcrossref{\sphinxcode{HorizontalBoundary}}}} for the available options.

\item {} 
\sphinxstyleliteralstrong{scheme} (\sphinxstyleliteralemphasis{str}) \textendash{} String specifying the numerical scheme carrying out the prognostic step of the dynamical core.
See {\hyperref[\detokenize{api:dycore.prognostic_isentropic.PrognosticIsentropic}]{\sphinxcrossref{\sphinxcode{PrognosticIsentropic}}}} for the available options.

\item {} 
\sphinxstyleliteralstrong{backend} (\sphinxstyleliteralemphasis{obj}) \textendash{} \sphinxcode{gridtools.mode} specifying the backend for the GT4Py’s stencils implementing the dynamical core.

\item {} 
\sphinxstyleliteralstrong{idamp} (\sphinxtitleref{bool}, optional) \textendash{} \sphinxcode{True} if vertical damping is enabled, \sphinxcode{False} otherwise. Default is \sphinxcode{True}.

\item {} 
\sphinxstyleliteralstrong{damp\_type} (\sphinxtitleref{str}, optional) \textendash{} String specifying the type of vertical damping to apply. Default is ‘rayleigh’.
See {\hyperref[\detokenize{api:dycore.vertical_damping.VerticalDamping}]{\sphinxcrossref{\sphinxcode{dycore.vertical\_damping.VerticalDamping}}}} for further details.

\item {} 
\sphinxstyleliteralstrong{damp\_depth} (\sphinxtitleref{int}, optional) \textendash{} Number of vertical layers in the damping region. Default is 15.

\item {} 
\sphinxstyleliteralstrong{damp\_max} (\sphinxtitleref{float}, optional) \textendash{} Maximum value for the damping coefficient. Default is 0.0002.

\item {} 
\sphinxstyleliteralstrong{idiff} (\sphinxtitleref{bool}, optional) \textendash{} \sphinxcode{True} if numerical diffusion is enabled, \sphinxcode{False} otherwise. Default is \sphinxcode{True}.

\item {} 
\sphinxstyleliteralstrong{diff\_damp\_depth} (\sphinxtitleref{int}, optional) \textendash{} Number of vertical layers in the diffusive damping region. Default is 10.

\item {} 
\sphinxstyleliteralstrong{diff\_coeff} (\sphinxtitleref{float}, optional) \textendash{} Diffusion coefficient. Default is 0.03.

\item {} 
\sphinxstyleliteralstrong{diff\_coeff\_moist} (\sphinxtitleref{float}, optional) \textendash{} Diffusion coefficient for the water constituents. Default is 0.03.

\item {} 
\sphinxstyleliteralstrong{diff\_max} (\sphinxtitleref{float}, optional) \textendash{} Maximum value for the diffusion coefficient. Default is 0.24. See {\hyperref[\detokenize{api:dycore.diffusion.Diffusion}]{\sphinxcrossref{\sphinxcode{Diffusion}}}}
for further details.

\end{itemize}

\end{description}\end{quote}

\end{fulllineitems}

\index{\_integrate\_dry() (dycore.dycore\_isentropic.DynamicalCoreIsentropic method)}

\begin{fulllineitems}
\phantomsection\label{\detokenize{api:dycore.dycore_isentropic.DynamicalCoreIsentropic._integrate_dry}}\pysiglinewithargsret{\sphinxbfcode{\_integrate\_dry}}{\emph{dt}, \emph{state}}{}
Method advancing the dry isentropic state by a single time step.
\begin{quote}\begin{description}
\item[{Parameters}] \leavevmode\begin{itemize}
\item {} 
\sphinxstyleliteralstrong{dt} (\sphinxstyleliteralemphasis{obj}) \textendash{} \sphinxcode{datetime.timedelta} representing the time step.

\item {} 
\sphinxstyleliteralstrong{state} (\sphinxstyleliteralemphasis{obj}) \textendash{} {\hyperref[\detokenize{api:storages.state_isentropic.StateIsentropic}]{\sphinxcrossref{\sphinxcode{StateIsentropic}}}} representing the current state.

\end{itemize}

\item[{Returns}] \leavevmode
{\hyperref[\detokenize{api:storages.state_isentropic.StateIsentropic}]{\sphinxcrossref{\sphinxcode{StateIsentropic}}}} representing the state at the next time level.

\item[{Return type}] \leavevmode
obj

\end{description}\end{quote}

\end{fulllineitems}

\index{\_integrate\_moist() (dycore.dycore\_isentropic.DynamicalCoreIsentropic method)}

\begin{fulllineitems}
\phantomsection\label{\detokenize{api:dycore.dycore_isentropic.DynamicalCoreIsentropic._integrate_moist}}\pysiglinewithargsret{\sphinxbfcode{\_integrate\_moist}}{\emph{dt}, \emph{state}}{}
Method advancing the moist isentropic state by a single time step.
\begin{quote}\begin{description}
\item[{Parameters}] \leavevmode\begin{itemize}
\item {} 
\sphinxstyleliteralstrong{dt} (\sphinxstyleliteralemphasis{obj}) \textendash{} \sphinxcode{datetime.timedelta} representing the time step.

\item {} 
\sphinxstyleliteralstrong{state} (\sphinxstyleliteralemphasis{obj}) \textendash{} {\hyperref[\detokenize{api:storages.state_isentropic.StateIsentropic}]{\sphinxcrossref{\sphinxcode{StateIsentropic}}}} representing the current state.

\end{itemize}

\item[{Returns}] \leavevmode
{\hyperref[\detokenize{api:storages.state_isentropic.StateIsentropic}]{\sphinxcrossref{\sphinxcode{StateIsentropic}}}} representing the state at the next time level.

\item[{Return type}] \leavevmode
obj

\end{description}\end{quote}

\end{fulllineitems}

\index{get\_initial\_state() (dycore.dycore\_isentropic.DynamicalCoreIsentropic method)}

\begin{fulllineitems}
\phantomsection\label{\detokenize{api:dycore.dycore_isentropic.DynamicalCoreIsentropic.get_initial_state}}\pysiglinewithargsret{\sphinxbfcode{get\_initial\_state}}{\emph{initial\_time}, \emph{initial\_state\_type}, \emph{**kwargs}}{}
Get the initial state, based on the identifier \sphinxcode{initial\_state\_type}. Particularly:
\begin{itemize}
\item {} 
if \sphinxcode{initial\_state\_type == 0}:
\begin{itemize}
\item {} 
\(u(x, \, y, \, \theta, \, 0) = u_0\) and \(v(x, \, y, \, \theta, \, 0) = v_0\);

\item {} 
all the other model variables (Exner function, pressure, Montgomery potential, height of the                          isentropes, isentropic density) are derived from the Brunt-Vaisala frequency \(N\).

\end{itemize}

\item {} 
if \sphinxcode{initial\_state\_type == 1}:
\begin{itemize}
\item {} 
\(u(x, \, y, \, \theta, \, 0) = u_0\) and \(v(x, \, y, \, \theta, \, 0) = v_0\);

\item {} 
\(T(x, \, y, \, \theta, \, 0) = T_0\).

\end{itemize}

\end{itemize}
\begin{quote}\begin{description}
\item[{Parameters}] \leavevmode\begin{itemize}
\item {} 
\sphinxstyleliteralstrong{initial\_time} (\sphinxstyleliteralemphasis{obj}) \textendash{} \sphinxcode{datetime.datetime} representing the initial simulation time.

\item {} 
\sphinxstyleliteralstrong{case} (\sphinxstyleliteralemphasis{int}) \textendash{} Identifier.

\end{itemize}

\item[{Keyword Arguments}] \leavevmode\begin{itemize}
\item {} 
\sphinxstyleliteralstrong{x\_velocity\_initial} (\sphinxstyleliteralemphasis{float}) \textendash{} The initial, uniform \(x\)-velocity \(u_0\). Default is \(10 m s^{-1}\).

\item {} 
\sphinxstyleliteralstrong{y\_velocity\_initial} (\sphinxstyleliteralemphasis{float}) \textendash{} The initial, uniform \(y\)-velocity \(v_0\). Default is \(0 m s^{-1}\).

\item {} 
\sphinxstyleliteralstrong{brunt\_vaisala\_initial} (\sphinxstyleliteralemphasis{float}) \textendash{} If \sphinxcode{initial\_state\_type == 0}, the uniform Brunt-Vaisala frequence \(N\). Default is \(0.01\).

\item {} 
\sphinxstyleliteralstrong{temperature\_initial} (\sphinxstyleliteralemphasis{float}) \textendash{} If \sphinxcode{initial\_state\_type == 1}, the uniform initial temperature \(T_0\). Default is \(250 K\).

\end{itemize}

\item[{Returns}] \leavevmode
{\hyperref[\detokenize{api:storages.state_isentropic.StateIsentropic}]{\sphinxcrossref{\sphinxcode{StateIsentropic}}}} representing the initial state.

\item[{Return type}] \leavevmode
obj

\end{description}\end{quote}

\end{fulllineitems}


\end{fulllineitems}



\subsection{Lateral boundary conditions}
\label{\detokenize{api:lateral-boundary-conditions}}\index{HorizontalBoundary (class in dycore.horizontal\_boundary)}

\begin{fulllineitems}
\phantomsection\label{\detokenize{api:dycore.horizontal_boundary.HorizontalBoundary}}\pysiglinewithargsret{\sphinxbfcode{class }\sphinxcode{dycore.horizontal\_boundary.}\sphinxbfcode{HorizontalBoundary}}{\emph{grid}, \emph{nb}}{}
Abstract base class whose derived classes implement different types of horizontal boundary conditions.
\begin{quote}\begin{description}
\item[{Variables}] \leavevmode
{\hyperref[\detokenize{api:dycore.prognostic_isentropic.PrognosticIsentropic.nb}]{\sphinxcrossref{\sphinxstyleliteralstrong{nb}}}} (\sphinxstyleliteralemphasis{int}) \textendash{} Number of boundary layers.

\end{description}\end{quote}
\index{\_\_init\_\_() (dycore.horizontal\_boundary.HorizontalBoundary method)}

\begin{fulllineitems}
\phantomsection\label{\detokenize{api:dycore.horizontal_boundary.HorizontalBoundary.__init__}}\pysiglinewithargsret{\sphinxbfcode{\_\_init\_\_}}{\emph{grid}, \emph{nb}}{}
Constructor.
\begin{quote}\begin{description}
\item[{Parameters}] \leavevmode\begin{itemize}
\item {} 
\sphinxstyleliteralstrong{grid} (\sphinxstyleliteralemphasis{obj}) \textendash{} The underlying grid, as an instance of {\hyperref[\detokenize{api:grids.grid_xyz.GridXYZ}]{\sphinxcrossref{\sphinxcode{GridXYZ}}}} or one of its derived classes.

\item {} 
\sphinxstyleliteralstrong{nb} (\sphinxstyleliteralemphasis{int}) \textendash{} Number of boundary layers.

\end{itemize}

\end{description}\end{quote}

\end{fulllineitems}

\index{apply() (dycore.horizontal\_boundary.HorizontalBoundary method)}

\begin{fulllineitems}
\phantomsection\label{\detokenize{api:dycore.horizontal_boundary.HorizontalBoundary.apply}}\pysiglinewithargsret{\sphinxbfcode{apply}}{\emph{phi\_new}, \emph{phi\_now}}{}
Apply the boundary conditions on the field \sphinxcode{phi\_new}, possibly relying upon the solution
\sphinxcode{phi\_now} at the current time.
As this method is marked as abstract, its implementation is delegated to the derived classes.
\begin{quote}\begin{description}
\item[{Parameters}] \leavevmode\begin{itemize}
\item {} 
\sphinxstyleliteralstrong{phi\_new} (\sphinxstyleliteralemphasis{array\_like}) \textendash{} \sphinxhref{https://docs.scipy.org/doc/numpy-1.13.0/reference/generated/numpy.ndarray.html\#numpy.ndarray}{\sphinxcode{numpy.ndarray}} representing the field on which applying the boundary conditions.

\item {} 
\sphinxstyleliteralstrong{phi\_now} (\sphinxstyleliteralemphasis{array\_like}) \textendash{} \sphinxhref{https://docs.scipy.org/doc/numpy-1.13.0/reference/generated/numpy.ndarray.html\#numpy.ndarray}{\sphinxcode{numpy.ndarray}} representing the field at the current time.

\end{itemize}

\end{description}\end{quote}

\end{fulllineitems}

\index{factory() (dycore.horizontal\_boundary.HorizontalBoundary static method)}

\begin{fulllineitems}
\phantomsection\label{\detokenize{api:dycore.horizontal_boundary.HorizontalBoundary.factory}}\pysiglinewithargsret{\sphinxbfcode{static }\sphinxbfcode{factory}}{\emph{horizontal\_boundary\_type}, \emph{grid}, \emph{nb}}{}
Static method which returns an instance of the derived class which implements the boundary
conditions specified by \sphinxcode{horizontal\_boundary\_type}.
\begin{quote}\begin{description}
\item[{Parameters}] \leavevmode\begin{itemize}
\item {} 
\sphinxstyleliteralstrong{horizontal\_boundary\_type} (\sphinxstyleliteralemphasis{str}) \textendash{} 
String specifying the type of boundary conditions to apply. Either:
\begin{itemize}
\item {} 
’periodic’, for periodic boundary conditions;

\item {} 
’relaxed’, for relaxed boundary conditions;

\item {} 
’relaxed-symmetric-xz’, for relaxed boundary conditions for a \(xz\)-symmetric field.

\item {} 
’relaxed-symmetric-yz’, for relaxed boundary conditions for a \(yz\)-symmetric field.

\end{itemize}


\item {} 
\sphinxstyleliteralstrong{grid} (\sphinxstyleliteralemphasis{obj}) \textendash{} The underlying grid, as an instance of {\hyperref[\detokenize{api:grids.grid_xyz.GridXYZ}]{\sphinxcrossref{\sphinxcode{GridXYZ}}}} or one of its derived classes.

\item {} 
\sphinxstyleliteralstrong{nb} (\sphinxstyleliteralemphasis{int}) \textendash{} Number of boundary layers.

\end{itemize}

\item[{Returns}] \leavevmode
An instance of the derived class implementing the boundary conditions specified by
\sphinxcode{horizontal\_boundary\_type}.

\item[{Return type}] \leavevmode
obj

\end{description}\end{quote}

\end{fulllineitems}

\index{from\_computational\_to\_physical\_domain() (dycore.horizontal\_boundary.HorizontalBoundary method)}

\begin{fulllineitems}
\phantomsection\label{\detokenize{api:dycore.horizontal_boundary.HorizontalBoundary.from_computational_to_physical_domain}}\pysiglinewithargsret{\sphinxbfcode{from\_computational\_to\_physical\_domain}}{\emph{phi\_}, \emph{out\_dims}, \emph{change\_sign}}{}
Given a \sphinxhref{https://docs.scipy.org/doc/numpy-1.13.0/reference/generated/numpy.ndarray.html\#numpy.ndarray}{\sphinxcode{numpy.ndarray}} representing the computational domain of a stencil, return the
associated physicalm field which may (or may not) satisfy the horizontal boundary conditions.
As this method is marked as abstract, its implementation is delegated to the derived classes.
\begin{quote}\begin{description}
\item[{Parameters}] \leavevmode\begin{itemize}
\item {} 
\sphinxstyleliteralstrong{phi} (\sphinxstyleliteralemphasis{array\_like}) \textendash{} \sphinxhref{https://docs.scipy.org/doc/numpy-1.13.0/reference/generated/numpy.ndarray.html\#numpy.ndarray}{\sphinxcode{numpy.ndarray}} representing the computational domain of a stencil.

\item {} 
\sphinxstyleliteralstrong{out\_dims} (\sphinxstyleliteralemphasis{tuple}) \textendash{} Tuple of the output array dimensions.

\item {} 
\sphinxstyleliteralstrong{change\_sign} (\sphinxstyleliteralemphasis{bool}) \textendash{} \sphinxcode{True} if the field should change sign through the symmetry plane (if any), \sphinxcode{False} otherwise.

\end{itemize}

\item[{Returns}] \leavevmode
\sphinxhref{https://docs.scipy.org/doc/numpy-1.13.0/reference/generated/numpy.ndarray.html\#numpy.ndarray}{\sphinxcode{numpy.ndarray}} representing the field defined over the physical domain.

\item[{Return type}] \leavevmode
array\_like

\end{description}\end{quote}

\end{fulllineitems}

\index{from\_physical\_to\_computational\_domain() (dycore.horizontal\_boundary.HorizontalBoundary method)}

\begin{fulllineitems}
\phantomsection\label{\detokenize{api:dycore.horizontal_boundary.HorizontalBoundary.from_physical_to_computational_domain}}\pysiglinewithargsret{\sphinxbfcode{from\_physical\_to\_computational\_domain}}{\emph{phi}}{}
Given a \sphinxhref{https://docs.scipy.org/doc/numpy-1.13.0/reference/generated/numpy.ndarray.html\#numpy.ndarray}{\sphinxcode{numpy.ndarray}} representing a physical field, return the associated stencils’ computational
domain, i.e., the \sphinxhref{https://docs.scipy.org/doc/numpy-1.13.0/reference/generated/numpy.ndarray.html\#numpy.ndarray}{\sphinxcode{numpy.ndarray}} (accomodating the boundary conditions) which will be input
to the stencils. If the physical and computational fields coincide, a deep copy of the physical
domain is returned.
As this method is marked as abstract, its implementation is delegated to the derived classes.
\begin{quote}\begin{description}
\item[{Parameters}] \leavevmode
\sphinxstyleliteralstrong{phi} (\sphinxstyleliteralemphasis{array\_like}) \textendash{} \sphinxhref{https://docs.scipy.org/doc/numpy-1.13.0/reference/generated/numpy.ndarray.html\#numpy.ndarray}{\sphinxcode{numpy.ndarray}} representing the physical field.

\item[{Returns}] \leavevmode
\sphinxhref{https://docs.scipy.org/doc/numpy-1.13.0/reference/generated/numpy.ndarray.html\#numpy.ndarray}{\sphinxcode{numpy.ndarray}} representing the stencils’ computational domain.

\item[{Return type}] \leavevmode
array\_like

\end{description}\end{quote}

\begin{sphinxadmonition}{note}{Note:}
The implementation should be designed to work with both staggered and unstaggared fields.
\end{sphinxadmonition}

\end{fulllineitems}

\index{set\_outermost\_layers\_x() (dycore.horizontal\_boundary.HorizontalBoundary method)}

\begin{fulllineitems}
\phantomsection\label{\detokenize{api:dycore.horizontal_boundary.HorizontalBoundary.set_outermost_layers_x}}\pysiglinewithargsret{\sphinxbfcode{set\_outermost\_layers\_x}}{\emph{phi\_new}, \emph{phi\_now}}{}
Set the outermost layers of \sphinxcode{phi\_new} in the \(x\)- (i.e., \sphinxcode{i}-) direction so to satisfy
the lateral boundary conditions. For this, possibly rely upon the field \sphinxcode{phi\_now} at the current time.
As this method is marked as abstract, its implementation is delegated to the derived classes.
\begin{quote}\begin{description}
\item[{Parameters}] \leavevmode\begin{itemize}
\item {} 
\sphinxstyleliteralstrong{phi\_new} (\sphinxstyleliteralemphasis{array\_like}) \textendash{} \sphinxhref{https://docs.scipy.org/doc/numpy-1.13.0/reference/generated/numpy.ndarray.html\#numpy.ndarray}{\sphinxcode{numpy.ndarray}} representing the field on which applying the boundary conditions.

\item {} 
\sphinxstyleliteralstrong{phi\_now} (\sphinxstyleliteralemphasis{array\_like}) \textendash{} \sphinxhref{https://docs.scipy.org/doc/numpy-1.13.0/reference/generated/numpy.ndarray.html\#numpy.ndarray}{\sphinxcode{numpy.ndarray}} representing the field at the current time.

\end{itemize}

\end{description}\end{quote}

\end{fulllineitems}

\index{set\_outermost\_layers\_y() (dycore.horizontal\_boundary.HorizontalBoundary method)}

\begin{fulllineitems}
\phantomsection\label{\detokenize{api:dycore.horizontal_boundary.HorizontalBoundary.set_outermost_layers_y}}\pysiglinewithargsret{\sphinxbfcode{set\_outermost\_layers\_y}}{\emph{phi\_new}, \emph{phi\_now}}{}
Set the outermost layers of \sphinxcode{phi\_new} in the \(y\)- (i.e., \sphinxcode{j}-) direction so to satisfy
the lateral boundary conditions. For this, possibly rely upon the field \sphinxcode{phi\_now} at the current time.
As this method is marked as abstract, its implementation is delegated to the derived classes.
\begin{quote}\begin{description}
\item[{Parameters}] \leavevmode\begin{itemize}
\item {} 
\sphinxstyleliteralstrong{phi\_new} (\sphinxstyleliteralemphasis{array\_like}) \textendash{} \sphinxhref{https://docs.scipy.org/doc/numpy-1.13.0/reference/generated/numpy.ndarray.html\#numpy.ndarray}{\sphinxcode{numpy.ndarray}} representing the field on which applying the boundary conditions.

\item {} 
\sphinxstyleliteralstrong{phi\_now} (\sphinxstyleliteralemphasis{array\_like}) \textendash{} \sphinxhref{https://docs.scipy.org/doc/numpy-1.13.0/reference/generated/numpy.ndarray.html\#numpy.ndarray}{\sphinxcode{numpy.ndarray}} representing the field at the current time.

\end{itemize}

\end{description}\end{quote}

\end{fulllineitems}


\end{fulllineitems}

\index{Periodic (class in dycore.horizontal\_boundary)}

\begin{fulllineitems}
\phantomsection\label{\detokenize{api:dycore.horizontal_boundary.Periodic}}\pysiglinewithargsret{\sphinxbfcode{class }\sphinxcode{dycore.horizontal\_boundary.}\sphinxbfcode{Periodic}}{\emph{grid}, \emph{nb}}{}
This class inherits {\hyperref[\detokenize{api:dycore.horizontal_boundary.HorizontalBoundary}]{\sphinxcrossref{\sphinxcode{HorizontalBoundary}}}} to implement horizontally periodic boundary conditions.
\begin{quote}\begin{description}
\item[{Variables}] \leavevmode
{\hyperref[\detokenize{api:dycore.prognostic_isentropic.PrognosticIsentropic.nb}]{\sphinxcrossref{\sphinxstyleliteralstrong{nb}}}} (\sphinxstyleliteralemphasis{int}) \textendash{} Number of boundary layers.

\end{description}\end{quote}
\index{\_\_init\_\_() (dycore.horizontal\_boundary.Periodic method)}

\begin{fulllineitems}
\phantomsection\label{\detokenize{api:dycore.horizontal_boundary.Periodic.__init__}}\pysiglinewithargsret{\sphinxbfcode{\_\_init\_\_}}{\emph{grid}, \emph{nb}}{}
Constructor.
\begin{quote}\begin{description}
\item[{Parameters}] \leavevmode\begin{itemize}
\item {} 
\sphinxstyleliteralstrong{grid} (\sphinxstyleliteralemphasis{obj}) \textendash{} The underlying grid, as an instance of {\hyperref[\detokenize{api:grids.grid_xyz.GridXYZ}]{\sphinxcrossref{\sphinxcode{GridXYZ}}}} or one of its derived classes.

\item {} 
\sphinxstyleliteralstrong{nb} (\sphinxstyleliteralemphasis{int}) \textendash{} Number of boundary layers.

\end{itemize}

\end{description}\end{quote}

\end{fulllineitems}

\index{apply() (dycore.horizontal\_boundary.Periodic method)}

\begin{fulllineitems}
\phantomsection\label{\detokenize{api:dycore.horizontal_boundary.Periodic.apply}}\pysiglinewithargsret{\sphinxbfcode{apply}}{\emph{phi\_new}, \emph{phi\_now=None}}{}
Apply horizontally periodic boundary conditions on \sphinxcode{phi\_new}.
\begin{quote}\begin{description}
\item[{Parameters}] \leavevmode\begin{itemize}
\item {} 
\sphinxstyleliteralstrong{phi\_new} (\sphinxstyleliteralemphasis{array\_like}) \textendash{} \sphinxhref{https://docs.scipy.org/doc/numpy-1.13.0/reference/generated/numpy.ndarray.html\#numpy.ndarray}{\sphinxcode{numpy.ndarray}} representing the field on which applying the boundary conditions.

\item {} 
\sphinxstyleliteralstrong{phi\_now} (\sphinxtitleref{array\_like}, optional) \textendash{} \sphinxhref{https://docs.scipy.org/doc/numpy-1.13.0/reference/generated/numpy.ndarray.html\#numpy.ndarray}{\sphinxcode{numpy.ndarray}} representing the field at the current time.

\end{itemize}

\end{description}\end{quote}

\begin{sphinxadmonition}{note}{Note:}
The argument \sphinxcode{phi\_now} is not required by the implementation, yet it is retained as optional
argument for compliancy with the class hierarchy interface.
\end{sphinxadmonition}

\end{fulllineitems}

\index{from\_computational\_to\_physical\_domain() (dycore.horizontal\_boundary.Periodic method)}

\begin{fulllineitems}
\phantomsection\label{\detokenize{api:dycore.horizontal_boundary.Periodic.from_computational_to_physical_domain}}\pysiglinewithargsret{\sphinxbfcode{from\_computational\_to\_physical\_domain}}{\emph{phi\_}, \emph{out\_dims=None}, \emph{change\_sign=True}}{}
Shrink the field \sphinxcode{phi\_} by removing the \sphinxcode{nb} outermost layers.
\begin{quote}\begin{description}
\item[{Parameters}] \leavevmode\begin{itemize}
\item {} 
\sphinxstyleliteralstrong{phi} (\sphinxstyleliteralemphasis{array\_like}) \textendash{} The \sphinxhref{https://docs.scipy.org/doc/numpy-1.13.0/reference/generated/numpy.ndarray.html\#numpy.ndarray}{\sphinxcode{numpy.ndarray}} to shrink.

\item {} 
\sphinxstyleliteralstrong{out\_dims} (\sphinxstyleliteralemphasis{tuple}) \textendash{} Tuple of the output array dimensions.

\item {} 
\sphinxstyleliteralstrong{change\_sign} (\sphinxstyleliteralemphasis{bool}) \textendash{} \sphinxcode{True} if the field should change sign through the symmetry plane (if any), \sphinxcode{False} otherwise.

\end{itemize}

\item[{Returns}] \leavevmode
The shrunk \sphinxhref{https://docs.scipy.org/doc/numpy-1.13.0/reference/generated/numpy.ndarray.html\#numpy.ndarray}{\sphinxcode{numpy.ndarray}}.

\item[{Return type}] \leavevmode
array\_like

\end{description}\end{quote}

\begin{sphinxadmonition}{note}{Note:}
The arguments \sphinxcode{out\_dims} and \sphinxcode{change\_sign} are not required by the implementation,
yet they are retained as optional arguments for compliancy with the class hierarchy interface.
\end{sphinxadmonition}

\end{fulllineitems}

\index{from\_physical\_to\_computational\_domain() (dycore.horizontal\_boundary.Periodic method)}

\begin{fulllineitems}
\phantomsection\label{\detokenize{api:dycore.horizontal_boundary.Periodic.from_physical_to_computational_domain}}\pysiglinewithargsret{\sphinxbfcode{from\_physical\_to\_computational\_domain}}{\emph{phi}}{}
Periodically extend the field \sphinxcode{phi} with \sphinxcode{nb} extra layers.
\begin{quote}\begin{description}
\item[{Parameters}] \leavevmode
\sphinxstyleliteralstrong{phi} (\sphinxstyleliteralemphasis{array\_like}) \textendash{} The \sphinxhref{https://docs.scipy.org/doc/numpy-1.13.0/reference/generated/numpy.ndarray.html\#numpy.ndarray}{\sphinxcode{numpy.ndarray}} to extend.

\item[{Returns}] \leavevmode
The extended \sphinxhref{https://docs.scipy.org/doc/numpy-1.13.0/reference/generated/numpy.ndarray.html\#numpy.ndarray}{\sphinxcode{numpy.ndarray}}.

\item[{Return type}] \leavevmode
array\_like

\end{description}\end{quote}

\end{fulllineitems}

\index{set\_outermost\_layers\_x() (dycore.horizontal\_boundary.Periodic method)}

\begin{fulllineitems}
\phantomsection\label{\detokenize{api:dycore.horizontal_boundary.Periodic.set_outermost_layers_x}}\pysiglinewithargsret{\sphinxbfcode{set\_outermost\_layers\_x}}{\emph{phi\_new}, \emph{phi\_now=None}}{}
Set the outermost layers of \sphinxcode{phi\_new} in the \sphinxcode{x}-direction so to satisfy the periodic
boundary conditions. For this, the field \sphinxcode{phi\_now} at the current time is not required. Yet,
it appears as (default) argument for compliancy with the general API.
\begin{quote}\begin{description}
\item[{Parameters}] \leavevmode\begin{itemize}
\item {} 
\sphinxstyleliteralstrong{phi\_new} (\sphinxstyleliteralemphasis{array\_like}) \textendash{} \sphinxhref{https://docs.scipy.org/doc/numpy-1.13.0/reference/generated/numpy.ndarray.html\#numpy.ndarray}{\sphinxcode{numpy.ndarray}} representing the field on which applying the boundary conditions.

\item {} 
\sphinxstyleliteralstrong{phi\_now} (\sphinxtitleref{array\_like}, optional) \textendash{} \sphinxhref{https://docs.scipy.org/doc/numpy-1.13.0/reference/generated/numpy.ndarray.html\#numpy.ndarray}{\sphinxcode{numpy.ndarray}} representing the field at the current time.

\end{itemize}

\end{description}\end{quote}

\begin{sphinxadmonition}{note}{Note:}
The argument \sphinxcode{phi\_now} is not required by the implementation, yet it is retained as optional
argument for compliancy with the class hierarchy interface.
\end{sphinxadmonition}

\end{fulllineitems}

\index{set\_outermost\_layers\_y() (dycore.horizontal\_boundary.Periodic method)}

\begin{fulllineitems}
\phantomsection\label{\detokenize{api:dycore.horizontal_boundary.Periodic.set_outermost_layers_y}}\pysiglinewithargsret{\sphinxbfcode{set\_outermost\_layers\_y}}{\emph{phi\_new}, \emph{phi\_now=None}}{}
Set the outermost layers of \sphinxcode{phi\_new} in the \sphinxcode{y}-direction so to satisfy the periodic
boundary conditions. For this, the field \sphinxcode{phi\_now} at the current time is not required. Yet,
it appears as (default) argument for compliancy with the general API.
\begin{quote}\begin{description}
\item[{Parameters}] \leavevmode\begin{itemize}
\item {} 
\sphinxstyleliteralstrong{phi\_new} (\sphinxstyleliteralemphasis{array\_like}) \textendash{} \sphinxhref{https://docs.scipy.org/doc/numpy-1.13.0/reference/generated/numpy.ndarray.html\#numpy.ndarray}{\sphinxcode{numpy.ndarray}} representing the field on which applying the boundary conditions.

\item {} 
\sphinxstyleliteralstrong{phi\_now} (\sphinxtitleref{array\_like}, optional) \textendash{} \sphinxhref{https://docs.scipy.org/doc/numpy-1.13.0/reference/generated/numpy.ndarray.html\#numpy.ndarray}{\sphinxcode{numpy.ndarray}} representing the field at the current time.

\end{itemize}

\end{description}\end{quote}

\begin{sphinxadmonition}{note}{Note:}
The argument \sphinxcode{phi\_now} is not required by the implementation, yet it is retained as optional
argument for compliancy with the class hierarchy interface.
\end{sphinxadmonition}

\end{fulllineitems}


\end{fulllineitems}

\index{Relaxed (class in dycore.horizontal\_boundary)}

\begin{fulllineitems}
\phantomsection\label{\detokenize{api:dycore.horizontal_boundary.Relaxed}}\pysiglinewithargsret{\sphinxbfcode{class }\sphinxcode{dycore.horizontal\_boundary.}\sphinxbfcode{Relaxed}}{\emph{grid}, \emph{nb}}{}
This class inherits {\hyperref[\detokenize{api:dycore.horizontal_boundary.HorizontalBoundary}]{\sphinxcrossref{\sphinxcode{HorizontalBoundary}}}} to implement horizontally relaxed boundary conditions.
\begin{quote}\begin{description}
\item[{Variables}] \leavevmode\begin{itemize}
\item {} 
{\hyperref[\detokenize{api:dycore.prognostic_isentropic.PrognosticIsentropic.nb}]{\sphinxcrossref{\sphinxstyleliteralstrong{nb}}}} (\sphinxstyleliteralemphasis{int}) \textendash{} Number of boundary layers.

\item {} 
\sphinxstyleliteralstrong{nr} (\sphinxstyleliteralemphasis{int}) \textendash{} Number of layers which will be affected by relaxation.

\end{itemize}

\end{description}\end{quote}
\index{\_\_init\_\_() (dycore.horizontal\_boundary.Relaxed method)}

\begin{fulllineitems}
\phantomsection\label{\detokenize{api:dycore.horizontal_boundary.Relaxed.__init__}}\pysiglinewithargsret{\sphinxbfcode{\_\_init\_\_}}{\emph{grid}, \emph{nb}}{}
Constructor.
\begin{quote}\begin{description}
\item[{Parameters}] \leavevmode\begin{itemize}
\item {} 
\sphinxstyleliteralstrong{grid} (\sphinxstyleliteralemphasis{obj}) \textendash{} The underlying grid, as an instance of {\hyperref[\detokenize{api:grids.grid_xyz.GridXYZ}]{\sphinxcrossref{\sphinxcode{GridXYZ}}}} or one of its derived classes.

\item {} 
\sphinxstyleliteralstrong{nb} (\sphinxstyleliteralemphasis{int}) \textendash{} Number of boundary layers.

\end{itemize}

\end{description}\end{quote}

\end{fulllineitems}

\index{apply() (dycore.horizontal\_boundary.Relaxed method)}

\begin{fulllineitems}
\phantomsection\label{\detokenize{api:dycore.horizontal_boundary.Relaxed.apply}}\pysiglinewithargsret{\sphinxbfcode{apply}}{\emph{phi\_new}, \emph{phi\_now}}{}
Apply relaxed lateral boundary conditions.
\begin{quote}\begin{description}
\item[{Parameters}] \leavevmode\begin{itemize}
\item {} 
\sphinxstyleliteralstrong{phi\_new} (\sphinxstyleliteralemphasis{array\_like}) \textendash{} \sphinxhref{https://docs.scipy.org/doc/numpy-1.13.0/reference/generated/numpy.ndarray.html\#numpy.ndarray}{\sphinxcode{numpy.ndarray}} representing the field on which applying the boundary conditions.

\item {} 
\sphinxstyleliteralstrong{phi\_now} (\sphinxstyleliteralemphasis{array\_like}) \textendash{} \sphinxhref{https://docs.scipy.org/doc/numpy-1.13.0/reference/generated/numpy.ndarray.html\#numpy.ndarray}{\sphinxcode{numpy.ndarray}} representing the field at the current time.

\end{itemize}

\end{description}\end{quote}

\begin{sphinxadmonition}{note}{Note:}
The Dirichlet conditions at the boundaries are assumed to be time-independent, so that they
can be inferred from the solution at current time.
\end{sphinxadmonition}

\end{fulllineitems}

\index{from\_computational\_to\_physical\_domain() (dycore.horizontal\_boundary.Relaxed method)}

\begin{fulllineitems}
\phantomsection\label{\detokenize{api:dycore.horizontal_boundary.Relaxed.from_computational_to_physical_domain}}\pysiglinewithargsret{\sphinxbfcode{from\_computational\_to\_physical\_domain}}{\emph{phi\_}, \emph{out\_dims=None}, \emph{change\_sign=True}}{}
As no extension is required to apply relaxed boundary conditions, return a deep copy of the
input field \sphinxcode{phi\_}.
\begin{quote}\begin{description}
\item[{Parameters}] \leavevmode\begin{itemize}
\item {} 
\sphinxstyleliteralstrong{phi} (\sphinxstyleliteralemphasis{array\_like}) \textendash{} A \sphinxhref{https://docs.scipy.org/doc/numpy-1.13.0/reference/generated/numpy.ndarray.html\#numpy.ndarray}{\sphinxcode{numpy.ndarray}}.

\item {} 
\sphinxstyleliteralstrong{out\_dims} (\sphinxtitleref{tuple}, optional) \textendash{} Tuple of the output array dimensions.

\item {} 
\sphinxstyleliteralstrong{change\_sign} (\sphinxtitleref{bool}, optional) \textendash{} \sphinxcode{True} if the field should change sign through the symmetry plane, \sphinxcode{False} otherwise.

\end{itemize}

\item[{Returns}] \leavevmode
A deep copy of \sphinxcode{phi\_}.

\item[{Return type}] \leavevmode
array\_like

\end{description}\end{quote}

\begin{sphinxadmonition}{note}{Note:}
The arguments \sphinxcode{out\_dims} and \sphinxcode{change\_sign} are not required by the implementation,
yet they are retained as optional arguments for compliancy with the class hierarchy interface.
\end{sphinxadmonition}

\end{fulllineitems}

\index{from\_physical\_to\_computational\_domain() (dycore.horizontal\_boundary.Relaxed method)}

\begin{fulllineitems}
\phantomsection\label{\detokenize{api:dycore.horizontal_boundary.Relaxed.from_physical_to_computational_domain}}\pysiglinewithargsret{\sphinxbfcode{from\_physical\_to\_computational\_domain}}{\emph{phi}}{}
As no extension is required to apply relaxed boundary conditions, return a deep copy of the
input field \sphinxcode{phi}.
\begin{quote}\begin{description}
\item[{Parameters}] \leavevmode
\sphinxstyleliteralstrong{phi} (\sphinxstyleliteralemphasis{array\_like}) \textendash{} A \sphinxhref{https://docs.scipy.org/doc/numpy-1.13.0/reference/generated/numpy.ndarray.html\#numpy.ndarray}{\sphinxcode{numpy.ndarray}}.

\item[{Returns}] \leavevmode
A deep copy of \sphinxcode{phi}.

\item[{Return type}] \leavevmode
array\_like

\end{description}\end{quote}

\end{fulllineitems}

\index{set\_outermost\_layers\_x() (dycore.horizontal\_boundary.Relaxed method)}

\begin{fulllineitems}
\phantomsection\label{\detokenize{api:dycore.horizontal_boundary.Relaxed.set_outermost_layers_x}}\pysiglinewithargsret{\sphinxbfcode{set\_outermost\_layers\_x}}{\emph{phi\_new}, \emph{phi\_now}}{}
Set the outermost layers of \sphinxcode{phi\_new} in the \sphinxcode{x}-direction equal to the corresponding
layers of \sphinxcode{phi\_now}. In other words, apply Dirichlet conditions in \sphinxcode{x}-direction.
\begin{quote}\begin{description}
\item[{Parameters}] \leavevmode\begin{itemize}
\item {} 
\sphinxstyleliteralstrong{phi\_new} (\sphinxstyleliteralemphasis{array\_like}) \textendash{} \sphinxhref{https://docs.scipy.org/doc/numpy-1.13.0/reference/generated/numpy.ndarray.html\#numpy.ndarray}{\sphinxcode{numpy.ndarray}} representing the field on which applying the boundary conditions.

\item {} 
\sphinxstyleliteralstrong{phi\_now} (\sphinxstyleliteralemphasis{array\_like}) \textendash{} \sphinxhref{https://docs.scipy.org/doc/numpy-1.13.0/reference/generated/numpy.ndarray.html\#numpy.ndarray}{\sphinxcode{numpy.ndarray}} representing the field at the current time.

\end{itemize}

\end{description}\end{quote}

\begin{sphinxadmonition}{note}{Note:}
The Dirichlet conditions at the boundaries are assumed to be time-independent, so that they
can be inferred from the solution at current time.
\end{sphinxadmonition}

\end{fulllineitems}

\index{set\_outermost\_layers\_y() (dycore.horizontal\_boundary.Relaxed method)}

\begin{fulllineitems}
\phantomsection\label{\detokenize{api:dycore.horizontal_boundary.Relaxed.set_outermost_layers_y}}\pysiglinewithargsret{\sphinxbfcode{set\_outermost\_layers\_y}}{\emph{phi\_new}, \emph{phi\_now}}{}
Set the outermost layers of \sphinxcode{phi\_new} in the \sphinxcode{y}-direction equal to the corresponding
layers of \sphinxcode{phi\_now}. In other words, apply Dirichelt conditions in \sphinxcode{y}-direction.
\begin{quote}\begin{description}
\item[{Parameters}] \leavevmode\begin{itemize}
\item {} 
\sphinxstyleliteralstrong{phi\_new} (\sphinxstyleliteralemphasis{array\_like}) \textendash{} \sphinxhref{https://docs.scipy.org/doc/numpy-1.13.0/reference/generated/numpy.ndarray.html\#numpy.ndarray}{\sphinxcode{numpy.ndarray}} representing the field on which applying the boundary conditions.

\item {} 
\sphinxstyleliteralstrong{phi\_now} (\sphinxstyleliteralemphasis{array\_like}) \textendash{} \sphinxhref{https://docs.scipy.org/doc/numpy-1.13.0/reference/generated/numpy.ndarray.html\#numpy.ndarray}{\sphinxcode{numpy.ndarray}} representing the field at the current time.

\end{itemize}

\end{description}\end{quote}

\begin{sphinxadmonition}{note}{Note:}
The Dirichlet conditions at the boundaries are assumed to be time-independent, so that they
can be inferred from the solution at current time.
\end{sphinxadmonition}

\end{fulllineitems}


\end{fulllineitems}

\index{RelaxedSymmetricXZ (class in dycore.horizontal\_boundary)}

\begin{fulllineitems}
\phantomsection\label{\detokenize{api:dycore.horizontal_boundary.RelaxedSymmetricXZ}}\pysiglinewithargsret{\sphinxbfcode{class }\sphinxcode{dycore.horizontal\_boundary.}\sphinxbfcode{RelaxedSymmetricXZ}}{\emph{grid}, \emph{nb}}{}
This class inherits {\hyperref[\detokenize{api:dycore.horizontal_boundary.Relaxed}]{\sphinxcrossref{\sphinxcode{Relaxed}}}} to implement horizontally relaxed boundary conditions
for fields symmetric with respect to the \(xz\)-plane \(y = y_c = 0.5 (a_y + b_y)\),
where \(a_y\) and \(b_y\) denote the extremes of the domain in the \(y\)-direction.
\begin{quote}\begin{description}
\item[{Variables}] \leavevmode\begin{itemize}
\item {} 
{\hyperref[\detokenize{api:dycore.prognostic_isentropic.PrognosticIsentropic.nb}]{\sphinxcrossref{\sphinxstyleliteralstrong{nb}}}} (\sphinxstyleliteralemphasis{int}) \textendash{} Number of boundary layers.

\item {} 
\sphinxstyleliteralstrong{nr} (\sphinxstyleliteralemphasis{int}) \textendash{} Number of layers which will be affected by relaxation.

\end{itemize}

\end{description}\end{quote}
\index{\_\_init\_\_() (dycore.horizontal\_boundary.RelaxedSymmetricXZ method)}

\begin{fulllineitems}
\phantomsection\label{\detokenize{api:dycore.horizontal_boundary.RelaxedSymmetricXZ.__init__}}\pysiglinewithargsret{\sphinxbfcode{\_\_init\_\_}}{\emph{grid}, \emph{nb}}{}
Constructor.
\begin{quote}\begin{description}
\item[{Parameters}] \leavevmode\begin{itemize}
\item {} 
\sphinxstyleliteralstrong{grid} (\sphinxstyleliteralemphasis{obj}) \textendash{} The underlying grid, as an instance of {\hyperref[\detokenize{api:grids.grid_xyz.GridXYZ}]{\sphinxcrossref{\sphinxcode{GridXYZ}}}} or one of its derived classes.

\item {} 
\sphinxstyleliteralstrong{nb} (\sphinxstyleliteralemphasis{int}) \textendash{} Number of boundary layers.

\end{itemize}

\end{description}\end{quote}

\end{fulllineitems}

\index{from\_computational\_to\_physical\_domain() (dycore.horizontal\_boundary.RelaxedSymmetricXZ method)}

\begin{fulllineitems}
\phantomsection\label{\detokenize{api:dycore.horizontal_boundary.RelaxedSymmetricXZ.from_computational_to_physical_domain}}\pysiglinewithargsret{\sphinxbfcode{from\_computational\_to\_physical\_domain}}{\emph{phi\_}, \emph{out\_dims}, \emph{change\_sign=False}}{}
Mirror the computational domain with respect to the \(xz\)-plane \(y = y_c\).
\begin{quote}\begin{description}
\item[{Parameters}] \leavevmode\begin{itemize}
\item {} 
\sphinxstyleliteralstrong{phi} (\sphinxstyleliteralemphasis{array\_like}) \textendash{} \sphinxhref{https://docs.scipy.org/doc/numpy-1.13.0/reference/generated/numpy.ndarray.html\#numpy.ndarray}{\sphinxcode{numpy.ndarray}} representing the computational domain of a stencil.

\item {} 
\sphinxstyleliteralstrong{out\_dims} (\sphinxstyleliteralemphasis{tuple}) \textendash{} Tuple of the output array dimensions.

\item {} 
\sphinxstyleliteralstrong{change\_sign} (\sphinxtitleref{bool}, optional) \textendash{} \sphinxcode{True} if the field should change sign through the symmetry plane, \sphinxcode{False} otherwise.
Default is false.

\end{itemize}

\item[{Returns}] \leavevmode
\sphinxhref{https://docs.scipy.org/doc/numpy-1.13.0/reference/generated/numpy.ndarray.html\#numpy.ndarray}{\sphinxcode{numpy.ndarray}} representing the field defined over the physical domain.

\item[{Return type}] \leavevmode
array\_like

\end{description}\end{quote}

\end{fulllineitems}

\index{from\_physical\_to\_computational\_domain() (dycore.horizontal\_boundary.RelaxedSymmetricXZ method)}

\begin{fulllineitems}
\phantomsection\label{\detokenize{api:dycore.horizontal_boundary.RelaxedSymmetricXZ.from_physical_to_computational_domain}}\pysiglinewithargsret{\sphinxbfcode{from\_physical\_to\_computational\_domain}}{\emph{phi}}{}
Return the \(y\)-lowermost half of the domain. To accomodate symmetric conditions,
we retain (at least) \sphinxcode{nb} additional layers in the positive direction of the \(y\)-axis.
\begin{quote}\begin{description}
\item[{Parameters}] \leavevmode
\sphinxstyleliteralstrong{phi} (\sphinxstyleliteralemphasis{array\_like}) \textendash{} \sphinxhref{https://docs.scipy.org/doc/numpy-1.13.0/reference/generated/numpy.ndarray.html\#numpy.ndarray}{\sphinxcode{numpy.ndarray}} representing the physical field.

\item[{Returns}] \leavevmode
\sphinxhref{https://docs.scipy.org/doc/numpy-1.13.0/reference/generated/numpy.ndarray.html\#numpy.ndarray}{\sphinxcode{numpy.ndarray}} representing the stencils’ computational domain.

\item[{Return type}] \leavevmode
array\_like

\end{description}\end{quote}

\end{fulllineitems}


\end{fulllineitems}

\index{RelaxedSymmetricYZ (class in dycore.horizontal\_boundary)}

\begin{fulllineitems}
\phantomsection\label{\detokenize{api:dycore.horizontal_boundary.RelaxedSymmetricYZ}}\pysiglinewithargsret{\sphinxbfcode{class }\sphinxcode{dycore.horizontal\_boundary.}\sphinxbfcode{RelaxedSymmetricYZ}}{\emph{grid}, \emph{nb}}{}
This class inherits {\hyperref[\detokenize{api:dycore.horizontal_boundary.Relaxed}]{\sphinxcrossref{\sphinxcode{Relaxed}}}} to implement horizontally relaxed boundary conditions
for fields symmetric with respect to the \(yz\)-plane \(x = x_c = 0.5 (a_x + b_x)\),
where \(a_x\) and \(b_x\) denote the extremes of the domain in the \(x\)-direction.
\begin{quote}\begin{description}
\item[{Variables}] \leavevmode\begin{itemize}
\item {} 
{\hyperref[\detokenize{api:dycore.prognostic_isentropic.PrognosticIsentropic.nb}]{\sphinxcrossref{\sphinxstyleliteralstrong{nb}}}} (\sphinxstyleliteralemphasis{int}) \textendash{} Number of boundary layers.

\item {} 
\sphinxstyleliteralstrong{nr} (\sphinxstyleliteralemphasis{int}) \textendash{} Number of layers which will be affected by relaxation.

\end{itemize}

\end{description}\end{quote}
\index{\_\_init\_\_() (dycore.horizontal\_boundary.RelaxedSymmetricYZ method)}

\begin{fulllineitems}
\phantomsection\label{\detokenize{api:dycore.horizontal_boundary.RelaxedSymmetricYZ.__init__}}\pysiglinewithargsret{\sphinxbfcode{\_\_init\_\_}}{\emph{grid}, \emph{nb}}{}
Constructor.
\begin{quote}\begin{description}
\item[{Parameters}] \leavevmode\begin{itemize}
\item {} 
\sphinxstyleliteralstrong{grid} (\sphinxstyleliteralemphasis{obj}) \textendash{} The underlying grid, as an instance of {\hyperref[\detokenize{api:grids.grid_xyz.GridXYZ}]{\sphinxcrossref{\sphinxcode{GridXYZ}}}} or one of its derived classes.

\item {} 
\sphinxstyleliteralstrong{nb} (\sphinxstyleliteralemphasis{int}) \textendash{} Number of boundary layers.

\end{itemize}

\end{description}\end{quote}

\end{fulllineitems}

\index{from\_computational\_to\_physical\_domain() (dycore.horizontal\_boundary.RelaxedSymmetricYZ method)}

\begin{fulllineitems}
\phantomsection\label{\detokenize{api:dycore.horizontal_boundary.RelaxedSymmetricYZ.from_computational_to_physical_domain}}\pysiglinewithargsret{\sphinxbfcode{from\_computational\_to\_physical\_domain}}{\emph{phi\_}, \emph{out\_dims}, \emph{change\_sign=False}}{}
Mirror the computational domain with respect to the \(yz\)-axis \(x = x_c\).
\begin{quote}\begin{description}
\item[{Parameters}] \leavevmode\begin{itemize}
\item {} 
\sphinxstyleliteralstrong{phi} (\sphinxstyleliteralemphasis{array\_like}) \textendash{} \sphinxhref{https://docs.scipy.org/doc/numpy-1.13.0/reference/generated/numpy.ndarray.html\#numpy.ndarray}{\sphinxcode{numpy.ndarray}} representing the computational domain of a stencil.

\item {} 
\sphinxstyleliteralstrong{out\_dims} (\sphinxstyleliteralemphasis{tuple}) \textendash{} Tuple of the output array dimensions.

\item {} 
\sphinxstyleliteralstrong{change\_sign} (\sphinxtitleref{bool}, optional) \textendash{} \sphinxcode{True} if the field should change sign through the symmetry plane, \sphinxcode{False} otherwise.
Default is false.

\end{itemize}

\item[{Returns}] \leavevmode
\sphinxhref{https://docs.scipy.org/doc/numpy-1.13.0/reference/generated/numpy.ndarray.html\#numpy.ndarray}{\sphinxcode{numpy.ndarray}} representing the field defined over the physical domain.

\item[{Return type}] \leavevmode
array\_like

\end{description}\end{quote}

\end{fulllineitems}

\index{from\_physical\_to\_computational\_domain() (dycore.horizontal\_boundary.RelaxedSymmetricYZ method)}

\begin{fulllineitems}
\phantomsection\label{\detokenize{api:dycore.horizontal_boundary.RelaxedSymmetricYZ.from_physical_to_computational_domain}}\pysiglinewithargsret{\sphinxbfcode{from\_physical\_to\_computational\_domain}}{\emph{phi}}{}
Return the \(x\)-lowermost half of the domain. To accomodate symmetric conditions,
we retain (at least) \sphinxcode{nb} additional layers in the positive direction of the \(x\)-axis.
\begin{quote}\begin{description}
\item[{Parameters}] \leavevmode
\sphinxstyleliteralstrong{phi} (\sphinxstyleliteralemphasis{array\_like}) \textendash{} \sphinxhref{https://docs.scipy.org/doc/numpy-1.13.0/reference/generated/numpy.ndarray.html\#numpy.ndarray}{\sphinxcode{numpy.ndarray}} representing the physical field.

\item[{Returns}] \leavevmode
\sphinxhref{https://docs.scipy.org/doc/numpy-1.13.0/reference/generated/numpy.ndarray.html\#numpy.ndarray}{\sphinxcode{numpy.ndarray}} representing the stencils’ computational domain.

\item[{Return type}] \leavevmode
array\_like

\end{description}\end{quote}

\end{fulllineitems}


\end{fulllineitems}



\subsection{Numerical diffusion}
\label{\detokenize{api:numerical-diffusion}}\index{Diffusion (class in dycore.diffusion)}

\begin{fulllineitems}
\phantomsection\label{\detokenize{api:dycore.diffusion.Diffusion}}\pysiglinewithargsret{\sphinxbfcode{class }\sphinxcode{dycore.diffusion.}\sphinxbfcode{Diffusion}}{\emph{dims}, \emph{grid}, \emph{diff\_damp\_depth}, \emph{diff\_coeff}, \emph{diff\_max}, \emph{backend}}{}
Abstract base class whose derived classes apply horizontal numerical diffusion to a generic (prognostic)
field by means of a GT4Py’s stencil.
\index{\_\_init\_\_() (dycore.diffusion.Diffusion method)}

\begin{fulllineitems}
\phantomsection\label{\detokenize{api:dycore.diffusion.Diffusion.__init__}}\pysiglinewithargsret{\sphinxbfcode{\_\_init\_\_}}{\emph{dims}, \emph{grid}, \emph{diff\_damp\_depth}, \emph{diff\_coeff}, \emph{diff\_max}, \emph{backend}}{}
Constructor.
\begin{quote}\begin{description}
\item[{Parameters}] \leavevmode\begin{itemize}
\item {} 
\sphinxstyleliteralstrong{dims} (\sphinxstyleliteralemphasis{tuple}) \textendash{} Tuple of the dimension of the arrays on which to apply numerical diffusion.

\item {} 
\sphinxstyleliteralstrong{grid} (\sphinxstyleliteralemphasis{obj}) \textendash{} The underlying grid, as an instance of {\hyperref[\detokenize{api:grids.grid_xyz.GridXYZ}]{\sphinxcrossref{\sphinxcode{GridXYZ}}}} or one of its derived classes.

\item {} 
\sphinxstyleliteralstrong{diff\_damp\_depth} (\sphinxstyleliteralemphasis{int}) \textendash{} Depth of the damping region, i.e., number of vertical layers in the damping region.

\item {} 
\sphinxstyleliteralstrong{diff\_coeff} (\sphinxstyleliteralemphasis{float}) \textendash{} Value for the diffusion coefficient far from the top boundary.

\item {} 
\sphinxstyleliteralstrong{diff\_max} (\sphinxstyleliteralemphasis{float}) \textendash{} Maximum value for the diffusion coefficient.

\item {} 
\sphinxstyleliteralstrong{backend} (\sphinxstyleliteralemphasis{obj}) \textendash{} \sphinxcode{gridtools.mode} specifying the backend for the GT4Py’s stencil implementing numerical diffusion.

\end{itemize}

\end{description}\end{quote}

\end{fulllineitems}

\index{apply() (dycore.diffusion.Diffusion method)}

\begin{fulllineitems}
\phantomsection\label{\detokenize{api:dycore.diffusion.Diffusion.apply}}\pysiglinewithargsret{\sphinxbfcode{apply}}{\emph{phi}}{}
Apply horizontal diffusion to a prognostic field.
As this method is marked as abstract, the implementation is delegated to the derived classes.
\begin{quote}\begin{description}
\item[{Parameters}] \leavevmode
\sphinxstyleliteralstrong{phi} (\sphinxstyleliteralemphasis{array\_like}) \textendash{} \sphinxhref{https://docs.scipy.org/doc/numpy-1.13.0/reference/generated/numpy.ndarray.html\#numpy.ndarray}{\sphinxcode{numpy.ndarray}} representing the field to diffuse.

\item[{Returns}] \leavevmode
\sphinxhref{https://docs.scipy.org/doc/numpy-1.13.0/reference/generated/numpy.ndarray.html\#numpy.ndarray}{\sphinxcode{numpy.ndarray}} representing the diffused field.

\item[{Return type}] \leavevmode
array\_like

\end{description}\end{quote}

\end{fulllineitems}

\index{factory() (dycore.diffusion.Diffusion static method)}

\begin{fulllineitems}
\phantomsection\label{\detokenize{api:dycore.diffusion.Diffusion.factory}}\pysiglinewithargsret{\sphinxbfcode{static }\sphinxbfcode{factory}}{\emph{dims}, \emph{grid}, \emph{diff\_damp\_depth}, \emph{diff\_coeff}, \emph{diff\_max}, \emph{backend}}{}
Static method returning an instance of the derived class appropriate to the field dimension.
\begin{quote}\begin{description}
\item[{Parameters}] \leavevmode\begin{itemize}
\item {} 
\sphinxstyleliteralstrong{dims} (\sphinxstyleliteralemphasis{tuple}) \textendash{} Tuple of the dimension of the arrays on which to apply numerical diffusion.

\item {} 
\sphinxstyleliteralstrong{grid} (\sphinxstyleliteralemphasis{obj}) \textendash{} The underlying grid, as an instance of {\hyperref[\detokenize{api:grids.grid_xyz.GridXYZ}]{\sphinxcrossref{\sphinxcode{GridXYZ}}}} or one of its derived classes.

\item {} 
\sphinxstyleliteralstrong{diff\_damp\_depth} (\sphinxstyleliteralemphasis{int}) \textendash{} Depth of the damping region, i.e., number of vertical layers in the damping region.

\item {} 
\sphinxstyleliteralstrong{diff\_coeff} (\sphinxstyleliteralemphasis{float}) \textendash{} Value for the diffusion coefficient far from the top boundary.

\item {} 
\sphinxstyleliteralstrong{diff\_max} (\sphinxstyleliteralemphasis{float}) \textendash{} Maximum value for the diffusion coefficient. For the sake of numerical stability.

\item {} 
\sphinxstyleliteralstrong{backend} (\sphinxstyleliteralemphasis{obj}) \textendash{} \sphinxcode{gridtools.mode} specifying the backend for the GT4Py’s stencil implementing numerical
diffusion. Default is \sphinxcode{gridtools.mode.NUMPY}.

\end{itemize}

\item[{Returns}] \leavevmode
Instance of the derived class appropriate to the field dimension.

\item[{Return type}] \leavevmode
obj

\end{description}\end{quote}

\end{fulllineitems}


\end{fulllineitems}

\index{DiffusionXYZ (class in dycore.diffusion)}

\begin{fulllineitems}
\phantomsection\label{\detokenize{api:dycore.diffusion.DiffusionXYZ}}\pysiglinewithargsret{\sphinxbfcode{class }\sphinxcode{dycore.diffusion.}\sphinxbfcode{DiffusionXYZ}}{\emph{dims}, \emph{grid}, \emph{diff\_damp\_depth=10}, \emph{diff\_coeff=0.03}, \emph{diff\_max=0.24}, \emph{backend=\textless{}Mode.NUMPY: 4\textgreater{}}}{}
This class inherits {\hyperref[\detokenize{api:dycore.diffusion.Diffusion}]{\sphinxcrossref{\sphinxcode{Diffusion}}}} to apply numerical diffusion to any three-dimensional
field with at least three elements in each direction.
\index{\_\_init\_\_() (dycore.diffusion.DiffusionXYZ method)}

\begin{fulllineitems}
\phantomsection\label{\detokenize{api:dycore.diffusion.DiffusionXYZ.__init__}}\pysiglinewithargsret{\sphinxbfcode{\_\_init\_\_}}{\emph{dims}, \emph{grid}, \emph{diff\_damp\_depth=10}, \emph{diff\_coeff=0.03}, \emph{diff\_max=0.24}, \emph{backend=\textless{}Mode.NUMPY: 4\textgreater{}}}{}
Constructor.
\begin{quote}\begin{description}
\item[{Parameters}] \leavevmode\begin{itemize}
\item {} 
\sphinxstyleliteralstrong{dims} (\sphinxstyleliteralemphasis{tuple}) \textendash{} Tuple of the dimension of the arrays on which to apply numerical diffusion.

\item {} 
\sphinxstyleliteralstrong{grid} (\sphinxstyleliteralemphasis{obj}) \textendash{} The underlying grid, as an instance of {\hyperref[\detokenize{api:grids.grid_xyz.GridXYZ}]{\sphinxcrossref{\sphinxcode{GridXYZ}}}} or one of its derived classes.

\item {} 
\sphinxstyleliteralstrong{diff\_damp\_depth} (\sphinxtitleref{int}, optional) \textendash{} Depth of the damping region, i.e., number of vertical layers in the damping region. Default is 10.

\item {} 
\sphinxstyleliteralstrong{diff\_coeff} (\sphinxtitleref{float}, optional) \textendash{} Value for the diffusion coefficient far from the top boundary. Default is 0.03.

\item {} 
\sphinxstyleliteralstrong{diff\_max} (\sphinxtitleref{float}, optional) \textendash{} Maximum value for the diffusion coefficient. For the sake of numerical stability, it should not
exceed 0.25. Default is 0.24.

\item {} 
\sphinxstyleliteralstrong{backend} (\sphinxtitleref{obj}, optional) \textendash{} \sphinxcode{gridtools.mode} specifying the backend for the GT4Py’s stencil implementing numerical
diffusion. Default is \sphinxcode{gridtools.mode.NUMPY}.

\end{itemize}

\end{description}\end{quote}

\end{fulllineitems}

\index{\_defs\_stencil() (dycore.diffusion.DiffusionXYZ method)}

\begin{fulllineitems}
\phantomsection\label{\detokenize{api:dycore.diffusion.DiffusionXYZ._defs_stencil}}\pysiglinewithargsret{\sphinxbfcode{\_defs\_stencil}}{\emph{in\_phi}, \emph{tau}}{}
The GT4Py’s stencil applying horizontal diffusion. A standard 5-points formula is used.
\begin{quote}\begin{description}
\item[{Parameters}] \leavevmode\begin{itemize}
\item {} 
\sphinxstyleliteralstrong{in\_phi} (\sphinxstyleliteralemphasis{obj}) \textendash{} \sphinxcode{gridtools.Equation} representing the input field to diffuse.

\item {} 
\sphinxstyleliteralstrong{tau} (\sphinxstyleliteralemphasis{obj}) \textendash{} \sphinxcode{gridtools.Equation} representing the diffusion coefficient.

\end{itemize}

\item[{Returns}] \leavevmode
\sphinxcode{gridtools.Equation} representing the diffused output field.

\item[{Return type}] \leavevmode
obj

\end{description}\end{quote}

\end{fulllineitems}

\index{\_initialize\_stencil() (dycore.diffusion.DiffusionXYZ method)}

\begin{fulllineitems}
\phantomsection\label{\detokenize{api:dycore.diffusion.DiffusionXYZ._initialize_stencil}}\pysiglinewithargsret{\sphinxbfcode{\_initialize\_stencil}}{\emph{phi}}{}
Initialize the GT4Py’s stencil applying horizontal diffusion.
\begin{quote}\begin{description}
\item[{Parameters}] \leavevmode
\sphinxstyleliteralstrong{phi} (\sphinxstyleliteralemphasis{array\_like}) \textendash{} \sphinxhref{https://docs.scipy.org/doc/numpy-1.13.0/reference/generated/numpy.ndarray.html\#numpy.ndarray}{\sphinxcode{numpy.ndarray}} representing the field to diffuse.

\end{description}\end{quote}

\end{fulllineitems}

\index{apply() (dycore.diffusion.DiffusionXYZ method)}

\begin{fulllineitems}
\phantomsection\label{\detokenize{api:dycore.diffusion.DiffusionXYZ.apply}}\pysiglinewithargsret{\sphinxbfcode{apply}}{\emph{phi}}{}
Apply horizontal diffusion to a prognostic field.
\begin{quote}\begin{description}
\item[{Parameters}] \leavevmode
\sphinxstyleliteralstrong{phi} (\sphinxstyleliteralemphasis{array\_like}) \textendash{} \sphinxhref{https://docs.scipy.org/doc/numpy-1.13.0/reference/generated/numpy.ndarray.html\#numpy.ndarray}{\sphinxcode{numpy.ndarray}} representing the field to diffuse.

\item[{Returns}] \leavevmode
\sphinxhref{https://docs.scipy.org/doc/numpy-1.13.0/reference/generated/numpy.ndarray.html\#numpy.ndarray}{\sphinxcode{numpy.ndarray}} representing the diffused field.

\item[{Return type}] \leavevmode
array\_like

\end{description}\end{quote}

\end{fulllineitems}


\end{fulllineitems}

\index{DiffusionXZ (class in dycore.diffusion)}

\begin{fulllineitems}
\phantomsection\label{\detokenize{api:dycore.diffusion.DiffusionXZ}}\pysiglinewithargsret{\sphinxbfcode{class }\sphinxcode{dycore.diffusion.}\sphinxbfcode{DiffusionXZ}}{\emph{dims}, \emph{grid}, \emph{diff\_damp\_depth=10}, \emph{diff\_coeff=0.03}, \emph{diff\_max=0.49}, \emph{backend=\textless{}Mode.NUMPY: 4\textgreater{}}}{}
This class inherits {\hyperref[\detokenize{api:dycore.diffusion.Diffusion}]{\sphinxcrossref{\sphinxcode{Diffusion}}}} to apply numerical diffusion to any three-dimensional
field with only one element in the \(y\)-direction.
\index{\_\_init\_\_() (dycore.diffusion.DiffusionXZ method)}

\begin{fulllineitems}
\phantomsection\label{\detokenize{api:dycore.diffusion.DiffusionXZ.__init__}}\pysiglinewithargsret{\sphinxbfcode{\_\_init\_\_}}{\emph{dims}, \emph{grid}, \emph{diff\_damp\_depth=10}, \emph{diff\_coeff=0.03}, \emph{diff\_max=0.49}, \emph{backend=\textless{}Mode.NUMPY: 4\textgreater{}}}{}
Constructor.
\begin{quote}\begin{description}
\item[{Parameters}] \leavevmode\begin{itemize}
\item {} 
\sphinxstyleliteralstrong{dims} (\sphinxstyleliteralemphasis{tuple}) \textendash{} Tuple of the dimension of the arrays on which to apply numerical diffusion.

\item {} 
\sphinxstyleliteralstrong{grid} (\sphinxstyleliteralemphasis{obj}) \textendash{} The underlying grid, as an instance of {\hyperref[\detokenize{api:grids.grid_xyz.GridXYZ}]{\sphinxcrossref{\sphinxcode{GridXYZ}}}} or one of its derived classes.

\item {} 
\sphinxstyleliteralstrong{diff\_damp\_depth} (\sphinxtitleref{int}, optional) \textendash{} Depth of the damping region, i.e., number of vertical layers in the damping region. Default is 10.

\item {} 
\sphinxstyleliteralstrong{diff\_coeff} (\sphinxtitleref{float}, optional) \textendash{} Value for the diffusion coefficient far from the top boundary. Default is 0.03.

\item {} 
\sphinxstyleliteralstrong{diff\_max} (\sphinxtitleref{float}, optional) \textendash{} Maximum value for the diffusion coefficient. For the sake of numerical stability, it should not
exceed 0.5. Default is 0.49.

\item {} 
\sphinxstyleliteralstrong{backend} (\sphinxtitleref{obj}, optional) \textendash{} \sphinxcode{gridtools.mode} specifying the backend for the GT4Py’s stencil implementing numerical
diffusion. Default is \sphinxcode{gridtools.mode.NUMPY}.

\end{itemize}

\end{description}\end{quote}

\end{fulllineitems}

\index{\_defs\_stencil() (dycore.diffusion.DiffusionXZ method)}

\begin{fulllineitems}
\phantomsection\label{\detokenize{api:dycore.diffusion.DiffusionXZ._defs_stencil}}\pysiglinewithargsret{\sphinxbfcode{\_defs\_stencil}}{\emph{in\_phi}, \emph{tau}}{}
The GT4Py’s stencil applying horizontal diffusion. A standard 3-points formula is used.
\begin{quote}\begin{description}
\item[{Parameters}] \leavevmode\begin{itemize}
\item {} 
\sphinxstyleliteralstrong{in\_phi} (\sphinxstyleliteralemphasis{obj}) \textendash{} \sphinxcode{gridtools.Equation} representing the input field to diffuse.

\item {} 
\sphinxstyleliteralstrong{tau} (\sphinxstyleliteralemphasis{obj}) \textendash{} \sphinxcode{gridtools.Equation} representing the diffusion coefficient.

\end{itemize}

\item[{Returns}] \leavevmode
\sphinxcode{gridtools.Equation} representing the diffused output field.

\item[{Return type}] \leavevmode
obj

\end{description}\end{quote}

\end{fulllineitems}

\index{\_initialize\_stencil() (dycore.diffusion.DiffusionXZ method)}

\begin{fulllineitems}
\phantomsection\label{\detokenize{api:dycore.diffusion.DiffusionXZ._initialize_stencil}}\pysiglinewithargsret{\sphinxbfcode{\_initialize\_stencil}}{\emph{phi}}{}
Initialize the GT4Py’s stencil applying horizontal diffusion.
\begin{quote}\begin{description}
\item[{Parameters}] \leavevmode
\sphinxstyleliteralstrong{phi} (\sphinxstyleliteralemphasis{array\_like}) \textendash{} \sphinxhref{https://docs.scipy.org/doc/numpy-1.13.0/reference/generated/numpy.ndarray.html\#numpy.ndarray}{\sphinxcode{numpy.ndarray}} representing the field to diffuse.

\end{description}\end{quote}

\end{fulllineitems}

\index{apply() (dycore.diffusion.DiffusionXZ method)}

\begin{fulllineitems}
\phantomsection\label{\detokenize{api:dycore.diffusion.DiffusionXZ.apply}}\pysiglinewithargsret{\sphinxbfcode{apply}}{\emph{phi}}{}
Apply horizontal diffusion to a prognostic field.
\begin{quote}\begin{description}
\item[{Parameters}] \leavevmode
\sphinxstyleliteralstrong{phi} (\sphinxstyleliteralemphasis{array\_like}) \textendash{} \sphinxhref{https://docs.scipy.org/doc/numpy-1.13.0/reference/generated/numpy.ndarray.html\#numpy.ndarray}{\sphinxcode{numpy.ndarray}} representing the field to diffuse.

\item[{Returns}] \leavevmode
\sphinxhref{https://docs.scipy.org/doc/numpy-1.13.0/reference/generated/numpy.ndarray.html\#numpy.ndarray}{\sphinxcode{numpy.ndarray}} representing the diffused field.

\item[{Return type}] \leavevmode
array\_like

\end{description}\end{quote}

\end{fulllineitems}


\end{fulllineitems}

\index{DiffusionYZ (class in dycore.diffusion)}

\begin{fulllineitems}
\phantomsection\label{\detokenize{api:dycore.diffusion.DiffusionYZ}}\pysiglinewithargsret{\sphinxbfcode{class }\sphinxcode{dycore.diffusion.}\sphinxbfcode{DiffusionYZ}}{\emph{dims}, \emph{grid}, \emph{diff\_damp\_depth=10}, \emph{diff\_coeff=0.03}, \emph{diff\_max=0.49}, \emph{backend=\textless{}Mode.NUMPY: 4\textgreater{}}}{}
This class inherits {\hyperref[\detokenize{api:dycore.diffusion.Diffusion}]{\sphinxcrossref{\sphinxcode{Diffusion}}}} to apply numerical diffusion to any three-dimensional
field with only one element in the \(x\)-direction.
\index{\_\_init\_\_() (dycore.diffusion.DiffusionYZ method)}

\begin{fulllineitems}
\phantomsection\label{\detokenize{api:dycore.diffusion.DiffusionYZ.__init__}}\pysiglinewithargsret{\sphinxbfcode{\_\_init\_\_}}{\emph{dims}, \emph{grid}, \emph{diff\_damp\_depth=10}, \emph{diff\_coeff=0.03}, \emph{diff\_max=0.49}, \emph{backend=\textless{}Mode.NUMPY: 4\textgreater{}}}{}
Constructor.
\begin{quote}\begin{description}
\item[{Parameters}] \leavevmode\begin{itemize}
\item {} 
\sphinxstyleliteralstrong{dims} (\sphinxstyleliteralemphasis{tuple}) \textendash{} Tuple of the dimension of the arrays on which to apply numerical diffusion.

\item {} 
\sphinxstyleliteralstrong{grid} (\sphinxstyleliteralemphasis{obj}) \textendash{} The underlying grid, as an instance of {\hyperref[\detokenize{api:grids.grid_xyz.GridXYZ}]{\sphinxcrossref{\sphinxcode{GridXYZ}}}} or one of its derived classes.

\item {} 
\sphinxstyleliteralstrong{diff\_damp\_depth} (\sphinxstyleliteralemphasis{int}) \textendash{} Depth of the damping region, i.e., number of vertical layers in the damping region. Default is 10.

\item {} 
\sphinxstyleliteralstrong{diff\_coeff} (\sphinxstyleliteralemphasis{float}) \textendash{} Value for the diffusion coefficient far from the top boundary. Default is 0.03.

\item {} 
\sphinxstyleliteralstrong{diff\_max} (\sphinxstyleliteralemphasis{float}) \textendash{} Maximum value for the diffusion coefficient. For the sake of numerical stability, it should not
exceed 0.5. Default is 0.49.

\item {} 
\sphinxstyleliteralstrong{backend} (\sphinxstyleliteralemphasis{obj}) \textendash{} \sphinxcode{gridtools.mode} specifying the backend for the GT4Py’s stencil implementing numerical
diffusion. Default is \sphinxcode{gridtools.mode.NUMPY}.

\end{itemize}

\end{description}\end{quote}

\end{fulllineitems}

\index{\_defs\_stencil() (dycore.diffusion.DiffusionYZ method)}

\begin{fulllineitems}
\phantomsection\label{\detokenize{api:dycore.diffusion.DiffusionYZ._defs_stencil}}\pysiglinewithargsret{\sphinxbfcode{\_defs\_stencil}}{\emph{in\_phi}, \emph{tau}}{}
The GT4Py’s stencil applying horizontal diffusion. A standard 3-points formula is used.
\begin{quote}\begin{description}
\item[{Parameters}] \leavevmode\begin{itemize}
\item {} 
\sphinxstyleliteralstrong{in\_phi} (\sphinxstyleliteralemphasis{obj}) \textendash{} \sphinxcode{gridtools.Equation} representing the input field to diffuse.

\item {} 
\sphinxstyleliteralstrong{tau} (\sphinxstyleliteralemphasis{obj}) \textendash{} \sphinxcode{gridtools.Equation} representing the diffusion coefficient.

\end{itemize}

\item[{Returns}] \leavevmode
\sphinxcode{gridtools.Equation} representing the diffused output field.

\item[{Return type}] \leavevmode
obj

\end{description}\end{quote}

\end{fulllineitems}

\index{\_initialize\_stencil() (dycore.diffusion.DiffusionYZ method)}

\begin{fulllineitems}
\phantomsection\label{\detokenize{api:dycore.diffusion.DiffusionYZ._initialize_stencil}}\pysiglinewithargsret{\sphinxbfcode{\_initialize\_stencil}}{\emph{phi}}{}
Initialize the GT4Py’s stencil applying horizontal diffusion.
\begin{quote}\begin{description}
\item[{Parameters}] \leavevmode
\sphinxstyleliteralstrong{phi} (\sphinxstyleliteralemphasis{array\_like}) \textendash{} \sphinxhref{https://docs.scipy.org/doc/numpy-1.13.0/reference/generated/numpy.ndarray.html\#numpy.ndarray}{\sphinxcode{numpy.ndarray}} representing the field to diffuse.

\end{description}\end{quote}

\end{fulllineitems}

\index{apply() (dycore.diffusion.DiffusionYZ method)}

\begin{fulllineitems}
\phantomsection\label{\detokenize{api:dycore.diffusion.DiffusionYZ.apply}}\pysiglinewithargsret{\sphinxbfcode{apply}}{\emph{phi}}{}
Apply horizontal diffusion to a prognostic field.
\begin{quote}\begin{description}
\item[{Parameters}] \leavevmode
\sphinxstyleliteralstrong{phi} (\sphinxstyleliteralemphasis{array\_like}) \textendash{} \sphinxhref{https://docs.scipy.org/doc/numpy-1.13.0/reference/generated/numpy.ndarray.html\#numpy.ndarray}{\sphinxcode{numpy.ndarray}} representing the field to diffuse.

\item[{Returns}] \leavevmode
\sphinxhref{https://docs.scipy.org/doc/numpy-1.13.0/reference/generated/numpy.ndarray.html\#numpy.ndarray}{\sphinxcode{numpy.ndarray}} representing the diffused field.

\item[{Return type}] \leavevmode
array\_like

\end{description}\end{quote}

\end{fulllineitems}


\end{fulllineitems}



\subsection{Numerical fluxes}
\label{\detokenize{api:numerical-fluxes}}\index{FluxIsentropic (class in dycore.flux\_isentropic)}

\begin{fulllineitems}
\phantomsection\label{\detokenize{api:dycore.flux_isentropic.FluxIsentropic}}\pysiglinewithargsret{\sphinxbfcode{class }\sphinxcode{dycore.flux\_isentropic.}\sphinxbfcode{FluxIsentropic}}{\emph{grid}, \emph{imoist}}{}
Abstract base class whose derived classes implement different schemes for computing the numerical fluxes for
the three-dimensional isentropic dynamical core. The conservative form of the governing equations is used.
\index{\_\_init\_\_() (dycore.flux\_isentropic.FluxIsentropic method)}

\begin{fulllineitems}
\phantomsection\label{\detokenize{api:dycore.flux_isentropic.FluxIsentropic.__init__}}\pysiglinewithargsret{\sphinxbfcode{\_\_init\_\_}}{\emph{grid}, \emph{imoist}}{}
Constructor.
\begin{quote}\begin{description}
\item[{Parameters}] \leavevmode\begin{itemize}
\item {} 
\sphinxstyleliteralstrong{grid} (\sphinxstyleliteralemphasis{obj}) \textendash{} {\hyperref[\detokenize{api:grids.grid_xyz.GridXYZ}]{\sphinxcrossref{\sphinxcode{GridXYZ}}}} representing the underlying grid.

\item {} 
\sphinxstyleliteralstrong{imoist} (\sphinxstyleliteralemphasis{bool}) \textendash{} \sphinxcode{True} for a moist dynamical core, \sphinxcode{False} otherwise.

\end{itemize}

\end{description}\end{quote}

\end{fulllineitems}

\index{\_compute\_fluxes() (dycore.flux\_isentropic.FluxIsentropic method)}

\begin{fulllineitems}
\phantomsection\label{\detokenize{api:dycore.flux_isentropic.FluxIsentropic._compute_fluxes}}\pysiglinewithargsret{\sphinxbfcode{\_compute\_fluxes}}{\emph{i}, \emph{j}, \emph{k}, \emph{dt}, \emph{s}, \emph{u}, \emph{v}, \emph{mtg}, \emph{U}, \emph{V}, \emph{Qv}, \emph{Qc}, \emph{Qr}}{}
Method computing the \sphinxcode{gridtools.Equation}s representing the \(x\)- and \(y\)-fluxes for all
the conservative prognostic variables. The :class:{\color{red}\bfseries{}{}`}gridtools.Equation{}`s are then set as instance attributes.
As this method is marked as abstract, the implementation is delegated to the derived classes.
\begin{quote}\begin{description}
\item[{Parameters}] \leavevmode\begin{itemize}
\item {} 
\sphinxstyleliteralstrong{i} (\sphinxstyleliteralemphasis{obj}) \textendash{} \sphinxcode{gridtools.Index} representing the index running along the \(x\)-axis.

\item {} 
\sphinxstyleliteralstrong{j} (\sphinxstyleliteralemphasis{obj}) \textendash{} \sphinxcode{gridtools.Index} representing the index running along the \(y\)-axis.

\item {} 
\sphinxstyleliteralstrong{k} (\sphinxstyleliteralemphasis{obj}) \textendash{} \sphinxcode{gridtools.Index} representing the index running along the \(\theta\)-axis.

\item {} 
\sphinxstyleliteralstrong{dt} (\sphinxstyleliteralemphasis{obj}) \textendash{} \sphinxcode{gridtools.Global} representing the time step.

\item {} 
\sphinxstyleliteralstrong{s} (\sphinxstyleliteralemphasis{obj}) \textendash{} \sphinxcode{gridtools.Equation} representing the isentropic density.

\item {} 
\sphinxstyleliteralstrong{u} (\sphinxstyleliteralemphasis{obj}) \textendash{} \sphinxcode{gridtools.Equation} representing the \(x\)-velocity.

\item {} 
\sphinxstyleliteralstrong{v} (\sphinxstyleliteralemphasis{obj}) \textendash{} \sphinxcode{gridtools.Equation} representing the \(y\)-velocity.

\item {} 
\sphinxstyleliteralstrong{mtg} (\sphinxstyleliteralemphasis{obj}) \textendash{} \sphinxcode{gridtools.Equation} representing the Montgomery potential.

\item {} 
\sphinxstyleliteralstrong{U} (\sphinxstyleliteralemphasis{obj}) \textendash{} \sphinxcode{gridtools.Equation} representing the \(x\)-momentum.

\item {} 
\sphinxstyleliteralstrong{V} (\sphinxstyleliteralemphasis{obj}) \textendash{} \sphinxcode{gridtools.Equation} representing the \(y\)-momentum.

\item {} 
\sphinxstyleliteralstrong{Qv} (\sphinxstyleliteralemphasis{obj}) \textendash{} \sphinxcode{gridtools.Equation} representing the mass of water vapour.

\item {} 
\sphinxstyleliteralstrong{Qc} (\sphinxstyleliteralemphasis{obj}) \textendash{} \sphinxcode{gridtools.Equation} representing the mass of cloud water.

\item {} 
\sphinxstyleliteralstrong{Qr} (\sphinxstyleliteralemphasis{obj}) \textendash{} \sphinxcode{gridtools.Equation} representing the mass of precipitation water.

\end{itemize}

\end{description}\end{quote}

\end{fulllineitems}

\index{factory() (dycore.flux\_isentropic.FluxIsentropic static method)}

\begin{fulllineitems}
\phantomsection\label{\detokenize{api:dycore.flux_isentropic.FluxIsentropic.factory}}\pysiglinewithargsret{\sphinxbfcode{static }\sphinxbfcode{factory}}{\emph{scheme}, \emph{grid}, \emph{imoist}}{}
Static method which returns an instance of the derived class implementing the numerical scheme
specified by \sphinxcode{scheme}.
\begin{quote}\begin{description}
\item[{Parameters}] \leavevmode\begin{itemize}
\item {} 
\sphinxstyleliteralstrong{scheme} (\sphinxstyleliteralemphasis{str}) \textendash{} 
String specifying the numerical scheme to implement. Either:
\begin{itemize}
\item {} 
’upwind’, for the upwind scheme;

\item {} 
’leapfrog’, for the leapfrog scheme;

\item {} 
’maccormack’, for the MacCormack scheme.

\end{itemize}


\item {} 
\sphinxstyleliteralstrong{grid} (\sphinxstyleliteralemphasis{obj}) \textendash{} {\hyperref[\detokenize{api:grids.grid_xyz.GridXYZ}]{\sphinxcrossref{\sphinxcode{GridXYZ}}}} representing the underlying grid.

\item {} 
\sphinxstyleliteralstrong{imoist} (\sphinxstyleliteralemphasis{bool}) \textendash{} \sphinxcode{True} for a moist dynamical core, \sphinxcode{False} otherwise.

\end{itemize}

\item[{Returns}] \leavevmode
Instance of the derived class implementing the scheme specified by \sphinxcode{scheme}.

\item[{Return type}] \leavevmode
obj

\end{description}\end{quote}

\end{fulllineitems}

\index{get\_fluxes() (dycore.flux\_isentropic.FluxIsentropic method)}

\begin{fulllineitems}
\phantomsection\label{\detokenize{api:dycore.flux_isentropic.FluxIsentropic.get_fluxes}}\pysiglinewithargsret{\sphinxbfcode{get\_fluxes}}{\emph{i}, \emph{j}, \emph{k}, \emph{dt}, \emph{s}, \emph{u}, \emph{v}, \emph{mtg}, \emph{U}, \emph{V}, \emph{Qv=None}, \emph{Qc=None}, \emph{Qr=None}}{}
The entry-point method returning the \sphinxcode{gridtools.Equation}s representing the \(x\)- and
\(y\)-fluxes for all the conservative model variables.
\begin{quote}\begin{description}
\item[{Parameters}] \leavevmode\begin{itemize}
\item {} 
\sphinxstyleliteralstrong{i} (\sphinxstyleliteralemphasis{obj}) \textendash{} \sphinxcode{gridtools.Index} representing the index running along the \(x\)-axis.

\item {} 
\sphinxstyleliteralstrong{j} (\sphinxstyleliteralemphasis{obj}) \textendash{} \sphinxcode{gridtools.Index} representing the index running along the \(y\)-axis.

\item {} 
\sphinxstyleliteralstrong{k} (\sphinxstyleliteralemphasis{obj}) \textendash{} \sphinxcode{gridtools.Index} representing the index running along the \(\theta\)-axis.

\item {} 
\sphinxstyleliteralstrong{dt} (\sphinxstyleliteralemphasis{obj}) \textendash{} \sphinxcode{gridtools.Global} representing the time step.

\item {} 
\sphinxstyleliteralstrong{s} (\sphinxstyleliteralemphasis{obj}) \textendash{} \sphinxcode{gridtools.Equation} representing the isentropic density.

\item {} 
\sphinxstyleliteralstrong{u} (\sphinxstyleliteralemphasis{obj}) \textendash{} \sphinxcode{gridtools.Equation} representing the \(x\)-velocity.

\item {} 
\sphinxstyleliteralstrong{v} (\sphinxstyleliteralemphasis{obj}) \textendash{} \sphinxcode{gridtools.Equation} representing the \(y\)-velocity.

\item {} 
\sphinxstyleliteralstrong{mtg} (\sphinxstyleliteralemphasis{obj}) \textendash{} \sphinxcode{gridtools.Equation} representing the Montgomery potential.

\item {} 
\sphinxstyleliteralstrong{U} (\sphinxstyleliteralemphasis{obj}) \textendash{} \sphinxcode{gridtools.Equation} representing the \(x\)-momentum.

\item {} 
\sphinxstyleliteralstrong{V} (\sphinxstyleliteralemphasis{obj}) \textendash{} \sphinxcode{gridtools.Equation} representing the \(y\)-momentum.

\item {} 
\sphinxstyleliteralstrong{Qv} (\sphinxtitleref{obj}, optional) \textendash{} \sphinxcode{gridtools.Equation} representing the mass of water vapour.

\item {} 
\sphinxstyleliteralstrong{Qc} (\sphinxtitleref{obj}, optional) \textendash{} \sphinxcode{gridtools.Equation} representing the mass of cloud water.

\item {} 
\sphinxstyleliteralstrong{Qr} (\sphinxtitleref{obj}, optional) \textendash{} \sphinxcode{gridtools.Equation} representing the mass of precipitation water.

\end{itemize}

\item[{Returns}] \leavevmode
\begin{itemize}
\item {} 
\sphinxstylestrong{flux\_s\_x} (\sphinxstyleemphasis{obj}) \textendash{} \sphinxcode{gridtools.Equation} representing the \(x\)-flux for the isentropic density.

\item {} 
\sphinxstylestrong{flux\_s\_y} (\sphinxstyleemphasis{obj}) \textendash{} \sphinxcode{gridtools.Equation} representing the \(y\)-flux for the isentropic density.

\item {} 
\sphinxstylestrong{flux\_U\_x} (\sphinxstyleemphasis{obj}) \textendash{} \sphinxcode{gridtools.Equation} representing the \(x\)-flux for the \(x\)-momentum.

\item {} 
\sphinxstylestrong{flux\_U\_y} (\sphinxstyleemphasis{obj}) \textendash{} \sphinxcode{gridtools.Equation} representing the \(y\)-flux for the \(x\)-momentum.

\item {} 
\sphinxstylestrong{flux\_V\_x} (\sphinxstyleemphasis{obj}) \textendash{} \sphinxcode{gridtools.Equation} representing the \(x\)-flux for the \(y\)-momentum.

\item {} 
\sphinxstylestrong{flux\_V\_y} (\sphinxstyleemphasis{obj}) \textendash{} \sphinxcode{gridtools.Equation} representing the \(y\)-flux for the \(y\)-momentum.

\item {} 
\sphinxstylestrong{flux\_Qv\_x} (\sphinxtitleref{obj}, optional) \textendash{} \sphinxcode{gridtools.Equation} representing the \(x\)-flux for the mass of water vapour.

\item {} 
\sphinxstylestrong{flux\_Qv\_y} (\sphinxtitleref{obj}, optional) \textendash{} \sphinxcode{gridtools.Equation} representing the \(y\)-flux for the mass of water vapour.

\item {} 
\sphinxstylestrong{flux\_Qc\_x} (\sphinxtitleref{obj}, optional) \textendash{} \sphinxcode{gridtools.Equation} representing the \(x\)-flux for the mass of cloud water.

\item {} 
\sphinxstylestrong{flux\_Qc\_y} (\sphinxtitleref{obj}, optional) \textendash{} \sphinxcode{gridtools.Equation} representing the \(y\)-flux for the mass of cloud water.

\item {} 
\sphinxstylestrong{flux\_Qr\_x} (\sphinxtitleref{obj}, optional) \textendash{} \sphinxcode{gridtools.Equation} representing the \(x\)-flux for the mass of precipitation water.

\item {} 
\sphinxstylestrong{flux\_Qr\_y} (\sphinxtitleref{obj}, optional) \textendash{} \sphinxcode{gridtools.Equation} representing the \(y\)-flux for the mass of precipitation water.

\end{itemize}


\end{description}\end{quote}

\end{fulllineitems}


\end{fulllineitems}

\index{FluxIsentropicUpwind (class in dycore.flux\_isentropic)}

\begin{fulllineitems}
\phantomsection\label{\detokenize{api:dycore.flux_isentropic.FluxIsentropicUpwind}}\pysiglinewithargsret{\sphinxbfcode{class }\sphinxcode{dycore.flux\_isentropic.}\sphinxbfcode{FluxIsentropicUpwind}}{\emph{grid}, \emph{imoist}}{}
Class which inherits {\hyperref[\detokenize{api:dycore.flux_isentropic.FluxIsentropic}]{\sphinxcrossref{\sphinxcode{FluxIsentropic}}}} to implement the upwind scheme applied to the governing equations
in conservative form.
\begin{quote}\begin{description}
\item[{Variables}] \leavevmode
{\hyperref[\detokenize{api:dycore.prognostic_isentropic.PrognosticIsentropic.nb}]{\sphinxcrossref{\sphinxstyleliteralstrong{nb}}}} (\sphinxstyleliteralemphasis{int}) \textendash{} Number of boundary layers.

\end{description}\end{quote}
\index{\_\_init\_\_() (dycore.flux\_isentropic.FluxIsentropicUpwind method)}

\begin{fulllineitems}
\phantomsection\label{\detokenize{api:dycore.flux_isentropic.FluxIsentropicUpwind.__init__}}\pysiglinewithargsret{\sphinxbfcode{\_\_init\_\_}}{\emph{grid}, \emph{imoist}}{}
Constructor.
\begin{quote}\begin{description}
\item[{Parameters}] \leavevmode\begin{itemize}
\item {} 
\sphinxstyleliteralstrong{grid} (\sphinxstyleliteralemphasis{obj}) \textendash{} {\hyperref[\detokenize{api:grids.grid_xyz.GridXYZ}]{\sphinxcrossref{\sphinxcode{GridXYZ}}}} representing the underlying grid.

\item {} 
\sphinxstyleliteralstrong{imoist} (\sphinxstyleliteralemphasis{bool}) \textendash{} \sphinxcode{True} for a moist dynamical core, \sphinxcode{False} otherwise.

\end{itemize}

\end{description}\end{quote}

\end{fulllineitems}

\index{\_compute\_fluxes() (dycore.flux\_isentropic.FluxIsentropicUpwind method)}

\begin{fulllineitems}
\phantomsection\label{\detokenize{api:dycore.flux_isentropic.FluxIsentropicUpwind._compute_fluxes}}\pysiglinewithargsret{\sphinxbfcode{\_compute\_fluxes}}{\emph{i}, \emph{j}, \emph{k}, \emph{dt}, \emph{s}, \emph{u}, \emph{v}, \emph{mtg}, \emph{U}, \emph{V}, \emph{Qv}, \emph{Qc}, \emph{Qr}}{}
Method computing the upwind \sphinxcode{gridtools.Equation}s representing the \(x\)- and \(y\)-fluxes for all
the conservative prognostic variables. The :class:{\color{red}\bfseries{}{}`}gridtools.Equation{}`s are then set as instance attributes.
\begin{quote}\begin{description}
\item[{Parameters}] \leavevmode\begin{itemize}
\item {} 
\sphinxstyleliteralstrong{i} (\sphinxstyleliteralemphasis{obj}) \textendash{} \sphinxcode{gridtools.Index} representing the index running along the \(x\)-axis.

\item {} 
\sphinxstyleliteralstrong{j} (\sphinxstyleliteralemphasis{obj}) \textendash{} \sphinxcode{gridtools.Index} representing the index running along the \(y\)-axis.

\item {} 
\sphinxstyleliteralstrong{k} (\sphinxstyleliteralemphasis{obj}) \textendash{} \sphinxcode{gridtools.Index} representing the index running along the \(\theta\)-axis.

\item {} 
\sphinxstyleliteralstrong{dt} (\sphinxstyleliteralemphasis{obj}) \textendash{} \sphinxcode{gridtools.Global} representing the time step.

\item {} 
\sphinxstyleliteralstrong{s} (\sphinxstyleliteralemphasis{obj}) \textendash{} \sphinxcode{gridtools.Equation} representing the isentropic density.

\item {} 
\sphinxstyleliteralstrong{u} (\sphinxstyleliteralemphasis{obj}) \textendash{} \sphinxcode{gridtools.Equation} representing the \(x\)-velocity.

\item {} 
\sphinxstyleliteralstrong{v} (\sphinxstyleliteralemphasis{obj}) \textendash{} \sphinxcode{gridtools.Equation} representing the \(y\)-velocity.

\item {} 
\sphinxstyleliteralstrong{mtg} (\sphinxstyleliteralemphasis{obj}) \textendash{} \sphinxcode{gridtools.Equation} representing the Montgomery potential.

\item {} 
\sphinxstyleliteralstrong{U} (\sphinxstyleliteralemphasis{obj}) \textendash{} \sphinxcode{gridtools.Equation} representing the \(x\)-momentum.

\item {} 
\sphinxstyleliteralstrong{V} (\sphinxstyleliteralemphasis{obj}) \textendash{} \sphinxcode{gridtools.Equation} representing the \(y\)-momentum.

\item {} 
\sphinxstyleliteralstrong{Qv} (\sphinxstyleliteralemphasis{obj}) \textendash{} \sphinxcode{gridtools.Equation} representing the mass of water vapour.

\item {} 
\sphinxstyleliteralstrong{Qc} (\sphinxstyleliteralemphasis{obj}) \textendash{} \sphinxcode{gridtools.Equation} representing the mass of cloud water.

\item {} 
\sphinxstyleliteralstrong{Qr} (\sphinxstyleliteralemphasis{obj}) \textendash{} \sphinxcode{gridtools.Equation} representing the mass of precipitation water.

\end{itemize}

\end{description}\end{quote}

\end{fulllineitems}

\index{\_get\_upwind\_flux\_x() (dycore.flux\_isentropic.FluxIsentropicUpwind method)}

\begin{fulllineitems}
\phantomsection\label{\detokenize{api:dycore.flux_isentropic.FluxIsentropicUpwind._get_upwind_flux_x}}\pysiglinewithargsret{\sphinxbfcode{\_get\_upwind\_flux\_x}}{\emph{i}, \emph{j}, \emph{k}, \emph{u}, \emph{phi}}{}
Get the \sphinxcode{gridtools.Equation} representing the upwind flux in \(x\)-direction for a generic
prognostic variable \(\phi\).
\begin{quote}\begin{description}
\item[{Parameters}] \leavevmode\begin{itemize}
\item {} 
\sphinxstyleliteralstrong{i} (\sphinxstyleliteralemphasis{obj}) \textendash{} \sphinxcode{gridtools.Index} representing the index running along the \(x\)-axis.

\item {} 
\sphinxstyleliteralstrong{j} (\sphinxstyleliteralemphasis{obj}) \textendash{} \sphinxcode{gridtools.Index} representing the index running along the \(y\)-axis.

\item {} 
\sphinxstyleliteralstrong{k} (\sphinxstyleliteralemphasis{obj}) \textendash{} \sphinxcode{gridtools.Index} representing the index running along the \(\theta\)-axis.

\item {} 
\sphinxstyleliteralstrong{u} (\sphinxstyleliteralemphasis{obj}) \textendash{} \sphinxcode{gridtools.Equation} representing the \(x\)-velocity.

\item {} 
\sphinxstyleliteralstrong{phi} (\sphinxstyleliteralemphasis{obj}) \textendash{} \sphinxcode{gridtools.Equation} representing the field \(\phi\).

\end{itemize}

\item[{Returns}] \leavevmode
\sphinxcode{gridtools.Equation} representing the upwind flux in \(x\)-direction for \(\phi\)

\item[{Return type}] \leavevmode
obj

\end{description}\end{quote}

\end{fulllineitems}

\index{\_get\_upwind\_flux\_y() (dycore.flux\_isentropic.FluxIsentropicUpwind method)}

\begin{fulllineitems}
\phantomsection\label{\detokenize{api:dycore.flux_isentropic.FluxIsentropicUpwind._get_upwind_flux_y}}\pysiglinewithargsret{\sphinxbfcode{\_get\_upwind\_flux\_y}}{\emph{i}, \emph{j}, \emph{k}, \emph{v}, \emph{phi}}{}
Get the \sphinxcode{gridtools.Equation} representing the upwind flux in \(y\)-direction for a generic
prognostic variable \(\phi\).
\begin{quote}\begin{description}
\item[{Parameters}] \leavevmode\begin{itemize}
\item {} 
\sphinxstyleliteralstrong{i} (\sphinxstyleliteralemphasis{obj}) \textendash{} \sphinxcode{gridtools.Index} representing the index running along the \(x\)-axis.

\item {} 
\sphinxstyleliteralstrong{j} (\sphinxstyleliteralemphasis{obj}) \textendash{} \sphinxcode{gridtools.Index} representing the index running along the \(y\)-axis.

\item {} 
\sphinxstyleliteralstrong{k} (\sphinxstyleliteralemphasis{obj}) \textendash{} \sphinxcode{gridtools.Index} representing the index running along the \(\theta\)-axis.

\item {} 
\sphinxstyleliteralstrong{v} (\sphinxstyleliteralemphasis{obj}) \textendash{} \sphinxcode{gridtools.Equation} representing the \(y\)-velocity.

\item {} 
\sphinxstyleliteralstrong{phi} (\sphinxstyleliteralemphasis{obj}) \textendash{} \sphinxcode{gridtools.Equation} representing the field \(\phi\).

\end{itemize}

\item[{Returns}] \leavevmode
\sphinxcode{gridtools.Equation} representing the upwind flux in \(y\)-direction for \(\phi\)

\item[{Return type}] \leavevmode
obj

\end{description}\end{quote}

\end{fulllineitems}


\end{fulllineitems}

\index{FluxIsentropicLeapfrog (class in dycore.flux\_isentropic)}

\begin{fulllineitems}
\phantomsection\label{\detokenize{api:dycore.flux_isentropic.FluxIsentropicLeapfrog}}\pysiglinewithargsret{\sphinxbfcode{class }\sphinxcode{dycore.flux\_isentropic.}\sphinxbfcode{FluxIsentropicLeapfrog}}{\emph{grid}, \emph{imoist}}{}
Class which inherits {\hyperref[\detokenize{api:dycore.flux_isentropic.FluxIsentropic}]{\sphinxcrossref{\sphinxcode{FluxIsentropic}}}} to implement the leapfrog scheme applied to the governing equations
in conservative form.
\begin{quote}\begin{description}
\item[{Variables}] \leavevmode
{\hyperref[\detokenize{api:dycore.prognostic_isentropic.PrognosticIsentropic.nb}]{\sphinxcrossref{\sphinxstyleliteralstrong{nb}}}} (\sphinxstyleliteralemphasis{int}) \textendash{} Number of boundary layers.

\end{description}\end{quote}
\index{\_\_init\_\_() (dycore.flux\_isentropic.FluxIsentropicLeapfrog method)}

\begin{fulllineitems}
\phantomsection\label{\detokenize{api:dycore.flux_isentropic.FluxIsentropicLeapfrog.__init__}}\pysiglinewithargsret{\sphinxbfcode{\_\_init\_\_}}{\emph{grid}, \emph{imoist}}{}
Constructor.
\begin{quote}\begin{description}
\item[{Parameters}] \leavevmode\begin{itemize}
\item {} 
\sphinxstyleliteralstrong{grid} (\sphinxstyleliteralemphasis{obj}) \textendash{} {\hyperref[\detokenize{api:grids.grid_xyz.GridXYZ}]{\sphinxcrossref{\sphinxcode{GridXYZ}}}} representing the underlying grid.

\item {} 
\sphinxstyleliteralstrong{imoist} (\sphinxstyleliteralemphasis{bool}) \textendash{} \sphinxcode{True} for a moist dynamical core, \sphinxcode{False} otherwise.

\end{itemize}

\end{description}\end{quote}

\end{fulllineitems}

\index{\_compute\_fluxes() (dycore.flux\_isentropic.FluxIsentropicLeapfrog method)}

\begin{fulllineitems}
\phantomsection\label{\detokenize{api:dycore.flux_isentropic.FluxIsentropicLeapfrog._compute_fluxes}}\pysiglinewithargsret{\sphinxbfcode{\_compute\_fluxes}}{\emph{i}, \emph{j}, \emph{k}, \emph{dt}, \emph{s}, \emph{u}, \emph{v}, \emph{mtg}, \emph{U}, \emph{V}, \emph{Qv}, \emph{Qc}, \emph{Qr}}{}
Method computing the leapfrog \sphinxcode{gridtools.Equation}s representing the \(x\)- and \(y\)-fluxes for all
the conservative prognostic variables. The :class:{\color{red}\bfseries{}{}`}gridtools.Equation{}`s are then set as instance attributes.
\begin{quote}\begin{description}
\item[{Parameters}] \leavevmode\begin{itemize}
\item {} 
\sphinxstyleliteralstrong{i} (\sphinxstyleliteralemphasis{obj}) \textendash{} \sphinxcode{gridtools.Index} representing the index running along the \(x\)-axis.

\item {} 
\sphinxstyleliteralstrong{j} (\sphinxstyleliteralemphasis{obj}) \textendash{} \sphinxcode{gridtools.Index} representing the index running along the \(y\)-axis.

\item {} 
\sphinxstyleliteralstrong{k} (\sphinxstyleliteralemphasis{obj}) \textendash{} \sphinxcode{gridtools.Index} representing the index running along the \(\theta\)-axis.

\item {} 
\sphinxstyleliteralstrong{dt} (\sphinxstyleliteralemphasis{obj}) \textendash{} \sphinxcode{gridtools.Global} representing the time step.

\item {} 
\sphinxstyleliteralstrong{s} (\sphinxstyleliteralemphasis{obj}) \textendash{} \sphinxcode{gridtools.Equation} representing the isentropic density.

\item {} 
\sphinxstyleliteralstrong{u} (\sphinxstyleliteralemphasis{obj}) \textendash{} \sphinxcode{gridtools.Equation} representing the \(x\)-velocity.

\item {} 
\sphinxstyleliteralstrong{v} (\sphinxstyleliteralemphasis{obj}) \textendash{} \sphinxcode{gridtools.Equation} representing the \(y\)-velocity.

\item {} 
\sphinxstyleliteralstrong{mtg} (\sphinxstyleliteralemphasis{obj}) \textendash{} \sphinxcode{gridtools.Equation} representing the Montgomery potential.

\item {} 
\sphinxstyleliteralstrong{U} (\sphinxstyleliteralemphasis{obj}) \textendash{} \sphinxcode{gridtools.Equation} representing the \(x\)-momentum.

\item {} 
\sphinxstyleliteralstrong{V} (\sphinxstyleliteralemphasis{obj}) \textendash{} \sphinxcode{gridtools.Equation} representing the \(y\)-momentum.

\item {} 
\sphinxstyleliteralstrong{Qv} (\sphinxstyleliteralemphasis{obj}) \textendash{} \sphinxcode{gridtools.Equation} representing the mass of water vapour.

\item {} 
\sphinxstyleliteralstrong{Qc} (\sphinxstyleliteralemphasis{obj}) \textendash{} \sphinxcode{gridtools.Equation} representing the mass of cloud water.

\item {} 
\sphinxstyleliteralstrong{Qr} (\sphinxstyleliteralemphasis{obj}) \textendash{} \sphinxcode{gridtools.Equation} representing the mass of precipitation water.

\end{itemize}

\end{description}\end{quote}

\end{fulllineitems}

\index{\_get\_leapfrog\_flux\_x() (dycore.flux\_isentropic.FluxIsentropicLeapfrog method)}

\begin{fulllineitems}
\phantomsection\label{\detokenize{api:dycore.flux_isentropic.FluxIsentropicLeapfrog._get_leapfrog_flux_x}}\pysiglinewithargsret{\sphinxbfcode{\_get\_leapfrog\_flux\_x}}{\emph{i}, \emph{j}, \emph{k}, \emph{u\_unstg}, \emph{phi}}{}
Get the \sphinxcode{gridtools.Equation} representing the leapfrog flux in \(x\)-direction for a generic
prognostic variable \(\phi\).
\begin{quote}\begin{description}
\item[{Parameters}] \leavevmode\begin{itemize}
\item {} 
\sphinxstyleliteralstrong{i} (\sphinxstyleliteralemphasis{obj}) \textendash{} \sphinxcode{gridtools.Index} representing the index running along the \(x\)-axis.

\item {} 
\sphinxstyleliteralstrong{j} (\sphinxstyleliteralemphasis{obj}) \textendash{} \sphinxcode{gridtools.Index} representing the index running along the \(y\)-axis.

\item {} 
\sphinxstyleliteralstrong{k} (\sphinxstyleliteralemphasis{obj}) \textendash{} \sphinxcode{gridtools.Index} representing the index running along the \(\theta\)-axis.

\item {} 
\sphinxstyleliteralstrong{u\_unstg} (\sphinxstyleliteralemphasis{obj}) \textendash{} \sphinxcode{gridtools.Equation} representing the unstaggered \(x\)-velocity.

\item {} 
\sphinxstyleliteralstrong{phi} (\sphinxstyleliteralemphasis{obj}) \textendash{} \sphinxcode{gridtools.Equation} representing the field \(\phi\).

\end{itemize}

\item[{Returns}] \leavevmode
\sphinxcode{gridtools.Equation} representing the leapfrog flux in \(x\)-direction for \(\phi\).

\item[{Return type}] \leavevmode
obj

\end{description}\end{quote}

\end{fulllineitems}

\index{\_get\_leapfrog\_flux\_x\_density() (dycore.flux\_isentropic.FluxIsentropicLeapfrog method)}

\begin{fulllineitems}
\phantomsection\label{\detokenize{api:dycore.flux_isentropic.FluxIsentropicLeapfrog._get_leapfrog_flux_x_density}}\pysiglinewithargsret{\sphinxbfcode{\_get\_leapfrog\_flux\_x\_density}}{\emph{i}, \emph{j}, \emph{k}, \emph{U}}{}
Get the \sphinxcode{gridtools.Equation} representing the leapfrog flux in \(x\)-direction for the isentropic density.
\begin{quote}\begin{description}
\item[{Parameters}] \leavevmode\begin{itemize}
\item {} 
\sphinxstyleliteralstrong{i} (\sphinxstyleliteralemphasis{obj}) \textendash{} \sphinxcode{gridtools.Index} representing the index running along the \(x\)-axis.

\item {} 
\sphinxstyleliteralstrong{j} (\sphinxstyleliteralemphasis{obj}) \textendash{} \sphinxcode{gridtools.Index} representing the index running along the \(y\)-axis.

\item {} 
\sphinxstyleliteralstrong{k} (\sphinxstyleliteralemphasis{obj}) \textendash{} \sphinxcode{gridtools.Index} representing the index running along the \(\theta\)-axis.

\item {} 
\sphinxstyleliteralstrong{U} (\sphinxstyleliteralemphasis{obj}) \textendash{} \sphinxcode{gridtools.Equation} representing the \(x\)-momentum.

\end{itemize}

\item[{Returns}] \leavevmode
\sphinxcode{gridtools.Equation} representing the leapfrog flux in \(x\)-direction for the isentropic density.

\item[{Return type}] \leavevmode
obj

\end{description}\end{quote}

\end{fulllineitems}

\index{\_get\_leapfrog\_flux\_y() (dycore.flux\_isentropic.FluxIsentropicLeapfrog method)}

\begin{fulllineitems}
\phantomsection\label{\detokenize{api:dycore.flux_isentropic.FluxIsentropicLeapfrog._get_leapfrog_flux_y}}\pysiglinewithargsret{\sphinxbfcode{\_get\_leapfrog\_flux\_y}}{\emph{i}, \emph{j}, \emph{k}, \emph{v\_unstg}, \emph{phi}}{}
Get the \sphinxcode{gridtools.Equation} representing the leapfrog flux in \(y\)-direction for a generic
prognostic variable \(\phi\).
\begin{quote}\begin{description}
\item[{Parameters}] \leavevmode\begin{itemize}
\item {} 
\sphinxstyleliteralstrong{i} (\sphinxstyleliteralemphasis{obj}) \textendash{} \sphinxcode{gridtools.Index} representing the index running along the \(x\)-axis.

\item {} 
\sphinxstyleliteralstrong{j} (\sphinxstyleliteralemphasis{obj}) \textendash{} \sphinxcode{gridtools.Index} representing the index running along the \(y\)-axis.

\item {} 
\sphinxstyleliteralstrong{k} (\sphinxstyleliteralemphasis{obj}) \textendash{} \sphinxcode{gridtools.Index} representing the index running along the \(\theta\)-axis.

\item {} 
\sphinxstyleliteralstrong{v\_unstg} (\sphinxstyleliteralemphasis{obj}) \textendash{} \sphinxcode{gridtools.Equation} representing the unstaggered \(y\)-velocity.

\item {} 
\sphinxstyleliteralstrong{phi} (\sphinxstyleliteralemphasis{obj}) \textendash{} \sphinxcode{gridtools.Equation} representing the field \(\phi\).

\end{itemize}

\item[{Returns}] \leavevmode
\sphinxcode{gridtools.Equation} representing the leapfrog flux in \(y\)-direction for \(\phi\).

\item[{Return type}] \leavevmode
obj

\end{description}\end{quote}

\end{fulllineitems}

\index{\_get\_leapfrog\_flux\_y\_density() (dycore.flux\_isentropic.FluxIsentropicLeapfrog method)}

\begin{fulllineitems}
\phantomsection\label{\detokenize{api:dycore.flux_isentropic.FluxIsentropicLeapfrog._get_leapfrog_flux_y_density}}\pysiglinewithargsret{\sphinxbfcode{\_get\_leapfrog\_flux\_y\_density}}{\emph{i}, \emph{j}, \emph{k}, \emph{V}}{}
Get the \sphinxcode{gridtools.Equation} representing the leapfrog flux in \(y\)-direction for the isentropic density.
\begin{quote}\begin{description}
\item[{Parameters}] \leavevmode\begin{itemize}
\item {} 
\sphinxstyleliteralstrong{i} (\sphinxstyleliteralemphasis{obj}) \textendash{} \sphinxcode{gridtools.Index} representing the index running along the \(x\)-axis.

\item {} 
\sphinxstyleliteralstrong{j} (\sphinxstyleliteralemphasis{obj}) \textendash{} \sphinxcode{gridtools.Index} representing the index running along the \(y\)-axis.

\item {} 
\sphinxstyleliteralstrong{k} (\sphinxstyleliteralemphasis{obj}) \textendash{} \sphinxcode{gridtools.Index} representing the index running along the \(\theta\)-axis.

\item {} 
\sphinxstyleliteralstrong{V} (\sphinxstyleliteralemphasis{obj}) \textendash{} \sphinxcode{gridtools.Equation} representing the \(y\)-momentum.

\end{itemize}

\item[{Returns}] \leavevmode
\sphinxcode{gridtools.Equation} representing the leapfrog flux in \(y\)-direction for the isentropic density.

\item[{Return type}] \leavevmode
obj

\end{description}\end{quote}

\end{fulllineitems}


\end{fulllineitems}

\index{FluxIsentropicMacCormack (class in dycore.flux\_isentropic)}

\begin{fulllineitems}
\phantomsection\label{\detokenize{api:dycore.flux_isentropic.FluxIsentropicMacCormack}}\pysiglinewithargsret{\sphinxbfcode{class }\sphinxcode{dycore.flux\_isentropic.}\sphinxbfcode{FluxIsentropicMacCormack}}{\emph{grid}, \emph{imoist}}{}
Class which inherits {\hyperref[\detokenize{api:dycore.flux_isentropic.FluxIsentropic}]{\sphinxcrossref{\sphinxcode{FluxIsentropic}}}} to implement the MacCormack scheme applied to the
governing equations in conservative form.
\begin{quote}\begin{description}
\item[{Variables}] \leavevmode
{\hyperref[\detokenize{api:dycore.prognostic_isentropic.PrognosticIsentropic.nb}]{\sphinxcrossref{\sphinxstyleliteralstrong{nb}}}} (\sphinxstyleliteralemphasis{int}) \textendash{} Number of boundary layers.

\end{description}\end{quote}
\index{\_\_init\_\_() (dycore.flux\_isentropic.FluxIsentropicMacCormack method)}

\begin{fulllineitems}
\phantomsection\label{\detokenize{api:dycore.flux_isentropic.FluxIsentropicMacCormack.__init__}}\pysiglinewithargsret{\sphinxbfcode{\_\_init\_\_}}{\emph{grid}, \emph{imoist}}{}
Constructor.
\begin{quote}\begin{description}
\item[{Parameters}] \leavevmode\begin{itemize}
\item {} 
\sphinxstyleliteralstrong{grid} (\sphinxstyleliteralemphasis{obj}) \textendash{} {\hyperref[\detokenize{api:grids.grid_xyz.GridXYZ}]{\sphinxcrossref{\sphinxcode{GridXYZ}}}} representing the underlying grid.

\item {} 
\sphinxstyleliteralstrong{imoist} (\sphinxstyleliteralemphasis{bool}) \textendash{} \sphinxcode{True} for a moist dynamical core, \sphinxcode{False} otherwise.

\end{itemize}

\end{description}\end{quote}

\end{fulllineitems}

\index{\_compute\_fluxes() (dycore.flux\_isentropic.FluxIsentropicMacCormack method)}

\begin{fulllineitems}
\phantomsection\label{\detokenize{api:dycore.flux_isentropic.FluxIsentropicMacCormack._compute_fluxes}}\pysiglinewithargsret{\sphinxbfcode{\_compute\_fluxes}}{\emph{i}, \emph{j}, \emph{k}, \emph{dt}, \emph{s}, \emph{u}, \emph{v}, \emph{mtg}, \emph{U}, \emph{V}, \emph{Qv}, \emph{Qc}, \emph{Qr}}{}
Method computing the MacCormack \sphinxcode{gridtools.Equation}s representing the \(x\)- and \(y\)-fluxes for all
the conservative prognostic variables. The :class:{\color{red}\bfseries{}{}`}gridtools.Equation{}`s are then set as instance attributes.
\begin{quote}\begin{description}
\item[{Parameters}] \leavevmode\begin{itemize}
\item {} 
\sphinxstyleliteralstrong{i} (\sphinxstyleliteralemphasis{obj}) \textendash{} \sphinxcode{gridtools.Index} representing the index running along the \(x\)-axis.

\item {} 
\sphinxstyleliteralstrong{j} (\sphinxstyleliteralemphasis{obj}) \textendash{} \sphinxcode{gridtools.Index} representing the index running along the \(y\)-axis.

\item {} 
\sphinxstyleliteralstrong{k} (\sphinxstyleliteralemphasis{obj}) \textendash{} \sphinxcode{gridtools.Index} representing the index running along the \(\theta\)-axis.

\item {} 
\sphinxstyleliteralstrong{dt} (\sphinxstyleliteralemphasis{obj}) \textendash{} \sphinxcode{gridtools.Global} representing the time step.

\item {} 
\sphinxstyleliteralstrong{s} (\sphinxstyleliteralemphasis{obj}) \textendash{} \sphinxcode{gridtools.Equation} representing the isentropic density.

\item {} 
\sphinxstyleliteralstrong{u} (\sphinxstyleliteralemphasis{obj}) \textendash{} \sphinxcode{gridtools.Equation} representing the \(x\)-velocity.

\item {} 
\sphinxstyleliteralstrong{v} (\sphinxstyleliteralemphasis{obj}) \textendash{} \sphinxcode{gridtools.Equation} representing the \(y\)-velocity.

\item {} 
\sphinxstyleliteralstrong{mtg} (\sphinxstyleliteralemphasis{obj}) \textendash{} \sphinxcode{gridtools.Equation} representing the Montgomery potential.

\item {} 
\sphinxstyleliteralstrong{U} (\sphinxstyleliteralemphasis{obj}) \textendash{} \sphinxcode{gridtools.Equation} representing the \(x\)-momentum.

\item {} 
\sphinxstyleliteralstrong{V} (\sphinxstyleliteralemphasis{obj}) \textendash{} \sphinxcode{gridtools.Equation} representing the \(y\)-momentum.

\item {} 
\sphinxstyleliteralstrong{Qv} (\sphinxstyleliteralemphasis{obj}) \textendash{} \sphinxcode{gridtools.Equation} representing the mass of water vapour.

\item {} 
\sphinxstyleliteralstrong{Qc} (\sphinxstyleliteralemphasis{obj}) \textendash{} \sphinxcode{gridtools.Equation} representing the mass of cloud water.

\item {} 
\sphinxstyleliteralstrong{Qr} (\sphinxstyleliteralemphasis{obj}) \textendash{} \sphinxcode{gridtools.Equation} representing the mass of precipitation water.

\end{itemize}

\end{description}\end{quote}

\end{fulllineitems}

\index{\_get\_maccormack\_flux\_x() (dycore.flux\_isentropic.FluxIsentropicMacCormack method)}

\begin{fulllineitems}
\phantomsection\label{\detokenize{api:dycore.flux_isentropic.FluxIsentropicMacCormack._get_maccormack_flux_x}}\pysiglinewithargsret{\sphinxbfcode{\_get\_maccormack\_flux\_x}}{\emph{i}, \emph{j}, \emph{k}, \emph{u\_unstg}, \emph{phi}, \emph{u\_unstg\_p}, \emph{phi\_p}}{}
Get the \sphinxcode{gridtools.Equation} representing the MacCormack flux in \(x\)-direction for a
generic prognostic variable \(\phi\).
\begin{quote}\begin{description}
\item[{Parameters}] \leavevmode\begin{itemize}
\item {} 
\sphinxstyleliteralstrong{i} (\sphinxstyleliteralemphasis{obj}) \textendash{} \sphinxcode{gridtools.Index} representing the index running along the \(x\)-axis.

\item {} 
\sphinxstyleliteralstrong{j} (\sphinxstyleliteralemphasis{obj}) \textendash{} \sphinxcode{gridtools.Index} representing the index running along the \(y\)-axis.

\item {} 
\sphinxstyleliteralstrong{k} (\sphinxstyleliteralemphasis{obj}) \textendash{} \sphinxcode{gridtools.Index} representing the index running along the \(\theta\)-axis.

\item {} 
\sphinxstyleliteralstrong{u\_unstg} (\sphinxstyleliteralemphasis{obj}) \textendash{} \sphinxcode{gridtools.Equation} representing the unstaggered \(x\)-velocity at the current time.

\item {} 
\sphinxstyleliteralstrong{phi} (\sphinxstyleliteralemphasis{obj}) \textendash{} \sphinxcode{gridtools.Equation} representing the field \(\phi\) at the current time.

\item {} 
\sphinxstyleliteralstrong{u\_unstg\_p} (\sphinxstyleliteralemphasis{obj}) \textendash{} \sphinxcode{gridtools.Equation} representing the predicted value for the unstaggered \(x\)-velocity.

\item {} 
\sphinxstyleliteralstrong{phi\_p} (\sphinxstyleliteralemphasis{obj}) \textendash{} \sphinxcode{gridtools.Equation} representing the predicted value for the field \(\phi\).

\end{itemize}

\item[{Returns}] \leavevmode
\sphinxcode{gridtools.Equation} representing the MacCormack flux in \(x\)-direction for \(\phi\).

\item[{Return type}] \leavevmode
obj

\end{description}\end{quote}

\end{fulllineitems}

\index{\_get\_maccormack\_flux\_x\_density() (dycore.flux\_isentropic.FluxIsentropicMacCormack method)}

\begin{fulllineitems}
\phantomsection\label{\detokenize{api:dycore.flux_isentropic.FluxIsentropicMacCormack._get_maccormack_flux_x_density}}\pysiglinewithargsret{\sphinxbfcode{\_get\_maccormack\_flux\_x\_density}}{\emph{i}, \emph{j}, \emph{k}, \emph{U}, \emph{U\_p}}{}
Get the \sphinxcode{gridtools.Equation} representing the MacCormack flux in \(x\)-direction for the
isentropic density.
\begin{quote}\begin{description}
\item[{Parameters}] \leavevmode\begin{itemize}
\item {} 
\sphinxstyleliteralstrong{i} (\sphinxstyleliteralemphasis{obj}) \textendash{} \sphinxcode{gridtools.Index} representing the index running along the \(x\)-axis.

\item {} 
\sphinxstyleliteralstrong{j} (\sphinxstyleliteralemphasis{obj}) \textendash{} \sphinxcode{gridtools.Index} representing the index running along the \(y\)-axis.

\item {} 
\sphinxstyleliteralstrong{k} (\sphinxstyleliteralemphasis{obj}) \textendash{} \sphinxcode{gridtools.Index} representing the index running along the \(\theta\)-axis.

\item {} 
\sphinxstyleliteralstrong{U} (\sphinxstyleliteralemphasis{obj}) \textendash{} \sphinxcode{gridtools.Equation} representing the \(x\)-momentum at the current time.

\item {} 
\sphinxstyleliteralstrong{U\_p} (\sphinxstyleliteralemphasis{obj}) \textendash{} \sphinxcode{gridtools.Equation} representing the predicted value for the \(x\)-momentum.

\end{itemize}

\item[{Returns}] \leavevmode
\sphinxcode{gridtools.Equation} representing the MacCormack flux in \(x\)-direction for the isentropic density.

\item[{Return type}] \leavevmode
obj

\end{description}\end{quote}

\end{fulllineitems}

\index{\_get\_maccormack\_flux\_y() (dycore.flux\_isentropic.FluxIsentropicMacCormack method)}

\begin{fulllineitems}
\phantomsection\label{\detokenize{api:dycore.flux_isentropic.FluxIsentropicMacCormack._get_maccormack_flux_y}}\pysiglinewithargsret{\sphinxbfcode{\_get\_maccormack\_flux\_y}}{\emph{i}, \emph{j}, \emph{k}, \emph{v\_unstg}, \emph{phi}, \emph{v\_unstg\_p}, \emph{phi\_p}}{}
Get the \sphinxcode{gridtools.Equation} representing the MacCormack flux in \(y\)-direction for a
generic prognostic variable \(\phi\).
\begin{quote}\begin{description}
\item[{Parameters}] \leavevmode\begin{itemize}
\item {} 
\sphinxstyleliteralstrong{i} (\sphinxstyleliteralemphasis{obj}) \textendash{} \sphinxcode{gridtools.Index} representing the index running along the \(x\)-axis.

\item {} 
\sphinxstyleliteralstrong{j} (\sphinxstyleliteralemphasis{obj}) \textendash{} \sphinxcode{gridtools.Index} representing the index running along the \(y\)-axis.

\item {} 
\sphinxstyleliteralstrong{k} (\sphinxstyleliteralemphasis{obj}) \textendash{} \sphinxcode{gridtools.Index} representing the index running along the \(\theta\)-axis.

\item {} 
\sphinxstyleliteralstrong{v\_unstg} (\sphinxstyleliteralemphasis{obj}) \textendash{} \sphinxcode{gridtools.Equation} representing the unstaggered \(y\)-velocity at the current time.

\item {} 
\sphinxstyleliteralstrong{phi} (\sphinxstyleliteralemphasis{obj}) \textendash{} \sphinxcode{gridtools.Equation} representing the field \(\phi\) at the current time.

\item {} 
\sphinxstyleliteralstrong{v\_unstg\_p} (\sphinxstyleliteralemphasis{obj}) \textendash{} \sphinxcode{gridtools.Equation} representing the predicted value for the unstaggered \(y\)-velocity.

\item {} 
\sphinxstyleliteralstrong{phi\_p} (\sphinxstyleliteralemphasis{obj}) \textendash{} \sphinxcode{gridtools.Equation} representing the predicted value for the field \(\phi\).

\end{itemize}

\item[{Returns}] \leavevmode
\sphinxcode{gridtools.Equation} representing the MacCormack flux in \(y\)-direction for \(\phi\).

\item[{Return type}] \leavevmode
obj

\end{description}\end{quote}

\end{fulllineitems}

\index{\_get\_maccormack\_flux\_y\_density() (dycore.flux\_isentropic.FluxIsentropicMacCormack method)}

\begin{fulllineitems}
\phantomsection\label{\detokenize{api:dycore.flux_isentropic.FluxIsentropicMacCormack._get_maccormack_flux_y_density}}\pysiglinewithargsret{\sphinxbfcode{\_get\_maccormack\_flux\_y\_density}}{\emph{i}, \emph{j}, \emph{k}, \emph{V}, \emph{V\_p}}{}
Get the \sphinxcode{gridtools.Equation} representing the MacCormack flux in \(y\)-direction for the
isentropic density.
\begin{quote}\begin{description}
\item[{Parameters}] \leavevmode\begin{itemize}
\item {} 
\sphinxstyleliteralstrong{i} (\sphinxstyleliteralemphasis{obj}) \textendash{} \sphinxcode{gridtools.Index} representing the index running along the \(x\)-axis.

\item {} 
\sphinxstyleliteralstrong{j} (\sphinxstyleliteralemphasis{obj}) \textendash{} \sphinxcode{gridtools.Index} representing the index running along the \(y\)-axis.

\item {} 
\sphinxstyleliteralstrong{k} (\sphinxstyleliteralemphasis{obj}) \textendash{} \sphinxcode{gridtools.Index} representing the index running along the \(\theta\)-axis.

\item {} 
\sphinxstyleliteralstrong{V} (\sphinxstyleliteralemphasis{obj}) \textendash{} \sphinxcode{gridtools.Equation} representing the \(y\)-momentum at the current time.

\item {} 
\sphinxstyleliteralstrong{V\_p} (\sphinxstyleliteralemphasis{obj}) \textendash{} \sphinxcode{gridtools.Equation} representing the predicted value for the \(y\)-momentum.

\end{itemize}

\item[{Returns}] \leavevmode
\sphinxcode{gridtools.Equation} representing the MacCormack flux in \(y\)-direction for the isentropic density.

\item[{Return type}] \leavevmode
obj

\end{description}\end{quote}

\end{fulllineitems}

\index{\_get\_maccormack\_predicted\_value\_constituent() (dycore.flux\_isentropic.FluxIsentropicMacCormack method)}

\begin{fulllineitems}
\phantomsection\label{\detokenize{api:dycore.flux_isentropic.FluxIsentropicMacCormack._get_maccormack_predicted_value_constituent}}\pysiglinewithargsret{\sphinxbfcode{\_get\_maccormack\_predicted\_value\_constituent}}{\emph{i}, \emph{j}, \emph{k}, \emph{dt}, \emph{u\_unstg}, \emph{v\_unstg}, \emph{Q}}{}
Get the \sphinxcode{gridtools.Equation} representing the predicted value for mass of a generic water constituent \(Q\).
\begin{quote}\begin{description}
\item[{Parameters}] \leavevmode\begin{itemize}
\item {} 
\sphinxstyleliteralstrong{i} (\sphinxstyleliteralemphasis{obj}) \textendash{} \sphinxcode{gridtools.Index} representing the index running along the \(x\)-axis.

\item {} 
\sphinxstyleliteralstrong{j} (\sphinxstyleliteralemphasis{obj}) \textendash{} \sphinxcode{gridtools.Index} representing the index running along the \(y\)-axis.

\item {} 
\sphinxstyleliteralstrong{k} (\sphinxstyleliteralemphasis{obj}) \textendash{} \sphinxcode{gridtools.Index} representing the index running along the \(\theta\)-axis.

\item {} 
\sphinxstyleliteralstrong{dt} (\sphinxstyleliteralemphasis{obj}) \textendash{} \sphinxcode{gridtools.Global} representing the time step.

\item {} 
\sphinxstyleliteralstrong{u\_unstg} (\sphinxstyleliteralemphasis{obj}) \textendash{} \sphinxcode{gridtools.Equation} representing the unstaggered \(x\)-velocity.

\item {} 
\sphinxstyleliteralstrong{v\_unstg} (\sphinxstyleliteralemphasis{obj}) \textendash{} \sphinxcode{gridtools.Equation} representing the unstaggered \(y\)-velocity.

\item {} 
\sphinxstyleliteralstrong{Q} (\sphinxstyleliteralemphasis{obj}) \textendash{} \sphinxcode{gridtools.Equation} representing the mass of a generic water constituent \(Q\).

\end{itemize}

\item[{Returns}] \leavevmode
\sphinxcode{gridtools.Equation} representing the predicted value for \(Q\).

\item[{Return type}] \leavevmode
obj

\end{description}\end{quote}

\end{fulllineitems}

\index{\_get\_maccormack\_predicted\_value\_density() (dycore.flux\_isentropic.FluxIsentropicMacCormack method)}

\begin{fulllineitems}
\phantomsection\label{\detokenize{api:dycore.flux_isentropic.FluxIsentropicMacCormack._get_maccormack_predicted_value_density}}\pysiglinewithargsret{\sphinxbfcode{\_get\_maccormack\_predicted\_value\_density}}{\emph{i}, \emph{j}, \emph{k}, \emph{dt}, \emph{s}, \emph{U}, \emph{V}}{}
Get the \sphinxcode{gridtools.Equation} representing the predicted value for the isentropic density.
\begin{quote}\begin{description}
\item[{Parameters}] \leavevmode\begin{itemize}
\item {} 
\sphinxstyleliteralstrong{i} (\sphinxstyleliteralemphasis{obj}) \textendash{} \sphinxcode{gridtools.Index} representing the index running along the \(x\)-axis.

\item {} 
\sphinxstyleliteralstrong{j} (\sphinxstyleliteralemphasis{obj}) \textendash{} \sphinxcode{gridtools.Index} representing the index running along the \(y\)-axis.

\item {} 
\sphinxstyleliteralstrong{k} (\sphinxstyleliteralemphasis{obj}) \textendash{} \sphinxcode{gridtools.Index} representing the index running along the \(\theta\)-axis.

\item {} 
\sphinxstyleliteralstrong{dt} (\sphinxstyleliteralemphasis{obj}) \textendash{} \sphinxcode{gridtools.Global} representing the time step.

\item {} 
\sphinxstyleliteralstrong{s} (\sphinxstyleliteralemphasis{obj}) \textendash{} \sphinxcode{gridtools.Equation} representing the isentropic density.

\item {} 
\sphinxstyleliteralstrong{U} (\sphinxstyleliteralemphasis{obj}) \textendash{} \sphinxcode{gridtools.Equation} representing the \(x\)-momentum.

\item {} 
\sphinxstyleliteralstrong{V} (\sphinxstyleliteralemphasis{obj}) \textendash{} \sphinxcode{gridtools.Equation} representing the \(y\)-momentum.

\end{itemize}

\item[{Returns}] \leavevmode
\sphinxcode{gridtools.Equation} representing the predicted value for the isentropic density.

\item[{Return type}] \leavevmode
obj

\end{description}\end{quote}

\end{fulllineitems}

\index{\_get\_maccormack\_predicted\_value\_momentum\_x() (dycore.flux\_isentropic.FluxIsentropicMacCormack method)}

\begin{fulllineitems}
\phantomsection\label{\detokenize{api:dycore.flux_isentropic.FluxIsentropicMacCormack._get_maccormack_predicted_value_momentum_x}}\pysiglinewithargsret{\sphinxbfcode{\_get\_maccormack\_predicted\_value\_momentum\_x}}{\emph{i}, \emph{j}, \emph{k}, \emph{dt}, \emph{s}, \emph{u\_unstg}, \emph{v\_unstg}, \emph{mtg}, \emph{U}}{}
Get the \sphinxcode{gridtools.Equation} representing the predicted value for the \(x\)-momentum.
\begin{quote}\begin{description}
\item[{Parameters}] \leavevmode\begin{itemize}
\item {} 
\sphinxstyleliteralstrong{i} (\sphinxstyleliteralemphasis{obj}) \textendash{} \sphinxcode{gridtools.Index} representing the index running along the \(x\)-axis.

\item {} 
\sphinxstyleliteralstrong{j} (\sphinxstyleliteralemphasis{obj}) \textendash{} \sphinxcode{gridtools.Index} representing the index running along the \(y\)-axis.

\item {} 
\sphinxstyleliteralstrong{k} (\sphinxstyleliteralemphasis{obj}) \textendash{} \sphinxcode{gridtools.Index} representing the index running along the \(\theta\)-axis.

\item {} 
\sphinxstyleliteralstrong{dt} (\sphinxstyleliteralemphasis{obj}) \textendash{} \sphinxcode{gridtools.Global} representing the time step.

\item {} 
\sphinxstyleliteralstrong{s} (\sphinxstyleliteralemphasis{obj}) \textendash{} \sphinxcode{gridtools.Equation} representing the isentropic density.

\item {} 
\sphinxstyleliteralstrong{u\_unstg} (\sphinxstyleliteralemphasis{obj}) \textendash{} \sphinxcode{gridtools.Equation} representing the unstaggered \(x\)-velocity.

\item {} 
\sphinxstyleliteralstrong{v\_unstg} (\sphinxstyleliteralemphasis{obj}) \textendash{} \sphinxcode{gridtools.Equation} representing the unstaggered \(y\)-velocity.

\item {} 
\sphinxstyleliteralstrong{mtg} (\sphinxstyleliteralemphasis{obj}) \textendash{} \sphinxcode{gridtools.Equation} representing the Montgomery potential.

\item {} 
\sphinxstyleliteralstrong{U} (\sphinxstyleliteralemphasis{obj}) \textendash{} \sphinxcode{gridtools.Equation} representing the \(x\)-momentum.

\end{itemize}

\item[{Returns}] \leavevmode
\sphinxcode{gridtools.Equation} representing the predicted value for the \(x\)-momentum.

\item[{Return type}] \leavevmode
obj

\end{description}\end{quote}

\end{fulllineitems}

\index{\_get\_maccormack\_predicted\_value\_momentum\_y() (dycore.flux\_isentropic.FluxIsentropicMacCormack method)}

\begin{fulllineitems}
\phantomsection\label{\detokenize{api:dycore.flux_isentropic.FluxIsentropicMacCormack._get_maccormack_predicted_value_momentum_y}}\pysiglinewithargsret{\sphinxbfcode{\_get\_maccormack\_predicted\_value\_momentum\_y}}{\emph{i}, \emph{j}, \emph{k}, \emph{dt}, \emph{s}, \emph{u\_unstg}, \emph{v\_unstg}, \emph{mtg}, \emph{V}}{}
Get the \sphinxcode{gridtools.Equation} representing the predicted value for the \(y\)-momentum.
\begin{quote}\begin{description}
\item[{Parameters}] \leavevmode\begin{itemize}
\item {} 
\sphinxstyleliteralstrong{i} (\sphinxstyleliteralemphasis{obj}) \textendash{} \sphinxcode{gridtools.Index} representing the index running along the \(x\)-axis.

\item {} 
\sphinxstyleliteralstrong{j} (\sphinxstyleliteralemphasis{obj}) \textendash{} \sphinxcode{gridtools.Index} representing the index running along the \(y\)-axis.

\item {} 
\sphinxstyleliteralstrong{k} (\sphinxstyleliteralemphasis{obj}) \textendash{} \sphinxcode{gridtools.Index} representing the index running along the \(\theta\)-axis.

\item {} 
\sphinxstyleliteralstrong{dt} (\sphinxstyleliteralemphasis{obj}) \textendash{} \sphinxcode{gridtools.Global} representing the time step.

\item {} 
\sphinxstyleliteralstrong{s} (\sphinxstyleliteralemphasis{obj}) \textendash{} \sphinxcode{gridtools.Equation} representing the isentropic density.

\item {} 
\sphinxstyleliteralstrong{u\_unstg} (\sphinxstyleliteralemphasis{obj}) \textendash{} \sphinxcode{gridtools.Equation} representing the unstaggered \(x\)-velocity.

\item {} 
\sphinxstyleliteralstrong{v\_unstg} (\sphinxstyleliteralemphasis{obj}) \textendash{} \sphinxcode{gridtools.Equation} representing the unstaggered \(y\)-velocity.

\item {} 
\sphinxstyleliteralstrong{mtg} (\sphinxstyleliteralemphasis{obj}) \textendash{} \sphinxcode{gridtools.Equation} representing the Montgomery potential.

\item {} 
\sphinxstyleliteralstrong{V} (\sphinxstyleliteralemphasis{obj}) \textendash{} \sphinxcode{gridtools.Equation} representing the \(y\)-momentum.

\end{itemize}

\item[{Returns}] \leavevmode
\sphinxcode{gridtools.Equation} representing the predicted value for the \(y\)-momentum.

\item[{Return type}] \leavevmode
obj

\end{description}\end{quote}

\end{fulllineitems}

\index{\_get\_velocity() (dycore.flux\_isentropic.FluxIsentropicMacCormack method)}

\begin{fulllineitems}
\phantomsection\label{\detokenize{api:dycore.flux_isentropic.FluxIsentropicMacCormack._get_velocity}}\pysiglinewithargsret{\sphinxbfcode{\_get\_velocity}}{\emph{i}, \emph{j}, \emph{k}, \emph{s}, \emph{mnt}}{}
Get the \sphinxcode{gridtools.Equation} representing an unstaggered water component.
\begin{quote}\begin{description}
\item[{Parameters}] \leavevmode\begin{itemize}
\item {} 
\sphinxstyleliteralstrong{i} (\sphinxstyleliteralemphasis{obj}) \textendash{} \sphinxcode{gridtools.Index} representing the index running along the \(x\)-axis.

\item {} 
\sphinxstyleliteralstrong{j} (\sphinxstyleliteralemphasis{obj}) \textendash{} \sphinxcode{gridtools.Index} representing the index running along the \(y\)-axis.

\item {} 
\sphinxstyleliteralstrong{k} (\sphinxstyleliteralemphasis{obj}) \textendash{} \sphinxcode{gridtools.Index} representing the index running along the \(\theta\)-axis.

\item {} 
\sphinxstyleliteralstrong{s} (\sphinxstyleliteralemphasis{obj}) \textendash{} \sphinxcode{gridtools.Equation} representing the isentropic density.

\item {} 
\sphinxstyleliteralstrong{mnt} (\sphinxstyleliteralemphasis{obj}) \textendash{} \sphinxcode{gridtools.Equation} representing either the \(x\)- or the \(y\)-momentum.

\end{itemize}

\item[{Returns}] \leavevmode
\sphinxcode{gridtools.Equation} representing the diagnosed unstaggered velocity component.

\item[{Return type}] \leavevmode
obj

\end{description}\end{quote}

\end{fulllineitems}


\end{fulllineitems}



\subsection{Prognostics}
\label{\detokenize{api:prognostics}}\index{PrognosticIsentropic (class in dycore.prognostic\_isentropic)}

\begin{fulllineitems}
\phantomsection\label{\detokenize{api:dycore.prognostic_isentropic.PrognosticIsentropic}}\pysiglinewithargsret{\sphinxbfcode{class }\sphinxcode{dycore.prognostic\_isentropic.}\sphinxbfcode{PrognosticIsentropic}}{\emph{grid}, \emph{imoist}, \emph{scheme}, \emph{backend}}{}
Abstract base class whose derived classes implement different schemes to carry out the prognostic step of
the three-dimensional moist isentropic dynamical core. The conservative form of the governing equations is used.
\index{\_\_init\_\_() (dycore.prognostic\_isentropic.PrognosticIsentropic method)}

\begin{fulllineitems}
\phantomsection\label{\detokenize{api:dycore.prognostic_isentropic.PrognosticIsentropic.__init__}}\pysiglinewithargsret{\sphinxbfcode{\_\_init\_\_}}{\emph{grid}, \emph{imoist}, \emph{scheme}, \emph{backend}}{}
Constructor.
\begin{quote}\begin{description}
\item[{Parameters}] \leavevmode\begin{itemize}
\item {} 
\sphinxstyleliteralstrong{grid} (\sphinxstyleliteralemphasis{obj}) \textendash{} {\hyperref[\detokenize{api:grids.grid_xyz.GridXYZ}]{\sphinxcrossref{\sphinxcode{GridXYZ}}}} representing the underlying grid.

\item {} 
\sphinxstyleliteralstrong{imoist} (\sphinxstyleliteralemphasis{bool}) \textendash{} \sphinxcode{True} for a moist dynamical core, \sphinxcode{False} otherwise.

\item {} 
\sphinxstyleliteralstrong{scheme} (\sphinxstyleliteralemphasis{str}) \textendash{} 
String specifying the scheme to use. Either:
\begin{itemize}
\item {} 
’upwind’, for the upwind scheme;

\item {} 
’leapfrog’, for the leapfrog scheme;

\item {} 
’maccormack’, for the MacCormack scheme.

\end{itemize}


\item {} 
\sphinxstyleliteralstrong{backend} (\sphinxstyleliteralemphasis{obj}) \textendash{} \sphinxcode{gridtools.mode} specifying the backend for the GT4Py’s stencils.

\end{itemize}

\end{description}\end{quote}

\end{fulllineitems}

\index{\_allocate\_inputs() (dycore.prognostic\_isentropic.PrognosticIsentropic method)}

\begin{fulllineitems}
\phantomsection\label{\detokenize{api:dycore.prognostic_isentropic.PrognosticIsentropic._allocate_inputs}}\pysiglinewithargsret{\sphinxbfcode{\_allocate\_inputs}}{\emph{s}, \emph{u}, \emph{v}}{}
Allocate (some of) the private instance attributes which will serve as inputs to the GT4Py’s stencils.
\begin{quote}\begin{description}
\item[{Parameters}] \leavevmode\begin{itemize}
\item {} 
\sphinxstyleliteralstrong{s} (\sphinxstyleliteralemphasis{array\_like}) \textendash{} \sphinxhref{https://docs.scipy.org/doc/numpy-1.13.0/reference/generated/numpy.ndarray.html\#numpy.ndarray}{\sphinxcode{numpy.ndarray}} representing the stencils’ computational domain for the isentropic density at current time.

\item {} 
\sphinxstyleliteralstrong{u} (\sphinxstyleliteralemphasis{array\_like}) \textendash{} \sphinxhref{https://docs.scipy.org/doc/numpy-1.13.0/reference/generated/numpy.ndarray.html\#numpy.ndarray}{\sphinxcode{numpy.ndarray}} representing the stencils’ computational domain for the \(x\)-velocity at current time.

\item {} 
\sphinxstyleliteralstrong{v} (\sphinxstyleliteralemphasis{array\_like}) \textendash{} \sphinxhref{https://docs.scipy.org/doc/numpy-1.13.0/reference/generated/numpy.ndarray.html\#numpy.ndarray}{\sphinxcode{numpy.ndarray}} representing the stencils’ computational domain for the \(y\)-velocity at current time.

\end{itemize}

\end{description}\end{quote}

\end{fulllineitems}

\index{\_allocate\_outputs() (dycore.prognostic\_isentropic.PrognosticIsentropic method)}

\begin{fulllineitems}
\phantomsection\label{\detokenize{api:dycore.prognostic_isentropic.PrognosticIsentropic._allocate_outputs}}\pysiglinewithargsret{\sphinxbfcode{\_allocate\_outputs}}{\emph{s}}{}
Allocate the Numpy arrays which will store the updated solution.
\begin{quote}\begin{description}
\item[{Parameters}] \leavevmode
\sphinxstyleliteralstrong{s} (\sphinxstyleliteralemphasis{array\_like}) \textendash{} \sphinxhref{https://docs.scipy.org/doc/numpy-1.13.0/reference/generated/numpy.ndarray.html\#numpy.ndarray}{\sphinxcode{numpy.ndarray}} representing the stencils’ computational domain for the isentropic density.

\end{description}\end{quote}

\end{fulllineitems}

\index{\_set\_inputs() (dycore.prognostic\_isentropic.PrognosticIsentropic method)}

\begin{fulllineitems}
\phantomsection\label{\detokenize{api:dycore.prognostic_isentropic.PrognosticIsentropic._set_inputs}}\pysiglinewithargsret{\sphinxbfcode{\_set\_inputs}}{\emph{dt}, \emph{s}, \emph{u}, \emph{v}, \emph{mtg}, \emph{U}, \emph{V}, \emph{Qv}, \emph{Qc}, \emph{Qr}}{}
Update (some of) the attributes which serve as inputs to the GT4Py’s stencils.
\begin{quote}\begin{description}
\item[{Parameters}] \leavevmode\begin{itemize}
\item {} 
\sphinxstyleliteralstrong{dt} (\sphinxstyleliteralemphasis{obj}) \textendash{} A \sphinxcode{datetime.timedelta} representing the time step.

\item {} 
\sphinxstyleliteralstrong{s} (\sphinxstyleliteralemphasis{array\_like}) \textendash{} \sphinxhref{https://docs.scipy.org/doc/numpy-1.13.0/reference/generated/numpy.ndarray.html\#numpy.ndarray}{\sphinxcode{numpy.ndarray}} representing the stencils’ computational domain for the isentropic density
at current time.

\item {} 
\sphinxstyleliteralstrong{u} (\sphinxstyleliteralemphasis{array\_like}) \textendash{} \sphinxhref{https://docs.scipy.org/doc/numpy-1.13.0/reference/generated/numpy.ndarray.html\#numpy.ndarray}{\sphinxcode{numpy.ndarray}} representing the stencils’ computational domain for the \(x\)-velocity
at current time.

\item {} 
\sphinxstyleliteralstrong{v} (\sphinxstyleliteralemphasis{array\_like}) \textendash{} \sphinxhref{https://docs.scipy.org/doc/numpy-1.13.0/reference/generated/numpy.ndarray.html\#numpy.ndarray}{\sphinxcode{numpy.ndarray}} representing the stencils’ computational domain for the \(y\)-velocity
at current time.

\item {} 
\sphinxstyleliteralstrong{p} (\sphinxstyleliteralemphasis{array\_like}) \textendash{} \sphinxhref{https://docs.scipy.org/doc/numpy-1.13.0/reference/generated/numpy.ndarray.html\#numpy.ndarray}{\sphinxcode{numpy.ndarray}} representing the stencils’ computational domain for the pressure at current time.

\item {} 
\sphinxstyleliteralstrong{mtg} (\sphinxstyleliteralemphasis{array\_like}) \textendash{} \sphinxhref{https://docs.scipy.org/doc/numpy-1.13.0/reference/generated/numpy.ndarray.html\#numpy.ndarray}{\sphinxcode{numpy.ndarray}} representing the stencils’ computational domain for the Montgomery potential
at current time.

\item {} 
\sphinxstyleliteralstrong{U} (\sphinxstyleliteralemphasis{array\_like}) \textendash{} \sphinxhref{https://docs.scipy.org/doc/numpy-1.13.0/reference/generated/numpy.ndarray.html\#numpy.ndarray}{\sphinxcode{numpy.ndarray}} representing the stencils’ computational domain for the \(x\)-momentum
at current time.

\item {} 
\sphinxstyleliteralstrong{V} (\sphinxstyleliteralemphasis{array\_like}) \textendash{} \sphinxhref{https://docs.scipy.org/doc/numpy-1.13.0/reference/generated/numpy.ndarray.html\#numpy.ndarray}{\sphinxcode{numpy.ndarray}} representing the stencils’ computational domain for the \(y\)-momentum
at current time.

\item {} 
\sphinxstyleliteralstrong{Qv} (\sphinxstyleliteralemphasis{array\_like}) \textendash{} \sphinxhref{https://docs.scipy.org/doc/numpy-1.13.0/reference/generated/numpy.ndarray.html\#numpy.ndarray}{\sphinxcode{numpy.ndarray}} representing the stencils’ computational domain for the mass of water vapour
at current time.

\item {} 
\sphinxstyleliteralstrong{Qc} (\sphinxstyleliteralemphasis{array\_like}) \textendash{} \sphinxhref{https://docs.scipy.org/doc/numpy-1.13.0/reference/generated/numpy.ndarray.html\#numpy.ndarray}{\sphinxcode{numpy.ndarray}} representing the stencils’ computational domain for the mass of cloud water
at current time.

\item {} 
\sphinxstyleliteralstrong{Qr} (\sphinxstyleliteralemphasis{array\_like}) \textendash{} \sphinxhref{https://docs.scipy.org/doc/numpy-1.13.0/reference/generated/numpy.ndarray.html\#numpy.ndarray}{\sphinxcode{numpy.ndarray}} representing the stencils’ computational domain for the mass of precipitation water
at current time.

\end{itemize}

\end{description}\end{quote}

\end{fulllineitems}

\index{factory() (dycore.prognostic\_isentropic.PrognosticIsentropic static method)}

\begin{fulllineitems}
\phantomsection\label{\detokenize{api:dycore.prognostic_isentropic.PrognosticIsentropic.factory}}\pysiglinewithargsret{\sphinxbfcode{static }\sphinxbfcode{factory}}{\emph{grid}, \emph{imoist}, \emph{scheme}, \emph{backend}}{}
Static method returning an instace of the derived class implementing the scheme specified by \sphinxcode{scheme}.
\begin{quote}\begin{description}
\item[{Parameters}] \leavevmode\begin{itemize}
\item {} 
\sphinxstyleliteralstrong{grid} (\sphinxstyleliteralemphasis{obj}) \textendash{} {\hyperref[\detokenize{api:grids.grid_xyz.GridXYZ}]{\sphinxcrossref{\sphinxcode{GridXYZ}}}} representing the underlying grid.

\item {} 
\sphinxstyleliteralstrong{imoist} (\sphinxstyleliteralemphasis{bool}) \textendash{} \sphinxcode{True} for a moist dynamical core, \sphinxcode{False} otherwise.

\item {} 
\sphinxstyleliteralstrong{scheme} (\sphinxstyleliteralemphasis{str}) \textendash{} 
String specifying the scheme to use. Either:
\begin{itemize}
\item {} 
’upwind’, for the upwind scheme;

\item {} 
’leapfrog’, for the leapfrog scheme;

\item {} 
’maccormack’, for the MacCormack scheme.

\end{itemize}


\item {} 
\sphinxstyleliteralstrong{backend} (\sphinxstyleliteralemphasis{obj}) \textendash{} \sphinxcode{gridtools.Mode} specifying the backend for the GT4Py’s stencils.

\end{itemize}

\item[{Returns}] \leavevmode
An instace of the derived class implementing the scheme specified by \sphinxcode{scheme}.

\item[{Return type}] \leavevmode
obj

\end{description}\end{quote}

\end{fulllineitems}

\index{nb (dycore.prognostic\_isentropic.PrognosticIsentropic attribute)}

\begin{fulllineitems}
\phantomsection\label{\detokenize{api:dycore.prognostic_isentropic.PrognosticIsentropic.nb}}\pysigline{\sphinxbfcode{nb}}
Get the number of boundary layers.
\begin{quote}\begin{description}
\item[{Returns}] \leavevmode
The number of boundary layers.

\item[{Return type}] \leavevmode
int

\end{description}\end{quote}

\end{fulllineitems}

\index{step\_forward() (dycore.prognostic\_isentropic.PrognosticIsentropic method)}

\begin{fulllineitems}
\phantomsection\label{\detokenize{api:dycore.prognostic_isentropic.PrognosticIsentropic.step_forward}}\pysiglinewithargsret{\sphinxbfcode{step\_forward}}{\emph{diagnostic}, \emph{boundary}, \emph{dt}, \emph{s}, \emph{u}, \emph{v}, \emph{p}, \emph{mtg}, \emph{U}, \emph{V}, \emph{Qv=None}, \emph{Qc=None}, \emph{Qr=None}, \emph{old\_s=None}, \emph{old\_U=None}, \emph{old\_V=None}, \emph{old\_Qv=None}, \emph{old\_Qc=None}, \emph{old\_Qr=None}}{}
Method advancing the conservative model variables one time step forward.
As this method is marked as abstract, its implementation is delegated to the derived classes.
\begin{quote}\begin{description}
\item[{Parameters}] \leavevmode\begin{itemize}
\item {} 
\sphinxstyleliteralstrong{diagnostic} (\sphinxstyleliteralemphasis{obj}) \textendash{} {\hyperref[\detokenize{api:dycore.diagnostic_isentropic.DiagnosticIsentropic}]{\sphinxcrossref{\sphinxcode{DiagnosticIsentropic}}}} performing the diagnostic steps
of the dynamical core.

\item {} 
\sphinxstyleliteralstrong{boundary} (\sphinxstyleliteralemphasis{obj}) \textendash{} An instance of one of the derived classes of {\hyperref[\detokenize{api:dycore.horizontal_boundary.HorizontalBoundary}]{\sphinxcrossref{\sphinxcode{HorizontalBoundary}}}},
implementing the lateral boundary conditions.

\item {} 
\sphinxstyleliteralstrong{dt} (\sphinxstyleliteralemphasis{obj}) \textendash{} A \sphinxcode{datetime.timedelta} representing the time step.

\item {} 
\sphinxstyleliteralstrong{s} (\sphinxstyleliteralemphasis{array\_like}) \textendash{} \sphinxhref{https://docs.scipy.org/doc/numpy-1.13.0/reference/generated/numpy.ndarray.html\#numpy.ndarray}{\sphinxcode{numpy.ndarray}} representing the stencils’ computational domain for the isentropic density
at current time.

\item {} 
\sphinxstyleliteralstrong{u} (\sphinxstyleliteralemphasis{array\_like}) \textendash{} \sphinxhref{https://docs.scipy.org/doc/numpy-1.13.0/reference/generated/numpy.ndarray.html\#numpy.ndarray}{\sphinxcode{numpy.ndarray}} representing the stencils’ computational domain for the \(x\)-velocity
at current time.

\item {} 
\sphinxstyleliteralstrong{v} (\sphinxstyleliteralemphasis{array\_like}) \textendash{} \sphinxhref{https://docs.scipy.org/doc/numpy-1.13.0/reference/generated/numpy.ndarray.html\#numpy.ndarray}{\sphinxcode{numpy.ndarray}} representing the stencils’ computational domain for the \(y\)-velocity
at current time.

\item {} 
\sphinxstyleliteralstrong{p} (\sphinxstyleliteralemphasis{array\_like}) \textendash{} \sphinxhref{https://docs.scipy.org/doc/numpy-1.13.0/reference/generated/numpy.ndarray.html\#numpy.ndarray}{\sphinxcode{numpy.ndarray}} representing the stencils’ computational domain for the pressure at current time.

\item {} 
\sphinxstyleliteralstrong{mtg} (\sphinxstyleliteralemphasis{array\_like}) \textendash{} \sphinxhref{https://docs.scipy.org/doc/numpy-1.13.0/reference/generated/numpy.ndarray.html\#numpy.ndarray}{\sphinxcode{numpy.ndarray}} representing the stencils’ computational domain for the Montgomery potential
at current time.

\item {} 
\sphinxstyleliteralstrong{U} (\sphinxstyleliteralemphasis{array\_like}) \textendash{} \sphinxhref{https://docs.scipy.org/doc/numpy-1.13.0/reference/generated/numpy.ndarray.html\#numpy.ndarray}{\sphinxcode{numpy.ndarray}} representing the stencils’ computational domain for the \(x\)-momentum
at current time.

\item {} 
\sphinxstyleliteralstrong{V} (\sphinxstyleliteralemphasis{array\_like}) \textendash{} \sphinxhref{https://docs.scipy.org/doc/numpy-1.13.0/reference/generated/numpy.ndarray.html\#numpy.ndarray}{\sphinxcode{numpy.ndarray}} representing the stencils’ computational domain for the \(y\)-momentum
at current time.

\item {} 
\sphinxstyleliteralstrong{Qv} (\sphinxtitleref{array\_like}, optional) \textendash{} \sphinxhref{https://docs.scipy.org/doc/numpy-1.13.0/reference/generated/numpy.ndarray.html\#numpy.ndarray}{\sphinxcode{numpy.ndarray}} representing the stencils’ computational domain for the mass of water vapour
at current time.

\item {} 
\sphinxstyleliteralstrong{Qc} (\sphinxtitleref{array\_like}, optional) \textendash{} \sphinxhref{https://docs.scipy.org/doc/numpy-1.13.0/reference/generated/numpy.ndarray.html\#numpy.ndarray}{\sphinxcode{numpy.ndarray}} representing the stencils’ computational domain for the mass of cloud water
at current time.

\item {} 
\sphinxstyleliteralstrong{Qr} (\sphinxtitleref{array\_like}, optional) \textendash{} \sphinxhref{https://docs.scipy.org/doc/numpy-1.13.0/reference/generated/numpy.ndarray.html\#numpy.ndarray}{\sphinxcode{numpy.ndarray}} representing the stencils’ computational domain for the mass of precipitation water
at current time.

\item {} 
\sphinxstyleliteralstrong{old\_s} (\sphinxtitleref{array\_like}, optional) \textendash{} \sphinxhref{https://docs.scipy.org/doc/numpy-1.13.0/reference/generated/numpy.ndarray.html\#numpy.ndarray}{\sphinxcode{numpy.ndarray}} representing the stencils’ computational domain for the isentropic density
at the previous time level.

\item {} 
\sphinxstyleliteralstrong{old\_U} (\sphinxtitleref{array\_like}, optional) \textendash{} \sphinxhref{https://docs.scipy.org/doc/numpy-1.13.0/reference/generated/numpy.ndarray.html\#numpy.ndarray}{\sphinxcode{numpy.ndarray}} representing the stencils’ computational domain for the \(x\)-momentum
at the previous time level.

\item {} 
\sphinxstyleliteralstrong{old\_V} (\sphinxtitleref{array\_like}, optional) \textendash{} \sphinxhref{https://docs.scipy.org/doc/numpy-1.13.0/reference/generated/numpy.ndarray.html\#numpy.ndarray}{\sphinxcode{numpy.ndarray}} representing the stencils’ computational domain for the \(y\)-momentum
at the previous time level.

\item {} 
\sphinxstyleliteralstrong{old\_Qv} (\sphinxtitleref{array\_like}, optional) \textendash{} \sphinxhref{https://docs.scipy.org/doc/numpy-1.13.0/reference/generated/numpy.ndarray.html\#numpy.ndarray}{\sphinxcode{numpy.ndarray}} representing the stencils’ computational domain for the mass of water vapour
at the previous time level.

\item {} 
\sphinxstyleliteralstrong{old\_Qc} (\sphinxtitleref{array\_like}, optional) \textendash{} \sphinxhref{https://docs.scipy.org/doc/numpy-1.13.0/reference/generated/numpy.ndarray.html\#numpy.ndarray}{\sphinxcode{numpy.ndarray}} representing the stencils’ computational domain for the mass of cloud water
at the previous time level.

\item {} 
\sphinxstyleliteralstrong{old\_Qr} (\sphinxtitleref{array\_like}, optional) \textendash{} \sphinxhref{https://docs.scipy.org/doc/numpy-1.13.0/reference/generated/numpy.ndarray.html\#numpy.ndarray}{\sphinxcode{numpy.ndarray}} representing the stencils’ computational domain for the mass of precipitation water
at the previous time level.

\end{itemize}

\item[{Returns}] \leavevmode
\begin{itemize}
\item {} 
\sphinxstylestrong{out\_s} (\sphinxstyleemphasis{array\_like}) \textendash{} \sphinxhref{https://docs.scipy.org/doc/numpy-1.13.0/reference/generated/numpy.ndarray.html\#numpy.ndarray}{\sphinxcode{numpy.ndarray}} representing the stencils’ computational domain for the isentropic density
at the next time level.

\item {} 
\sphinxstylestrong{out\_U} (\sphinxstyleemphasis{array\_like}) \textendash{} \sphinxhref{https://docs.scipy.org/doc/numpy-1.13.0/reference/generated/numpy.ndarray.html\#numpy.ndarray}{\sphinxcode{numpy.ndarray}} representing the stencils’ computational domain for the \(x\)-momentum
at the next time level.

\item {} 
\sphinxstylestrong{out\_V} (\sphinxstyleemphasis{array\_like}) \textendash{} \sphinxhref{https://docs.scipy.org/doc/numpy-1.13.0/reference/generated/numpy.ndarray.html\#numpy.ndarray}{\sphinxcode{numpy.ndarray}} representing the stencils’ computational domain for the \(y\)-momentum
at the next time level.

\item {} 
\sphinxstylestrong{out\_Qv} (\sphinxtitleref{array\_like}, optional) \textendash{} \sphinxhref{https://docs.scipy.org/doc/numpy-1.13.0/reference/generated/numpy.ndarray.html\#numpy.ndarray}{\sphinxcode{numpy.ndarray}} representing the stencils’ computational domain for the mass of water vapour
at the next time level.

\item {} 
\sphinxstylestrong{out\_Qc} (\sphinxtitleref{array\_like}, optional) \textendash{} \sphinxhref{https://docs.scipy.org/doc/numpy-1.13.0/reference/generated/numpy.ndarray.html\#numpy.ndarray}{\sphinxcode{numpy.ndarray}} representing the stencils’ computational domain for the mass of cloud water
at the next time level.

\item {} 
\sphinxstylestrong{out\_Qr} (\sphinxtitleref{array\_like}, optional) \textendash{} \sphinxhref{https://docs.scipy.org/doc/numpy-1.13.0/reference/generated/numpy.ndarray.html\#numpy.ndarray}{\sphinxcode{numpy.ndarray}} representing the stencils’ computational domain for the mass of precipitation water
at the next time level.

\end{itemize}


\end{description}\end{quote}

\end{fulllineitems}


\end{fulllineitems}

\index{PrognosticIsentropicTL1 (class in dycore.prognostic\_isentropic)}

\begin{fulllineitems}
\phantomsection\label{\detokenize{api:dycore.prognostic_isentropic.PrognosticIsentropicTL1}}\pysiglinewithargsret{\sphinxbfcode{class }\sphinxcode{dycore.prognostic\_isentropic.}\sphinxbfcode{PrognosticIsentropicTL1}}{\emph{grid}, \emph{imoist}, \emph{scheme}, \emph{backend}}{}
This class inherits {\hyperref[\detokenize{api:dycore.prognostic_isentropic.PrognosticIsentropic}]{\sphinxcrossref{\sphinxcode{PrognosticIsentropic}}}} to implement
a standard one-time-level scheme carrying out the prognostic step of the three-dimensional
moist isentropic dynamical core.
\begin{quote}\begin{description}
\item[{Variables}] \leavevmode\begin{itemize}
\item {} 
\sphinxstyleliteralstrong{time\_levels} (\sphinxstyleliteralemphasis{int}) \textendash{} Number of time levels the scheme relies on.

\item {} 
\sphinxstyleliteralstrong{steps} (\sphinxstyleliteralemphasis{int}) \textendash{} Number of steps the scheme entails.

\end{itemize}

\end{description}\end{quote}
\index{\_\_init\_\_() (dycore.prognostic\_isentropic.PrognosticIsentropicTL1 method)}

\begin{fulllineitems}
\phantomsection\label{\detokenize{api:dycore.prognostic_isentropic.PrognosticIsentropicTL1.__init__}}\pysiglinewithargsret{\sphinxbfcode{\_\_init\_\_}}{\emph{grid}, \emph{imoist}, \emph{scheme}, \emph{backend}}{}
Constructor.
\begin{quote}\begin{description}
\item[{Parameters}] \leavevmode\begin{itemize}
\item {} 
\sphinxstyleliteralstrong{grid} (\sphinxstyleliteralemphasis{obj}) \textendash{} {\hyperref[\detokenize{api:grids.grid_xyz.GridXYZ}]{\sphinxcrossref{\sphinxcode{GridXYZ}}}} representing the underlying grid.

\item {} 
\sphinxstyleliteralstrong{imoist} (\sphinxstyleliteralemphasis{bool}) \textendash{} \sphinxcode{True} for a moist dynamical core, \sphinxcode{False} otherwise.

\item {} 
\sphinxstyleliteralstrong{scheme} (\sphinxstyleliteralemphasis{str}) \textendash{} 
String specifying the one-time-level scheme to use. Either:
\begin{itemize}
\item {} 
’upwind’, for the upwind scheme;

\item {} 
’maccormack’, for the MacCormack scheme.

\end{itemize}


\item {} 
\sphinxstyleliteralstrong{backend} (\sphinxstyleliteralemphasis{obj}) \textendash{} \sphinxcode{gridtools.mode} specifying the backend for the GT\$Py’s stencils.

\end{itemize}

\end{description}\end{quote}

\begin{sphinxadmonition}{note}{Note:}
To instantiate an object of this class, one should prefer the static method
{\hyperref[\detokenize{api:dycore.prognostic_isentropic.PrognosticIsentropic.factory}]{\sphinxcrossref{\sphinxcode{factory()}}}} of
{\hyperref[\detokenize{api:dycore.prognostic_isentropic.PrognosticIsentropic}]{\sphinxcrossref{\sphinxcode{PrognosticIsentropic}}}}.
\end{sphinxadmonition}

\end{fulllineitems}

\index{\_allocate\_temporaries() (dycore.prognostic\_isentropic.PrognosticIsentropicTL1 method)}

\begin{fulllineitems}
\phantomsection\label{\detokenize{api:dycore.prognostic_isentropic.PrognosticIsentropicTL1._allocate_temporaries}}\pysiglinewithargsret{\sphinxbfcode{\_allocate\_temporaries}}{\emph{s}}{}
Allocate the Numpy arrays which will store temporary fields.
\begin{quote}\begin{description}
\item[{Parameters}] \leavevmode
\sphinxstyleliteralstrong{s} (\sphinxstyleliteralemphasis{array\_like}) \textendash{} \sphinxhref{https://docs.scipy.org/doc/numpy-1.13.0/reference/generated/numpy.ndarray.html\#numpy.ndarray}{\sphinxcode{numpy.ndarray}} representing the stencils’ computational domain for the isentropic density
at current time.

\end{description}\end{quote}

\end{fulllineitems}

\index{\_defs\_stencil\_isentropic\_density\_and\_water\_constituents() (dycore.prognostic\_isentropic.PrognosticIsentropicTL1 method)}

\begin{fulllineitems}
\phantomsection\label{\detokenize{api:dycore.prognostic_isentropic.PrognosticIsentropicTL1._defs_stencil_isentropic_density_and_water_constituents}}\pysiglinewithargsret{\sphinxbfcode{\_defs\_stencil\_isentropic\_density\_and\_water\_constituents}}{\emph{dt}, \emph{in\_s}, \emph{in\_u}, \emph{in\_v}, \emph{in\_mtg}, \emph{in\_U}, \emph{in\_V}, \emph{in\_Qv=None}, \emph{in\_Qc=None}, \emph{in\_Qr=None}}{}
GT4Py’s stencil stepping the isentropic density and the water constituents via a one-time-level scheme.
Further, it computes the provisional values for the momentums.
\begin{quote}\begin{description}
\item[{Parameters}] \leavevmode\begin{itemize}
\item {} 
\sphinxstyleliteralstrong{dt} (\sphinxstyleliteralemphasis{obj}) \textendash{} \sphinxcode{gridtools.Global} representing the time step.

\item {} 
\sphinxstyleliteralstrong{in\_s} (\sphinxstyleliteralemphasis{obj}) \textendash{} \sphinxcode{gridtools.Equation} representing the isentropic density at the current time.

\item {} 
\sphinxstyleliteralstrong{in\_u} (\sphinxstyleliteralemphasis{obj}) \textendash{} \sphinxcode{gridtools.Equation} representing the \(x\)-velocity at the current time.

\item {} 
\sphinxstyleliteralstrong{in\_v} (\sphinxstyleliteralemphasis{obj}) \textendash{} \sphinxcode{gridtools.Equation} representing the \(y\)-velocity at the current time.

\item {} 
\sphinxstyleliteralstrong{in\_mtg} (\sphinxstyleliteralemphasis{obj}) \textendash{} \sphinxcode{gridtools.Equation} representing the Montgomery potential at the current time.

\item {} 
\sphinxstyleliteralstrong{in\_U} (\sphinxstyleliteralemphasis{obj}) \textendash{} \sphinxcode{gridtools.Equation} representing the \(x\)-momentum at the current time.

\item {} 
\sphinxstyleliteralstrong{in\_V} (\sphinxstyleliteralemphasis{obj}) \textendash{} \sphinxcode{gridtools.Equation} representing the \(y\)-momentum at the current time.

\item {} 
\sphinxstyleliteralstrong{in\_Qv} (\sphinxtitleref{obj}, optional) \textendash{} \sphinxcode{gridtools.Equation} representing the mass of water vapour at the current time.

\item {} 
\sphinxstyleliteralstrong{in\_Qc} (\sphinxtitleref{obj}, optional) \textendash{} \sphinxcode{gridtools.Equation} representing the mass of cloud water at the current time.

\item {} 
\sphinxstyleliteralstrong{in\_Qr} (\sphinxtitleref{obj}, optional) \textendash{} \sphinxcode{gridtools.Equation} representing the mass of precipitation water at the current time.

\end{itemize}

\item[{Returns}] \leavevmode
\begin{itemize}
\item {} 
\sphinxstylestrong{out\_s} (\sphinxstyleemphasis{obj}) \textendash{} \sphinxcode{gridtools.Equation} representing the stepped isentropic density.

\item {} 
\sphinxstylestrong{out\_U} (\sphinxstyleemphasis{obj}) \textendash{} \sphinxcode{gridtools.Equation} representing the provisional \(x\)-momentum.

\item {} 
\sphinxstylestrong{out\_V} (\sphinxstyleemphasis{obj}) \textendash{} \sphinxcode{gridtools.Equation} representing the provisional \(y\)-momentum.

\item {} 
\sphinxstylestrong{out\_Qv} (\sphinxtitleref{obj}, optional) \textendash{} \sphinxcode{gridtools.Equation} representing the stepped mass of water vapour.

\item {} 
\sphinxstylestrong{out\_Qc} (\sphinxtitleref{obj}, optional) \textendash{} \sphinxcode{gridtools.Equation} representing the stepped mass of cloud water.

\item {} 
\sphinxstylestrong{out\_Qr} (\sphinxtitleref{obj}, optional) \textendash{} \sphinxcode{gridtools.Equation} representing the stepped mass of precipitation water.

\end{itemize}


\end{description}\end{quote}

\end{fulllineitems}

\index{\_defs\_stencil\_momentums() (dycore.prognostic\_isentropic.PrognosticIsentropicTL1 method)}

\begin{fulllineitems}
\phantomsection\label{\detokenize{api:dycore.prognostic_isentropic.PrognosticIsentropicTL1._defs_stencil_momentums}}\pysiglinewithargsret{\sphinxbfcode{\_defs\_stencil\_momentums}}{\emph{dt}, \emph{in\_s}, \emph{in\_mtg}, \emph{in\_U}, \emph{in\_V}}{}
GT4Py’s stencil stepping the momentums via a one-time-level scheme.
\begin{quote}\begin{description}
\item[{Parameters}] \leavevmode\begin{itemize}
\item {} 
\sphinxstyleliteralstrong{dt} (\sphinxstyleliteralemphasis{obj}) \textendash{} \sphinxcode{gridtools.Global} representing the time step.

\item {} 
\sphinxstyleliteralstrong{in\_s} (\sphinxstyleliteralemphasis{obj}) \textendash{} \sphinxcode{gridtools.Equation} representing the stepped isentropic density.

\item {} 
\sphinxstyleliteralstrong{in\_mtg} (\sphinxstyleliteralemphasis{obj}) \textendash{} \sphinxcode{gridtools.Equation} representing the Montgomery potential diagnosed from the stepped isentropic density.

\item {} 
\sphinxstyleliteralstrong{in\_U} (\sphinxstyleliteralemphasis{obj}) \textendash{} \sphinxcode{gridtools.Equation} representing the provisional \(x\)-momentum.

\item {} 
\sphinxstyleliteralstrong{in\_V} (\sphinxstyleliteralemphasis{obj}) \textendash{} \sphinxcode{gridtools.Equation} representing the provisional \(y\)-momentum.

\end{itemize}

\item[{Returns}] \leavevmode
\begin{itemize}
\item {} 
\sphinxstylestrong{out\_U} (\sphinxstyleemphasis{obj}) \textendash{} \sphinxcode{gridtools.Equation} representing the stepped \(x\)-momentum.

\item {} 
\sphinxstylestrong{out\_V} (\sphinxstyleemphasis{obj}) \textendash{} \sphinxcode{gridtools.Equation} representing the stepped \(y\)-momentum.

\end{itemize}


\end{description}\end{quote}

\end{fulllineitems}

\index{\_initialize\_stencils() (dycore.prognostic\_isentropic.PrognosticIsentropicTL1 method)}

\begin{fulllineitems}
\phantomsection\label{\detokenize{api:dycore.prognostic_isentropic.PrognosticIsentropicTL1._initialize_stencils}}\pysiglinewithargsret{\sphinxbfcode{\_initialize\_stencils}}{\emph{s}, \emph{u}, \emph{v}}{}
Initialize the GT4Py’s stencil implementing the one-time-level scheme.
\begin{quote}\begin{description}
\item[{Parameters}] \leavevmode\begin{itemize}
\item {} 
\sphinxstyleliteralstrong{s} (\sphinxstyleliteralemphasis{array\_like}) \textendash{} \sphinxhref{https://docs.scipy.org/doc/numpy-1.13.0/reference/generated/numpy.ndarray.html\#numpy.ndarray}{\sphinxcode{numpy.ndarray}} representing the stencils’ computational domain for the isentropic density
at current time.

\item {} 
\sphinxstyleliteralstrong{u} (\sphinxstyleliteralemphasis{array\_like}) \textendash{} \sphinxhref{https://docs.scipy.org/doc/numpy-1.13.0/reference/generated/numpy.ndarray.html\#numpy.ndarray}{\sphinxcode{numpy.ndarray}} representing the stencils’ computational domain for the \(x\)-velocity
at current time.

\item {} 
\sphinxstyleliteralstrong{v} (\sphinxstyleliteralemphasis{array\_like}) \textendash{} \sphinxhref{https://docs.scipy.org/doc/numpy-1.13.0/reference/generated/numpy.ndarray.html\#numpy.ndarray}{\sphinxcode{numpy.ndarray}} representing the stencils’ computational domain for the \(y\)-velocity
at current time.

\end{itemize}

\end{description}\end{quote}

\end{fulllineitems}

\index{step\_forward() (dycore.prognostic\_isentropic.PrognosticIsentropicTL1 method)}

\begin{fulllineitems}
\phantomsection\label{\detokenize{api:dycore.prognostic_isentropic.PrognosticIsentropicTL1.step_forward}}\pysiglinewithargsret{\sphinxbfcode{step\_forward}}{\emph{diagnostic}, \emph{boundary}, \emph{dt}, \emph{s}, \emph{u}, \emph{v}, \emph{p}, \emph{mtg}, \emph{U}, \emph{V}, \emph{Qv=None}, \emph{Qc=None}, \emph{Qr=None}, \emph{old\_s=None}, \emph{old\_U=None}, \emph{old\_V=None}, \emph{old\_Qv=None}, \emph{old\_Qc=None}, \emph{old\_Qr=None}}{}
Method advancing the conservative model variables one time step forward via a one-time-level scheme.
\begin{quote}\begin{description}
\item[{Parameters}] \leavevmode\begin{itemize}
\item {} 
\sphinxstyleliteralstrong{diagnostic} (\sphinxstyleliteralemphasis{obj}) \textendash{} {\hyperref[\detokenize{api:dycore.diagnostic_isentropic.DiagnosticIsentropic}]{\sphinxcrossref{\sphinxcode{DiagnosticIsentropic}}}} performing the diagnostic steps
of the dynamical core.

\item {} 
\sphinxstyleliteralstrong{boundary} (\sphinxstyleliteralemphasis{obj}) \textendash{} An instance of one of the derived classes of {\hyperref[\detokenize{api:dycore.horizontal_boundary.HorizontalBoundary}]{\sphinxcrossref{\sphinxcode{HorizontalBoundary}}}},
implementing the lateral boundary conditions.

\item {} 
\sphinxstyleliteralstrong{dt} (\sphinxstyleliteralemphasis{obj}) \textendash{} A \sphinxcode{datetime.timedelta} representing the time step.

\item {} 
\sphinxstyleliteralstrong{s} (\sphinxstyleliteralemphasis{array\_like}) \textendash{} \sphinxhref{https://docs.scipy.org/doc/numpy-1.13.0/reference/generated/numpy.ndarray.html\#numpy.ndarray}{\sphinxcode{numpy.ndarray}} representing the stencils’ computational domain for the isentropic density
at current time.

\item {} 
\sphinxstyleliteralstrong{u} (\sphinxstyleliteralemphasis{array\_like}) \textendash{} \sphinxhref{https://docs.scipy.org/doc/numpy-1.13.0/reference/generated/numpy.ndarray.html\#numpy.ndarray}{\sphinxcode{numpy.ndarray}} representing the stencils’ computational domain for the \(x\)-velocity
at current time.

\item {} 
\sphinxstyleliteralstrong{v} (\sphinxstyleliteralemphasis{array\_like}) \textendash{} \sphinxhref{https://docs.scipy.org/doc/numpy-1.13.0/reference/generated/numpy.ndarray.html\#numpy.ndarray}{\sphinxcode{numpy.ndarray}} representing the stencils’ computational domain for the \(y\)-velocity
at current time.

\item {} 
\sphinxstyleliteralstrong{p} (\sphinxstyleliteralemphasis{array\_like}) \textendash{} \sphinxhref{https://docs.scipy.org/doc/numpy-1.13.0/reference/generated/numpy.ndarray.html\#numpy.ndarray}{\sphinxcode{numpy.ndarray}} representing the stencils’ computational domain for the pressure at current time.

\item {} 
\sphinxstyleliteralstrong{mtg} (\sphinxstyleliteralemphasis{array\_like}) \textendash{} \sphinxhref{https://docs.scipy.org/doc/numpy-1.13.0/reference/generated/numpy.ndarray.html\#numpy.ndarray}{\sphinxcode{numpy.ndarray}} representing the stencils’ computational domain for the Montgomery potential
at current time.

\item {} 
\sphinxstyleliteralstrong{U} (\sphinxstyleliteralemphasis{array\_like}) \textendash{} \sphinxhref{https://docs.scipy.org/doc/numpy-1.13.0/reference/generated/numpy.ndarray.html\#numpy.ndarray}{\sphinxcode{numpy.ndarray}} representing the stencils’ computational domain for the \(x\)-momentum
at current time.

\item {} 
\sphinxstyleliteralstrong{V} (\sphinxstyleliteralemphasis{array\_like}) \textendash{} \sphinxhref{https://docs.scipy.org/doc/numpy-1.13.0/reference/generated/numpy.ndarray.html\#numpy.ndarray}{\sphinxcode{numpy.ndarray}} representing the stencils’ computational domain for the \(y\)-momentum
at current time.

\item {} 
\sphinxstyleliteralstrong{Qv} (\sphinxtitleref{array\_like}, optional) \textendash{} \sphinxhref{https://docs.scipy.org/doc/numpy-1.13.0/reference/generated/numpy.ndarray.html\#numpy.ndarray}{\sphinxcode{numpy.ndarray}} representing the stencils’ computational domain for the mass of water vapour
at current time.

\item {} 
\sphinxstyleliteralstrong{Qc} (\sphinxtitleref{array\_like}, optional) \textendash{} \sphinxhref{https://docs.scipy.org/doc/numpy-1.13.0/reference/generated/numpy.ndarray.html\#numpy.ndarray}{\sphinxcode{numpy.ndarray}} representing the stencils’ computational domain for the mass of cloud water
at current time.

\item {} 
\sphinxstyleliteralstrong{Qr} (\sphinxtitleref{array\_like}, optional) \textendash{} \sphinxhref{https://docs.scipy.org/doc/numpy-1.13.0/reference/generated/numpy.ndarray.html\#numpy.ndarray}{\sphinxcode{numpy.ndarray}} representing the stencils’ computational domain for the mass of precipitation water
at current time.

\item {} 
\sphinxstyleliteralstrong{old\_s} (\sphinxtitleref{array\_like}, optional) \textendash{} \sphinxhref{https://docs.scipy.org/doc/numpy-1.13.0/reference/generated/numpy.ndarray.html\#numpy.ndarray}{\sphinxcode{numpy.ndarray}} representing the stencils’ computational domain for the isentropic density
at the previous time level.

\item {} 
\sphinxstyleliteralstrong{old\_U} (\sphinxtitleref{array\_like}, optional) \textendash{} \sphinxhref{https://docs.scipy.org/doc/numpy-1.13.0/reference/generated/numpy.ndarray.html\#numpy.ndarray}{\sphinxcode{numpy.ndarray}} representing the stencils’ computational domain for the \(x\)-momentum
at the previous time level.

\item {} 
\sphinxstyleliteralstrong{old\_V} (\sphinxtitleref{array\_like}, optional) \textendash{} \sphinxhref{https://docs.scipy.org/doc/numpy-1.13.0/reference/generated/numpy.ndarray.html\#numpy.ndarray}{\sphinxcode{numpy.ndarray}} representing the stencils’ computational domain for the \(y\)-momentum
at the previous time level.

\item {} 
\sphinxstyleliteralstrong{old\_Qv} (\sphinxtitleref{array\_like}, optional) \textendash{} \sphinxhref{https://docs.scipy.org/doc/numpy-1.13.0/reference/generated/numpy.ndarray.html\#numpy.ndarray}{\sphinxcode{numpy.ndarray}} representing the stencils’ computational domain for the mass of water vapour
at the previous time level.

\item {} 
\sphinxstyleliteralstrong{old\_Qc} (\sphinxtitleref{array\_like}, optional) \textendash{} \sphinxhref{https://docs.scipy.org/doc/numpy-1.13.0/reference/generated/numpy.ndarray.html\#numpy.ndarray}{\sphinxcode{numpy.ndarray}} representing the stencils’ computational domain for the mass of cloud water
at the previous time level.

\item {} 
\sphinxstyleliteralstrong{old\_Qr} (\sphinxtitleref{array\_like}, optional) \textendash{} \sphinxhref{https://docs.scipy.org/doc/numpy-1.13.0/reference/generated/numpy.ndarray.html\#numpy.ndarray}{\sphinxcode{numpy.ndarray}} representing the stencils’ computational domain for the mass of precipitation water
at the previous time level.

\end{itemize}

\item[{Returns}] \leavevmode
\begin{itemize}
\item {} 
\sphinxstylestrong{out\_s} (\sphinxstyleemphasis{array\_like}) \textendash{} \sphinxhref{https://docs.scipy.org/doc/numpy-1.13.0/reference/generated/numpy.ndarray.html\#numpy.ndarray}{\sphinxcode{numpy.ndarray}} representing the stencils’ computational domain for the isentropic density
at the next time level.

\item {} 
\sphinxstylestrong{out\_U} (\sphinxstyleemphasis{array\_like}) \textendash{} \sphinxhref{https://docs.scipy.org/doc/numpy-1.13.0/reference/generated/numpy.ndarray.html\#numpy.ndarray}{\sphinxcode{numpy.ndarray}} representing the stencils’ computational domain for the \(x\)-momentum
at the next time level.

\item {} 
\sphinxstylestrong{out\_V} (\sphinxstyleemphasis{array\_like}) \textendash{} \sphinxhref{https://docs.scipy.org/doc/numpy-1.13.0/reference/generated/numpy.ndarray.html\#numpy.ndarray}{\sphinxcode{numpy.ndarray}} representing the stencils’ computational domain for the \(y\)-momentum
at the next time level.

\item {} 
\sphinxstylestrong{out\_Qv} (\sphinxtitleref{array\_like}, optional) \textendash{} \sphinxhref{https://docs.scipy.org/doc/numpy-1.13.0/reference/generated/numpy.ndarray.html\#numpy.ndarray}{\sphinxcode{numpy.ndarray}} representing the stencils’ computational domain for the mass of water vapour
at the next time level.

\item {} 
\sphinxstylestrong{out\_Qc} (\sphinxtitleref{array\_like}, optional) \textendash{} \sphinxhref{https://docs.scipy.org/doc/numpy-1.13.0/reference/generated/numpy.ndarray.html\#numpy.ndarray}{\sphinxcode{numpy.ndarray}} representing the stencils’ computational domain for the mass of cloud water
at the next time level.

\item {} 
\sphinxstylestrong{out\_Qr} (\sphinxtitleref{array\_like}, optional) \textendash{} \sphinxhref{https://docs.scipy.org/doc/numpy-1.13.0/reference/generated/numpy.ndarray.html\#numpy.ndarray}{\sphinxcode{numpy.ndarray}} representing the stencils’ computational domain for the mass of precipitation water
at the next time level.

\end{itemize}


\end{description}\end{quote}

\end{fulllineitems}


\end{fulllineitems}

\index{PrognosticIsentropicTL2 (class in dycore.prognostic\_isentropic)}

\begin{fulllineitems}
\phantomsection\label{\detokenize{api:dycore.prognostic_isentropic.PrognosticIsentropicTL2}}\pysiglinewithargsret{\sphinxbfcode{class }\sphinxcode{dycore.prognostic\_isentropic.}\sphinxbfcode{PrognosticIsentropicTL2}}{\emph{grid}, \emph{imoist}, \emph{scheme}, \emph{backend}}{}
This class inherits {\hyperref[\detokenize{api:dycore.prognostic_isentropic.PrognosticIsentropic}]{\sphinxcrossref{\sphinxcode{PrognosticIsentropic}}}} to
implement a standard two-time-levels scheme carrying out the prognostic step of the three-dimensional
moist isentropic dynamical core. An example of this kind of schemes is the leapfrog scheme.
\begin{quote}\begin{description}
\item[{Variables}] \leavevmode\begin{itemize}
\item {} 
\sphinxstyleliteralstrong{time\_levels} (\sphinxstyleliteralemphasis{int}) \textendash{} Number of time levels the scheme relies on.

\item {} 
\sphinxstyleliteralstrong{steps} (\sphinxstyleliteralemphasis{int}) \textendash{} Number of steps the scheme entails.

\end{itemize}

\end{description}\end{quote}
\index{\_\_init\_\_() (dycore.prognostic\_isentropic.PrognosticIsentropicTL2 method)}

\begin{fulllineitems}
\phantomsection\label{\detokenize{api:dycore.prognostic_isentropic.PrognosticIsentropicTL2.__init__}}\pysiglinewithargsret{\sphinxbfcode{\_\_init\_\_}}{\emph{grid}, \emph{imoist}, \emph{scheme}, \emph{backend}}{}
Constructor.
\begin{quote}\begin{description}
\item[{Parameters}] \leavevmode\begin{itemize}
\item {} 
\sphinxstyleliteralstrong{grid} (\sphinxstyleliteralemphasis{obj}) \textendash{} {\hyperref[\detokenize{api:grids.grid_xyz.GridXYZ}]{\sphinxcrossref{\sphinxcode{GridXYZ}}}} representing the underlying grid.

\item {} 
\sphinxstyleliteralstrong{imoist} (\sphinxstyleliteralemphasis{bool}) \textendash{} \sphinxcode{True} for a moist dynamical core, \sphinxcode{False} otherwise.

\item {} 
\sphinxstyleliteralstrong{scheme} (\sphinxstyleliteralemphasis{str}) \textendash{} 
String specifying the two-time-level scheme to use. Either:
\begin{itemize}
\item {} 
’leapfrog’, for the leapfrog scheme.

\end{itemize}


\item {} 
\sphinxstyleliteralstrong{backend} (\sphinxstyleliteralemphasis{obj}) \textendash{} \sphinxcode{gridtools.mode} specifying the backend for the GT4Py’s stencils.

\end{itemize}

\end{description}\end{quote}

\begin{sphinxadmonition}{note}{Note:}
To instantiate an object of this class, one should prefer the static method
{\hyperref[\detokenize{api:dycore.prognostic_isentropic.PrognosticIsentropic.factory}]{\sphinxcrossref{\sphinxcode{factory()}}}} of
{\hyperref[\detokenize{api:dycore.prognostic_isentropic.PrognosticIsentropic}]{\sphinxcrossref{\sphinxcode{PrognosticIsentropic}}}}.
\end{sphinxadmonition}

\end{fulllineitems}

\index{\_allocate\_inputs() (dycore.prognostic\_isentropic.PrognosticIsentropicTL2 method)}

\begin{fulllineitems}
\phantomsection\label{\detokenize{api:dycore.prognostic_isentropic.PrognosticIsentropicTL2._allocate_inputs}}\pysiglinewithargsret{\sphinxbfcode{\_allocate\_inputs}}{\emph{s}, \emph{u}, \emph{v}}{}
Allocate the attributes which will serve as inputs to the GT4Py’s stencil.
\begin{quote}\begin{description}
\item[{Parameters}] \leavevmode\begin{itemize}
\item {} 
\sphinxstyleliteralstrong{s} (\sphinxstyleliteralemphasis{array\_like}) \textendash{} \sphinxhref{https://docs.scipy.org/doc/numpy-1.13.0/reference/generated/numpy.ndarray.html\#numpy.ndarray}{\sphinxcode{numpy.ndarray}} representing the stencils’ computational domain for the isentropic density at current time.

\item {} 
\sphinxstyleliteralstrong{u} (\sphinxstyleliteralemphasis{array\_like}) \textendash{} \sphinxhref{https://docs.scipy.org/doc/numpy-1.13.0/reference/generated/numpy.ndarray.html\#numpy.ndarray}{\sphinxcode{numpy.ndarray}} representing the stencils’ computational domain for the \(x\)-velocity at current time.

\item {} 
\sphinxstyleliteralstrong{v} (\sphinxstyleliteralemphasis{array\_like}) \textendash{} \sphinxhref{https://docs.scipy.org/doc/numpy-1.13.0/reference/generated/numpy.ndarray.html\#numpy.ndarray}{\sphinxcode{numpy.ndarray}} representing the stencils’ computational domain for the \(y\)-velocity at current time.

\end{itemize}

\end{description}\end{quote}

\end{fulllineitems}

\index{\_defs\_stencil() (dycore.prognostic\_isentropic.PrognosticIsentropicTL2 method)}

\begin{fulllineitems}
\phantomsection\label{\detokenize{api:dycore.prognostic_isentropic.PrognosticIsentropicTL2._defs_stencil}}\pysiglinewithargsret{\sphinxbfcode{\_defs\_stencil}}{\emph{dt}, \emph{in\_s}, \emph{in\_u}, \emph{in\_v}, \emph{in\_mtg}, \emph{in\_U}, \emph{in\_V}, \emph{in\_Qv=None}, \emph{in\_Qc=None}, \emph{in\_Qr=None}, \emph{old\_s=None}, \emph{old\_U=None}, \emph{old\_V=None}, \emph{old\_Qv=None}, \emph{old\_Qc=None}, \emph{old\_Qr=None}}{}
GT4Py’s stencil implementing a two-time-levels scheme.
\begin{quote}\begin{description}
\item[{Parameters}] \leavevmode\begin{itemize}
\item {} 
\sphinxstyleliteralstrong{dt} (\sphinxstyleliteralemphasis{obj}) \textendash{} \sphinxcode{gridtools.Global} representing the time step.

\item {} 
\sphinxstyleliteralstrong{in\_s} (\sphinxstyleliteralemphasis{obj}) \textendash{} \sphinxcode{gridtools.Equation} representing the isentropic density at the current time.

\item {} 
\sphinxstyleliteralstrong{in\_u} (\sphinxstyleliteralemphasis{obj}) \textendash{} \sphinxcode{gridtools.Equation} representing the \(x\)-velocity at the current time.

\item {} 
\sphinxstyleliteralstrong{in\_v} (\sphinxstyleliteralemphasis{obj}) \textendash{} \sphinxcode{gridtools.Equation} representing the \(y\)-velocity at the current time.

\item {} 
\sphinxstyleliteralstrong{in\_mtg} (\sphinxstyleliteralemphasis{obj}) \textendash{} \sphinxcode{gridtools.Equation} representing the Montgomery potential at the current time.

\item {} 
\sphinxstyleliteralstrong{in\_U} (\sphinxstyleliteralemphasis{obj}) \textendash{} \sphinxcode{gridtools.Equation} representing the \(x\)-momentum at the current time.

\item {} 
\sphinxstyleliteralstrong{in\_V} (\sphinxstyleliteralemphasis{obj}) \textendash{} \sphinxcode{gridtools.Equation} representing the \(y\)-momentum at the current time.

\item {} 
\sphinxstyleliteralstrong{in\_Qv} (\sphinxtitleref{obj}, optional) \textendash{} \sphinxcode{gridtools.Equation} representing the mass of water vapour at the current time.

\item {} 
\sphinxstyleliteralstrong{in\_Qc} (\sphinxtitleref{obj}, optional) \textendash{} \sphinxcode{gridtools.Equation} representing the mass of cloud water at the current time.

\item {} 
\sphinxstyleliteralstrong{in\_Qr} (\sphinxtitleref{obj}, optional) \textendash{} \sphinxcode{gridtools.Equation} representing the mass of precipitation water at the current time.

\item {} 
\sphinxstyleliteralstrong{old\_s} (\sphinxtitleref{obj}, optional) \textendash{} \sphinxcode{gridtools.Equation} representing the isentropic density at the previous time level.

\item {} 
\sphinxstyleliteralstrong{old\_U} (\sphinxtitleref{obj}, optional) \textendash{} \sphinxcode{gridtools.Equation} representing the \(x\)-momentum at the previous time level.

\item {} 
\sphinxstyleliteralstrong{old\_V} (\sphinxtitleref{obj}, optional) \textendash{} \sphinxcode{gridtools.Equation} representing the \(y\)-momentum at the previous time level.

\item {} 
\sphinxstyleliteralstrong{old\_Qv} (\sphinxtitleref{obj}, optional) \textendash{} \sphinxcode{gridtools.Equation} representing the mass of water vapour at the previous time level.

\item {} 
\sphinxstyleliteralstrong{old\_Qc} (\sphinxtitleref{obj}, optional) \textendash{} \sphinxcode{gridtools.Equation} representing the mass of cloud water at the previous time level.

\item {} 
\sphinxstyleliteralstrong{old\_Qr} (\sphinxtitleref{obj}, optional) \textendash{} \sphinxcode{gridtools.Equation} representing the mass of precipitation water at the previous time level.

\end{itemize}

\item[{Returns}] \leavevmode
\begin{itemize}
\item {} 
\sphinxstylestrong{out\_s} (\sphinxstyleemphasis{obj}) \textendash{} \sphinxcode{gridtools.Equation} representing the stepped isentropic density.

\item {} 
\sphinxstylestrong{out\_U} (\sphinxstyleemphasis{obj}) \textendash{} \sphinxcode{gridtools.Equation} representing the provisional \(x\)-momentum.

\item {} 
\sphinxstylestrong{out\_V} (\sphinxstyleemphasis{obj}) \textendash{} \sphinxcode{gridtools.Equation} representing the provisional \(y\)-momentum.

\item {} 
\sphinxstylestrong{out\_Qv} (\sphinxtitleref{obj}, optional) \textendash{} \sphinxcode{gridtools.Equation} representing the stepped mass of water vapour.

\item {} 
\sphinxstylestrong{out\_Qc} (\sphinxtitleref{obj}, optional) \textendash{} \sphinxcode{gridtools.Equation} representing the stepped mass of cloud water.

\item {} 
\sphinxstylestrong{out\_Qr} (\sphinxtitleref{obj}, optional) \textendash{} \sphinxcode{gridtools.Equation} representing the stepped mass of precipitation water.

\end{itemize}


\end{description}\end{quote}

\end{fulllineitems}

\index{\_initialize\_stencil() (dycore.prognostic\_isentropic.PrognosticIsentropicTL2 method)}

\begin{fulllineitems}
\phantomsection\label{\detokenize{api:dycore.prognostic_isentropic.PrognosticIsentropicTL2._initialize_stencil}}\pysiglinewithargsret{\sphinxbfcode{\_initialize\_stencil}}{\emph{s}, \emph{u}, \emph{v}}{}
Initialize the GT4Py’s stencil implementing the two-time-levels scheme.
\begin{quote}\begin{description}
\item[{Parameters}] \leavevmode\begin{itemize}
\item {} 
\sphinxstyleliteralstrong{s} (\sphinxstyleliteralemphasis{array\_like}) \textendash{} \sphinxhref{https://docs.scipy.org/doc/numpy-1.13.0/reference/generated/numpy.ndarray.html\#numpy.ndarray}{\sphinxcode{numpy.ndarray}} representing the stencils’ computational domain for the isentropic density at current time.

\item {} 
\sphinxstyleliteralstrong{u} (\sphinxstyleliteralemphasis{array\_like}) \textendash{} \sphinxhref{https://docs.scipy.org/doc/numpy-1.13.0/reference/generated/numpy.ndarray.html\#numpy.ndarray}{\sphinxcode{numpy.ndarray}} representing the stencils’ computational domain for the \(x\)-velocity at current time.

\item {} 
\sphinxstyleliteralstrong{v} (\sphinxstyleliteralemphasis{array\_like}) \textendash{} \sphinxhref{https://docs.scipy.org/doc/numpy-1.13.0/reference/generated/numpy.ndarray.html\#numpy.ndarray}{\sphinxcode{numpy.ndarray}} representing the stencils’ computational domain for the \(y\)-velocity at current time.

\end{itemize}

\end{description}\end{quote}

\end{fulllineitems}

\index{\_set\_inputs() (dycore.prognostic\_isentropic.PrognosticIsentropicTL2 method)}

\begin{fulllineitems}
\phantomsection\label{\detokenize{api:dycore.prognostic_isentropic.PrognosticIsentropicTL2._set_inputs}}\pysiglinewithargsret{\sphinxbfcode{\_set\_inputs}}{\emph{dt}, \emph{s}, \emph{u}, \emph{v}, \emph{mtg}, \emph{U}, \emph{V}, \emph{Qv}, \emph{Qc}, \emph{Qr}, \emph{old\_s}, \emph{old\_U}, \emph{old\_V}, \emph{old\_Qv}, \emph{old\_Qc}, \emph{old\_Qr}}{}
Update the attributes which serve as inputs to the GT4Py’s stencil.
\begin{quote}\begin{description}
\item[{Parameters}] \leavevmode\begin{itemize}
\item {} 
\sphinxstyleliteralstrong{dt} (\sphinxstyleliteralemphasis{obj}) \textendash{} A \sphinxcode{datetime.timedelta} representing the time step.

\item {} 
\sphinxstyleliteralstrong{s} (\sphinxstyleliteralemphasis{array\_like}) \textendash{} \sphinxhref{https://docs.scipy.org/doc/numpy-1.13.0/reference/generated/numpy.ndarray.html\#numpy.ndarray}{\sphinxcode{numpy.ndarray}} representing the stencils’ computational domain for the isentropic density
at current time.

\item {} 
\sphinxstyleliteralstrong{u} (\sphinxstyleliteralemphasis{array\_like}) \textendash{} \sphinxhref{https://docs.scipy.org/doc/numpy-1.13.0/reference/generated/numpy.ndarray.html\#numpy.ndarray}{\sphinxcode{numpy.ndarray}} representing the stencils’ computational domain for the \(x\)-velocity
at current time.

\item {} 
\sphinxstyleliteralstrong{v} (\sphinxstyleliteralemphasis{array\_like}) \textendash{} \sphinxhref{https://docs.scipy.org/doc/numpy-1.13.0/reference/generated/numpy.ndarray.html\#numpy.ndarray}{\sphinxcode{numpy.ndarray}} representing the stencils’ computational domain for the \(y\)-velocity
at current time.

\item {} 
\sphinxstyleliteralstrong{p} (\sphinxstyleliteralemphasis{array\_like}) \textendash{} \sphinxhref{https://docs.scipy.org/doc/numpy-1.13.0/reference/generated/numpy.ndarray.html\#numpy.ndarray}{\sphinxcode{numpy.ndarray}} representing the stencils’ computational domain for the pressure at current time.

\item {} 
\sphinxstyleliteralstrong{mtg} (\sphinxstyleliteralemphasis{array\_like}) \textendash{} \sphinxhref{https://docs.scipy.org/doc/numpy-1.13.0/reference/generated/numpy.ndarray.html\#numpy.ndarray}{\sphinxcode{numpy.ndarray}} representing the stencils’ computational domain for the Montgomery potential
at current time.

\item {} 
\sphinxstyleliteralstrong{U} (\sphinxstyleliteralemphasis{array\_like}) \textendash{} \sphinxhref{https://docs.scipy.org/doc/numpy-1.13.0/reference/generated/numpy.ndarray.html\#numpy.ndarray}{\sphinxcode{numpy.ndarray}} representing the stencils’ computational domain for the \(x\)-momentum
at current time.

\item {} 
\sphinxstyleliteralstrong{V} (\sphinxstyleliteralemphasis{array\_like}) \textendash{} \sphinxhref{https://docs.scipy.org/doc/numpy-1.13.0/reference/generated/numpy.ndarray.html\#numpy.ndarray}{\sphinxcode{numpy.ndarray}} representing the stencils’ computational domain for the \(y\)-momentum
at current time.

\item {} 
\sphinxstyleliteralstrong{Qv} (\sphinxstyleliteralemphasis{array\_like}) \textendash{} \sphinxhref{https://docs.scipy.org/doc/numpy-1.13.0/reference/generated/numpy.ndarray.html\#numpy.ndarray}{\sphinxcode{numpy.ndarray}} representing the stencils’ computational domain for the mass of water vapour
at current time.

\item {} 
\sphinxstyleliteralstrong{Qc} (\sphinxstyleliteralemphasis{array\_like}) \textendash{} \sphinxhref{https://docs.scipy.org/doc/numpy-1.13.0/reference/generated/numpy.ndarray.html\#numpy.ndarray}{\sphinxcode{numpy.ndarray}} representing the stencils’ computational domain for the mass of cloud water
at current time.

\item {} 
\sphinxstyleliteralstrong{Qr} (\sphinxstyleliteralemphasis{array\_like}) \textendash{} \sphinxhref{https://docs.scipy.org/doc/numpy-1.13.0/reference/generated/numpy.ndarray.html\#numpy.ndarray}{\sphinxcode{numpy.ndarray}} representing the stencils’ computational domain for the mass of precipitation water
at current time.

\item {} 
\sphinxstyleliteralstrong{old\_s} (\sphinxtitleref{array\_like}, optional) \textendash{} \sphinxhref{https://docs.scipy.org/doc/numpy-1.13.0/reference/generated/numpy.ndarray.html\#numpy.ndarray}{\sphinxcode{numpy.ndarray}} representing the stencils’ computational domain for the isentropic density
at the previous time level.

\item {} 
\sphinxstyleliteralstrong{old\_U} (\sphinxtitleref{array\_like}, optional) \textendash{} \sphinxhref{https://docs.scipy.org/doc/numpy-1.13.0/reference/generated/numpy.ndarray.html\#numpy.ndarray}{\sphinxcode{numpy.ndarray}} representing the stencils’ computational domain for the \(x\)-momentum
at the previous time level.

\item {} 
\sphinxstyleliteralstrong{old\_V} (\sphinxtitleref{array\_like}, optional) \textendash{} \sphinxhref{https://docs.scipy.org/doc/numpy-1.13.0/reference/generated/numpy.ndarray.html\#numpy.ndarray}{\sphinxcode{numpy.ndarray}} representing the stencils’ computational domain for the \(y\)-momentum
at the previous time level.

\item {} 
\sphinxstyleliteralstrong{old\_Qv} (\sphinxtitleref{array\_like}, optional) \textendash{} \sphinxhref{https://docs.scipy.org/doc/numpy-1.13.0/reference/generated/numpy.ndarray.html\#numpy.ndarray}{\sphinxcode{numpy.ndarray}} representing the stencils’ computational domain for the mass of water vapour
at the previous time level.

\item {} 
\sphinxstyleliteralstrong{old\_Qc} (\sphinxtitleref{array\_like}, optional) \textendash{} \sphinxhref{https://docs.scipy.org/doc/numpy-1.13.0/reference/generated/numpy.ndarray.html\#numpy.ndarray}{\sphinxcode{numpy.ndarray}} representing the stencils’ computational domain for the mass of cloud water
at the previous time level.

\item {} 
\sphinxstyleliteralstrong{old\_Qr} (\sphinxtitleref{array\_like}, optional) \textendash{} \sphinxhref{https://docs.scipy.org/doc/numpy-1.13.0/reference/generated/numpy.ndarray.html\#numpy.ndarray}{\sphinxcode{numpy.ndarray}} representing the stencils’ computational domain for the mass of precipitation water
at the previous time level.

\end{itemize}

\end{description}\end{quote}

\end{fulllineitems}

\index{step\_forward() (dycore.prognostic\_isentropic.PrognosticIsentropicTL2 method)}

\begin{fulllineitems}
\phantomsection\label{\detokenize{api:dycore.prognostic_isentropic.PrognosticIsentropicTL2.step_forward}}\pysiglinewithargsret{\sphinxbfcode{step\_forward}}{\emph{boundary}, \emph{diagnostic}, \emph{dt}, \emph{s}, \emph{u}, \emph{v}, \emph{p}, \emph{mtg}, \emph{U}, \emph{V}, \emph{Qv=None}, \emph{Qc=None}, \emph{Qr=None}, \emph{old\_s=None}, \emph{old\_U=None}, \emph{old\_V=None}, \emph{old\_Qv=None}, \emph{old\_Qc=None}, \emph{old\_Qr=None}}{}
Method advancing the conservative model variables one time step forward via a two-time-level scheme.
\begin{quote}\begin{description}
\item[{Parameters}] \leavevmode\begin{itemize}
\item {} 
\sphinxstyleliteralstrong{diagnostic} (\sphinxstyleliteralemphasis{obj}) \textendash{} {\hyperref[\detokenize{api:dycore.diagnostic_isentropic.DiagnosticIsentropic}]{\sphinxcrossref{\sphinxcode{DiagnosticIsentropic}}}} performing the diagnostic steps
of the dynamical core.

\item {} 
\sphinxstyleliteralstrong{boundary} (\sphinxstyleliteralemphasis{obj}) \textendash{} An instance of one of the derived classes of {\hyperref[\detokenize{api:dycore.horizontal_boundary.HorizontalBoundary}]{\sphinxcrossref{\sphinxcode{HorizontalBoundary}}}},
implementing the lateral boundary conditions.

\item {} 
\sphinxstyleliteralstrong{dt} (\sphinxstyleliteralemphasis{obj}) \textendash{} A \sphinxcode{datetime.timedelta} representing the time step.

\item {} 
\sphinxstyleliteralstrong{s} (\sphinxstyleliteralemphasis{array\_like}) \textendash{} \sphinxhref{https://docs.scipy.org/doc/numpy-1.13.0/reference/generated/numpy.ndarray.html\#numpy.ndarray}{\sphinxcode{numpy.ndarray}} representing the stencils’ computational domain for the isentropic density
at current time.

\item {} 
\sphinxstyleliteralstrong{u} (\sphinxstyleliteralemphasis{array\_like}) \textendash{} \sphinxhref{https://docs.scipy.org/doc/numpy-1.13.0/reference/generated/numpy.ndarray.html\#numpy.ndarray}{\sphinxcode{numpy.ndarray}} representing the stencils’ computational domain for the \(x\)-velocity
at current time.

\item {} 
\sphinxstyleliteralstrong{v} (\sphinxstyleliteralemphasis{array\_like}) \textendash{} \sphinxhref{https://docs.scipy.org/doc/numpy-1.13.0/reference/generated/numpy.ndarray.html\#numpy.ndarray}{\sphinxcode{numpy.ndarray}} representing the stencils’ computational domain for the \(y\)-velocity
at current time.

\item {} 
\sphinxstyleliteralstrong{p} (\sphinxstyleliteralemphasis{array\_like}) \textendash{} \sphinxhref{https://docs.scipy.org/doc/numpy-1.13.0/reference/generated/numpy.ndarray.html\#numpy.ndarray}{\sphinxcode{numpy.ndarray}} representing the stencils’ computational domain for the pressure at current time.

\item {} 
\sphinxstyleliteralstrong{mtg} (\sphinxstyleliteralemphasis{array\_like}) \textendash{} \sphinxhref{https://docs.scipy.org/doc/numpy-1.13.0/reference/generated/numpy.ndarray.html\#numpy.ndarray}{\sphinxcode{numpy.ndarray}} representing the stencils’ computational domain for the Montgomery potential
at current time.

\item {} 
\sphinxstyleliteralstrong{U} (\sphinxstyleliteralemphasis{array\_like}) \textendash{} \sphinxhref{https://docs.scipy.org/doc/numpy-1.13.0/reference/generated/numpy.ndarray.html\#numpy.ndarray}{\sphinxcode{numpy.ndarray}} representing the stencils’ computational domain for the \(x\)-momentum
at current time.

\item {} 
\sphinxstyleliteralstrong{V} (\sphinxstyleliteralemphasis{array\_like}) \textendash{} \sphinxhref{https://docs.scipy.org/doc/numpy-1.13.0/reference/generated/numpy.ndarray.html\#numpy.ndarray}{\sphinxcode{numpy.ndarray}} representing the stencils’ computational domain for the \(y\)-momentum
at current time.

\item {} 
\sphinxstyleliteralstrong{Qv} (\sphinxtitleref{array\_like}, optional) \textendash{} \sphinxhref{https://docs.scipy.org/doc/numpy-1.13.0/reference/generated/numpy.ndarray.html\#numpy.ndarray}{\sphinxcode{numpy.ndarray}} representing the stencils’ computational domain for the mass of water vapour
at current time.

\item {} 
\sphinxstyleliteralstrong{Qc} (\sphinxtitleref{array\_like}, optional) \textendash{} \sphinxhref{https://docs.scipy.org/doc/numpy-1.13.0/reference/generated/numpy.ndarray.html\#numpy.ndarray}{\sphinxcode{numpy.ndarray}} representing the stencils’ computational domain for the mass of cloud water
at current time.

\item {} 
\sphinxstyleliteralstrong{Qr} (\sphinxtitleref{array\_like}, optional) \textendash{} \sphinxhref{https://docs.scipy.org/doc/numpy-1.13.0/reference/generated/numpy.ndarray.html\#numpy.ndarray}{\sphinxcode{numpy.ndarray}} representing the stencils’ computational domain for the mass of precipitation water
at current time.

\item {} 
\sphinxstyleliteralstrong{old\_s} (\sphinxtitleref{array\_like}, optional) \textendash{} \sphinxhref{https://docs.scipy.org/doc/numpy-1.13.0/reference/generated/numpy.ndarray.html\#numpy.ndarray}{\sphinxcode{numpy.ndarray}} representing the stencils’ computational domain for the isentropic density
at the previous time level.

\item {} 
\sphinxstyleliteralstrong{old\_U} (\sphinxtitleref{array\_like}, optional) \textendash{} \sphinxhref{https://docs.scipy.org/doc/numpy-1.13.0/reference/generated/numpy.ndarray.html\#numpy.ndarray}{\sphinxcode{numpy.ndarray}} representing the stencils’ computational domain for the \(x\)-momentum
at the previous time level.

\item {} 
\sphinxstyleliteralstrong{old\_V} (\sphinxtitleref{array\_like}, optional) \textendash{} \sphinxhref{https://docs.scipy.org/doc/numpy-1.13.0/reference/generated/numpy.ndarray.html\#numpy.ndarray}{\sphinxcode{numpy.ndarray}} representing the stencils’ computational domain for the \(y\)-momentum
at the previous time level.

\item {} 
\sphinxstyleliteralstrong{old\_Qv} (\sphinxtitleref{array\_like}, optional) \textendash{} \sphinxhref{https://docs.scipy.org/doc/numpy-1.13.0/reference/generated/numpy.ndarray.html\#numpy.ndarray}{\sphinxcode{numpy.ndarray}} representing the stencils’ computational domain for the mass of water vapour
at the previous time level.

\item {} 
\sphinxstyleliteralstrong{old\_Qc} (\sphinxtitleref{array\_like}, optional) \textendash{} \sphinxhref{https://docs.scipy.org/doc/numpy-1.13.0/reference/generated/numpy.ndarray.html\#numpy.ndarray}{\sphinxcode{numpy.ndarray}} representing the stencils’ computational domain for the mass of cloud water
at the previous time level.

\item {} 
\sphinxstyleliteralstrong{old\_Qr} (\sphinxtitleref{array\_like}, optional) \textendash{} \sphinxhref{https://docs.scipy.org/doc/numpy-1.13.0/reference/generated/numpy.ndarray.html\#numpy.ndarray}{\sphinxcode{numpy.ndarray}} representing the stencils’ computational domain for the mass of precipitation water
at the previous time level.

\end{itemize}

\item[{Returns}] \leavevmode
\begin{itemize}
\item {} 
\sphinxstylestrong{out\_s} (\sphinxstyleemphasis{array\_like}) \textendash{} \sphinxhref{https://docs.scipy.org/doc/numpy-1.13.0/reference/generated/numpy.ndarray.html\#numpy.ndarray}{\sphinxcode{numpy.ndarray}} representing the stencils’ computational domain for the isentropic density
at the next time level.

\item {} 
\sphinxstylestrong{out\_U} (\sphinxstyleemphasis{array\_like}) \textendash{} \sphinxhref{https://docs.scipy.org/doc/numpy-1.13.0/reference/generated/numpy.ndarray.html\#numpy.ndarray}{\sphinxcode{numpy.ndarray}} representing the stencils’ computational domain for the \(x\)-momentum
at the next time level.

\item {} 
\sphinxstylestrong{out\_V} (\sphinxstyleemphasis{array\_like}) \textendash{} \sphinxhref{https://docs.scipy.org/doc/numpy-1.13.0/reference/generated/numpy.ndarray.html\#numpy.ndarray}{\sphinxcode{numpy.ndarray}} representing the stencils’ computational domain for the \(y\)-momentum
at the next time level.

\item {} 
\sphinxstylestrong{out\_Qv} (\sphinxtitleref{array\_like}, optional) \textendash{} \sphinxhref{https://docs.scipy.org/doc/numpy-1.13.0/reference/generated/numpy.ndarray.html\#numpy.ndarray}{\sphinxcode{numpy.ndarray}} representing the stencils’ computational domain for the mass of water vapour
at the next time level.

\item {} 
\sphinxstylestrong{out\_Qc} (\sphinxtitleref{array\_like}, optional) \textendash{} \sphinxhref{https://docs.scipy.org/doc/numpy-1.13.0/reference/generated/numpy.ndarray.html\#numpy.ndarray}{\sphinxcode{numpy.ndarray}} representing the stencils’ computational domain for the mass of cloud water
at the next time level.

\item {} 
\sphinxstylestrong{out\_Qr} (\sphinxtitleref{array\_like}, optional) \textendash{} \sphinxhref{https://docs.scipy.org/doc/numpy-1.13.0/reference/generated/numpy.ndarray.html\#numpy.ndarray}{\sphinxcode{numpy.ndarray}} representing the stencils’ computational domain for the mass of precipitation water
at the next time level.

\end{itemize}


\end{description}\end{quote}

\end{fulllineitems}


\end{fulllineitems}



\subsection{Wave absorber}
\label{\detokenize{api:wave-absorber}}\index{VerticalDamping (class in dycore.vertical\_damping)}

\begin{fulllineitems}
\phantomsection\label{\detokenize{api:dycore.vertical_damping.VerticalDamping}}\pysiglinewithargsret{\sphinxbfcode{class }\sphinxcode{dycore.vertical\_damping.}\sphinxbfcode{VerticalDamping}}{\emph{grid}, \emph{damp\_depth}, \emph{damp\_max}, \emph{backend}}{}
Abstract base class whose derived classes implement different vertical damping, i.e., wave absorbing, techniques.
\index{\_\_init\_\_() (dycore.vertical\_damping.VerticalDamping method)}

\begin{fulllineitems}
\phantomsection\label{\detokenize{api:dycore.vertical_damping.VerticalDamping.__init__}}\pysiglinewithargsret{\sphinxbfcode{\_\_init\_\_}}{\emph{grid}, \emph{damp\_depth}, \emph{damp\_max}, \emph{backend}}{}
Constructor.
\begin{quote}\begin{description}
\item[{Parameters}] \leavevmode\begin{itemize}
\item {} 
\sphinxstyleliteralstrong{grid} (\sphinxstyleliteralemphasis{obj}) \textendash{} The underlying grid, as an instance of {\hyperref[\detokenize{api:grids.grid_xyz.GridXYZ}]{\sphinxcrossref{\sphinxcode{GridXYZ}}}} or one of its derived classes.

\item {} 
\sphinxstyleliteralstrong{damp\_depth} (\sphinxstyleliteralemphasis{int}) \textendash{} Number of vertical layers in the damping region.

\item {} 
\sphinxstyleliteralstrong{damp\_max} (\sphinxstyleliteralemphasis{float}) \textendash{} Maximum value for the damping coefficient.

\item {} 
\sphinxstyleliteralstrong{backend} (\sphinxstyleliteralemphasis{obj}) \textendash{} \sphinxcode{gridtools.mode} specifying the backend for the GT4Py’s stencils implementing the dynamical core.

\end{itemize}

\end{description}\end{quote}

\end{fulllineitems}

\index{apply() (dycore.vertical\_damping.VerticalDamping method)}

\begin{fulllineitems}
\phantomsection\label{\detokenize{api:dycore.vertical_damping.VerticalDamping.apply}}\pysiglinewithargsret{\sphinxbfcode{apply}}{\emph{dt}, \emph{phi\_now}, \emph{phi\_new}, \emph{phi\_ref}}{}
Apply vertical damping to a generic field \(\phi\).
As this method is marked as abstract, its implementation is delegated to the derived classes.
\begin{quote}\begin{description}
\item[{Parameters}] \leavevmode\begin{itemize}
\item {} 
\sphinxstyleliteralstrong{dt} (\sphinxstyleliteralemphasis{obj}) \textendash{} \sphinxcode{datetime.timedelta} representing the time step.

\item {} 
\sphinxstyleliteralstrong{phi\_now} (\sphinxstyleliteralemphasis{array\_like}) \textendash{} \sphinxhref{https://docs.scipy.org/doc/numpy-1.13.0/reference/generated/numpy.ndarray.html\#numpy.ndarray}{\sphinxcode{numpy.ndarray}} representing the field \(\phi\) at the current time level.

\item {} 
\sphinxstyleliteralstrong{phi\_new} (\sphinxstyleliteralemphasis{array\_like}) \textendash{} \sphinxhref{https://docs.scipy.org/doc/numpy-1.13.0/reference/generated/numpy.ndarray.html\#numpy.ndarray}{\sphinxcode{numpy.ndarray}} representing the field \(\phi\) at the next time level, on
which the absorber will be applied.

\item {} 
\sphinxstyleliteralstrong{phi\_ref} (\sphinxstyleliteralemphasis{array\_like}) \textendash{} \sphinxhref{https://docs.scipy.org/doc/numpy-1.13.0/reference/generated/numpy.ndarray.html\#numpy.ndarray}{\sphinxcode{numpy.ndarray}} representing a reference value for \(\phi\).

\end{itemize}

\item[{Returns}] \leavevmode
\sphinxhref{https://docs.scipy.org/doc/numpy-1.13.0/reference/generated/numpy.ndarray.html\#numpy.ndarray}{\sphinxcode{numpy.ndarray}} representing the damped field \(\phi\).

\item[{Return type}] \leavevmode
array\_like

\end{description}\end{quote}

\end{fulllineitems}

\index{factory() (dycore.vertical\_damping.VerticalDamping static method)}

\begin{fulllineitems}
\phantomsection\label{\detokenize{api:dycore.vertical_damping.VerticalDamping.factory}}\pysiglinewithargsret{\sphinxbfcode{static }\sphinxbfcode{factory}}{\emph{damp\_type}, \emph{grid}, \emph{damp\_depth}, \emph{damp\_max}, \emph{backend}}{}
Static method which returns an instance of the derived class implementing the damping method
specified by \sphinxcode{damp\_type}.
\begin{quote}\begin{description}
\item[{Parameters}] \leavevmode\begin{itemize}
\item {} 
\sphinxstyleliteralstrong{damp\_type} (\sphinxstyleliteralemphasis{str}) \textendash{} 
String specifying the damper to implement. Either:
\begin{itemize}
\item {} 
’rayleigh’, for a Rayleigh damper.

\end{itemize}


\item {} 
\sphinxstyleliteralstrong{grid} (\sphinxstyleliteralemphasis{obj}) \textendash{} The underlying grid, as an instance of {\hyperref[\detokenize{api:grids.grid_xyz.GridXYZ}]{\sphinxcrossref{\sphinxcode{GridXYZ}}}} or one of its derived classes.

\item {} 
\sphinxstyleliteralstrong{damp\_depth} (\sphinxstyleliteralemphasis{int}) \textendash{} Number of vertical layers in the damping region. Default is 15.

\item {} 
\sphinxstyleliteralstrong{damp\_max} (\sphinxstyleliteralemphasis{float}) \textendash{} Maximum value for the damping coefficient. Default is 0.0002.

\item {} 
\sphinxstyleliteralstrong{backend} (\sphinxstyleliteralemphasis{obj}) \textendash{} \sphinxcode{gridtools.mode} specifying the backend for the GT4Py’s stencils implementing the dynamical core.

\end{itemize}

\item[{Returns}] \leavevmode
An instance of the derived class implementing the damping method specified by \sphinxcode{damp\_type}.

\item[{Return type}] \leavevmode
obj

\end{description}\end{quote}

\end{fulllineitems}


\end{fulllineitems}

\index{Rayleigh (class in dycore.vertical\_damping)}

\begin{fulllineitems}
\phantomsection\label{\detokenize{api:dycore.vertical_damping.Rayleigh}}\pysiglinewithargsret{\sphinxbfcode{class }\sphinxcode{dycore.vertical\_damping.}\sphinxbfcode{Rayleigh}}{\emph{grid}, \emph{damp\_depth}, \emph{damp\_max}, \emph{backend}}{}
This class inherits {\hyperref[\detokenize{api:dycore.vertical_damping.VerticalDamping}]{\sphinxcrossref{\sphinxcode{VerticalDamping}}}} to implement a Rayleigh absorber.
\index{\_\_init\_\_() (dycore.vertical\_damping.Rayleigh method)}

\begin{fulllineitems}
\phantomsection\label{\detokenize{api:dycore.vertical_damping.Rayleigh.__init__}}\pysiglinewithargsret{\sphinxbfcode{\_\_init\_\_}}{\emph{grid}, \emph{damp\_depth}, \emph{damp\_max}, \emph{backend}}{}
Constructor.
\begin{quote}\begin{description}
\item[{Parameters}] \leavevmode\begin{itemize}
\item {} 
\sphinxstyleliteralstrong{grid} (\sphinxstyleliteralemphasis{obj}) \textendash{} The underlying grid, as an instance of {\hyperref[\detokenize{api:grids.grid_xyz.GridXYZ}]{\sphinxcrossref{\sphinxcode{GridXYZ}}}} or one of its derived classes.

\item {} 
\sphinxstyleliteralstrong{damp\_depth} (\sphinxstyleliteralemphasis{int}) \textendash{} Number of vertical layers in the damping region.

\item {} 
\sphinxstyleliteralstrong{damp\_max} (\sphinxstyleliteralemphasis{float}) \textendash{} Maximum value for the damping coefficient.

\item {} 
\sphinxstyleliteralstrong{backend} (\sphinxstyleliteralemphasis{obj}) \textendash{} \sphinxcode{gridtools.mode} specifying the backend for the GT4Py’s stencils implementing the dynamical core.

\end{itemize}

\end{description}\end{quote}

\end{fulllineitems}

\index{\_defs\_stencil() (dycore.vertical\_damping.Rayleigh method)}

\begin{fulllineitems}
\phantomsection\label{\detokenize{api:dycore.vertical_damping.Rayleigh._defs_stencil}}\pysiglinewithargsret{\sphinxbfcode{\_defs\_stencil}}{\emph{dt}, \emph{phi\_now}, \emph{phi\_new}, \emph{phi\_ref}, \emph{R}}{}
The GT4Py’s stencil applying Rayleigh vertical damping.
\begin{quote}\begin{description}
\item[{Parameters}] \leavevmode\begin{itemize}
\item {} 
\sphinxstyleliteralstrong{dt} (\sphinxstyleliteralemphasis{obj}) \textendash{} \sphinxcode{gridtools.Global} representing the time step.

\item {} 
\sphinxstyleliteralstrong{phi\_now} (\sphinxstyleliteralemphasis{ibj}) \textendash{} \sphinxcode{gridtools.Equation} representing the field \(\phi\) at the current time level.

\item {} 
\sphinxstyleliteralstrong{phi\_new} (\sphinxstyleliteralemphasis{obj}) \textendash{} \sphinxcode{gridtools.Equation} representing the field \(\phi\) at the next time level, on
which the absorber will be applied.

\item {} 
\sphinxstyleliteralstrong{phi\_ref} (\sphinxstyleliteralemphasis{obj}) \textendash{} \sphinxcode{gridtools.Equation} representing a reference value for \(\phi\).

\end{itemize}

\item[{Returns}] \leavevmode
\sphinxcode{gridtools.Equation} representing the damped field \(\phi\).

\item[{Return type}] \leavevmode
obj

\end{description}\end{quote}

\end{fulllineitems}

\index{\_initialize\_stencil() (dycore.vertical\_damping.Rayleigh method)}

\begin{fulllineitems}
\phantomsection\label{\detokenize{api:dycore.vertical_damping.Rayleigh._initialize_stencil}}\pysiglinewithargsret{\sphinxbfcode{\_initialize\_stencil}}{\emph{phi\_now}}{}
Initialize the GT4Py’s stencil applying Rayleigh vertical damping.
\begin{quote}\begin{description}
\item[{Parameters}] \leavevmode
\sphinxstyleliteralstrong{phi\_now} (\sphinxstyleliteralemphasis{array\_like}) \textendash{} \sphinxhref{https://docs.scipy.org/doc/numpy-1.13.0/reference/generated/numpy.ndarray.html\#numpy.ndarray}{\sphinxcode{numpy.ndarray}} representing the field \(\phi\) at the current time level.

\end{description}\end{quote}

\end{fulllineitems}

\index{\_set\_inputs() (dycore.vertical\_damping.Rayleigh method)}

\begin{fulllineitems}
\phantomsection\label{\detokenize{api:dycore.vertical_damping.Rayleigh._set_inputs}}\pysiglinewithargsret{\sphinxbfcode{\_set\_inputs}}{\emph{dt}, \emph{phi\_now}, \emph{phi\_new}, \emph{phi\_ref}}{}
Update the attributes which stores the stencil’s input fields.
\begin{quote}\begin{description}
\item[{Parameters}] \leavevmode\begin{itemize}
\item {} 
\sphinxstyleliteralstrong{dt} (\sphinxstyleliteralemphasis{obj}) \textendash{} \sphinxcode{datetime.timedelta} representing the time step.

\item {} 
\sphinxstyleliteralstrong{phi\_now} (\sphinxstyleliteralemphasis{array\_like}) \textendash{} \sphinxhref{https://docs.scipy.org/doc/numpy-1.13.0/reference/generated/numpy.ndarray.html\#numpy.ndarray}{\sphinxcode{numpy.ndarray}} representing the field \(\phi\) at the current time level.

\item {} 
\sphinxstyleliteralstrong{phi\_new} (\sphinxstyleliteralemphasis{array\_like}) \textendash{} \sphinxhref{https://docs.scipy.org/doc/numpy-1.13.0/reference/generated/numpy.ndarray.html\#numpy.ndarray}{\sphinxcode{numpy.ndarray}} representing the field \(\phi\) at the next time level, on
which the absorber will be applied.

\item {} 
\sphinxstyleliteralstrong{phi\_ref} (\sphinxstyleliteralemphasis{array\_like}) \textendash{} \sphinxhref{https://docs.scipy.org/doc/numpy-1.13.0/reference/generated/numpy.ndarray.html\#numpy.ndarray}{\sphinxcode{numpy.ndarray}} representing a reference value for \(\phi\).

\end{itemize}

\end{description}\end{quote}

\end{fulllineitems}

\index{apply() (dycore.vertical\_damping.Rayleigh method)}

\begin{fulllineitems}
\phantomsection\label{\detokenize{api:dycore.vertical_damping.Rayleigh.apply}}\pysiglinewithargsret{\sphinxbfcode{apply}}{\emph{dt}, \emph{phi\_now}, \emph{phi\_new}, \emph{phi\_ref}}{}
Apply vertical damping to a generic field \(\phi\).
\begin{quote}\begin{description}
\item[{Parameters}] \leavevmode\begin{itemize}
\item {} 
\sphinxstyleliteralstrong{dt} (\sphinxstyleliteralemphasis{obj}) \textendash{} \sphinxcode{datetime.timedelta} representing the time step.

\item {} 
\sphinxstyleliteralstrong{phi\_now} (\sphinxstyleliteralemphasis{array\_like}) \textendash{} \sphinxhref{https://docs.scipy.org/doc/numpy-1.13.0/reference/generated/numpy.ndarray.html\#numpy.ndarray}{\sphinxcode{numpy.ndarray}} representing the field \(\phi\) at the current time level.

\item {} 
\sphinxstyleliteralstrong{phi\_new} (\sphinxstyleliteralemphasis{array\_like}) \textendash{} \sphinxhref{https://docs.scipy.org/doc/numpy-1.13.0/reference/generated/numpy.ndarray.html\#numpy.ndarray}{\sphinxcode{numpy.ndarray}} representing the field \(\phi\) at the next time level, on
which the absorber will be applied.

\item {} 
\sphinxstyleliteralstrong{phi\_ref} (\sphinxstyleliteralemphasis{array\_like}) \textendash{} \sphinxhref{https://docs.scipy.org/doc/numpy-1.13.0/reference/generated/numpy.ndarray.html\#numpy.ndarray}{\sphinxcode{numpy.ndarray}} representing a reference value for \(\phi\).

\end{itemize}

\item[{Returns}] \leavevmode
\sphinxhref{https://docs.scipy.org/doc/numpy-1.13.0/reference/generated/numpy.ndarray.html\#numpy.ndarray}{\sphinxcode{numpy.ndarray}} representing the damped field \(\phi\).

\item[{Return type}] \leavevmode
array\_like

\end{description}\end{quote}

\end{fulllineitems}


\end{fulllineitems}



\section{Grids}
\label{\detokenize{api:grids}}

\subsection{Two-dimensional grids}
\label{\detokenize{api:two-dimensional-grids}}\index{GridXY (class in grids.grid\_xy)}

\begin{fulllineitems}
\phantomsection\label{\detokenize{api:grids.grid_xy.GridXY}}\pysiglinewithargsret{\sphinxbfcode{class }\sphinxcode{grids.grid\_xy.}\sphinxbfcode{GridXY}}{\emph{domain\_x}, \emph{nx}, \emph{domain\_y}, \emph{ny}, \emph{units\_x='degrees\_east'}, \emph{dims\_x='longitude'}, \emph{units\_y='degrees\_north'}, \emph{dims\_y='latitude'}}{}
Rectangular and regular two-dimensional grid embedded in a reference system whose coordinates are,
in the order, \(x\) and \(y\). No assumption is made on the nature of the coordinates. For
instance, \(x\) may be the longitude, in which case \(x \equiv \lambda\), and \(y\) may
be the latitude, in which case \(y \equiv \phi\).
\begin{quote}\begin{description}
\item[{Variables}] \leavevmode\begin{itemize}
\item {} 
{\hyperref[\detokenize{api:grids.grid_xyz.GridXYZ.x}]{\sphinxcrossref{\sphinxstyleliteralstrong{x}}}} (\sphinxstyleliteralemphasis{obj}) \textendash{} {\hyperref[\detokenize{api:grids.axis.Axis}]{\sphinxcrossref{\sphinxcode{Axis}}}} representing the \(x\) main levels.

\item {} 
{\hyperref[\detokenize{api:grids.grid_xyz.GridXYZ.x_half_levels}]{\sphinxcrossref{\sphinxstyleliteralstrong{x\_half\_levels}}}} (\sphinxstyleliteralemphasis{obj}) \textendash{} {\hyperref[\detokenize{api:grids.axis.Axis}]{\sphinxcrossref{\sphinxcode{Axis}}}} representing the \(x\) half levels.

\item {} 
{\hyperref[\detokenize{api:grids.grid_xyz.GridXYZ.nx}]{\sphinxcrossref{\sphinxstyleliteralstrong{nx}}}} (\sphinxstyleliteralemphasis{int}) \textendash{} Number of grid points along \(x\).

\item {} 
{\hyperref[\detokenize{api:grids.grid_xyz.GridXYZ.dx}]{\sphinxcrossref{\sphinxstyleliteralstrong{dx}}}} (\sphinxstyleliteralemphasis{float}) \textendash{} The \(x\)-spacing.

\item {} 
{\hyperref[\detokenize{api:grids.grid_xyz.GridXYZ.y}]{\sphinxcrossref{\sphinxstyleliteralstrong{y}}}} (\sphinxstyleliteralemphasis{obj}) \textendash{} {\hyperref[\detokenize{api:grids.axis.Axis}]{\sphinxcrossref{\sphinxcode{Axis}}}} representing the \(y\) main levels.

\item {} 
{\hyperref[\detokenize{api:grids.grid_xyz.GridXYZ.y_half_levels}]{\sphinxcrossref{\sphinxstyleliteralstrong{y\_half\_levels}}}} (\sphinxstyleliteralemphasis{obj}) \textendash{} {\hyperref[\detokenize{api:grids.axis.Axis}]{\sphinxcrossref{\sphinxcode{Axis}}}} representing the \(y\) half levels.

\item {} 
{\hyperref[\detokenize{api:grids.grid_xyz.GridXYZ.ny}]{\sphinxcrossref{\sphinxstyleliteralstrong{ny}}}} (\sphinxstyleliteralemphasis{int}) \textendash{} Number of grid points along \(y\).

\item {} 
{\hyperref[\detokenize{api:grids.grid_xyz.GridXYZ.dy}]{\sphinxcrossref{\sphinxstyleliteralstrong{dy}}}} (\sphinxstyleliteralemphasis{float}) \textendash{} The \(y\)-spacing.

\end{itemize}

\end{description}\end{quote}
\index{\_\_init\_\_() (grids.grid\_xy.GridXY method)}

\begin{fulllineitems}
\phantomsection\label{\detokenize{api:grids.grid_xy.GridXY.__init__}}\pysiglinewithargsret{\sphinxbfcode{\_\_init\_\_}}{\emph{domain\_x}, \emph{nx}, \emph{domain\_y}, \emph{ny}, \emph{units\_x='degrees\_east'}, \emph{dims\_x='longitude'}, \emph{units\_y='degrees\_north'}, \emph{dims\_y='latitude'}}{}
Constructor.
\begin{quote}\begin{description}
\item[{Parameters}] \leavevmode\begin{itemize}
\item {} 
\sphinxstyleliteralstrong{domain\_x} (\sphinxstyleliteralemphasis{tuple}) \textendash{} Tuple in the form \((x_{start}, ~ x_{stop})\).

\item {} 
\sphinxstyleliteralstrong{nx} (\sphinxstyleliteralemphasis{int}) \textendash{} Number of grid points along \(x\).

\item {} 
\sphinxstyleliteralstrong{domain\_y} (\sphinxstyleliteralemphasis{tuple}) \textendash{} Tuple in the form \((y_{start}, ~ y_{stop})\).

\item {} 
\sphinxstyleliteralstrong{ny} (\sphinxstyleliteralemphasis{int}) \textendash{} Number of grid points along \(y\).

\item {} 
\sphinxstyleliteralstrong{units\_x} (\sphinxtitleref{str}, optional) \textendash{} Units for the \(x\)-coordinate.

\item {} 
\sphinxstyleliteralstrong{dims\_x} (\sphinxtitleref{str}, optional) \textendash{} Label for the \(x\)-coordinate.

\item {} 
\sphinxstyleliteralstrong{units\_y} (\sphinxtitleref{str}, optional) \textendash{} Units for the \(y\)-coordinate.

\item {} 
\sphinxstyleliteralstrong{dims\_y} (\sphinxtitleref{str}, optional) \textendash{} Label for the \(y\)-coordinate.

\end{itemize}

\end{description}\end{quote}

\begin{sphinxadmonition}{note}{Note:}
Axes labels should use the \sphinxhref{http://cfconventions.org}{CF Conventions}.
\end{sphinxadmonition}

\end{fulllineitems}


\end{fulllineitems}

\index{GridXZ (class in grids.grid\_xz)}

\begin{fulllineitems}
\phantomsection\label{\detokenize{api:grids.grid_xz.GridXZ}}\pysiglinewithargsret{\sphinxbfcode{class }\sphinxcode{grids.grid\_xz.}\sphinxbfcode{GridXZ}}{\emph{domain\_x}, \emph{nx}, \emph{domain\_z}, \emph{nz}, \emph{units\_x='m'}, \emph{dims\_x='x'}, \emph{units\_z='m'}, \emph{dims\_z='z'}, \emph{z\_interface=None}, \emph{topo\_type='terrain\_flat'}, \emph{topo\_time=datetime.timedelta(0)}, \emph{**kwargs}}{}
Rectangular and regular two-dimensional grid embedded in a reference system whose coordinates are
\begin{itemize}
\item {} 
the horizontal coordinate \(x\);

\item {} 
the vertical (terrain-following) coordinate \(z\).

\end{itemize}

The vertical coordinate \(z\) may be formulated to define a hybrid terrain-following coordinate system
with terrain-following coordinate lines between the surface terrain-height and \(z = z_F\), where
\(z\)-coordinate lines change back to flat horizontal lines. However, no assumption is made on the actual
nature of \(z\) which may be either pressure-based or height-based.
\begin{quote}\begin{description}
\item[{Variables}] \leavevmode\begin{itemize}
\item {} 
{\hyperref[\detokenize{api:grids.grid_xyz.GridXYZ.x}]{\sphinxcrossref{\sphinxstyleliteralstrong{x}}}} (\sphinxstyleliteralemphasis{obj}) \textendash{} {\hyperref[\detokenize{api:grids.axis.Axis}]{\sphinxcrossref{\sphinxcode{Axis}}}} representing the \(x\)-axis.

\item {} 
{\hyperref[\detokenize{api:grids.grid_xyz.GridXYZ.nx}]{\sphinxcrossref{\sphinxstyleliteralstrong{nx}}}} (\sphinxstyleliteralemphasis{int}) \textendash{} Number of grid points along \(x\).

\item {} 
{\hyperref[\detokenize{api:grids.grid_xyz.GridXYZ.dx}]{\sphinxcrossref{\sphinxstyleliteralstrong{dx}}}} (\sphinxstyleliteralemphasis{float}) \textendash{} The \(x\)-spacing.

\item {} 
\sphinxstyleliteralstrong{z} (\sphinxstyleliteralemphasis{obj}) \textendash{} {\hyperref[\detokenize{api:grids.axis.Axis}]{\sphinxcrossref{\sphinxcode{Axis}}}} representing the \(z\)-main levels.

\item {} 
\sphinxstyleliteralstrong{z\_half\_levels} (\sphinxstyleliteralemphasis{obj}) \textendash{} {\hyperref[\detokenize{api:grids.axis.Axis}]{\sphinxcrossref{\sphinxcode{Axis}}}} representing the \(z\)-half levels.

\item {} 
\sphinxstyleliteralstrong{nz} (\sphinxstyleliteralemphasis{int}) \textendash{} Number of vertical main levels.

\item {} 
\sphinxstyleliteralstrong{dz} (\sphinxstyleliteralemphasis{float}) \textendash{} The \(z\)-spacing.

\item {} 
\sphinxstyleliteralstrong{z\_interface} (\sphinxstyleliteralemphasis{float}) \textendash{} The interface coordinate \(z_F\).

\end{itemize}

\end{description}\end{quote}

\begin{sphinxadmonition}{note}{Note:}
For the sake of compliancy with the \sphinxhref{http://www.cosmo-model.org}{COSMO model}, the vertical grid points are
ordered from the top of the domain to the surface.
\end{sphinxadmonition}
\index{\_\_init\_\_() (grids.grid\_xz.GridXZ method)}

\begin{fulllineitems}
\phantomsection\label{\detokenize{api:grids.grid_xz.GridXZ.__init__}}\pysiglinewithargsret{\sphinxbfcode{\_\_init\_\_}}{\emph{domain\_x}, \emph{nx}, \emph{domain\_z}, \emph{nz}, \emph{units\_x='m'}, \emph{dims\_x='x'}, \emph{units\_z='m'}, \emph{dims\_z='z'}, \emph{z\_interface=None}, \emph{topo\_type='terrain\_flat'}, \emph{topo\_time=datetime.timedelta(0)}, \emph{**kwargs}}{}
Constructor.
\begin{quote}\begin{description}
\item[{Parameters}] \leavevmode\begin{itemize}
\item {} 
\sphinxstyleliteralstrong{domain\_x} (\sphinxstyleliteralemphasis{tuple}) \textendash{} Tuple in the form \((x_{left}, ~ x_{right})\).

\item {} 
\sphinxstyleliteralstrong{nx} (\sphinxstyleliteralemphasis{int}) \textendash{} Number of grid points in the \(x\)-direction.

\item {} 
\sphinxstyleliteralstrong{domain\_z} (\sphinxstyleliteralemphasis{tuple}) \textendash{} Tuple in the form \((z_{top}, ~ z_{surface})\).

\item {} 
\sphinxstyleliteralstrong{nz} (\sphinxstyleliteralemphasis{int}) \textendash{} Number of vertical main levels.

\item {} 
\sphinxstyleliteralstrong{units\_x} (\sphinxtitleref{str}, optional) \textendash{} 
Units for the \(x\)-coordinate. Must be compliant with the \sphinxhref{http://cfconventions.org}{CF Conventions}
(see also {\hyperref[\detokenize{api:grids.axis.Axis.__init__}]{\sphinxcrossref{\sphinxcode{grids.axis.Axis.\_\_init\_\_()}}}}).


\item {} 
\sphinxstyleliteralstrong{dims\_x} (\sphinxtitleref{str}, optional) \textendash{} Label for the \(x\)-coordinate.

\item {} 
\sphinxstyleliteralstrong{units\_z} (\sphinxtitleref{str}, optional) \textendash{} 
Units for the \(z\)-coordinate. Must be compliant with the \sphinxhref{http://cfconventions.org}{CF Conventions}
(see also {\hyperref[\detokenize{api:grids.axis.Axis.__init__}]{\sphinxcrossref{\sphinxcode{grids.axis.Axis.\_\_init\_\_()}}}}).


\item {} 
\sphinxstyleliteralstrong{dims\_z} (\sphinxtitleref{str}, optional) \textendash{} Label for the \(z\)-coordinate.

\item {} 
\sphinxstyleliteralstrong{z\_interface} (\sphinxtitleref{float}, optional) \textendash{} Interface value \(z_F\). If not specified, it is assumed that \(z_F = z_T\), with \(z_T\) the
value of \(z\) at the top of the domain. In other words, a fully terrain-following coordinate system is
supposed.

\item {} 
\sphinxstyleliteralstrong{topo\_type} (\sphinxtitleref{str}, optional) \textendash{} Topography type. See {\hyperref[\detokenize{api:module-grids.topography}]{\sphinxcrossref{\sphinxcode{grids.topography}}}} for further details.

\item {} 
\sphinxstyleliteralstrong{topo\_time} (\sphinxtitleref{obj}, optional) \textendash{} \sphinxcode{datetime.timedelta} representing the simulation time after which the topography should stop
increasing. Default is 0, corresponding to a time-invariant terrain surface-height. See
{\hyperref[\detokenize{api:module-grids.topography}]{\sphinxcrossref{\sphinxcode{grids.topography}}}} for further details.

\end{itemize}

\item[{Keyword Arguments}] \leavevmode
\sphinxstyleliteralstrong{kwargs} \textendash{} Keyword arguments to be forwarded to the constructor of {\hyperref[\detokenize{api:grids.topography.Topography1d}]{\sphinxcrossref{\sphinxcode{Topography1d}}}}.

\end{description}\end{quote}

\end{fulllineitems}

\index{topography\_height (grids.grid\_xz.GridXZ attribute)}

\begin{fulllineitems}
\phantomsection\label{\detokenize{api:grids.grid_xz.GridXZ.topography_height}}\pysigline{\sphinxbfcode{topography\_height}}
Get the topography (i.e., terrain-surface) height.
\begin{quote}\begin{description}
\item[{Returns}] \leavevmode
One-dimensional \sphinxhref{https://docs.scipy.org/doc/numpy-1.13.0/reference/generated/numpy.ndarray.html\#numpy.ndarray}{\sphinxcode{numpy.ndarray}} representing the topography height.

\item[{Return type}] \leavevmode
array\_like

\end{description}\end{quote}

\end{fulllineitems}

\index{update\_topography() (grids.grid\_xz.GridXZ method)}

\begin{fulllineitems}
\phantomsection\label{\detokenize{api:grids.grid_xz.GridXZ.update_topography}}\pysiglinewithargsret{\sphinxbfcode{update\_topography}}{\emph{time}}{}
Update the (time-dependent) topography.
\begin{quote}\begin{description}
\item[{Parameters}] \leavevmode
\sphinxstyleliteralstrong{time} (\sphinxstyleliteralemphasis{obj}) \textendash{} \sphinxcode{datetime.timedelta} representing the elapsed simulation time.

\end{description}\end{quote}

\end{fulllineitems}


\end{fulllineitems}

\index{Sigma2d (class in grids.sigma)}

\begin{fulllineitems}
\phantomsection\label{\detokenize{api:grids.sigma.Sigma2d}}\pysiglinewithargsret{\sphinxbfcode{class }\sphinxcode{grids.sigma.}\sphinxbfcode{Sigma2d}}{\emph{domain\_x}, \emph{nx}, \emph{domain\_z}, \emph{nz}, \emph{units\_x='m'}, \emph{dims\_x='x'}, \emph{z\_interface=None}, \emph{topo\_type='flat\_terrain'}, \emph{topo\_time=datetime.timedelta(0)}, \emph{**kwargs}}{}
This class inherits {\hyperref[\detokenize{api:grids.grid_xz.GridXZ}]{\sphinxcrossref{\sphinxcode{GridXZ}}}} to represent a rectangular and regular
two-dimensional grid embedded in a reference system whose coordinates are
\begin{itemize}
\item {} 
the horizontal coordinate \(x\);

\item {} 
the pressure-based terrain-following coordinate \(\sigma = p / p_{SL}\),                  where \(p\) is the pressure and \(p_{SL}\) the pressure at the sea level.

\end{itemize}

The vertical coordinate \(\sigma\) may be formulated to define a hybrid terrain-following coordinate system
with terrain-following coordinate lines between the surface terrain-height and \(\sigma = \sigma_F\), where
\(\sigma\)-coordinate lines change back to flat horizontal lines.
\begin{quote}\begin{description}
\item[{Variables}] \leavevmode\begin{itemize}
\item {} 
{\hyperref[\detokenize{api:grids.grid_xyz.GridXYZ.x}]{\sphinxcrossref{\sphinxstyleliteralstrong{x}}}} (\sphinxstyleliteralemphasis{obj}) \textendash{} {\hyperref[\detokenize{api:grids.axis.Axis}]{\sphinxcrossref{\sphinxcode{Axis}}}} representing the \(x\)-axis.

\item {} 
{\hyperref[\detokenize{api:grids.grid_xyz.GridXYZ.nx}]{\sphinxcrossref{\sphinxstyleliteralstrong{nx}}}} (\sphinxstyleliteralemphasis{int}) \textendash{} Number of grid points along \(x\).

\item {} 
{\hyperref[\detokenize{api:grids.grid_xyz.GridXYZ.dx}]{\sphinxcrossref{\sphinxstyleliteralstrong{dx}}}} (\sphinxstyleliteralemphasis{float}) \textendash{} The \(x\)-spacing.

\item {} 
\sphinxstyleliteralstrong{z} (\sphinxstyleliteralemphasis{obj}) \textendash{} {\hyperref[\detokenize{api:grids.axis.Axis}]{\sphinxcrossref{\sphinxcode{Axis}}}} representing the \(\sigma\)-main levels.

\item {} 
\sphinxstyleliteralstrong{z\_half\_levels} (\sphinxstyleliteralemphasis{obj}) \textendash{} {\hyperref[\detokenize{api:grids.axis.Axis}]{\sphinxcrossref{\sphinxcode{Axis}}}} representing the \(\sigma\)-half levels.

\item {} 
\sphinxstyleliteralstrong{nz} (\sphinxstyleliteralemphasis{int}) \textendash{} Number of vertical main levels.

\item {} 
\sphinxstyleliteralstrong{dz} (\sphinxstyleliteralemphasis{float}) \textendash{} The \(\sigma\)-spacing.

\item {} 
\sphinxstyleliteralstrong{z\_interface} (\sphinxstyleliteralemphasis{float}) \textendash{} The interface coordinate \(\sigma_F\).

\item {} 
\sphinxstyleliteralstrong{height} (\sphinxstyleliteralemphasis{obj}) \textendash{} \sphinxhref{http://xarray.pydata.org/en/stable/generated/xarray.DataArray.html\#xarray.DataArray}{\sphinxcode{xarray.DataArray}} representing the geometric height of the main levels.

\item {} 
\sphinxstyleliteralstrong{height\_half\_levels} (\sphinxstyleliteralemphasis{obj}) \textendash{} \sphinxhref{http://xarray.pydata.org/en/stable/generated/xarray.DataArray.html\#xarray.DataArray}{\sphinxcode{xarray.DataArray}} representing the geometric height of the half levels.

\item {} 
\sphinxstyleliteralstrong{height\_interface} (\sphinxstyleliteralemphasis{float}) \textendash{} Geometric height corresponding to \(\sigma = \sigma_F\).

\item {} 
\sphinxstyleliteralstrong{reference\_pressure} (\sphinxstyleliteralemphasis{obj}) \textendash{} \sphinxhref{http://xarray.pydata.org/en/stable/generated/xarray.DataArray.html\#xarray.DataArray}{\sphinxcode{xarray.DataArray}} representing the reference pressure at the main levels.

\item {} 
\sphinxstyleliteralstrong{reference\_pressure\_half\_levels} (\sphinxstyleliteralemphasis{obj}) \textendash{} \sphinxhref{http://xarray.pydata.org/en/stable/generated/xarray.DataArray.html\#xarray.DataArray}{\sphinxcode{xarray.DataArray}} representing the reference pressure at the half levels.

\end{itemize}

\end{description}\end{quote}
\index{\_\_init\_\_() (grids.sigma.Sigma2d method)}

\begin{fulllineitems}
\phantomsection\label{\detokenize{api:grids.sigma.Sigma2d.__init__}}\pysiglinewithargsret{\sphinxbfcode{\_\_init\_\_}}{\emph{domain\_x}, \emph{nx}, \emph{domain\_z}, \emph{nz}, \emph{units\_x='m'}, \emph{dims\_x='x'}, \emph{z\_interface=None}, \emph{topo\_type='flat\_terrain'}, \emph{topo\_time=datetime.timedelta(0)}, \emph{**kwargs}}{}
Constructor.
\begin{quote}\begin{description}
\item[{Parameters}] \leavevmode\begin{itemize}
\item {} 
\sphinxstyleliteralstrong{domain\_x} (\sphinxstyleliteralemphasis{tuple}) \textendash{} Tuple in the form \((x_{left}, ~ x_{right})\).

\item {} 
\sphinxstyleliteralstrong{nx} (\sphinxstyleliteralemphasis{int}) \textendash{} Number of grid points in the \(x\)-direction.

\item {} 
\sphinxstyleliteralstrong{domain\_z} (\sphinxstyleliteralemphasis{tuple}) \textendash{} Tuple in the form \((\sigma_{top}, ~ \sigma_{surface})\).

\item {} 
\sphinxstyleliteralstrong{nz} (\sphinxstyleliteralemphasis{int}) \textendash{} Number of vertical main levels.

\item {} 
\sphinxstyleliteralstrong{units\_x} (\sphinxtitleref{str}, optional) \textendash{} 
Units for the \(x\)-coordinate. Must be compliant with the \sphinxhref{http://cfconventions.org}{CF Conventions}
(see also {\hyperref[\detokenize{api:grids.axis.Axis.__init__}]{\sphinxcrossref{\sphinxcode{grids.axis.Axis.\_\_init\_\_()}}}}).


\item {} 
\sphinxstyleliteralstrong{dims\_x} (\sphinxtitleref{str}, optional) \textendash{} Label for the \(x\)-coordinate.

\item {} 
\sphinxstyleliteralstrong{z\_interface} (\sphinxtitleref{float}, optional) \textendash{} Interface value \(\sigma_F\). If not specified, it is assumed that \(\sigma_F = \sigma_T\),
with \(\sigma_T\) the value of \(\sigma\) at the top of the domain. In other words, a fully
terrain-following coordinate system is supposed.

\item {} 
\sphinxstyleliteralstrong{topo\_type} (\sphinxtitleref{str}, optional) \textendash{} Topography type. Default is ‘flat\_terrain’. See {\hyperref[\detokenize{api:module-grids.topography}]{\sphinxcrossref{\sphinxcode{grids.topography}}}} for further details.

\item {} 
\sphinxstyleliteralstrong{topo\_time} (\sphinxtitleref{obj}, optional) \textendash{} \sphinxcode{datetime.timedelta} representing the simulation time after which the topography should stop
increasing. Default is 0, corresponding to a time-invariant terrain surface-height. See {\hyperref[\detokenize{api:module-grids.topography}]{\sphinxcrossref{\sphinxcode{grids.topography}}}}
for further details.

\end{itemize}

\item[{Keyword Arguments}] \leavevmode
\sphinxstyleliteralstrong{**kwargs} \textendash{} Keyword arguments to be forwarded to the constructor of {\hyperref[\detokenize{api:grids.topography.Topography1d}]{\sphinxcrossref{\sphinxcode{Topography1d}}}}.

\end{description}\end{quote}

\end{fulllineitems}

\index{\_update\_metric\_terms() (grids.sigma.Sigma2d method)}

\begin{fulllineitems}
\phantomsection\label{\detokenize{api:grids.sigma.Sigma2d._update_metric_terms}}\pysiglinewithargsret{\sphinxbfcode{\_update\_metric\_terms}}{}{}
Update the class by computing the metric terms, i.e., the geometric height and the reference pressure,
at both half and main levels. In doing this, a logarithmic vertical profile of reference pressure is assumed.
This method should be called every time the topography is updated or changed.

\end{fulllineitems}

\index{plot() (grids.sigma.Sigma2d method)}

\begin{fulllineitems}
\phantomsection\label{\detokenize{api:grids.sigma.Sigma2d.plot}}\pysiglinewithargsret{\sphinxbfcode{plot}}{\emph{**kwargs}}{}
Plot the grid half levels using \sphinxhref{https://matplotlib.org/2.1.1/api/\_as\_gen/matplotlib.pyplot.html\#module-matplotlib.pyplot}{\sphinxcode{matplotlib.pyplot}}’s utilities.
\begin{quote}\begin{description}
\item[{Keyword Arguments}] \leavevmode
\sphinxstyleliteralstrong{**kwargs} \textendash{} Keyword arguments to be forwarded to \sphinxhref{https://matplotlib.org/2.1.1/api/\_as\_gen/matplotlib.pyplot.subplots.html\#matplotlib.pyplot.subplots}{\sphinxcode{matplotlib.pyplot.subplots()}}.

\end{description}\end{quote}

\begin{sphinxadmonition}{note}{Note:}
For the sake of compliancy with the notation employed by \sphinxhref{http://www.cosmo-model.org}{COSMO},
the vertical geometric height is denoted by \(z\).
\end{sphinxadmonition}

\end{fulllineitems}

\index{update\_topography() (grids.sigma.Sigma2d method)}

\begin{fulllineitems}
\phantomsection\label{\detokenize{api:grids.sigma.Sigma2d.update_topography}}\pysiglinewithargsret{\sphinxbfcode{update\_topography}}{\emph{time}}{}
Update the (time-dependent) topography. In turn, the metric terms are re-computed.
\begin{quote}\begin{description}
\item[{Parameters}] \leavevmode
\sphinxstyleliteralstrong{time} (\sphinxstyleliteralemphasis{obj}) \textendash{} \sphinxcode{datetime.timedelta} representing the elapsed simulation time.

\end{description}\end{quote}

\end{fulllineitems}


\end{fulllineitems}

\index{GalChen2d (class in grids.gal\_chen)}

\begin{fulllineitems}
\phantomsection\label{\detokenize{api:grids.gal_chen.GalChen2d}}\pysiglinewithargsret{\sphinxbfcode{class }\sphinxcode{grids.gal\_chen.}\sphinxbfcode{GalChen2d}}{\emph{domain\_x}, \emph{nx}, \emph{domain\_z}, \emph{nz}, \emph{units\_x='m'}, \emph{dims\_x='x'}, \emph{z\_interface=None}, \emph{topo\_type='flat\_terrain'}, \emph{topo\_time=datetime.timedelta(0)}, \emph{**kwargs}}{}
This class inherits \sphinxcode{GridXZ} to represent a rectangular and regular two-dimensional
grid embedded in a reference system whose coordinates are
\begin{itemize}
\item {} 
the horizontal coordinate \(x\);

\item {} 
the height-based Gal-Chen terrain-following coordinate \(\mu\).

\end{itemize}

The vertical coordinate \(\mu\) may be formulated to define a hybrid terrain-following coordinate system
with terrain-following coordinate lines between the surface terrain-height and \(\mu = \mu_F\), where
\(\mu\)-coordinate lines change back to flat horizontal lines.
\begin{quote}\begin{description}
\item[{Variables}] \leavevmode\begin{itemize}
\item {} 
{\hyperref[\detokenize{api:grids.grid_xyz.GridXYZ.x}]{\sphinxcrossref{\sphinxstyleliteralstrong{x}}}} (\sphinxstyleliteralemphasis{obj}) \textendash{} {\hyperref[\detokenize{api:grids.axis.Axis}]{\sphinxcrossref{\sphinxcode{Axis}}}} object representing the \(x\)-axis.

\item {} 
{\hyperref[\detokenize{api:grids.grid_xyz.GridXYZ.nx}]{\sphinxcrossref{\sphinxstyleliteralstrong{nx}}}} (\sphinxstyleliteralemphasis{int}) \textendash{} Number of grid points along \(x\).

\item {} 
{\hyperref[\detokenize{api:grids.grid_xyz.GridXYZ.dx}]{\sphinxcrossref{\sphinxstyleliteralstrong{dx}}}} (\sphinxstyleliteralemphasis{float}) \textendash{} The \(x\)-spacing.

\item {} 
\sphinxstyleliteralstrong{z} (\sphinxstyleliteralemphasis{obj}) \textendash{} {\hyperref[\detokenize{api:grids.axis.Axis}]{\sphinxcrossref{\sphinxcode{Axis}}}} representing the \(\mu\)-main levels.

\item {} 
\sphinxstyleliteralstrong{z\_half\_levels} (\sphinxstyleliteralemphasis{obj}) \textendash{} {\hyperref[\detokenize{api:grids.axis.Axis}]{\sphinxcrossref{\sphinxcode{Axis}}}} representing the \(\mu\)-half levels.

\item {} 
\sphinxstyleliteralstrong{nz} (\sphinxstyleliteralemphasis{int}) \textendash{} Number of vertical main levels.

\item {} 
\sphinxstyleliteralstrong{dz} (\sphinxstyleliteralemphasis{float}) \textendash{} The \(\mu\)-spacing.

\item {} 
\sphinxstyleliteralstrong{z\_interface} (\sphinxstyleliteralemphasis{float}) \textendash{} The interface coordinate \(\mu_F\).

\item {} 
\sphinxstyleliteralstrong{height} (\sphinxstyleliteralemphasis{obj}) \textendash{} \sphinxhref{http://xarray.pydata.org/en/stable/generated/xarray.DataArray.html\#xarray.DataArray}{\sphinxcode{xarray.DataArray}} representing the geometric height of the main levels.

\item {} 
\sphinxstyleliteralstrong{height\_half\_levels} (\sphinxstyleliteralemphasis{obj}) \textendash{} \sphinxhref{http://xarray.pydata.org/en/stable/generated/xarray.DataArray.html\#xarray.DataArray}{\sphinxcode{xarray.DataArray}} representing the geometric height of the half levels.

\item {} 
\sphinxstyleliteralstrong{height\_interface} (\sphinxstyleliteralemphasis{float}) \textendash{} Geometric height corresponding to \(\mu = \mu_F\).

\item {} 
\sphinxstyleliteralstrong{reference\_pressure} (\sphinxstyleliteralemphasis{obj}) \textendash{} \sphinxhref{http://xarray.pydata.org/en/stable/generated/xarray.DataArray.html\#xarray.DataArray}{\sphinxcode{xarray.DataArray}} representing the reference pressure at the main levels.

\item {} 
\sphinxstyleliteralstrong{reference\_pressure\_half\_levels} (\sphinxstyleliteralemphasis{obj}) \textendash{} \sphinxhref{http://xarray.pydata.org/en/stable/generated/xarray.DataArray.html\#xarray.DataArray}{\sphinxcode{xarray.DataArray}} representing the reference pressure at the half levels.

\end{itemize}

\end{description}\end{quote}
\index{\_\_init\_\_() (grids.gal\_chen.GalChen2d method)}

\begin{fulllineitems}
\phantomsection\label{\detokenize{api:grids.gal_chen.GalChen2d.__init__}}\pysiglinewithargsret{\sphinxbfcode{\_\_init\_\_}}{\emph{domain\_x}, \emph{nx}, \emph{domain\_z}, \emph{nz}, \emph{units\_x='m'}, \emph{dims\_x='x'}, \emph{z\_interface=None}, \emph{topo\_type='flat\_terrain'}, \emph{topo\_time=datetime.timedelta(0)}, \emph{**kwargs}}{}
Constructor.
\begin{quote}\begin{description}
\item[{Parameters}] \leavevmode\begin{itemize}
\item {} 
\sphinxstyleliteralstrong{domain\_x} (\sphinxstyleliteralemphasis{tuple}) \textendash{} Tuple in the form \((x_{left}, ~ x_{right})\).

\item {} 
\sphinxstyleliteralstrong{nx} (\sphinxstyleliteralemphasis{int}) \textendash{} Number of grid points in the \(x\)-direction.

\item {} 
\sphinxstyleliteralstrong{domain\_z} (\sphinxstyleliteralemphasis{tuple}) \textendash{} Tuple in the form \((\mu_{top}, ~ \mu_{surface})\).

\item {} 
\sphinxstyleliteralstrong{nz} (\sphinxstyleliteralemphasis{int}) \textendash{} Number of vertical main levels.

\item {} 
\sphinxstyleliteralstrong{units\_x} (\sphinxtitleref{str}, optional) \textendash{} 
Units for the \(x\)-coordinate. Must be compliant with the \sphinxhref{http://cfconventions.org}{CF Conventions}
(see also {\hyperref[\detokenize{api:grids.axis.Axis.__init__}]{\sphinxcrossref{\sphinxcode{grids.axis.Axis.\_\_init\_\_()}}}}).


\item {} 
\sphinxstyleliteralstrong{dims\_x} (\sphinxtitleref{str}, optional) \textendash{} Label for the \(x\)-coordinate.

\item {} 
\sphinxstyleliteralstrong{z\_interface} (\sphinxtitleref{float}, optional) \textendash{} Interface value \(\mu_F\). If not specified, it is assumed that \(\mu_F = \mu_T\), with \(\mu_T\)
the value of \(\mu\) at the top of the domain. In other words, a fully terrain-following coordinate system
is supposed.

\item {} 
\sphinxstyleliteralstrong{topo\_type} (\sphinxtitleref{str}, optional) \textendash{} Topography type. Default is ‘flat\_terrain’. See {\hyperref[\detokenize{api:module-grids.topography}]{\sphinxcrossref{\sphinxcode{grids.topography}}}} for further details.

\item {} 
\sphinxstyleliteralstrong{topo\_time} (\sphinxtitleref{obj}, optional) \textendash{} \sphinxcode{datetime.timedelta} representing the simulation time after which the topography should stop increasing.
Default is 0, corresponding to a time-invariant terrain surface-height. See {\hyperref[\detokenize{api:module-grids.topography}]{\sphinxcrossref{\sphinxcode{grids.topography}}}} for further
details.

\end{itemize}

\item[{Keyword Arguments}] \leavevmode
\sphinxstyleliteralstrong{**kwargs} \textendash{} Keyword arguments to be forwarded to the constructor of {\hyperref[\detokenize{api:grids.topography.Topography1d}]{\sphinxcrossref{\sphinxcode{Topography1d}}}}.

\end{description}\end{quote}

\end{fulllineitems}

\index{\_update\_metric\_terms() (grids.gal\_chen.GalChen2d method)}

\begin{fulllineitems}
\phantomsection\label{\detokenize{api:grids.gal_chen.GalChen2d._update_metric_terms}}\pysiglinewithargsret{\sphinxbfcode{\_update\_metric\_terms}}{}{}
Update the class by computing the metric terms, i.e., the geometric height and the reference pressure,
at both half and main levels. In doing this, a logarithmic vertical profile of reference pressure is assumed.
This method should be called every time the topography is updated or changed.

\end{fulllineitems}

\index{plot() (grids.gal\_chen.GalChen2d method)}

\begin{fulllineitems}
\phantomsection\label{\detokenize{api:grids.gal_chen.GalChen2d.plot}}\pysiglinewithargsret{\sphinxbfcode{plot}}{\emph{**kwargs}}{}
Plot the grid half levels using \sphinxhref{https://matplotlib.org/2.1.1/api/\_as\_gen/matplotlib.pyplot.html\#module-matplotlib.pyplot}{\sphinxcode{matplotlib.pyplot}}’s utilities.
\begin{quote}\begin{description}
\item[{Keyword Arguments}] \leavevmode
\sphinxstyleliteralstrong{**kwargs} \textendash{} Keyword arguments to be forwarded to \sphinxhref{https://matplotlib.org/2.1.1/api/\_as\_gen/matplotlib.pyplot.subplots.html\#matplotlib.pyplot.subplots}{\sphinxcode{matplotlib.pyplot.subplots()}}.

\end{description}\end{quote}

\begin{sphinxadmonition}{note}{Note:}
For the sake of compliancy with the notation employed by \sphinxhref{http://www.cosmo-model.org}{COSMO},
the vertical geometric height is denoted by \(z\).
\end{sphinxadmonition}

\end{fulllineitems}

\index{update\_topography() (grids.gal\_chen.GalChen2d method)}

\begin{fulllineitems}
\phantomsection\label{\detokenize{api:grids.gal_chen.GalChen2d.update_topography}}\pysiglinewithargsret{\sphinxbfcode{update\_topography}}{\emph{time}}{}
Update the (time-dependent) topography. In turn, the metric terms are re-computed.
\begin{quote}\begin{description}
\item[{Parameters}] \leavevmode
\sphinxstyleliteralstrong{time} (\sphinxstyleliteralemphasis{obj}) \textendash{} \sphinxcode{datetime.timedelta} representing the elapsed simulation time.

\end{description}\end{quote}

\end{fulllineitems}


\end{fulllineitems}

\index{SLEVE2d (class in grids.sleve)}

\begin{fulllineitems}
\phantomsection\label{\detokenize{api:grids.sleve.SLEVE2d}}\pysiglinewithargsret{\sphinxbfcode{class }\sphinxcode{grids.sleve.}\sphinxbfcode{SLEVE2d}}{\emph{domain\_x}, \emph{nx}, \emph{domain\_z}, \emph{nz}, \emph{units\_x='m'}, \emph{dims\_x='x'}, \emph{z\_interface=None}, \emph{N=100}, \emph{s1=8000.0}, \emph{s2=5000.0}, \emph{topo\_type='flat\_terrain'}, \emph{topo\_time=datetime.timedelta(0)}, \emph{**kwargs}}{}
This class inherits {\hyperref[\detokenize{api:grids.grid_xz.GridXZ}]{\sphinxcrossref{\sphinxcode{GridXZ}}}} to represent a rectangular and regular two-dimensional
grid embedded in a reference system whose coordinates are
\begin{itemize}
\item {} 
the horizontal coordinate \(x\);

\item {} 
the height-based SLEVE terrain-following coordinate \(\mu\).

\end{itemize}

The vertical coordinate \(\mu\) may be formulated to define a hybrid terrain-following coordinate system
with terrain-following coordinate lines between the surface terrain-height and \(\mu = \mu_F\), where
\(\mu\)-coordinate lines change back to flat horizontal lines.
\begin{quote}\begin{description}
\item[{Variables}] \leavevmode\begin{itemize}
\item {} 
{\hyperref[\detokenize{api:grids.grid_xyz.GridXYZ.x}]{\sphinxcrossref{\sphinxstyleliteralstrong{x}}}} (\sphinxstyleliteralemphasis{obj}) \textendash{} {\hyperref[\detokenize{api:grids.axis.Axis}]{\sphinxcrossref{\sphinxcode{Axis}}}} representing the \(x\)-axis.

\item {} 
{\hyperref[\detokenize{api:grids.grid_xyz.GridXYZ.nx}]{\sphinxcrossref{\sphinxstyleliteralstrong{nx}}}} (\sphinxstyleliteralemphasis{int}) \textendash{} Number of grid points along \(x\).

\item {} 
{\hyperref[\detokenize{api:grids.grid_xyz.GridXYZ.dx}]{\sphinxcrossref{\sphinxstyleliteralstrong{dx}}}} (\sphinxstyleliteralemphasis{float}) \textendash{} The \(x\)-spacing.

\item {} 
\sphinxstyleliteralstrong{z} (\sphinxstyleliteralemphasis{obj}) \textendash{} {\hyperref[\detokenize{api:grids.axis.Axis}]{\sphinxcrossref{\sphinxcode{Axis}}}} representing the \(\mu\)-main levels.

\item {} 
\sphinxstyleliteralstrong{z\_half\_levels} (\sphinxstyleliteralemphasis{obj}) \textendash{} {\hyperref[\detokenize{api:grids.axis.Axis}]{\sphinxcrossref{\sphinxcode{Axis}}}} representing the \(\mu\)-half levels.

\item {} 
\sphinxstyleliteralstrong{nz} (\sphinxstyleliteralemphasis{int}) \textendash{} Number of vertical main levels.

\item {} 
\sphinxstyleliteralstrong{dz} (\sphinxstyleliteralemphasis{float}) \textendash{} The \(\mu\)-spacing.

\item {} 
\sphinxstyleliteralstrong{z\_interface} (\sphinxstyleliteralemphasis{float}) \textendash{} The interface coordinate \(\mu_F\).

\item {} 
\sphinxstyleliteralstrong{height} (\sphinxstyleliteralemphasis{obj}) \textendash{} \sphinxhref{http://xarray.pydata.org/en/stable/generated/xarray.DataArray.html\#xarray.DataArray}{\sphinxcode{xarray.DataArray}} representing the geometric height of the main levels.

\item {} 
\sphinxstyleliteralstrong{height\_half\_levels} (\sphinxstyleliteralemphasis{obj}) \textendash{} \sphinxhref{http://xarray.pydata.org/en/stable/generated/xarray.DataArray.html\#xarray.DataArray}{\sphinxcode{xarray.DataArray}} representing the geometric height of the half levels.

\item {} 
\sphinxstyleliteralstrong{height\_interface} (\sphinxstyleliteralemphasis{float}) \textendash{} Geometric height corresponding to \(\mu = \mu_F\).

\item {} 
\sphinxstyleliteralstrong{reference\_pressure} (\sphinxstyleliteralemphasis{obj}) \textendash{} \sphinxhref{http://xarray.pydata.org/en/stable/generated/xarray.DataArray.html\#xarray.DataArray}{\sphinxcode{xarray.DataArray}} representing the reference pressure at the main levels.

\item {} 
\sphinxstyleliteralstrong{reference\_pressure\_half\_levels} (\sphinxstyleliteralemphasis{obj}) \textendash{} \sphinxhref{http://xarray.pydata.org/en/stable/generated/xarray.DataArray.html\#xarray.DataArray}{\sphinxcode{xarray.DataArray}} representing the reference pressure at the half levels.

\end{itemize}

\end{description}\end{quote}
\index{\_\_init\_\_() (grids.sleve.SLEVE2d method)}

\begin{fulllineitems}
\phantomsection\label{\detokenize{api:grids.sleve.SLEVE2d.__init__}}\pysiglinewithargsret{\sphinxbfcode{\_\_init\_\_}}{\emph{domain\_x}, \emph{nx}, \emph{domain\_z}, \emph{nz}, \emph{units\_x='m'}, \emph{dims\_x='x'}, \emph{z\_interface=None}, \emph{N=100}, \emph{s1=8000.0}, \emph{s2=5000.0}, \emph{topo\_type='flat\_terrain'}, \emph{topo\_time=datetime.timedelta(0)}, \emph{**kwargs}}{}
Constructor.
\begin{quote}\begin{description}
\item[{Parameters}] \leavevmode\begin{itemize}
\item {} 
\sphinxstyleliteralstrong{domain\_x} (\sphinxstyleliteralemphasis{tuple}) \textendash{} Tuple in the form \((x_{left}, ~ x_{right})\).

\item {} 
\sphinxstyleliteralstrong{nx} (\sphinxstyleliteralemphasis{int}) \textendash{} Number of grid points in the \(x\)-direction.

\item {} 
\sphinxstyleliteralstrong{domain\_z} (\sphinxstyleliteralemphasis{tuple}) \textendash{} Tuple in the form \((\mu_{top}, ~ \mu_{surface})\).

\item {} 
\sphinxstyleliteralstrong{nz} (\sphinxstyleliteralemphasis{int}) \textendash{} Number of vertical main levels.

\item {} 
\sphinxstyleliteralstrong{units\_x} (\sphinxtitleref{str}, optional) \textendash{} 
Units for the \(x\)-coordinate. Must be compliant with the \sphinxhref{http://cfconventions.org}{CF Conventions}
(see also {\hyperref[\detokenize{api:grids.axis.Axis.__init__}]{\sphinxcrossref{\sphinxcode{grids.axis.Axis.\_\_init\_\_()}}}}).


\item {} 
\sphinxstyleliteralstrong{dims\_x} (\sphinxtitleref{str}, optional) \textendash{} Label for the \(x\)-coordinate.

\item {} 
\sphinxstyleliteralstrong{z\_interface} (\sphinxtitleref{float}, optional) \textendash{} Interface value \(\mu_F\). If not specified, it is assumed that \(\mu_F = \mu_T\), with
\(\mu_T\) the value of \(\mu\) at the top of the domain. In other words, a fully terrain-following
coordinate system is supposed.

\item {} 
\sphinxstyleliteralstrong{N} (\sphinxtitleref{int}, optional) \textendash{} Number of filter iterations performed to extract the large-scale component of the surface terrain-height.
Defaults to 100.

\item {} 
\sphinxstyleliteralstrong{s1} (\sphinxtitleref{float}, optional) \textendash{} Large-scale decay constant. Defaults to \(8000 ~ m\).

\item {} 
\sphinxstyleliteralstrong{s2} (\sphinxtitleref{float}, optional) \textendash{} Small-scale decay constant. Defaults to \(5000 ~ m\).

\item {} 
\sphinxstyleliteralstrong{topo\_type} (\sphinxtitleref{str}, optional) \textendash{} Topography type. Defaults to ‘flat\_terrain’. See {\hyperref[\detokenize{api:module-grids.topography}]{\sphinxcrossref{\sphinxcode{grids.topography}}}} for further details.

\item {} 
\sphinxstyleliteralstrong{topo\_time} (\sphinxtitleref{obj}, optional) \textendash{} \sphinxcode{datetime.timedelta} representing the simulation time after which the topography should stop
increasing. Default is 0, corresponding to a time-invariant terrain surface-height. See {\hyperref[\detokenize{api:module-grids.topography}]{\sphinxcrossref{\sphinxcode{grids.topography}}}}
for further details.

\end{itemize}

\item[{Keyword Arguments}] \leavevmode
\sphinxstyleliteralstrong{**kwargs} \textendash{} Keyword arguments to be forwarded to the constructor of {\hyperref[\detokenize{api:grids.topography.Topography1d}]{\sphinxcrossref{\sphinxcode{Topography1d}}}}.

\end{description}\end{quote}

\end{fulllineitems}

\index{\_update\_metric\_terms() (grids.sleve.SLEVE2d method)}

\begin{fulllineitems}
\phantomsection\label{\detokenize{api:grids.sleve.SLEVE2d._update_metric_terms}}\pysiglinewithargsret{\sphinxbfcode{\_update\_metric\_terms}}{}{}
Update the class by computing the metric terms, i.e., the geometric height and the reference pressure,
at both half and main levels. In doing this, a logarithmic vertical profile of reference pressure is assumed.
This method should be called every time the topography is updated or changed.

\end{fulllineitems}

\index{plot() (grids.sleve.SLEVE2d method)}

\begin{fulllineitems}
\phantomsection\label{\detokenize{api:grids.sleve.SLEVE2d.plot}}\pysiglinewithargsret{\sphinxbfcode{plot}}{\emph{**kwargs}}{}
Plot the grid half levels using \sphinxhref{https://matplotlib.org/2.1.1/api/\_as\_gen/matplotlib.pyplot.html\#module-matplotlib.pyplot}{\sphinxcode{matplotlib.pyplot}}’s utilities.
\begin{quote}\begin{description}
\item[{Keyword Arguments}] \leavevmode
\sphinxstyleliteralstrong{**kwargs} \textendash{} Keyword arguments to be forwarded to \sphinxhref{https://matplotlib.org/2.1.1/api/\_as\_gen/matplotlib.pyplot.subplots.html\#matplotlib.pyplot.subplots}{\sphinxcode{matplotlib.pyplot.subplots()}}.

\end{description}\end{quote}

\begin{sphinxadmonition}{note}{Note:}
For the sake of compliancy with the notation employed by \sphinxhref{http://www.cosmo-model.org}{COSMO},
the vertical geometric height is denoted by \(z\).
\end{sphinxadmonition}

\end{fulllineitems}

\index{update\_topography() (grids.sleve.SLEVE2d method)}

\begin{fulllineitems}
\phantomsection\label{\detokenize{api:grids.sleve.SLEVE2d.update_topography}}\pysiglinewithargsret{\sphinxbfcode{update\_topography}}{\emph{time}}{}
Update the (time-dependent) topography. In turn, the metric terms are re-computed.
\begin{quote}\begin{description}
\item[{Parameters}] \leavevmode
\sphinxstyleliteralstrong{time} (\sphinxstyleliteralemphasis{obj}) \textendash{} \sphinxcode{datetime.timedelta} representing the elasped simulation time.

\end{description}\end{quote}

\end{fulllineitems}


\end{fulllineitems}



\subsection{Three-dimensional grids}
\label{\detokenize{api:three-dimensional-grids}}\index{GridXYZ (class in grids.grid\_xyz)}

\begin{fulllineitems}
\phantomsection\label{\detokenize{api:grids.grid_xyz.GridXYZ}}\pysiglinewithargsret{\sphinxbfcode{class }\sphinxcode{grids.grid\_xyz.}\sphinxbfcode{GridXYZ}}{\emph{domain\_x}, \emph{nx}, \emph{domain\_y}, \emph{ny}, \emph{domain\_z}, \emph{nz}, \emph{units\_x='degrees\_east'}, \emph{dims\_x='longitude'}, \emph{units\_y='degrees\_north'}, \emph{dims\_y='latitude'}, \emph{units\_z='m'}, \emph{dims\_z='z'}, \emph{z\_interface=None}, \emph{topo\_type='flat\_terrain'}, \emph{topo\_time=datetime.timedelta(0)}, \emph{**kwargs}}{}
Rectangular and regular three-dimensional grid embedded in a reference system whose coordinates are
\begin{itemize}
\item {} 
the first horizontal coordinate \(x\);

\item {} 
the second horizontal coordinate \(y\);

\item {} 
the vertical (terrain-following) coordinate \(z\).

\end{itemize}

The vertical coordinate \(z\) may be formulated to define a hybrid terrain-following coordinate system
with terrain-following coordinate lines between the surface terrain-height and \(z = z_F\), where
\(z\)-coordinate lines change back to flat horizontal lines. However, no assumption is made on the actual
nature of \(z\) which may be either pressure-based or height-based.
\begin{quote}\begin{description}
\item[{Variables}] \leavevmode\begin{itemize}
\item {} 
\sphinxstyleliteralstrong{xy\_grid} (\sphinxstyleliteralemphasis{obj}) \textendash{} \sphinxtitleref{\textasciitilde{}grids.grid\_xy.GridXY} representing the horizontal grid..

\item {} 
\sphinxstyleliteralstrong{z} (\sphinxstyleliteralemphasis{obj}) \textendash{} {\hyperref[\detokenize{api:grids.axis.Axis}]{\sphinxcrossref{\sphinxcode{Axis}}}} representing the \(z\)-main levels.

\item {} 
\sphinxstyleliteralstrong{z\_half\_levels} (\sphinxstyleliteralemphasis{obj}) \textendash{} {\hyperref[\detokenize{api:grids.axis.Axis}]{\sphinxcrossref{\sphinxcode{Axis}}}} representing the \(z\)-half levels.

\item {} 
\sphinxstyleliteralstrong{nz} (\sphinxstyleliteralemphasis{int}) \textendash{} Number of vertical main levels.

\item {} 
\sphinxstyleliteralstrong{dz} (\sphinxstyleliteralemphasis{float}) \textendash{} The \(z\)-spacing.

\item {} 
\sphinxstyleliteralstrong{z\_interface} (\sphinxstyleliteralemphasis{float}) \textendash{} The interface coordinate \(z_F\).

\end{itemize}

\end{description}\end{quote}

\begin{sphinxadmonition}{note}{Note:}
For the sake of compliancy with the \sphinxhref{http://cosmo-model.org}{COSMO model}, the vertical grid points are ordered
from the top of the domain to the surface.
\end{sphinxadmonition}
\index{\_\_init\_\_() (grids.grid\_xyz.GridXYZ method)}

\begin{fulllineitems}
\phantomsection\label{\detokenize{api:grids.grid_xyz.GridXYZ.__init__}}\pysiglinewithargsret{\sphinxbfcode{\_\_init\_\_}}{\emph{domain\_x}, \emph{nx}, \emph{domain\_y}, \emph{ny}, \emph{domain\_z}, \emph{nz}, \emph{units\_x='degrees\_east'}, \emph{dims\_x='longitude'}, \emph{units\_y='degrees\_north'}, \emph{dims\_y='latitude'}, \emph{units\_z='m'}, \emph{dims\_z='z'}, \emph{z\_interface=None}, \emph{topo\_type='flat\_terrain'}, \emph{topo\_time=datetime.timedelta(0)}, \emph{**kwargs}}{}
Constructor.
\begin{quote}\begin{description}
\item[{Parameters}] \leavevmode\begin{itemize}
\item {} 
\sphinxstyleliteralstrong{domain\_x} (\sphinxstyleliteralemphasis{tuple}) \textendash{} Tuple in the form \((x_{start}, ~ x_{stop})\).

\item {} 
\sphinxstyleliteralstrong{nx} (\sphinxstyleliteralemphasis{int}) \textendash{} Number of grid points in the \(x\)-direction.

\item {} 
\sphinxstyleliteralstrong{domain\_y} (\sphinxstyleliteralemphasis{tuple}) \textendash{} Tuple in the form \((y_{start}, ~ y_{stop})\).

\item {} 
\sphinxstyleliteralstrong{ny} (\sphinxstyleliteralemphasis{int}) \textendash{} Number of grid points in the \(y\)-direction.

\item {} 
\sphinxstyleliteralstrong{domain\_z} (\sphinxstyleliteralemphasis{tuple}) \textendash{} Tuple in the form \((z_{top}, ~ z_{surface})\).

\item {} 
\sphinxstyleliteralstrong{nz} (\sphinxstyleliteralemphasis{int}) \textendash{} Number of vertical main levels.

\item {} 
\sphinxstyleliteralstrong{units\_x} (\sphinxtitleref{str}, optional) \textendash{} 
Units for the \(x\)-coordinate. Must be compliant with the \sphinxhref{cfconventions.org}{CF Conventions}.


\item {} 
\sphinxstyleliteralstrong{dims\_x} (\sphinxtitleref{str}, optional) \textendash{} Label for the \(x\)-coordinate.

\item {} 
\sphinxstyleliteralstrong{units\_y} (\sphinxtitleref{str}, optional) \textendash{} 
Units for the \(y\)-coordinate. Must be compliant with the \sphinxhref{cfconventions.org}{CF Conventions}.


\item {} 
\sphinxstyleliteralstrong{dims\_y} (\sphinxtitleref{str}, optional) \textendash{} Label for the \(y\)-coordinate.

\item {} 
\sphinxstyleliteralstrong{units\_z} (\sphinxtitleref{str}, optional) \textendash{} 
Units for the \(z\)-coordinate. Must be compliant with the \sphinxhref{cfconventions.org}{CF Conventions}.


\item {} 
\sphinxstyleliteralstrong{dims\_z} (\sphinxtitleref{str}, optional) \textendash{} Label for the \(z\)-coordinate.

\item {} 
\sphinxstyleliteralstrong{z\_interface} (\sphinxtitleref{float}, optional) \textendash{} Interface value \(z_F\). If not specified, it is assumed that \(z_F = z_T\), with \(z_T\)
the value of \(z\) at the top of the domain. In other words, a fully terrain-following coordinate
system is supposed.

\item {} 
\sphinxstyleliteralstrong{topo\_type} (\sphinxtitleref{str}, optional) \textendash{} Topography type. Default is ‘flat\_terrain’. See {\hyperref[\detokenize{api:module-grids.topography}]{\sphinxcrossref{\sphinxcode{grids.topography}}}} for further details.

\item {} 
\sphinxstyleliteralstrong{topo\_time} (\sphinxtitleref{obj}, optional) \textendash{} \sphinxcode{datetime.timedelta} representing the simulation time after which the topography should stop
increasing. Default is 0, corresponding to a time-invariant terrain surface-height. See
{\hyperref[\detokenize{api:module-grids.topography}]{\sphinxcrossref{\sphinxcode{grids.topography}}}} for further details.

\end{itemize}

\item[{Keyword Arguments}] \leavevmode
\sphinxstyleliteralstrong{kwargs} \textendash{} Keyword arguments to be forwarded to the constructor of {\hyperref[\detokenize{api:grids.topography.Topography2d}]{\sphinxcrossref{\sphinxcode{Topography2d}}}}.

\end{description}\end{quote}

\end{fulllineitems}

\index{dx (grids.grid\_xyz.GridXYZ attribute)}

\begin{fulllineitems}
\phantomsection\label{\detokenize{api:grids.grid_xyz.GridXYZ.dx}}\pysigline{\sphinxbfcode{dx}}
Get the \(x\)-spacing.
\begin{quote}\begin{description}
\item[{Returns}] \leavevmode
The \(x\)-spacing.

\item[{Return type}] \leavevmode
float

\end{description}\end{quote}

\end{fulllineitems}

\index{dy (grids.grid\_xyz.GridXYZ attribute)}

\begin{fulllineitems}
\phantomsection\label{\detokenize{api:grids.grid_xyz.GridXYZ.dy}}\pysigline{\sphinxbfcode{dy}}
Get the \(y\)-spacing.
\begin{quote}\begin{description}
\item[{Returns}] \leavevmode
The \(y\)-spacing.

\item[{Return type}] \leavevmode
float

\end{description}\end{quote}

\end{fulllineitems}

\index{nx (grids.grid\_xyz.GridXYZ attribute)}

\begin{fulllineitems}
\phantomsection\label{\detokenize{api:grids.grid_xyz.GridXYZ.nx}}\pysigline{\sphinxbfcode{nx}}
Get the number of grid points in the \(x\)-direction.
\begin{quote}\begin{description}
\item[{Returns}] \leavevmode
Number of grid points in the \(x\)-direction.

\item[{Return type}] \leavevmode
int

\end{description}\end{quote}

\end{fulllineitems}

\index{ny (grids.grid\_xyz.GridXYZ attribute)}

\begin{fulllineitems}
\phantomsection\label{\detokenize{api:grids.grid_xyz.GridXYZ.ny}}\pysigline{\sphinxbfcode{ny}}
Get the number of grid points in the \(y\)-direction.
\begin{quote}\begin{description}
\item[{Returns}] \leavevmode
Number of grid points in the \(y\)-direction.

\item[{Return type}] \leavevmode
int

\end{description}\end{quote}

\end{fulllineitems}

\index{topography\_height (grids.grid\_xyz.GridXYZ attribute)}

\begin{fulllineitems}
\phantomsection\label{\detokenize{api:grids.grid_xyz.GridXYZ.topography_height}}\pysigline{\sphinxbfcode{topography\_height}}
Get the topography (i.e., terrain-surface) height.
\begin{quote}\begin{description}
\item[{Returns}] \leavevmode
Two-dimensional \sphinxhref{https://docs.scipy.org/doc/numpy-1.13.0/reference/generated/numpy.ndarray.html\#numpy.ndarray}{\sphinxcode{numpy.ndarray}} representing the topography height.

\item[{Return type}] \leavevmode
array\_like

\end{description}\end{quote}

\end{fulllineitems}

\index{update\_topography() (grids.grid\_xyz.GridXYZ method)}

\begin{fulllineitems}
\phantomsection\label{\detokenize{api:grids.grid_xyz.GridXYZ.update_topography}}\pysiglinewithargsret{\sphinxbfcode{update\_topography}}{\emph{time}}{}
Update the (time-dependent) topography.
\begin{quote}\begin{description}
\item[{Parameters}] \leavevmode
\sphinxstyleliteralstrong{time} (\sphinxstyleliteralemphasis{obj}) \textendash{} \sphinxcode{datetime.timedelta} representing the elapsed simulation time.

\end{description}\end{quote}

\end{fulllineitems}

\index{x (grids.grid\_xyz.GridXYZ attribute)}

\begin{fulllineitems}
\phantomsection\label{\detokenize{api:grids.grid_xyz.GridXYZ.x}}\pysigline{\sphinxbfcode{x}}
Get the \(x\)-axis.
\begin{quote}\begin{description}
\item[{Returns}] \leavevmode
{\hyperref[\detokenize{api:grids.axis.Axis}]{\sphinxcrossref{\sphinxcode{Axis}}}} representing the \(x\)-axis.

\item[{Return type}] \leavevmode
obj

\end{description}\end{quote}

\end{fulllineitems}

\index{x\_half\_levels (grids.grid\_xyz.GridXYZ attribute)}

\begin{fulllineitems}
\phantomsection\label{\detokenize{api:grids.grid_xyz.GridXYZ.x_half_levels}}\pysigline{\sphinxbfcode{x\_half\_levels}}
Get the \(x\)-half levels.
\begin{quote}\begin{description}
\item[{Returns}] \leavevmode
{\hyperref[\detokenize{api:grids.axis.Axis}]{\sphinxcrossref{\sphinxcode{Axis}}}} representing the \(x\)-half levels.

\item[{Return type}] \leavevmode
obj

\end{description}\end{quote}

\end{fulllineitems}

\index{y (grids.grid\_xyz.GridXYZ attribute)}

\begin{fulllineitems}
\phantomsection\label{\detokenize{api:grids.grid_xyz.GridXYZ.y}}\pysigline{\sphinxbfcode{y}}
Get the \(y\)-axis.
\begin{quote}\begin{description}
\item[{Returns}] \leavevmode
{\hyperref[\detokenize{api:grids.axis.Axis}]{\sphinxcrossref{\sphinxcode{Axis}}}} representing the \(y\)-axis.

\item[{Return type}] \leavevmode
obj

\end{description}\end{quote}

\end{fulllineitems}

\index{y\_half\_levels (grids.grid\_xyz.GridXYZ attribute)}

\begin{fulllineitems}
\phantomsection\label{\detokenize{api:grids.grid_xyz.GridXYZ.y_half_levels}}\pysigline{\sphinxbfcode{y\_half\_levels}}
Get the \(y\)-half levels.
\begin{quote}\begin{description}
\item[{Returns}] \leavevmode
{\hyperref[\detokenize{api:grids.axis.Axis}]{\sphinxcrossref{\sphinxcode{Axis}}}} representing the \(y\)-half levels.

\item[{Return type}] \leavevmode
obj

\end{description}\end{quote}

\end{fulllineitems}


\end{fulllineitems}

\index{Sigma3d (class in grids.sigma)}

\begin{fulllineitems}
\phantomsection\label{\detokenize{api:grids.sigma.Sigma3d}}\pysiglinewithargsret{\sphinxbfcode{class }\sphinxcode{grids.sigma.}\sphinxbfcode{Sigma3d}}{\emph{domain\_x}, \emph{nx}, \emph{domain\_y}, \emph{ny}, \emph{domain\_z}, \emph{nz}, \emph{units\_x='degrees\_east'}, \emph{dims\_x='longitude'}, \emph{units\_y='degrees\_north'}, \emph{dims\_y='latitude'}, \emph{z\_interface=None}, \emph{topo\_type='flat\_terrain'}, \emph{topo\_time=datetime.timedelta(0)}, \emph{**kwargs}}{}
This class inherits {\hyperref[\detokenize{api:grids.grid_xyz.GridXYZ}]{\sphinxcrossref{\sphinxcode{GridXYZ}}}} to represent a rectangular and regular computational grid
embedded in a three-dimensional terrain-following reference system, whose coordinates are:
\begin{itemize}
\item {} 
first horizontal coordinate \(x\), e.g., the longitude;

\item {} 
second horizontal coordinate \(y\), e.g., the latitude;

\item {} 
the pressure-based terrain-following coordinate \(\sigma = p / p_{SL}\),                  where \(p\) is the pressure and \(p_{SL}\) the pressure at the sea level.

\end{itemize}

The vertical coordinate \(\sigma\) may be formulated to define a hybrid terrain-following coordinate system
with terrain-following coordinate lines between the surface terrain-height and \(\sigma = \sigma_F\), where
\(\sigma\)-coordinate lines change back to flat horizontal lines.
\begin{quote}\begin{description}
\item[{Variables}] \leavevmode\begin{itemize}
\item {} 
\sphinxstyleliteralstrong{xy\_grid} (\sphinxstyleliteralemphasis{obj}) \textendash{} {\hyperref[\detokenize{api:grids.grid_xy.GridXY}]{\sphinxcrossref{\sphinxcode{GridXY}}}} representing the horizontal grid.

\item {} 
\sphinxstyleliteralstrong{z} (\sphinxstyleliteralemphasis{obj}) \textendash{} {\hyperref[\detokenize{api:grids.axis.Axis}]{\sphinxcrossref{\sphinxcode{Axis}}}} representing the \(\sigma\)-main levels.

\item {} 
\sphinxstyleliteralstrong{z\_half\_levels} (\sphinxstyleliteralemphasis{obj}) \textendash{} {\hyperref[\detokenize{api:grids.axis.Axis}]{\sphinxcrossref{\sphinxcode{Axis}}}} representing the \(\sigma\)-half levels.

\item {} 
\sphinxstyleliteralstrong{nz} (\sphinxstyleliteralemphasis{int}) \textendash{} Number of vertical main levels.

\item {} 
\sphinxstyleliteralstrong{dz} (\sphinxstyleliteralemphasis{float}) \textendash{} The \(\sigma\)-spacing.

\item {} 
\sphinxstyleliteralstrong{z\_interface} (\sphinxstyleliteralemphasis{float}) \textendash{} The interface coordinate \(\sigma_F\).

\item {} 
\sphinxstyleliteralstrong{height} (\sphinxstyleliteralemphasis{obj}) \textendash{} \sphinxhref{http://xarray.pydata.org/en/stable/generated/xarray.DataArray.html\#xarray.DataArray}{\sphinxcode{xarray.DataArray}} representing the geometric height of the main levels.

\item {} 
\sphinxstyleliteralstrong{height\_half\_levels} (\sphinxstyleliteralemphasis{obj}) \textendash{} \sphinxhref{http://xarray.pydata.org/en/stable/generated/xarray.DataArray.html\#xarray.DataArray}{\sphinxcode{xarray.DataArray}} representing the geometric height of the half levels.

\item {} 
\sphinxstyleliteralstrong{height\_interface} (\sphinxstyleliteralemphasis{float}) \textendash{} Geometric height corresponding to \(\sigma = \sigma_F\).

\item {} 
\sphinxstyleliteralstrong{reference\_pressure} (\sphinxstyleliteralemphasis{obj}) \textendash{} \sphinxhref{http://xarray.pydata.org/en/stable/generated/xarray.DataArray.html\#xarray.DataArray}{\sphinxcode{xarray.DataArray}} storing the reference pressure at the main levels.

\item {} 
\sphinxstyleliteralstrong{reference\_pressure\_half\_levels} (\sphinxstyleliteralemphasis{obj}) \textendash{} \sphinxhref{http://xarray.pydata.org/en/stable/generated/xarray.DataArray.html\#xarray.DataArray}{\sphinxcode{xarray.DataArray}} storing the reference pressure at the half levels.

\end{itemize}

\end{description}\end{quote}
\index{\_\_init\_\_() (grids.sigma.Sigma3d method)}

\begin{fulllineitems}
\phantomsection\label{\detokenize{api:grids.sigma.Sigma3d.__init__}}\pysiglinewithargsret{\sphinxbfcode{\_\_init\_\_}}{\emph{domain\_x}, \emph{nx}, \emph{domain\_y}, \emph{ny}, \emph{domain\_z}, \emph{nz}, \emph{units\_x='degrees\_east'}, \emph{dims\_x='longitude'}, \emph{units\_y='degrees\_north'}, \emph{dims\_y='latitude'}, \emph{z\_interface=None}, \emph{topo\_type='flat\_terrain'}, \emph{topo\_time=datetime.timedelta(0)}, \emph{**kwargs}}{}
Constructor.
\begin{quote}\begin{description}
\item[{Parameters}] \leavevmode\begin{itemize}
\item {} 
\sphinxstyleliteralstrong{domain\_x} (\sphinxstyleliteralemphasis{tuple}) \textendash{} Tuple in the form \((x_{left}, ~ x_{right})\).

\item {} 
\sphinxstyleliteralstrong{nx} (\sphinxstyleliteralemphasis{int}) \textendash{} Number of grid points in the \(x\)-direction.

\item {} 
\sphinxstyleliteralstrong{domain\_y} (\sphinxstyleliteralemphasis{tuple}) \textendash{} Tuple in the form \((y_{left}, ~ y_{right})\).

\item {} 
\sphinxstyleliteralstrong{ny} (\sphinxstyleliteralemphasis{int}) \textendash{} Number of grid points in the \(y\)-direction.

\item {} 
\sphinxstyleliteralstrong{domain\_z} (\sphinxstyleliteralemphasis{tuple}) \textendash{} Tuple in the form \((\sigma_{top}, ~ \sigma_{surface})\).

\item {} 
\sphinxstyleliteralstrong{nz} (\sphinxstyleliteralemphasis{int}) \textendash{} Number of vertical main levels.

\item {} 
\sphinxstyleliteralstrong{units\_x} (\sphinxtitleref{str}, optional) \textendash{} 
Units for the \(x\)-coordinate. Must be compliant with the \sphinxhref{http://cfconventions.org}{CF Conventions}
(see also {\hyperref[\detokenize{api:grids.axis.Axis.__init__}]{\sphinxcrossref{\sphinxcode{grids.axis.Axis.\_\_init\_\_()}}}}).


\item {} 
\sphinxstyleliteralstrong{dims\_x} (\sphinxtitleref{str}, optional) \textendash{} Label for the \(x\)-coordinate.

\item {} 
\sphinxstyleliteralstrong{units\_y} (\sphinxtitleref{str}, optional) \textendash{} 
Units for the \(y\)-coordinate. Must be compliant with the \sphinxhref{http://cfconventions.org}{CF Conventions}
(see also {\hyperref[\detokenize{api:grids.axis.Axis.__init__}]{\sphinxcrossref{\sphinxcode{grids.axis.Axis.\_\_init\_\_()}}}}).


\item {} 
\sphinxstyleliteralstrong{dims\_y} (\sphinxtitleref{str}, optional) \textendash{} Label for the \(y\)-coordinate.

\item {} 
\sphinxstyleliteralstrong{z\_interface} (\sphinxtitleref{float}, optional) \textendash{} Interface value \(\sigma_F\). If not specified, it is assumed that \(\sigma_F = \sigma_T\),
with \(\sigma_T\) the value of  \(\sigma\) at the top of the domain. In other words, a fully
terrain-following coordinate system is supposed.

\item {} 
\sphinxstyleliteralstrong{topo\_type} (\sphinxtitleref{str}, optional) \textendash{} Topography type. Default is ‘flat\_terrain’. See {\hyperref[\detokenize{api:module-grids.topography}]{\sphinxcrossref{\sphinxcode{grids.topography}}}} for further details.

\item {} 
\sphinxstyleliteralstrong{topo\_time} (\sphinxtitleref{obj}, optional) \textendash{} \sphinxcode{datetime.timedelta} representing the simulation time after which the topography should stop
increasing. Default is 0, corresponding to a time-invariant terrain surface-height.
See {\hyperref[\detokenize{api:module-grids.topography}]{\sphinxcrossref{\sphinxcode{grids.topography}}}} for further details.

\end{itemize}

\item[{Keyword Arguments}] \leavevmode
\sphinxstyleliteralstrong{**kwargs} \textendash{} Keyword arguments to be forwarded to the constructor of {\hyperref[\detokenize{api:grids.topography.Topography2d}]{\sphinxcrossref{\sphinxcode{Topography2d}}}}.

\end{description}\end{quote}

\end{fulllineitems}

\index{\_update\_metric\_terms() (grids.sigma.Sigma3d method)}

\begin{fulllineitems}
\phantomsection\label{\detokenize{api:grids.sigma.Sigma3d._update_metric_terms}}\pysiglinewithargsret{\sphinxbfcode{\_update\_metric\_terms}}{}{}
Update the class by computing the metric terms, i.e., the geometric height and the reference pressure,
at both half and main levels. In doing this, a logarithmic vertical profile of reference pressure is assumed.
This method should be called every time the topography is updated or changed.

\end{fulllineitems}

\index{update\_topography() (grids.sigma.Sigma3d method)}

\begin{fulllineitems}
\phantomsection\label{\detokenize{api:grids.sigma.Sigma3d.update_topography}}\pysiglinewithargsret{\sphinxbfcode{update\_topography}}{\emph{time}}{}
Update the (time-dependent) topography. In turn, the metric terms are re-computed.
\begin{quote}\begin{description}
\item[{Parameters}] \leavevmode
\sphinxstyleliteralstrong{time} (\sphinxstyleliteralemphasis{obj}) \textendash{} \sphinxcode{datetime.timedelta} representing the elapsed simulation time.

\end{description}\end{quote}

\end{fulllineitems}


\end{fulllineitems}

\index{GalChen3d (class in grids.gal\_chen)}

\begin{fulllineitems}
\phantomsection\label{\detokenize{api:grids.gal_chen.GalChen3d}}\pysiglinewithargsret{\sphinxbfcode{class }\sphinxcode{grids.gal\_chen.}\sphinxbfcode{GalChen3d}}{\emph{domain\_x}, \emph{nx}, \emph{domain\_y}, \emph{ny}, \emph{domain\_z}, \emph{nz}, \emph{units\_x='degrees\_east'}, \emph{dims\_x='longitude'}, \emph{units\_y='degrees\_north'}, \emph{dims\_y='latitude'}, \emph{z\_interface=None}, \emph{topo\_type='flat\_terrain'}, \emph{topo\_time=datetime.timedelta(0)}, \emph{**kwargs}}{}
This class inherits {\hyperref[\detokenize{api:grids.grid_xyz.GridXYZ}]{\sphinxcrossref{\sphinxcode{GridXYZ}}}} to represent a rectangular and regular computational grid
embedded in a three-dimensional terrain-following reference system, whose coordinates are:
\begin{itemize}
\item {} 
first horizontal coordinate \(x\), e.g., the longitude;

\item {} 
second horizontal coordinate \(y\), e.g., the latitude;

\item {} 
the Gal-Chen terrain-following coordinate \(\mu\).

\end{itemize}

The vertical coordinate \(\mu\) may be formulated to define a hybrid terrain-following coordinate system
with terrain-following coordinate lines between the surface terrain-height and \(\mu = \mu_F\), where
\(\mu\)-coordinate lines change back to flat horizontal lines.
\begin{quote}\begin{description}
\item[{Variables}] \leavevmode\begin{itemize}
\item {} 
\sphinxstyleliteralstrong{xy\_grid} (\sphinxstyleliteralemphasis{obj}) \textendash{} {\hyperref[\detokenize{api:grids.grid_xy.GridXY}]{\sphinxcrossref{\sphinxcode{GridXY}}}} representing the horizontal grid.

\item {} 
\sphinxstyleliteralstrong{z} (\sphinxstyleliteralemphasis{obj}) \textendash{} {\hyperref[\detokenize{api:grids.axis.Axis}]{\sphinxcrossref{\sphinxcode{Axis}}}} representing the \(z\)-main levels.

\item {} 
\sphinxstyleliteralstrong{z\_half\_levels} (\sphinxstyleliteralemphasis{obj}) \textendash{} {\hyperref[\detokenize{api:grids.axis.Axis}]{\sphinxcrossref{\sphinxcode{Axis}}}} representing the \(z\)-half levels.

\item {} 
\sphinxstyleliteralstrong{nz} (\sphinxstyleliteralemphasis{int}) \textendash{} Number of vertical main levels.

\item {} 
\sphinxstyleliteralstrong{dz} (\sphinxstyleliteralemphasis{float}) \textendash{} The \(z\)-spacing.

\item {} 
\sphinxstyleliteralstrong{z\_interface} (\sphinxstyleliteralemphasis{float}) \textendash{} The interface coordinate \(z_F\).

\item {} 
\sphinxstyleliteralstrong{height} (\sphinxstyleliteralemphasis{obj}) \textendash{} \sphinxhref{http://xarray.pydata.org/en/stable/generated/xarray.DataArray.html\#xarray.DataArray}{\sphinxcode{xarray.DataArray}} representing the geometric height of the main levels.

\item {} 
\sphinxstyleliteralstrong{height\_half\_levels} (\sphinxstyleliteralemphasis{obj}) \textendash{} \sphinxhref{http://xarray.pydata.org/en/stable/generated/xarray.DataArray.html\#xarray.DataArray}{\sphinxcode{xarray.DataArray}} representing the geometric height of the half levels.

\item {} 
\sphinxstyleliteralstrong{height\_interface} (\sphinxstyleliteralemphasis{float}) \textendash{} Geometric height corresponding to \(\mu = \mu_F\).

\item {} 
\sphinxstyleliteralstrong{reference\_pressure} (\sphinxstyleliteralemphasis{obj}) \textendash{} \sphinxhref{http://xarray.pydata.org/en/stable/generated/xarray.DataArray.html\#xarray.DataArray}{\sphinxcode{xarray.DataArray}} representing the reference pressure at the main levels.

\item {} 
\sphinxstyleliteralstrong{reference\_pressure\_half\_levels} (\sphinxstyleliteralemphasis{obj}) \textendash{} \sphinxhref{http://xarray.pydata.org/en/stable/generated/xarray.DataArray.html\#xarray.DataArray}{\sphinxcode{xarray.DataArray}} representing the reference pressure at the half levels.

\end{itemize}

\end{description}\end{quote}
\index{\_\_init\_\_() (grids.gal\_chen.GalChen3d method)}

\begin{fulllineitems}
\phantomsection\label{\detokenize{api:grids.gal_chen.GalChen3d.__init__}}\pysiglinewithargsret{\sphinxbfcode{\_\_init\_\_}}{\emph{domain\_x}, \emph{nx}, \emph{domain\_y}, \emph{ny}, \emph{domain\_z}, \emph{nz}, \emph{units\_x='degrees\_east'}, \emph{dims\_x='longitude'}, \emph{units\_y='degrees\_north'}, \emph{dims\_y='latitude'}, \emph{z\_interface=None}, \emph{topo\_type='flat\_terrain'}, \emph{topo\_time=datetime.timedelta(0)}, \emph{**kwargs}}{}
Constructor.
\begin{quote}\begin{description}
\item[{Parameters}] \leavevmode\begin{itemize}
\item {} 
\sphinxstyleliteralstrong{domain\_x} (\sphinxstyleliteralemphasis{tuple}) \textendash{} Tuple in the form \((x_{left}, ~ x_{right})\).

\item {} 
\sphinxstyleliteralstrong{nx} (\sphinxstyleliteralemphasis{int}) \textendash{} Number of grid points in the \(x\)-direction.

\item {} 
\sphinxstyleliteralstrong{domain\_y} (\sphinxstyleliteralemphasis{tuple}) \textendash{} Tuple in the form \((y_{left}, ~ y_{right})\).

\item {} 
\sphinxstyleliteralstrong{ny} (\sphinxstyleliteralemphasis{int}) \textendash{} Number of grid points in the \(y\)-direction.

\item {} 
\sphinxstyleliteralstrong{domain\_z} (\sphinxstyleliteralemphasis{tuple}) \textendash{} Tuple in the form \((\mu_{top}, ~ \mu_{surface})\).

\item {} 
\sphinxstyleliteralstrong{nz} (\sphinxstyleliteralemphasis{int}) \textendash{} Number of vertical main levels.

\item {} 
\sphinxstyleliteralstrong{units\_x} (\sphinxtitleref{str}, optional) \textendash{} 
Units for the \(x\)-coordinate. Must be compliant with the \sphinxhref{http://cfconventions.org}{CF Conventions}
(see also {\hyperref[\detokenize{api:grids.axis.Axis.__init__}]{\sphinxcrossref{\sphinxcode{grids.axis.Axis.\_\_init\_\_()}}}}).


\item {} 
\sphinxstyleliteralstrong{dims\_x} (\sphinxtitleref{str}, optional) \textendash{} Label for the \(x\)-coordinate.

\item {} 
\sphinxstyleliteralstrong{str}\sphinxstyleliteralstrong{, }\sphinxstyleliteralstrong{optional} (\sphinxstyleliteralemphasis{units\_y}) \textendash{} 
Units for the \(y\)-coordinate. Must be compliant with the \sphinxhref{http://cfconventions.org}{CF Conventions}
(see also {\hyperref[\detokenize{api:grids.axis.Axis.__init__}]{\sphinxcrossref{\sphinxcode{grids.axis.Axis.\_\_init\_\_()}}}}).


\item {} 
\sphinxstyleliteralstrong{dims\_y} (\sphinxtitleref{str}, optional) \textendash{} Label for the \(y\)-coordinate.

\item {} 
\sphinxstyleliteralstrong{z\_interface} (\sphinxtitleref{float}, optional) \textendash{} Interface value \(zmu_F = \mu_F\). If not specified, it is assumed that \(\mu_F = \mu_T\), with
\(\mu_T\) the value of \(\mu\) at the top of the domain. In other words, a fully terrain-following
coordinate nsystem is supposed.

\item {} 
\sphinxstyleliteralstrong{topo\_type} (\sphinxtitleref{str}, optional) \textendash{} Topography type. Default is ‘flat\_terrain’. See {\hyperref[\detokenize{api:module-grids.topography}]{\sphinxcrossref{\sphinxcode{grids.topography}}}} for further details.

\item {} 
\sphinxstyleliteralstrong{topo\_time} (\sphinxtitleref{obj}, optional) \textendash{} \sphinxcode{datetime.timedelta} representing the simulation time after which the topography should stop increasing.
Default is 0, corresponding to a time-invariant terrain surface-height. See {\hyperref[\detokenize{api:module-grids.topography}]{\sphinxcrossref{\sphinxcode{grids.topography}}}} for further
details.

\end{itemize}

\item[{Keyword Arguments}] \leavevmode
\sphinxstyleliteralstrong{**kwargs} \textendash{} Keyword arguments to be forwarded to the constructor of {\hyperref[\detokenize{api:grids.topography.Topography2d}]{\sphinxcrossref{\sphinxcode{Topography2d}}}}.

\end{description}\end{quote}

\end{fulllineitems}

\index{\_update\_metric\_terms() (grids.gal\_chen.GalChen3d method)}

\begin{fulllineitems}
\phantomsection\label{\detokenize{api:grids.gal_chen.GalChen3d._update_metric_terms}}\pysiglinewithargsret{\sphinxbfcode{\_update\_metric\_terms}}{}{}
Update the class by computing the metric terms, i.e., the geometric height and the reference pressure,
at both half and main levels. In doing this, a logarithmic vertical profile of reference pressure is assumed.
This method should be called every time the topography is updated or changed.

\end{fulllineitems}

\index{update\_topography() (grids.gal\_chen.GalChen3d method)}

\begin{fulllineitems}
\phantomsection\label{\detokenize{api:grids.gal_chen.GalChen3d.update_topography}}\pysiglinewithargsret{\sphinxbfcode{update\_topography}}{\emph{time}}{}
Update the (time-dependent) topography. In turn, the metric terms are re-computed.
\begin{quote}\begin{description}
\item[{Parameters}] \leavevmode
\sphinxstyleliteralstrong{time} (\sphinxstyleliteralemphasis{obj}) \textendash{} \sphinxcode{datetime.timedelta} representing the elapsed simulation time.

\end{description}\end{quote}

\end{fulllineitems}


\end{fulllineitems}

\index{SLEVE3d (class in grids.sleve)}

\begin{fulllineitems}
\phantomsection\label{\detokenize{api:grids.sleve.SLEVE3d}}\pysiglinewithargsret{\sphinxbfcode{class }\sphinxcode{grids.sleve.}\sphinxbfcode{SLEVE3d}}{\emph{domain\_x}, \emph{nx}, \emph{domain\_y}, \emph{ny}, \emph{domain\_z}, \emph{nz}, \emph{units\_x='degrees\_east'}, \emph{dims\_x='longitude'}, \emph{units\_y='degrees\_north'}, \emph{dims\_y='latitude'}, \emph{z\_interface=None}, \emph{N=100}, \emph{s1=8000.0}, \emph{s2=5000.0}, \emph{topo\_type='flat\_terrain'}, \emph{topo\_time=datetime.timedelta(0)}, \emph{**kwargs}}{}
This class inherits {\hyperref[\detokenize{api:grids.grid_xyz.GridXYZ}]{\sphinxcrossref{\sphinxcode{GridXYZ}}}} to represent a rectangular and regular computational grid
embedded in a three-dimensional terrain-following reference system, whose coordinates are:
\begin{itemize}
\item {} 
first horizontal coordinate \(x\), e.g., the longitude;

\item {} 
second horizontal coordinate \(y\), e.g., the latitude;

\item {} 
the SLEVE terrain-following coordinate \(\mu\).

\end{itemize}

The vertical coordinate \(\mu\) may be formulated to define a hybrid terrain-following coordinate system
with terrain-following coordinate lines between the surface terrain-height and \(\mu = \mu_F\), where
\(\mu\)-coordinate lines change back to flat horizontal lines.
\begin{quote}\begin{description}
\item[{Variables}] \leavevmode\begin{itemize}
\item {} 
\sphinxstyleliteralstrong{xy\_grid} (\sphinxstyleliteralemphasis{obj}) \textendash{} {\hyperref[\detokenize{api:grids.grid_xy.GridXY}]{\sphinxcrossref{\sphinxcode{GridXY}}}} representing the horizontal grid.

\item {} 
\sphinxstyleliteralstrong{z} (\sphinxstyleliteralemphasis{obj}) \textendash{} {\hyperref[\detokenize{api:grids.axis.Axis}]{\sphinxcrossref{\sphinxcode{Axis}}}} representing the \(z\)-main levels.

\item {} 
\sphinxstyleliteralstrong{z\_half\_levels} (\sphinxstyleliteralemphasis{obj}) \textendash{} {\hyperref[\detokenize{api:grids.axis.Axis}]{\sphinxcrossref{\sphinxcode{Axis}}}} representing the \(z\)-half levels.

\item {} 
\sphinxstyleliteralstrong{nz} (\sphinxstyleliteralemphasis{int}) \textendash{} Number of vertical main levels.

\item {} 
\sphinxstyleliteralstrong{dz} (\sphinxstyleliteralemphasis{float}) \textendash{} The \(z\)-spacing.

\item {} 
\sphinxstyleliteralstrong{z\_interface} (\sphinxstyleliteralemphasis{float}) \textendash{} The interface coordinate \(z_F\).

\item {} 
\sphinxstyleliteralstrong{height} (\sphinxstyleliteralemphasis{obj}) \textendash{} \sphinxhref{http://xarray.pydata.org/en/stable/generated/xarray.DataArray.html\#xarray.DataArray}{\sphinxcode{xarray.DataArray}} representing the geometric height of the main levels.

\item {} 
\sphinxstyleliteralstrong{height\_half\_levels} (\sphinxstyleliteralemphasis{obj}) \textendash{} \sphinxhref{http://xarray.pydata.org/en/stable/generated/xarray.DataArray.html\#xarray.DataArray}{\sphinxcode{xarray.DataArray}} representing the geometric height of the half levels.

\item {} 
\sphinxstyleliteralstrong{height\_interface} (\sphinxstyleliteralemphasis{float}) \textendash{} Geometric height corresponding to \(\mu = \mu_F\).

\item {} 
\sphinxstyleliteralstrong{reference\_pressure} (\sphinxstyleliteralemphasis{obj}) \textendash{} \sphinxhref{http://xarray.pydata.org/en/stable/generated/xarray.DataArray.html\#xarray.DataArray}{\sphinxcode{xarray.DataArray}} representing the reference pressure at the main levels.

\item {} 
\sphinxstyleliteralstrong{reference\_pressure\_half\_levels} (\sphinxstyleliteralemphasis{obj}) \textendash{} \sphinxhref{http://xarray.pydata.org/en/stable/generated/xarray.DataArray.html\#xarray.DataArray}{\sphinxcode{xarray.DataArray}} representing the reference pressure at the half levels.

\end{itemize}

\end{description}\end{quote}
\index{\_\_init\_\_() (grids.sleve.SLEVE3d method)}

\begin{fulllineitems}
\phantomsection\label{\detokenize{api:grids.sleve.SLEVE3d.__init__}}\pysiglinewithargsret{\sphinxbfcode{\_\_init\_\_}}{\emph{domain\_x}, \emph{nx}, \emph{domain\_y}, \emph{ny}, \emph{domain\_z}, \emph{nz}, \emph{units\_x='degrees\_east'}, \emph{dims\_x='longitude'}, \emph{units\_y='degrees\_north'}, \emph{dims\_y='latitude'}, \emph{z\_interface=None}, \emph{N=100}, \emph{s1=8000.0}, \emph{s2=5000.0}, \emph{topo\_type='flat\_terrain'}, \emph{topo\_time=datetime.timedelta(0)}, \emph{**kwargs}}{}
Constructor.
\begin{quote}\begin{description}
\item[{Parameters}] \leavevmode\begin{itemize}
\item {} 
\sphinxstyleliteralstrong{domain\_x} (\sphinxstyleliteralemphasis{tuple}) \textendash{} Tuple in the form \((x_{left}, ~ x_{right})\).

\item {} 
\sphinxstyleliteralstrong{nx} (\sphinxstyleliteralemphasis{int}) \textendash{} Number of grid points in the \(x\)-direction.

\item {} 
\sphinxstyleliteralstrong{domain\_y} (\sphinxstyleliteralemphasis{tuple}) \textendash{} Tuple in the form \((y_{left}, ~ y_{right})\).

\item {} 
\sphinxstyleliteralstrong{ny} (\sphinxstyleliteralemphasis{int}) \textendash{} Number of grid points in the \(y\)-direction.

\item {} 
\sphinxstyleliteralstrong{domain\_z} (\sphinxstyleliteralemphasis{tuple}) \textendash{} Tuple in the form \((\mu_{top}, ~ \mu_{surface})\).

\item {} 
\sphinxstyleliteralstrong{nz} (\sphinxstyleliteralemphasis{int}) \textendash{} Number of vertical main levels.

\item {} 
\sphinxstyleliteralstrong{units\_x} (\sphinxtitleref{str}, optional) \textendash{} 
Units for the \(x\)-coordinate. Must be compliant with the \sphinxhref{http://cfconventions.org}{CF Conventions}
(see also {\hyperref[\detokenize{api:grids.axis.Axis.__init__}]{\sphinxcrossref{\sphinxcode{grids.axis.Axis.\_\_init\_\_()}}}}).


\item {} 
\sphinxstyleliteralstrong{dims\_x} (\sphinxtitleref{str}, optional) \textendash{} Label for the \(x\)-coordinate.

\item {} 
\sphinxstyleliteralstrong{units\_y} (\sphinxtitleref{str}, optional) \textendash{} 
Units for the \(y\)-coordinate. Must be compliant with the \sphinxhref{http://cfconventions.org}{CF Conventions}
(see also {\hyperref[\detokenize{api:grids.axis.Axis.__init__}]{\sphinxcrossref{\sphinxcode{grids.axis.Axis.\_\_init\_\_()}}}}).


\item {} 
\sphinxstyleliteralstrong{dims\_y} (\sphinxtitleref{str}, optional) \textendash{} Label for the \(y\)-coordinate.

\item {} 
\sphinxstyleliteralstrong{z\_interface} (\sphinxtitleref{float}, optional) \textendash{} Interface value \(zmu_F = \mu_F\). If not specified, it is assumed that \(\mu_F = \mu_T\), with
\(\mu_T\) the value of \(\mu\) at the top of the domain. In other words, a fully terrain-following
coordinate nsystem is supposed.

\item {} 
\sphinxstyleliteralstrong{N} (\sphinxtitleref{int}, optional) \textendash{} Number of filter iterations performed to determine the large-scale component of the surface terrain-height.
Defaults to 100.

\item {} 
\sphinxstyleliteralstrong{s1} (\sphinxtitleref{float}, optional) \textendash{} Large-scale decay constant. Defaults to \(8000 ~ m\).

\item {} 
\sphinxstyleliteralstrong{s2} (\sphinxtitleref{float}, optional) \textendash{} Small-scale decay constant. Defaults to \(5000 ~ m\).

\item {} 
\sphinxstyleliteralstrong{topo\_type} (\sphinxtitleref{str}, optional) \textendash{} Topography type. Defaults to ‘flat\_terrain’. See {\hyperref[\detokenize{api:module-grids.topography}]{\sphinxcrossref{\sphinxcode{grids.topography}}}} for further details.

\item {} 
\sphinxstyleliteralstrong{topo\_time} (\sphinxtitleref{obj}, optional) \textendash{} \sphinxcode{datetime.timedelta} representing the simulation time after which the topography should stop
increasing. Default is 0, corresponding to a time-invariant terrain surface-height. See {\hyperref[\detokenize{api:module-grids.topography}]{\sphinxcrossref{\sphinxcode{grids.topography}}}}
for further details.

\end{itemize}

\item[{Keyword Arguments}] \leavevmode
\sphinxstyleliteralstrong{**kwargs} \textendash{} Keyword arguments to be forwarded to the constructor of {\hyperref[\detokenize{api:grids.topography.Topography2d}]{\sphinxcrossref{\sphinxcode{Topography2d}}}}.

\end{description}\end{quote}

\end{fulllineitems}

\index{\_update\_metric\_terms() (grids.sleve.SLEVE3d method)}

\begin{fulllineitems}
\phantomsection\label{\detokenize{api:grids.sleve.SLEVE3d._update_metric_terms}}\pysiglinewithargsret{\sphinxbfcode{\_update\_metric\_terms}}{}{}
Update the class by computing the metric terms, i.e., the geometric height and the reference pressure,
at both half and main levels. In doing this, a logarithmic vertical profile of reference pressure is assumed.
This method should be called every time the topography is updated or changed.

\end{fulllineitems}

\index{update\_topography() (grids.sleve.SLEVE3d method)}

\begin{fulllineitems}
\phantomsection\label{\detokenize{api:grids.sleve.SLEVE3d.update_topography}}\pysiglinewithargsret{\sphinxbfcode{update\_topography}}{\emph{time}}{}
Update the (time-dependent) topography. In turn, the metric terms are re-computed.
\begin{quote}\begin{description}
\item[{Parameters}] \leavevmode
\sphinxstyleliteralstrong{time} (\sphinxstyleliteralemphasis{obj}) \textendash{} \sphinxcode{datetime.timedelta} representing the elapsed simulation time.

\end{description}\end{quote}

\end{fulllineitems}


\end{fulllineitems}



\section{Model}
\label{\detokenize{api:model}}\index{Model (class in model)}

\begin{fulllineitems}
\phantomsection\label{\detokenize{api:model.Model}}\pysiglinewithargsret{\sphinxbfcode{class }\sphinxcode{model.}\sphinxbfcode{Model}}{\emph{dynamical\_core}, \emph{diagnostics={[}{]}}}{}
This class is intended to represent and run a generic climate or meteorological numerical model.
\index{\_\_call\_\_() (model.Model method)}

\begin{fulllineitems}
\phantomsection\label{\detokenize{api:model.Model.__call__}}\pysiglinewithargsret{\sphinxbfcode{\_\_call\_\_}}{\emph{dt}, \emph{simulation\_time}, \emph{state}, \emph{save\_iterations={[}{]}}}{}
Call operator integrating the model forward in time.
\begin{quote}\begin{description}
\item[{Parameters}] \leavevmode\begin{itemize}
\item {} 
\sphinxstyleliteralstrong{dt} (\sphinxstyleliteralemphasis{obj}) \textendash{} \sphinxcode{datetime.timedelta} representing the time step.

\item {} 
\sphinxstyleliteralstrong{simulation\_time} (\sphinxstyleliteralemphasis{obj}) \textendash{} \sphinxcode{datetime.timedelta} representing the simulation time.

\item {} 
\sphinxstyleliteralstrong{state} (\sphinxstyleliteralemphasis{obj}) \textendash{} The initial state, as an instance of {\hyperref[\detokenize{api:storages.grid_data.GridData}]{\sphinxcrossref{\sphinxcode{GridData}}}} or one of its derived classes.

\item {} 
\sphinxstyleliteralstrong{save\_freq} (\sphinxtitleref{tuple}, optional) \textendash{} The iterations at which the state should be saved. Default is empty, meaning that only the initial and
final states are saved.

\end{itemize}

\item[{Returns}] \leavevmode
\begin{itemize}
\item {} 
\sphinxstylestrong{state\_out} (\sphinxstyleemphasis{obj}) \textendash{} The final state, of the same class of \sphinxcode{state}.

\item {} 
\sphinxstylestrong{state\_save} (\sphinxstyleemphasis{obj}) \textendash{} The sequence of saved states, of the same class of \sphinxcode{state}.

\end{itemize}


\end{description}\end{quote}

\end{fulllineitems}

\index{\_\_init\_\_() (model.Model method)}

\begin{fulllineitems}
\phantomsection\label{\detokenize{api:model.Model.__init__}}\pysiglinewithargsret{\sphinxbfcode{\_\_init\_\_}}{\emph{dynamical\_core}, \emph{diagnostics={[}{]}}}{}
Constructor.
\begin{quote}\begin{description}
\item[{Parameters}] \leavevmode\begin{itemize}
\item {} 
\sphinxstyleliteralstrong{dynamical\_core} (\sphinxstyleliteralemphasis{obj}) \textendash{} An instance of {\hyperref[\detokenize{api:dycore.dycore.DynamicalCore}]{\sphinxcrossref{\sphinxcode{DynamicalCore}}}} or one of its derived classes, implementing a dynamical core.

\item {} 
\sphinxstyleliteralstrong{diagnostics} (\sphinxtitleref{list}, optional) \textendash{} List of diagnostics. Default is empty.

\end{itemize}

\end{description}\end{quote}

\end{fulllineitems}


\end{fulllineitems}



\section{Namelist}
\label{\detokenize{api:namelist}}\label{\detokenize{api:module-namelist}}\index{namelist (module)}
Configuration and global variables used throughout the package.
\begin{description}
\item[{Physical constants:}] \leavevmode\begin{itemize}
\item {} 
\sphinxcode{namelist.p\_ref}: Reference pressure ({[}\(Pa\){]}).

\item {} 
\sphinxcode{namelist.p\_sl}: Reference pressure at sea level ({[}\(Pa\){]}).

\item {} 
\sphinxcode{namelist.T\_sl}: Reference temperature at sea level ({[}\(K\){]}).

\item {} 
\sphinxcode{namelist.beta}: Rate of increase in reference temperature with the logarithm           of reference pressure ({[}\(K ~ Pa^{-1}\){]}).

\item {} 
\sphinxcode{namelist.Rd}: Gas constant for dry airi ({[}\(J ~ K^{-1} ~ Kg^{-1}\){]}).

\item {} 
\sphinxcode{namelist.cp}: Specific heat of dry air at constant pressure ({[}\(J ~ K^{-1} ~ Kg^{-1}\){]}).

\item {} 
\sphinxcode{namelist.g}: Mean gravitational acceleration ({[}\(m ~ s^{-2}\){]}).

\end{itemize}

\item[{Grid settings:}] \leavevmode\begin{itemize}
\item {} 
\sphinxcode{namelist.domain\_x}: Tuple storing the boundaries of the domain in the \(x\)-direction              in the form (\(x_{west}\), \(x_{east}\)).

\item {} 
\sphinxcode{namelist.nx}: Number of grid points in the \(x\)-direction.

\item {} 
\sphinxcode{namelist.domain\_y}: Tuple storing the boundaries of the domain in the \(y\)-direction              in the form (\(y_{south}\), \(y_{north}\)).

\end{itemize}

\end{description}
\begin{itemize}
\item {} 
\sphinxcode{namelist.ny}: Number of grid points in the \(y\)-direction.

\item {} 
\sphinxcode{namelist.domain\_z}: Tuple storing the boundaries of the domain in the \(z\)-direction      in the form (\(z_{top}\), \(z_{bottom}\)).

\item {} 
\sphinxcode{namelist.nz}: Number of grid points in the \(z\)-direction.

\item {} 
\sphinxcode{namelist.z\_interface}: For a hybrid coordinate system, interface level at which terrain-following      \(z\)-coordinate lines get back to horizontal lines.

\item {} 
\sphinxcode{namelist.topo\_type}: Topography type. Available options are:
\begin{itemize}
\item {} 
‘flat\_terrain’;

\item {} 
‘gaussian’;

\item {} 
‘schaer’;

\item {} 
‘user\_defined’.

\end{itemize}

\item {} 
\sphinxcode{namelist.topo\_time}: \sphinxcode{datetime.timedelta} object representing the elapsed simulation time                       after which the topography should stop increasing.

\item {} 
\sphinxcode{namelist.topo\_max\_height}: When \sphinxcode{topo\_type} is ‘gaussian’, maximum mountain height ({[}\(m\){]}).

\item {} 
\sphinxcode{namelist.topo\_width\_x}: When \sphinxcode{topo\_type} is ‘gaussian’, mountain half-width in \(x\)-direction     ({[}\(m\){]}).

\item {} 
\sphinxcode{namelist.topo\_width\_y}: When \sphinxcode{topo\_type} is ‘gaussian’, mountain half-width in \(y\)-direction     ({[}\(m\){]}).

\item {} 
\sphinxcode{namelist.topo\_str}: When \sphinxcode{topo\_type} is ‘user\_defined’, terrain profile expression in the independent  variables \(x\) and \(y\). Must be fully C++-compliant.

\item {} 
\sphinxcode{namelist.topo\_kwargs}: Dictionary storing \sphinxcode{topo\_max\_height}, \sphinxcode{topo\_width\_x}, \sphinxcode{topo\_width\_y} and \sphinxcode{topo\_str}.

\end{itemize}

Model settings:
* \sphinxcode{namelist.model\_name}: Name of the model to implement. Available options are:
\begin{itemize}
\item {} 
‘isentropic’, for the isentropic model.

\end{itemize}
\begin{itemize}
\item {} 
\sphinxcode{namelist.imoist}: \sphinxcode{True} if water constituents should be taken into account, \sphinxcode{False} otherwise.

\item {} 
\sphinxcode{namelist.horizontal\_boundary\_type}: Horizontal boundary conditions. Available options are:
\begin{itemize}
\item {} 
‘periodic’, for periodic boundary conditions;

\item {} 
‘relaxed’, for relaxed boundary conditions.

\end{itemize}

\end{itemize}

Numerical settings:
* \sphinxcode{namelist.scheme}: Numerical scheme to implement. For the isentropic model, available options are:
\begin{itemize}
\item {} 
‘upwind’, for the first-order upwind scheme;

\item {} 
‘leapfrog’, for the second-order leapfrog scheme;

\item {} 
‘maccormack’, for the second-order maccormack scheme.

\end{itemize}
\begin{itemize}
\item {} 
\sphinxcode{namelist.idamp}: \sphinxcode{True} if (explicit) vertical damping should be applied, \sphinxcode{False} otherwise.       Note that when vertical damping is switched off, the numerical diffusion is monotonically increased towards     the top of the model, so to act as a diffusive wave absorber.

\item {} 
\sphinxcode{namelist.damp\_type}: Type of vertical damping to apply. Available options are:
\begin{itemize}
\item {} 
‘rayleigh’, for Rayleigh vertical damping.

\end{itemize}

\item {} 
\sphinxcode{namelist.damp\_depth}: Number of levels (either main levels or half levels) in the absorbing region.

\item {} 
\sphinxcode{namelist.damp\_max}: Maximum value which should be assumed by the damping coefficient.

\item {} 
\sphinxcode{namelist.idiff}: \sphinxcode{True} to add numerical horizontal diffusion, \sphinxcode{False} otherwise.

\item {} 
\sphinxcode{namelist.diff\_coeff}: The diffusion coefficient, i.e., the diffusivity.

\item {} 
\sphinxcode{namelist.diff\_coeff\_moist}: The diffusion coefficient, i.e., the diffusivity, for the moisture components.

\item {} 
\sphinxcode{namelist.diff\_max}: Maximum value which should be assumed by the diffusivity when diffusive vertical damping is applied.

\end{itemize}

Simulation settings:
* \sphinxcode{namelist.dt}: \sphinxcode{datetime.timedelta} object representing the timestep.
* \sphinxcode{namelist.initial\_time}: \sphinxcode{datetime.datetime} representing the initial simulation time.
* \sphinxcode{namelist.simulation\_time}: \sphinxcode{datetime.timedelta} object representing the simulation time.
* \sphinxcode{namelist.initial\_state\_type}: Integer identifying the initial state. See the documentation for the method      \sphinxcode{get\_initial\_state()} of \sphinxcode{DycoreIsentropic}.
* \sphinxcode{namelist.x\_velocity\_initial}: The initial, uniform \(x\)-velocity ({[}\(m s^{-1}\){]}).
* \sphinxcode{namelist.y\_velocity\_initial}: The initial, uniform \(y\)-velocity ({[}\(m s^{-1}\){]}).
* \sphinxcode{namelist.brunt\_vaisala\_initial}: The initial, uniform Brunt-Vaisala frequency.
* \sphinxcode{namelist.temperature\_initial}: The initial, uniform temperature ({[}\(K\){]}).
* \sphinxcode{namelist.initial\_state\_kwargs}: Dictionary storing \sphinxcode{x\_velocity\_initial}, \sphinxcode{y\_velocity\_initial}, \sphinxcode{brunt\_vaisala\_initial} and \sphinxcode{temperature\_initial}.
* \sphinxcode{namelist.backend}: GT4Py’s backend to use. Available options are:
\begin{itemize}
\item {} 
\sphinxcode{gridtools.mode.NUMPY}: Numpy (i.e., vectorized) backend.

\end{itemize}
\begin{itemize}
\item {} 
\sphinxcode{namelist.save\_iterations}: List of the iterations at which the state should be saved.

\item {} 
\sphinxcode{namelist.save\_dest}: Path to the location where results should be saved.

\item {} 
\sphinxcode{namelist.tol}: Tolerance used to compare floats (see {\hyperref[\detokenize{api:module-utils}]{\sphinxcrossref{\sphinxcode{utils}}}}).

\item {} 
\sphinxcode{namelist.datatype}: Datatype for \sphinxhref{https://docs.scipy.org/doc/numpy-1.13.0/reference/generated/numpy.ndarray.html\#numpy.ndarray}{\sphinxcode{numpy.ndarray}}. Either \sphinxcode{np.float32}     or \sphinxcode{np.float64}.

\end{itemize}


\section{Storages}
\label{\detokenize{api:storages}}\index{GridData (class in storages.grid\_data)}

\begin{fulllineitems}
\phantomsection\label{\detokenize{api:storages.grid_data.GridData}}\pysiglinewithargsret{\sphinxbfcode{class }\sphinxcode{storages.grid\_data.}\sphinxbfcode{GridData}}{\emph{time}, \emph{grid}, \emph{**kwargs}}{}
Class storing and handling time-dependent variables defined on a grid. Ideally, this class should be used to
represent the state, or a sequence of states at different time levels, of a \sphinxstyleemphasis{generic} climate or meteorological model.
The model variables, in the shape of \sphinxhref{https://docs.scipy.org/doc/numpy-1.13.0/reference/generated/numpy.ndarray.html\#numpy.ndarray}{\sphinxcode{numpy.ndarray}}s, are passed to the constructor as keyword arguments.
After conversion to \sphinxhref{http://xarray.pydata.org/en/stable/generated/xarray.DataArray.html\#xarray.DataArray}{\sphinxcode{xarray.DataArray}}s, the variables are packed in a dictionary whose keys are the input keywords.
The class attribute \sphinxcode{units} lists, for any admissible keyword, the units in which the associated field should
be expressed. Any variable can be accessed in read-only mode via the accessor operator by specifying the corresponding
keyword. Other methods are provided to update the state, or to create a sequence of states (useful for animation purposes).
This class is designed to be as general as possible. Hence, it is not endowed with any method whose
implementation depends on the variables actually stored by the class. This kind of methods might be provided by some
derived classes, each one representing the state of a \sphinxstyleemphasis{specific} model.
\begin{quote}\begin{description}
\item[{Variables}] \leavevmode
\sphinxstyleliteralstrong{grid} (\sphinxstyleliteralemphasis{obj}) \textendash{} The underlying grid, as an instance of {\hyperref[\detokenize{api:grids.grid_xyz.GridXYZ}]{\sphinxcrossref{\sphinxcode{GridXYZ}}}} or one of its derived classes.

\end{description}\end{quote}
\index{\_\_getitem\_\_() (storages.grid\_data.GridData method)}

\begin{fulllineitems}
\phantomsection\label{\detokenize{api:storages.grid_data.GridData.__getitem__}}\pysiglinewithargsret{\sphinxbfcode{\_\_getitem\_\_}}{\emph{key}}{}
Get a deep copy of a gridded variable.
\begin{quote}\begin{description}
\item[{Parameters}] \leavevmode
\sphinxstyleliteralstrong{key} (\sphinxstyleliteralemphasis{str}) \textendash{} The key corresponding to the variable to access.

\item[{Returns}] \leavevmode
Deep copy of the \sphinxhref{http://xarray.pydata.org/en/stable/generated/xarray.DataArray.html\#xarray.DataArray}{\sphinxcode{xarray.DataArray}} representing the variable.

\item[{Return type}] \leavevmode
obj

\end{description}\end{quote}

\end{fulllineitems}

\index{\_\_init\_\_() (storages.grid\_data.GridData method)}

\begin{fulllineitems}
\phantomsection\label{\detokenize{api:storages.grid_data.GridData.__init__}}\pysiglinewithargsret{\sphinxbfcode{\_\_init\_\_}}{\emph{time}, \emph{grid}, \emph{**kwargs}}{}
Constructor.
\begin{quote}\begin{description}
\item[{Parameters}] \leavevmode\begin{itemize}
\item {} 
\sphinxstyleliteralstrong{time} (\sphinxstyleliteralemphasis{obj}) \textendash{} \sphinxcode{datetime.datetime} representing the time instant at which the variables are defined.

\item {} 
\sphinxstyleliteralstrong{grid} (\sphinxstyleliteralemphasis{obj}) \textendash{} The underlying grid, as an instance of {\hyperref[\detokenize{api:grids.grid_xyz.GridXYZ}]{\sphinxcrossref{\sphinxcode{GridXYZ}}}} or one of its derived classes.

\item {} 
\sphinxstyleliteralstrong{**kwargs} (\sphinxstyleliteralemphasis{array\_like}) \textendash{} \sphinxhref{https://docs.scipy.org/doc/numpy-1.13.0/reference/generated/numpy.ndarray.html\#numpy.ndarray}{\sphinxcode{numpy.ndarray}} representing a gridded variable.

\end{itemize}

\end{description}\end{quote}

\end{fulllineitems}

\index{append() (storages.grid\_data.GridData method)}

\begin{fulllineitems}
\phantomsection\label{\detokenize{api:storages.grid_data.GridData.append}}\pysiglinewithargsret{\sphinxbfcode{append}}{\emph{other}}{}
Append a new state to the sequence of states.
\begin{quote}\begin{description}
\item[{Parameters}] \leavevmode
\sphinxstyleliteralstrong{other} (\sphinxstyleliteralemphasis{obj}) \textendash{} Another {\hyperref[\detokenize{api:storages.grid_data.GridData}]{\sphinxcrossref{\sphinxcode{GridData}}}} (or a derived class), whose \sphinxhref{http://xarray.pydata.org/en/stable/generated/xarray.DataArray.html\#xarray.DataArray}{\sphinxcode{xarray.DataArray}}s
will be concatenated along the temporal axis to the corresponding ones in the current object.

\end{description}\end{quote}

\begin{sphinxadmonition}{note}{Note:}
\sphinxcode{other} is supposed to contain exactly the same variables stored by the current object.
\end{sphinxadmonition}

\end{fulllineitems}

\index{get\_max() (storages.grid\_data.GridData method)}

\begin{fulllineitems}
\phantomsection\label{\detokenize{api:storages.grid_data.GridData.get_max}}\pysiglinewithargsret{\sphinxbfcode{get\_max}}{\emph{key}}{}
Get the maximum value of a variable.
\begin{quote}\begin{description}
\item[{Parameters}] \leavevmode
\sphinxstyleliteralstrong{key} (\sphinxstyleliteralemphasis{str}) \textendash{} Key identifying the variable of interest.

\item[{Returns}] \leavevmode
The maximum value of the variable of interest.

\item[{Return type}] \leavevmode
float

\end{description}\end{quote}

\end{fulllineitems}

\index{get\_min() (storages.grid\_data.GridData method)}

\begin{fulllineitems}
\phantomsection\label{\detokenize{api:storages.grid_data.GridData.get_min}}\pysiglinewithargsret{\sphinxbfcode{get\_min}}{\emph{key}}{}
Get the minimum value of a variable.
\begin{quote}\begin{description}
\item[{Parameters}] \leavevmode
\sphinxstyleliteralstrong{key} (\sphinxstyleliteralemphasis{str}) \textendash{} Key identifying the variable of interest.

\item[{Returns}] \leavevmode
The minimum value of the variable of interest.

\item[{Return type}] \leavevmode
float

\end{description}\end{quote}

\end{fulllineitems}

\index{time (storages.grid\_data.GridData attribute)}

\begin{fulllineitems}
\phantomsection\label{\detokenize{api:storages.grid_data.GridData.time}}\pysigline{\sphinxbfcode{time}}
Return the time at which the data are defined.
\begin{quote}\begin{description}
\item[{Returns}] \leavevmode
\sphinxcode{datetime.datetime} representing the time at which the data are defined.

\item[{Return type}] \leavevmode
obj

\end{description}\end{quote}

\end{fulllineitems}

\index{update() (storages.grid\_data.GridData method)}

\begin{fulllineitems}
\phantomsection\label{\detokenize{api:storages.grid_data.GridData.update}}\pysiglinewithargsret{\sphinxbfcode{update}}{\emph{other}}{}
Update (some of) the stored variables by syncing the current object with another {\hyperref[\detokenize{api:storages.grid_data.GridData}]{\sphinxcrossref{\sphinxcode{GridData}}}}
(or a derived class).
\begin{quote}\begin{description}
\item[{Parameters}] \leavevmode
\sphinxstyleliteralstrong{other} (\sphinxstyleliteralemphasis{obj}) \textendash{} Another {\hyperref[\detokenize{api:storages.grid_data.GridData}]{\sphinxcrossref{\sphinxcode{GridData}}}} (or a derived class) with which the current object will be synced.

\end{description}\end{quote}

\begin{sphinxadmonition}{note}{Note:}
\sphinxcode{other} is not required to contain \sphinxstyleemphasis{all} the variables stored by the current object, yet it cannot contain
variables not included in the current object.
\end{sphinxadmonition}

\end{fulllineitems}


\end{fulllineitems}

\index{StateIsentropic (class in storages.state\_isentropic)}

\begin{fulllineitems}
\phantomsection\label{\detokenize{api:storages.state_isentropic.StateIsentropic}}\pysiglinewithargsret{\sphinxbfcode{class }\sphinxcode{storages.state\_isentropic.}\sphinxbfcode{StateIsentropic}}{\emph{time}, \emph{grid}, \emph{isentropic\_density}, \emph{x\_velocity}, \emph{x\_momentum}, \emph{y\_velocity}, \emph{y\_momentum}, \emph{pressure}, \emph{exner\_function}, \emph{montgomery\_potential}, \emph{height}, \emph{water\_vapour=None}, \emph{cloud\_water=None}, \emph{precipitation\_water=None}}{}
This class inherits {\hyperref[\detokenize{api:storages.grid_data.GridData}]{\sphinxcrossref{\sphinxcode{GridData}}}} to represent the state of the three-dimensional
(moist) isentropic model.
\begin{quote}\begin{description}
\item[{Variables}] \leavevmode
\sphinxstyleliteralstrong{grid} (\sphinxstyleliteralemphasis{obj}) \textendash{} {\hyperref[\detokenize{api:grids.grid_xyz.GridXYZ}]{\sphinxcrossref{\sphinxcode{GridXYZ}}}} representing the underlying grid.

\end{description}\end{quote}
\index{\_\_init\_\_() (storages.state\_isentropic.StateIsentropic method)}

\begin{fulllineitems}
\phantomsection\label{\detokenize{api:storages.state_isentropic.StateIsentropic.__init__}}\pysiglinewithargsret{\sphinxbfcode{\_\_init\_\_}}{\emph{time}, \emph{grid}, \emph{isentropic\_density}, \emph{x\_velocity}, \emph{x\_momentum}, \emph{y\_velocity}, \emph{y\_momentum}, \emph{pressure}, \emph{exner\_function}, \emph{montgomery\_potential}, \emph{height}, \emph{water\_vapour=None}, \emph{cloud\_water=None}, \emph{precipitation\_water=None}}{}
Constructor.
\begin{quote}\begin{description}
\item[{Parameters}] \leavevmode\begin{itemize}
\item {} 
\sphinxstyleliteralstrong{time} (\sphinxstyleliteralemphasis{obj}) \textendash{} \sphinxcode{datetime.timedelta} representing the time instant at which the state is defined.

\item {} 
\sphinxstyleliteralstrong{grid} (\sphinxstyleliteralemphasis{obj}) \textendash{} {\hyperref[\detokenize{api:grids.grid_xyz.GridXYZ}]{\sphinxcrossref{\sphinxcode{GridXYZ}}}} representing the underlying grid.

\item {} 
\sphinxstyleliteralstrong{isentropic\_density} (\sphinxstyleliteralemphasis{array\_like}) \textendash{} \sphinxhref{https://docs.scipy.org/doc/numpy-1.13.0/reference/generated/numpy.ndarray.html\#numpy.ndarray}{\sphinxcode{numpy.ndarray}} representing the isentropic density.

\item {} 
\sphinxstyleliteralstrong{x\_velocity} (\sphinxstyleliteralemphasis{array\_like}) \textendash{} \sphinxhref{https://docs.scipy.org/doc/numpy-1.13.0/reference/generated/numpy.ndarray.html\#numpy.ndarray}{\sphinxcode{numpy.ndarray}} representing the \(x\)-velocity.

\item {} 
\sphinxstyleliteralstrong{x\_momentum} (\sphinxstyleliteralemphasis{array\_like}) \textendash{} \sphinxhref{https://docs.scipy.org/doc/numpy-1.13.0/reference/generated/numpy.ndarray.html\#numpy.ndarray}{\sphinxcode{numpy.ndarray}} representing the \(x\)-momentum.

\item {} 
\sphinxstyleliteralstrong{y\_velocity} (\sphinxstyleliteralemphasis{array\_like}) \textendash{} \sphinxhref{https://docs.scipy.org/doc/numpy-1.13.0/reference/generated/numpy.ndarray.html\#numpy.ndarray}{\sphinxcode{numpy.ndarray}} representing the \(y\)-velocity.

\item {} 
\sphinxstyleliteralstrong{y\_momentum} (\sphinxstyleliteralemphasis{array\_like}) \textendash{} \sphinxhref{https://docs.scipy.org/doc/numpy-1.13.0/reference/generated/numpy.ndarray.html\#numpy.ndarray}{\sphinxcode{numpy.ndarray}} representing the \(y\)-momentum.

\item {} 
\sphinxstyleliteralstrong{pressure} (\sphinxstyleliteralemphasis{array\_like}) \textendash{} \sphinxhref{https://docs.scipy.org/doc/numpy-1.13.0/reference/generated/numpy.ndarray.html\#numpy.ndarray}{\sphinxcode{numpy.ndarray}} representing the pressure.

\item {} 
\sphinxstyleliteralstrong{exner\_function} (\sphinxstyleliteralemphasis{array\_like}) \textendash{} \sphinxhref{https://docs.scipy.org/doc/numpy-1.13.0/reference/generated/numpy.ndarray.html\#numpy.ndarray}{\sphinxcode{numpy.ndarray}} representing the Exner function.

\item {} 
\sphinxstyleliteralstrong{montgomery\_potential} (\sphinxstyleliteralemphasis{array\_like}) \textendash{} \sphinxhref{https://docs.scipy.org/doc/numpy-1.13.0/reference/generated/numpy.ndarray.html\#numpy.ndarray}{\sphinxcode{numpy.ndarray}} representing the Montgomery potential.

\item {} 
\sphinxstyleliteralstrong{height} (\sphinxstyleliteralemphasis{array\_like}) \textendash{} \sphinxhref{https://docs.scipy.org/doc/numpy-1.13.0/reference/generated/numpy.ndarray.html\#numpy.ndarray}{\sphinxcode{numpy.ndarray}} representing the height of the potential temperature surfaces.

\item {} 
\sphinxstyleliteralstrong{water\_vapour} (\sphinxtitleref{array\_like}, optional) \textendash{} \sphinxhref{https://docs.scipy.org/doc/numpy-1.13.0/reference/generated/numpy.ndarray.html\#numpy.ndarray}{\sphinxcode{numpy.ndarray}} representing the mass fraction of water vapour.

\item {} 
\sphinxstyleliteralstrong{cloud\_water} (\sphinxtitleref{array\_like}, optional) \textendash{} \sphinxhref{https://docs.scipy.org/doc/numpy-1.13.0/reference/generated/numpy.ndarray.html\#numpy.ndarray}{\sphinxcode{numpy.ndarray}} representing the mass fraction of cloud water.

\item {} 
\sphinxstyleliteralstrong{precipitation\_water} (\sphinxtitleref{array\_like}, optional) \textendash{} \sphinxhref{https://docs.scipy.org/doc/numpy-1.13.0/reference/generated/numpy.ndarray.html\#numpy.ndarray}{\sphinxcode{numpy.ndarray}} representing the mass fraction of precipitation water.

\end{itemize}

\end{description}\end{quote}

\end{fulllineitems}

\index{contourf\_xy() (storages.state\_isentropic.StateIsentropic method)}

\begin{fulllineitems}
\phantomsection\label{\detokenize{api:storages.state_isentropic.StateIsentropic.contourf_xy}}\pysiglinewithargsret{\sphinxbfcode{contourf\_xy}}{\emph{field\_to\_plot}, \emph{z\_level}, \emph{time\_level}, \emph{**kwargs}}{}
Plot a field at the level \(\theta = \bar{\theta}\).
\begin{quote}\begin{description}
\item[{Parameters}] \leavevmode\begin{itemize}
\item {} 
\sphinxstyleliteralstrong{field\_to\_plot} (\sphinxstyleliteralemphasis{str}) \textendash{} 
String specifying the field to plot. This might be:
\begin{itemize}
\item {} 
the name of a variable stored in the current object.

\item {} 
’horizontal\_velocity’, for the horizontal velocity.

\end{itemize}


\item {} 
\sphinxstyleliteralstrong{z\_level} (\sphinxstyleliteralemphasis{int}) \textendash{} Index corresponding to the \(\theta\)-level identifying the cross-section to plot.

\item {} 
\sphinxstyleliteralstrong{time\_level} (\sphinxstyleliteralemphasis{int}) \textendash{} The index corresponding to the time level to plot.

\end{itemize}

\item[{Keyword Arguments}] \leavevmode\begin{itemize}
\item {} 
\sphinxstyleliteralstrong{figsize} (\sphinxstyleliteralemphasis{tuple}) \textendash{} Figure size. Default is {[}8,8{]}.

\item {} 
\sphinxstyleliteralstrong{title} (\sphinxstyleliteralemphasis{str}) \textendash{} The title for the plot. By default, it coincides with \sphinxcode{field\_to\_plot}.

\item {} 
\sphinxstyleliteralstrong{x\_label} (\sphinxstyleliteralemphasis{str}) \textendash{} The label for the \sphinxcode{x}-axis of the plot.

\item {} 
\sphinxstyleliteralstrong{x\_factor} (\sphinxstyleliteralemphasis{float}) \textendash{} Factor for the \sphinxcode{x}-axis of the plot. Default is 1.

\item {} 
\sphinxstyleliteralstrong{y\_label} (\sphinxstyleliteralemphasis{str}) \textendash{} The label for the \sphinxcode{y}-axis of the plot.

\item {} 
\sphinxstyleliteralstrong{y\_factor} (\sphinxstyleliteralemphasis{float}) \textendash{} Factor for the \sphinxcode{y}-axis of the plot. Default is 1.

\item {} 
\sphinxstyleliteralstrong{cmap\_name} (\sphinxstyleliteralemphasis{str}) \textendash{} Name of the Matplotlib’s colormap to use. Default is ‘RdYlBu’.

\item {} 
\sphinxstyleliteralstrong{cmap\_levels} (\sphinxstyleliteralemphasis{int}) \textendash{} Number of levels for the colormap. Default is 31.

\item {} 
\sphinxstyleliteralstrong{cmap\_center} (\sphinxstyleliteralemphasis{float}) \textendash{} The central value for the colormap. If not specified, the colormap ranges from the minimum to the maximum
of the field to plot.

\item {} 
\sphinxstyleliteralstrong{cmap\_half\_width} (\sphinxstyleliteralemphasis{float}) \textendash{} Half width of the colormap range. If not specified, the colormap ranges from the minimum to the maximum
of the field to plot.

\end{itemize}

\end{description}\end{quote}

\end{fulllineitems}

\index{contourf\_xz() (storages.state\_isentropic.StateIsentropic method)}

\begin{fulllineitems}
\phantomsection\label{\detokenize{api:storages.state_isentropic.StateIsentropic.contourf_xz}}\pysiglinewithargsret{\sphinxbfcode{contourf\_xz}}{\emph{field\_to\_plot}, \emph{y\_level}, \emph{time\_level}, \emph{**kwargs}}{}
Plot a field in the cross-section \(y = \bar{y}\).
\begin{quote}\begin{description}
\item[{Parameters}] \leavevmode\begin{itemize}
\item {} 
\sphinxstyleliteralstrong{field\_to\_plot} (\sphinxstyleliteralemphasis{str}) \textendash{} 
String specifying the field to plot. This might be:
\begin{itemize}
\item {} 
the name of a variable stored in the current object;

\item {} 
’vertical\_velocity’, for the vertical velocity.

\item {} 
’temperature’, for the temperature.

\end{itemize}


\item {} 
\sphinxstyleliteralstrong{y\_level} (\sphinxstyleliteralemphasis{int}) \textendash{} Index corresponding to the \(y\)-level identifying the cross-section to plot.

\item {} 
\sphinxstyleliteralstrong{time\_level} (\sphinxstyleliteralemphasis{int}) \textendash{} The index corresponding to the time level to plot.

\end{itemize}

\item[{Keyword Arguments}] \leavevmode\begin{itemize}
\item {} 
\sphinxstyleliteralstrong{figsize} (\sphinxstyleliteralemphasis{tuple}) \textendash{} Figure size. Default is {[}8,8{]}.

\item {} 
\sphinxstyleliteralstrong{title} (\sphinxstyleliteralemphasis{str}) \textendash{} The title for the plot. By default, it coincides with \sphinxcode{field\_to\_plot}.

\item {} 
\sphinxstyleliteralstrong{x\_label} (\sphinxstyleliteralemphasis{str}) \textendash{} The label for the \sphinxcode{x}-axis of the plot.

\item {} 
\sphinxstyleliteralstrong{x\_factor} (\sphinxstyleliteralemphasis{float}) \textendash{} Factor for the \sphinxcode{x}-axis of the plot. Default is 1.

\item {} 
\sphinxstyleliteralstrong{z\_label} (\sphinxstyleliteralemphasis{str}) \textendash{} The label for the \sphinxcode{y}-axis of the plot, i.e., the \(z\)-axis.

\item {} 
\sphinxstyleliteralstrong{z\_factor} (\sphinxstyleliteralemphasis{float}) \textendash{} Factor for the \sphinxcode{y}-axis of the plot. Default is 1.

\item {} 
\sphinxstyleliteralstrong{cmap\_name} (\sphinxstyleliteralemphasis{str}) \textendash{} Name of the Matplotlib’s colormap to use. Default is ‘RdYlBu’.

\item {} 
\sphinxstyleliteralstrong{cmap\_levels} (\sphinxstyleliteralemphasis{int}) \textendash{} Number of levels for the colormap. Default is 31.

\item {} 
\sphinxstyleliteralstrong{cmap\_center} (\sphinxstyleliteralemphasis{float}) \textendash{} The central value for the colormap. If not specified, the colormap ranges from the minimum to the maximum
of the field to plot.

\item {} 
\sphinxstyleliteralstrong{cmap\_half\_width} (\sphinxstyleliteralemphasis{float}) \textendash{} Half width of the colormap range. If not specified, the colormap ranges from the minimum to the maximum
of the field to plot.

\end{itemize}

\end{description}\end{quote}

\end{fulllineitems}

\index{get\_cfl() (storages.state\_isentropic.StateIsentropic method)}

\begin{fulllineitems}
\phantomsection\label{\detokenize{api:storages.state_isentropic.StateIsentropic.get_cfl}}\pysiglinewithargsret{\sphinxbfcode{get\_cfl}}{\emph{dt}}{}
Compute the CFL number.
\begin{quote}\begin{description}
\item[{Parameters}] \leavevmode
\sphinxstyleliteralstrong{dt} (\sphinxstyleliteralemphasis{obj}) \textendash{} \sphinxcode{datetime.timedelta} representing the time step.

\item[{Returns}] \leavevmode
The CFL number.

\item[{Return type}] \leavevmode
float

\end{description}\end{quote}

\begin{sphinxadmonition}{note}{Note:}
If the CFL number exceeds the unity, i.e., if the CFL condition is violated, the method throws a warning.
\end{sphinxadmonition}

\end{fulllineitems}

\index{quiver\_xy() (storages.state\_isentropic.StateIsentropic method)}

\begin{fulllineitems}
\phantomsection\label{\detokenize{api:storages.state_isentropic.StateIsentropic.quiver_xy}}\pysiglinewithargsret{\sphinxbfcode{quiver\_xy}}{\emph{field\_to\_plot}, \emph{z\_level}, \emph{time\_level}, \emph{**kwargs}}{}
Generate the quiver-plot for a vector field at the level \(\theta = \bar{\theta}\).
\begin{quote}\begin{description}
\item[{Parameters}] \leavevmode\begin{itemize}
\item {} 
\sphinxstyleliteralstrong{field\_to\_plot} (\sphinxstyleliteralemphasis{str}) \textendash{} 
String specifying the field to plot. This might be:
\begin{itemize}
\item {} 
’horizontal\_velocity’, for the horizontal velocity.

\end{itemize}


\item {} 
\sphinxstyleliteralstrong{z\_level} (\sphinxstyleliteralemphasis{int}) \textendash{} Index corresponding to the \(\theta\)-level identifying the cross-section to plot.

\item {} 
\sphinxstyleliteralstrong{time\_level} (\sphinxstyleliteralemphasis{int}) \textendash{} The index corresponding to the time level to plot.

\end{itemize}

\item[{Keyword Arguments}] \leavevmode\begin{itemize}
\item {} 
\sphinxstyleliteralstrong{figsize} (\sphinxstyleliteralemphasis{tuple}) \textendash{} Figure size. Default is {[}8,8{]}.

\item {} 
\sphinxstyleliteralstrong{title} (\sphinxstyleliteralemphasis{str}) \textendash{} The title for the plot. By default, it coincides with \sphinxcode{field\_to\_plot}.

\item {} 
\sphinxstyleliteralstrong{x\_label} (\sphinxstyleliteralemphasis{str}) \textendash{} The label for the \sphinxcode{x}-axis of the plot.

\item {} 
\sphinxstyleliteralstrong{x\_factor} (\sphinxstyleliteralemphasis{float}) \textendash{} Factor for the \sphinxcode{x}-axis of the plot. Default is 1.

\item {} 
\sphinxstyleliteralstrong{y\_label} (\sphinxstyleliteralemphasis{str}) \textendash{} The label for the \sphinxcode{y}-axis of the plot.

\item {} 
\sphinxstyleliteralstrong{y\_factor} (\sphinxstyleliteralemphasis{float}) \textendash{} Factor for the \sphinxcode{y}-axis of the plot. Default is 1.

\item {} 
\sphinxstyleliteralstrong{cmap\_name} (\sphinxstyleliteralemphasis{str}) \textendash{} Name of the Matplotlib’s colormap to use. Default is ‘RdYlBu’.

\item {} 
\sphinxstyleliteralstrong{cmap\_levels} (\sphinxstyleliteralemphasis{int}) \textendash{} Number of levels for the colormap. Default is 31.

\item {} 
\sphinxstyleliteralstrong{cmap\_center} (\sphinxstyleliteralemphasis{float}) \textendash{} The central value for the colormap. If not specified, the colormap ranges from the minimum to the maximum
of the field to plot.

\item {} 
\sphinxstyleliteralstrong{cmap\_half\_width} (\sphinxstyleliteralemphasis{float}) \textendash{} Half width of the colormap range. If not specified, the colormap ranges from the minimum to the maximum
of the field to plot.

\end{itemize}

\end{description}\end{quote}

\end{fulllineitems}


\end{fulllineitems}



\section{Topography}
\label{\detokenize{api:module-grids.topography}}\label{\detokenize{api:topography}}\index{grids.topography (module)}
Classes representing one- and two-dimensional topographies, possibly time-dependent.
Indeed, although clearly not physical, a terrain surface (slowly) growing in the early stages
of a simulation may help to retrieve numerical stability, as it prevents steep gradients
in the first few iterations.

Letting \(h_s = h_s(x)\) be a one-dimensional topography, with \(x \in [a,b]\),
the user may choose among:
\begin{itemize}
\item {} 
a flat terrain, i.e., \(h_s(x) \equiv 0\);

\item {} 
a Gaussian-shaped mountain, i.e.,
\begin{quote}
\begin{equation*}
\begin{split}h_s(x) = h_{max} \exp{\left[ - \left( \frac{x - c}{\sigma_x} \right)^2 \right]},\end{split}
\end{equation*}\end{quote}

where \(c = 0.5 (a + b)\).

\end{itemize}

For the two-dimensional case, letting \(h_s = h_s(x,y)\) be the topography, with
\(x \in [a_x,b_x]\) and \(y \in [a_y,b_y]\), the following profiles are provided:
\begin{itemize}
\item {} 
flat terrain, i.e., \(h_s(x,y) \equiv 0\);

\item {} 
Gaussian shaped-mountain, i.e.
\begin{quote}
\begin{equation*}
\begin{split}h_s(x,y) = h_{max} \exp{\left[ - \left( \frac{x - c_x}{\sigma_x} \right)^2 - \left( \frac{y - c_y}{\sigma_y} \right)^2 \right]} ,\end{split}
\end{equation*}\end{quote}

where \(c_x = 0.5 (a_x + b_x)\) and \(c_y = 0.5 (a_y + b_y)\);

\item {} 
modified Gaussian-shaped mountain proposed by Schaer and Durran (1997),
\begin{quote}
\begin{equation*}
\begin{split}h_s(x,y) = \frac{h_{max}}{\left[ 1 + \left( \frac{x}{\sigma_x} \right)^2 + \left( \frac{y}{\sigma_y} \right)^2 \right]^{3/2}}.\end{split}
\end{equation*}\end{quote}

\end{itemize}

Yet, user-defined profiles are supported as well, provided that they admit an analytical expression.
This is passed to the class as a string, which is then parsed in C++ via \sphinxhref{http://cython.org}{Cython}
(see \sphinxcode{parser\_1d} and \sphinxcode{parser\_2d}). Hence, the string itself must be
fully C++-compliant.
\paragraph{References}

Schaer, C., and Durran, D. R. (1997). \sphinxstyleemphasis{Vortex formation and vortex shedding in continuosly stratified flows     past isolated topography}. Journal of Atmospheric Sciences, 54:534-554.
\index{Topography1d (class in grids.topography)}

\begin{fulllineitems}
\phantomsection\label{\detokenize{api:grids.topography.Topography1d}}\pysiglinewithargsret{\sphinxbfcode{class }\sphinxcode{grids.topography.}\sphinxbfcode{Topography1d}}{\emph{x}, \emph{topo\_type='flat\_terrain'}, \emph{topo\_time=datetime.timedelta(0)}, \emph{**kwargs}}{}
Class representing a one-dimensional topography.
\begin{quote}\begin{description}
\item[{Variables}] \leavevmode\begin{itemize}
\item {} 
\sphinxstyleliteralstrong{topo} (\sphinxstyleliteralemphasis{array\_like}) \textendash{} \sphinxhref{http://xarray.pydata.org/en/stable/generated/xarray.DataArray.html\#xarray.DataArray}{\sphinxcode{xarray.DataArray}} representing the topography (in meters).

\item {} 
\sphinxstyleliteralstrong{topo\_type} (\sphinxstyleliteralemphasis{str}) \textendash{} 
Topography type. Either:
\begin{itemize}
\item {} 
’flat\_terrain’;

\item {} 
’gaussian’;

\item {} 
’user\_defined’.

\end{itemize}


\item {} 
\sphinxstyleliteralstrong{topo\_time} (\sphinxstyleliteralemphasis{obj}) \textendash{} \sphinxcode{datetime.timedelta} object representing the elapsed simulation time after which the topography
should stop increasing.

\item {} 
\sphinxstyleliteralstrong{topo\_fact} (\sphinxstyleliteralemphasis{float}) \textendash{} Topography factor. It runs in between 0 (at the beginning of the simulation) and 1 (once the simulation
has been run for \sphinxcode{topo\_time}).

\end{itemize}

\end{description}\end{quote}
\index{\_\_init\_\_() (grids.topography.Topography1d method)}

\begin{fulllineitems}
\phantomsection\label{\detokenize{api:grids.topography.Topography1d.__init__}}\pysiglinewithargsret{\sphinxbfcode{\_\_init\_\_}}{\emph{x}, \emph{topo\_type='flat\_terrain'}, \emph{topo\_time=datetime.timedelta(0)}, \emph{**kwargs}}{}
Constructor.
\begin{quote}\begin{description}
\item[{Parameters}] \leavevmode\begin{itemize}
\item {} 
\sphinxstyleliteralstrong{x} (\sphinxstyleliteralemphasis{obj}) \textendash{} {\hyperref[\detokenize{api:grids.axis.Axis}]{\sphinxcrossref{\sphinxcode{Axis}}}} representing the underlying horizontal axis.

\item {} 
\sphinxstyleliteralstrong{topo\_type} (\sphinxtitleref{str}, optional) \textendash{} 
Topography type. Either:
\begin{itemize}
\item {} 
’flat\_terrain’ (default);

\item {} 
’gaussian’;

\item {} 
’user\_defined’.

\end{itemize}


\item {} 
\sphinxstyleliteralstrong{topo\_time} (\sphinxstyleliteralemphasis{obj}) \textendash{} class:\sphinxtitleref{datetime.timedelta} representing the elapsed simulation time after which the topography
should stop increasing. Default is 0, corresponding to a time-invariant terrain surface-height.

\end{itemize}

\item[{Keyword Arguments}] \leavevmode\begin{itemize}
\item {} 
\sphinxstyleliteralstrong{topo\_max\_height} (\sphinxstyleliteralemphasis{float}) \textendash{} When \sphinxcode{topo\_type} is ‘gaussian’, maximum mountain height (in meters). Default is 500.

\item {} 
\sphinxstyleliteralstrong{topo\_width\_x} (\sphinxstyleliteralemphasis{float}) \textendash{} When \sphinxcode{topo\_type} is ‘gaussian’, mountain half-width (in meters). Default is 10000.

\item {} 
\sphinxstyleliteralstrong{topo\_str} (\sphinxstyleliteralemphasis{str}) \textendash{} When \sphinxcode{topo\_type} is ‘user\_defined’, terrain profile expression in the independent variable \(x\).
Must be fully C++-compliant.

\end{itemize}

\end{description}\end{quote}

\end{fulllineitems}

\index{update() (grids.topography.Topography1d method)}

\begin{fulllineitems}
\phantomsection\label{\detokenize{api:grids.topography.Topography1d.update}}\pysiglinewithargsret{\sphinxbfcode{update}}{\emph{time}}{}
Update topography at current simulation time.
\begin{quote}\begin{description}
\item[{Parameters}] \leavevmode
\sphinxstyleliteralstrong{time} (\sphinxstyleliteralemphasis{obj}) \textendash{} \sphinxcode{datetime.timedelta} representing the elapsed simulation time.

\end{description}\end{quote}

\end{fulllineitems}


\end{fulllineitems}

\index{Topography2d (class in grids.topography)}

\begin{fulllineitems}
\phantomsection\label{\detokenize{api:grids.topography.Topography2d}}\pysiglinewithargsret{\sphinxbfcode{class }\sphinxcode{grids.topography.}\sphinxbfcode{Topography2d}}{\emph{grid}, \emph{topo\_type='flat\_terrain'}, \emph{topo\_time=datetime.timedelta(0)}, \emph{**kwargs}}{}
Class representing a two-dimensional topography.
\begin{quote}\begin{description}
\item[{Variables}] \leavevmode\begin{itemize}
\item {} 
\sphinxstyleliteralstrong{topo} (\sphinxstyleliteralemphasis{array\_like}) \textendash{} \sphinxhref{http://xarray.pydata.org/en/stable/generated/xarray.DataArray.html\#xarray.DataArray}{\sphinxcode{xarray.DataArray}} representing the topography (in meters).

\item {} 
\sphinxstyleliteralstrong{topo\_type} (\sphinxstyleliteralemphasis{str}) \textendash{} 
Topography type. Either:
\begin{itemize}
\item {} 
’flat\_terrain’;

\item {} 
’gaussian’;

\item {} 
’schaer’;

\item {} 
’user\_defined’.

\end{itemize}


\item {} 
\sphinxstyleliteralstrong{topo\_time} (\sphinxstyleliteralemphasis{obj}) \textendash{} \sphinxcode{datetime.timedelta} representing the elapsed simulation time after which the topography
should stop increasing.

\item {} 
\sphinxstyleliteralstrong{topo\_fact} (\sphinxstyleliteralemphasis{float}) \textendash{} Topography factor. It runs in between 0 (at the beginning of the simulation) and 1 (once the simulation
has been run for \sphinxcode{topo\_time}).

\end{itemize}

\end{description}\end{quote}
\index{\_\_init\_\_() (grids.topography.Topography2d method)}

\begin{fulllineitems}
\phantomsection\label{\detokenize{api:grids.topography.Topography2d.__init__}}\pysiglinewithargsret{\sphinxbfcode{\_\_init\_\_}}{\emph{grid}, \emph{topo\_type='flat\_terrain'}, \emph{topo\_time=datetime.timedelta(0)}, \emph{**kwargs}}{}
Constructor.
\begin{quote}\begin{description}
\item[{Parameters}] \leavevmode\begin{itemize}
\item {} 
\sphinxstyleliteralstrong{grid} (\sphinxstyleliteralemphasis{obj}) \textendash{} {\hyperref[\detokenize{api:grids.grid_xy.GridXY}]{\sphinxcrossref{\sphinxcode{GridXY}}}} representing the underlying grid.

\item {} 
\sphinxstyleliteralstrong{topo\_type} (\sphinxtitleref{str}, optional) \textendash{} 
Topography type. Either:
\begin{itemize}
\item {} 
’flat\_terrain’ (default);

\item {} 
’gaussian’;

\item {} 
’schaer’;

\item {} 
’user\_defined’.

\end{itemize}


\item {} 
\sphinxstyleliteralstrong{topo\_time} (\sphinxstyleliteralemphasis{obj}) \textendash{} \sphinxcode{datetime.timedelta} representing the elapsed simulation time after which the topography
should stop increasing. Default is 0, corresponding to a time-invariant terrain surface-height.

\end{itemize}

\item[{Keyword Arguments}] \leavevmode\begin{itemize}
\item {} 
\sphinxstyleliteralstrong{topo\_max\_height} (\sphinxstyleliteralemphasis{float}) \textendash{} When \sphinxcode{topo\_type} is either ‘gaussian’ or ‘schaer’, maximum mountain height (in meters).
Default is 500.

\item {} 
\sphinxstyleliteralstrong{topo\_width\_x} (\sphinxstyleliteralemphasis{float}) \textendash{} When \sphinxcode{topo\_type} is either ‘gaussian’ or ‘schaer’, mountain half-width in \(x\)-direction
(in meters). Default is 10000.

\item {} 
\sphinxstyleliteralstrong{topo\_width\_y} (\sphinxstyleliteralemphasis{float}) \textendash{} When \sphinxcode{topo\_type} is either ‘gaussian’ or ‘schaer’, mountain half-width in \(y\)-direction
(in meters). Default is 10000.

\item {} 
\sphinxstyleliteralstrong{topo\_str} (\sphinxstyleliteralemphasis{str}) \textendash{} When \sphinxcode{topo\_type} is ‘user\_defined’, terrain profile expression in the independent variables
\(x\) and \(y\). Must be fully C++-compliant.

\end{itemize}

\end{description}\end{quote}

\end{fulllineitems}

\index{plot() (grids.topography.Topography2d method)}

\begin{fulllineitems}
\phantomsection\label{\detokenize{api:grids.topography.Topography2d.plot}}\pysiglinewithargsret{\sphinxbfcode{plot}}{\emph{grid}, \emph{**kwargs}}{}
Plot the topography using the \sphinxhref{https://matplotlib.org/tutorials/toolkits/mplot3d.html}{mplot3d toolkit}.
\begin{quote}\begin{description}
\item[{Parameters}] \leavevmode
\sphinxstyleliteralstrong{grid} (\sphinxstyleliteralemphasis{obj}) \textendash{} {\hyperref[\detokenize{api:grids.grid_xy.GridXY}]{\sphinxcrossref{\sphinxcode{GridXY}}}} representing the underlying grid.

\item[{Keyword Arguments}] \leavevmode
\sphinxstyleliteralstrong{**kwargs} \textendash{} Keyword arguments to be forwarded to \sphinxhref{https://matplotlib.org/2.1.1/api/\_as\_gen/matplotlib.pyplot.figure.html\#matplotlib.pyplot.figure}{\sphinxcode{matplotlib.pyplot.figure()}}.

\end{description}\end{quote}

\end{fulllineitems}

\index{update() (grids.topography.Topography2d method)}

\begin{fulllineitems}
\phantomsection\label{\detokenize{api:grids.topography.Topography2d.update}}\pysiglinewithargsret{\sphinxbfcode{update}}{\emph{time}}{}
Update topography at current simulation time.
\begin{quote}\begin{description}
\item[{Parameters}] \leavevmode
\sphinxstyleliteralstrong{time} (\sphinxstyleliteralemphasis{obj}) \textendash{} \sphinxcode{datetime.timedelta} representing the elapsed simulation time.

\end{description}\end{quote}

\end{fulllineitems}


\end{fulllineitems}



\subsection{Parsers}
\label{\detokenize{api:parsers}}\index{Parser1d (class in grids.parser.parser\_1d)}

\begin{fulllineitems}
\phantomsection\label{\detokenize{api:grids.parser.parser_1d.Parser1d}}\pysigline{\sphinxbfcode{class }\sphinxcode{grids.parser.parser\_1d.}\sphinxbfcode{Parser1d}}
Cython wrapper for the C++ class \sphinxcode{parser\_1d\_cpp}.
\begin{quote}\begin{description}
\item[{Variables}] \leavevmode
\sphinxstyleliteralstrong{parser} (\sphinxstyleliteralemphasis{obj}) \textendash{} Pointer to a \sphinxcode{parser\_1d\_cpp} object.

\end{description}\end{quote}
\index{evaluate() (grids.parser.parser\_1d.Parser1d method)}

\begin{fulllineitems}
\phantomsection\label{\detokenize{api:grids.parser.parser_1d.Parser1d.evaluate}}\pysiglinewithargsret{\sphinxbfcode{evaluate}}{}{}
Evaluate the expression.
\begin{quote}\begin{description}
\item[{Returns}] \leavevmode
\sphinxhref{https://docs.scipy.org/doc/numpy-1.13.0/reference/generated/numpy.ndarray.html\#numpy.ndarray}{\sphinxcode{numpy.ndarray}} of the evaluations.

\end{description}\end{quote}

\end{fulllineitems}


\end{fulllineitems}

\index{Parser2d (class in grids.parser.parser\_2d)}

\begin{fulllineitems}
\phantomsection\label{\detokenize{api:grids.parser.parser_2d.Parser2d}}\pysigline{\sphinxbfcode{class }\sphinxcode{grids.parser.parser\_2d.}\sphinxbfcode{Parser2d}}
Cython wrapper for the C++ class \sphinxcode{parser\_2d\_cpp}.
\begin{quote}\begin{description}
\item[{Variables}] \leavevmode
\sphinxstyleliteralstrong{parser} (\sphinxstyleliteralemphasis{obj}) \textendash{} Pointer to a \sphinxcode{parser\_2d\_cpp} object.

\end{description}\end{quote}
\index{evaluate() (grids.parser.parser\_2d.Parser2d method)}

\begin{fulllineitems}
\phantomsection\label{\detokenize{api:grids.parser.parser_2d.Parser2d.evaluate}}\pysiglinewithargsret{\sphinxbfcode{evaluate}}{}{}
Evaluate the expression.
\begin{quote}\begin{description}
\item[{Returns}] \leavevmode
Two-dimensional \sphinxhref{https://docs.scipy.org/doc/numpy-1.13.0/reference/generated/numpy.ndarray.html\#numpy.ndarray}{\sphinxcode{numpy.ndarray}} of the evaluations.

\end{description}\end{quote}

\end{fulllineitems}


\end{fulllineitems}



\section{Utilities}
\label{\detokenize{api:module-utils}}\label{\detokenize{api:utilities}}\index{utils (module)}
Some useful utilities.
\index{\_get\_prefix() (in module utils)}

\begin{fulllineitems}
\phantomsection\label{\detokenize{api:utils._get_prefix}}\pysiglinewithargsret{\sphinxcode{utils.}\sphinxbfcode{\_get\_prefix}}{\emph{units}}{}
Extract the prefix from the units name.
\begin{quote}\begin{description}
\item[{Parameters}] \leavevmode
\sphinxstyleliteralstrong{units} (\sphinxstyleliteralemphasis{str}) \textendash{} The units.

\item[{Returns}] \leavevmode
The prefix.

\item[{Return type}] \leavevmode
str

\end{description}\end{quote}

\end{fulllineitems}

\index{convert\_datetime64\_to\_datetime() (in module utils)}

\begin{fulllineitems}
\phantomsection\label{\detokenize{api:utils.convert_datetime64_to_datetime}}\pysiglinewithargsret{\sphinxcode{utils.}\sphinxbfcode{convert\_datetime64\_to\_datetime}}{\emph{time}}{}
Convert \sphinxcode{numpy.datetime64} to \sphinxcode{datetime.datetime}.
\begin{quote}\begin{description}
\item[{Parameters}] \leavevmode
\sphinxstyleliteralstrong{time} (\sphinxstyleliteralemphasis{obj}) \textendash{} The \sphinxcode{numpy.datetime64} object to convert.

\item[{Returns}] \leavevmode
The converted \sphinxcode{datetime.datetime} object.

\item[{Return type}] \leavevmode
obj

\end{description}\end{quote}
\paragraph{References}

\sphinxurl{https://stackoverflow.com/questions/13703720/converting-between-datetime-timestamp-and-datetime64}.

\end{fulllineitems}

\index{equal\_to() (in module utils)}

\begin{fulllineitems}
\phantomsection\label{\detokenize{api:utils.equal_to}}\pysiglinewithargsret{\sphinxcode{utils.}\sphinxbfcode{equal\_to}}{\emph{a}, \emph{b}, \emph{tol=None}}{}
Compare floating point numbers (or arrays of floating point numbers), properly accounting for round-off errors.
\begin{quote}\begin{description}
\item[{Parameters}] \leavevmode\begin{itemize}
\item {} 
\sphinxstyleliteralstrong{a} (\sphinxtitleref{float} or \sphinxtitleref{array\_like}) \textendash{} Left-hand side.

\item {} 
\sphinxstyleliteralstrong{b} (\sphinxtitleref{float} or \sphinxtitleref{array\_like}) \textendash{} Right-hand side.

\item {} 
\sphinxstyleliteralstrong{tol} (\sphinxtitleref{float}, optional) \textendash{} Tolerance.

\end{itemize}

\item[{Returns}] \leavevmode
\sphinxcode{True} if \sphinxcode{a} is equal to \sphinxcode{b} up to \sphinxcode{tol}, \sphinxcode{False} otherwise.

\item[{Return type}] \leavevmode
bool

\end{description}\end{quote}

\end{fulllineitems}

\index{get\_factor() (in module utils)}

\begin{fulllineitems}
\phantomsection\label{\detokenize{api:utils.get_factor}}\pysiglinewithargsret{\sphinxcode{utils.}\sphinxbfcode{get\_factor}}{\emph{units}}{}
Convert units prefix to the corresponding factor.
For the conversion, the \sphinxhref{http://cfconventions.org/}{CF Conventions} are used.
\begin{quote}\begin{description}
\item[{Parameters}] \leavevmode
\sphinxstyleliteralstrong{units} (\sphinxstyleliteralemphasis{str}) \textendash{} The units.

\item[{Returns}] \leavevmode
The factor.

\item[{Return type}] \leavevmode
float

\end{description}\end{quote}

\end{fulllineitems}

\index{greater\_or\_equal\_than() (in module utils)}

\begin{fulllineitems}
\phantomsection\label{\detokenize{api:utils.greater_or_equal_than}}\pysiglinewithargsret{\sphinxcode{utils.}\sphinxbfcode{greater\_or\_equal\_than}}{\emph{a}, \emph{b}, \emph{tol=None}}{}
Compare floating point numbers (or arrays of floating point numbers), properly accounting for round-off errors.
\begin{quote}\begin{description}
\item[{Parameters}] \leavevmode\begin{itemize}
\item {} 
\sphinxstyleliteralstrong{a} (\sphinxtitleref{float} or \sphinxtitleref{array\_like}) \textendash{} Left-hand side.

\item {} 
\sphinxstyleliteralstrong{b} (\sphinxtitleref{float} or \sphinxtitleref{array\_like}) \textendash{} Right-hand side.

\item {} 
\sphinxstyleliteralstrong{tol} (\sphinxtitleref{float}, optional) \textendash{} Tolerance.

\end{itemize}

\item[{Returns}] \leavevmode
\sphinxcode{True} if \sphinxcode{a} is greater than or equal to \sphinxcode{b} up to \sphinxcode{tol}, \sphinxcode{False} otherwise.

\item[{Return type}] \leavevmode
bool

\end{description}\end{quote}

\end{fulllineitems}

\index{greater\_than() (in module utils)}

\begin{fulllineitems}
\phantomsection\label{\detokenize{api:utils.greater_than}}\pysiglinewithargsret{\sphinxcode{utils.}\sphinxbfcode{greater\_than}}{\emph{a}, \emph{b}, \emph{tol=None}}{}
Compare floating point numbers (or arrays of floating point numbers), properly accounting for round-off errors.
\begin{quote}\begin{description}
\item[{Parameters}] \leavevmode\begin{itemize}
\item {} 
\sphinxstyleliteralstrong{a} (\sphinxtitleref{float} or \sphinxtitleref{array\_like}) \textendash{} Left-hand side.

\item {} 
\sphinxstyleliteralstrong{b} (\sphinxtitleref{float} or \sphinxtitleref{array\_like}) \textendash{} Right-hand side.

\item {} 
\sphinxstyleliteralstrong{tol} (\sphinxtitleref{float}, optional) \textendash{} Tolerance.

\end{itemize}

\item[{Returns}] \leavevmode
\sphinxcode{True} if \sphinxcode{a} is greater than \sphinxcode{b} up to \sphinxcode{tol}, \sphinxcode{False} otherwise.

\item[{Return type}] \leavevmode
bool

\end{description}\end{quote}

\end{fulllineitems}

\index{reverse\_colormap() (in module utils)}

\begin{fulllineitems}
\phantomsection\label{\detokenize{api:utils.reverse_colormap}}\pysiglinewithargsret{\sphinxcode{utils.}\sphinxbfcode{reverse\_colormap}}{\emph{cmap}, \emph{name=None}}{}
Reverse a Matplotlib colormap.
\begin{quote}\begin{description}
\item[{Parameters}] \leavevmode\begin{itemize}
\item {} 
\sphinxstyleliteralstrong{cmap} (\sphinxstyleliteralemphasis{obj}) \textendash{} The \sphinxhref{https://matplotlib.org/2.1.1/api/\_as\_gen/matplotlib.colors.LinearSegmentedColormap.html\#matplotlib.colors.LinearSegmentedColormap}{\sphinxcode{matplotlib.colors.LinearSegmentedColormap}} to invert.

\item {} 
\sphinxstyleliteralstrong{name} (\sphinxtitleref{str}, optional) \textendash{} The name of the reversed colormap. By default, this is obtained by appending ‘\_r’ to the name of the input colormap.

\end{itemize}

\item[{Returns}] \leavevmode
The reversed \sphinxhref{https://matplotlib.org/2.1.1/api/\_as\_gen/matplotlib.colors.LinearSegmentedColormap.html\#matplotlib.colors.LinearSegmentedColormap}{\sphinxcode{matplotlib.colors.LinearSegmentedColormap}}.

\item[{Return type}] \leavevmode
obj

\end{description}\end{quote}
\paragraph{References}

\sphinxurl{https://stackoverflow.com/questions/3279560/invert-colormap-in-matplotlib}.

\end{fulllineitems}

\index{set\_namelist() (in module utils)}

\begin{fulllineitems}
\phantomsection\label{\detokenize{api:utils.set_namelist}}\pysiglinewithargsret{\sphinxcode{utils.}\sphinxbfcode{set\_namelist}}{\emph{user\_namelist=None}}{}
Place the user-defined namelist module in the Python search path.
This is achieved by physically copying the content of the user-provided module into GT4ESS\_ROOT/namelist.py.
\begin{quote}\begin{description}
\item[{Parameters}] \leavevmode
\sphinxstyleliteralstrong{user\_namelist} (\sphinxstyleliteralemphasis{str}) \textendash{} Path to the user-defined namelist. If not specified, the default namelist GT4ESS\_ROOT/\_namelist.py is used.

\end{description}\end{quote}

\end{fulllineitems}

\index{smaller\_or\_equal\_than() (in module utils)}

\begin{fulllineitems}
\phantomsection\label{\detokenize{api:utils.smaller_or_equal_than}}\pysiglinewithargsret{\sphinxcode{utils.}\sphinxbfcode{smaller\_or\_equal\_than}}{\emph{a}, \emph{b}, \emph{tol=None}}{}
Compare floating point numbers (or arrays of floating point numbers), properly accounting for round-off errors.
\begin{quote}\begin{description}
\item[{Parameters}] \leavevmode\begin{itemize}
\item {} 
\sphinxstyleliteralstrong{a} (\sphinxtitleref{float} or \sphinxtitleref{array\_like}) \textendash{} Left-hand side.

\item {} 
\sphinxstyleliteralstrong{b} (\sphinxtitleref{float} or \sphinxtitleref{array\_like}) \textendash{} Right-hand side.

\item {} 
\sphinxstyleliteralstrong{tol} (\sphinxtitleref{float}, optional) \textendash{} Tolerance.

\end{itemize}

\item[{Returns}] \leavevmode
\sphinxcode{True} if \sphinxcode{a} is smaller than or equal to \sphinxcode{b} up to \sphinxcode{tol}, \sphinxcode{False} otherwise.

\item[{Return type}] \leavevmode
bool

\end{description}\end{quote}

\end{fulllineitems}

\index{smaller\_than() (in module utils)}

\begin{fulllineitems}
\phantomsection\label{\detokenize{api:utils.smaller_than}}\pysiglinewithargsret{\sphinxcode{utils.}\sphinxbfcode{smaller\_than}}{\emph{a}, \emph{b}, \emph{tol=None}}{}
Compare floating point numbers (or arrays of floating point numbers), properly accounting for round-off errors.
\begin{quote}\begin{description}
\item[{Parameters}] \leavevmode\begin{itemize}
\item {} 
\sphinxstyleliteralstrong{a} (\sphinxtitleref{float} or \sphinxtitleref{array\_like}) \textendash{} Left-hand side.

\item {} 
\sphinxstyleliteralstrong{b} (\sphinxtitleref{float} or \sphinxtitleref{array\_like}) \textendash{} Right-hand side.

\item {} 
\sphinxstyleliteralstrong{tol} (\sphinxtitleref{float}, optional) \textendash{} Tolerance.

\end{itemize}

\item[{Returns}] \leavevmode
\sphinxcode{True} if \sphinxcode{a} is smaller than \sphinxcode{b} up to \sphinxcode{tol}, \sphinxcode{False} otherwise.

\item[{Return type}] \leavevmode
bool

\end{description}\end{quote}

\end{fulllineitems}



\chapter{Indices and tables}
\label{\detokenize{index:indices-and-tables}}\begin{itemize}
\item {} 
\DUrole{xref,std,std-ref}{genindex}

\item {} 
\DUrole{xref,std,std-ref}{modindex}

\item {} 
\DUrole{xref,std,std-ref}{search}

\end{itemize}


\renewcommand{\indexname}{Python Module Index}
\begin{sphinxtheindex}
\def\bigletter#1{{\Large\sffamily#1}\nopagebreak\vspace{1mm}}
\bigletter{g}
\item {\sphinxstyleindexentry{grids.topography}}\sphinxstyleindexpageref{api:\detokenize{module-grids.topography}}
\indexspace
\bigletter{n}
\item {\sphinxstyleindexentry{namelist}}\sphinxstyleindexpageref{api:\detokenize{module-namelist}}
\indexspace
\bigletter{u}
\item {\sphinxstyleindexentry{utils}}\sphinxstyleindexpageref{api:\detokenize{module-utils}}
\end{sphinxtheindex}

\renewcommand{\indexname}{Index}
\printindex
\end{document}